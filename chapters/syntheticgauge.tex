\chapter[Review of synthetic gauge fields in photonics]{Review of synthetic gauge fields in photonics\footnote{This chapter is reproduced in part with permission, from Hey \& Li \textit{``Advances in Synthetic Gauge Fields through Dynamic Modulation"} 2017, submitted to \textit{Optics Communications} for publication.}}
\label{chapter:synthetic}
To achieve fine-control over light one may use so-called synthetic gauge fields. A synthetic gauge field is the tailoring of specific conditions such that some quantity of neutral particles emulates the dynamics of charged particles in a magnetic field. The rotation of a trapping of neutral atoms constitutes one such field: the Coriolis force, $F_C = -2 m \Omega \times \bm{v}$ couples to the atoms in a manner analogous to the classical Lorentz force, $F_L = q(\bm{v} \times \bm{B})$  \cite{Dalibard2015b}. Conversely, the mechanism for generating a synthetic gauge field for cold atoms relies on changing some internal degree of freedom to impart a phase to the wave function over time \cite{Dalibard2011b}. Similar synthetic gauge fields have also been explored in opto-mechanics \cite{Walter2016, Yang2017}  and acoustics \cite{Miri2017a,Yang2016b}. This lends itself to the question of whether it is possible to achieve similar conditions for light, and has motivated significant research into the field of synthetic gauge fields in photonics \cite{Fang2013b,Fan2015}. 

These synthetic gauge fields have proved instrumental in the nascent field of topological photonics, where a transfer of edifice from electronic systems has created a wealth of research ideas, ranging from the observation of protected edge states of light, where light moves along the edge of a system \cite{Wang2009b,Poshakinskiy2014,Rechtsman2013c,Raghu2008,Barik2016} to Floquet topological insulators, devices that are characterised by edge states immune to disorder \cite{Chen2014,Lumer2013,Zhang2015,Leykam2016,Maczewsky2016,Khanikaev2013}. In photonics, spatial periodicity of a lattice is combined with a synthetic gauge field, leading to 2D energy bands that are distinguished by the topological invariant known as the first Chern number. Suitable engineering of synthetic gauge fields thus has the potential to explore higher dimensional topologies, and the first experimental work on 3D lattices has unveiled particularly intriguing topological features, including the elusive Weyl points. Going further, recent work has begun on examining 4D topologically non-trivial effects of light \cite{Ozawa2016}.

\section{Inducing a photonic gauge field}

\label{sec:inducing}
Originally, the concept of generating a synthetic gauge field for photons was limited to a static ring resonator lattice that is carefully engineered to impart direction-dependent phases to photons with opposite spins \cite{Umucallar2011a}. On a single dielectric ring, light will resonate when it constructively interferes with itself after making a full round-trip \cite{Bogaerts2012}, in what are known as `whispering gallery' modes, so named for their acoustic origin in St Paul's cathedral where whispered sounds propagate along the circumference of the circular interior. In a periodic array of these resonators, it is possible for photons to `hop' between different rings, in a manner that is analogous to electrons tunnelling between atoms in a crystal. As a result, the edge modes of opposite spins propagate in opposite directions, realising a photonic analogue of the quantum spin Hall effect, a phenomena typically associated with electrons that is characterised by propagation only on the surface of a material. However, structures engineered in this way are susceptible to backscattering, since TR-symmetry is not broken. 


\begin{figure}[t]
	\centering
	\def\svgwidth{0.5\textwidth}
	\begin{normalsize}
		\input{figures/DynamicLattice.pdf_tex}
	\end{normalsize}
	\caption[A 2D dynamically modulated lattice of photonic resonators]{A 2D dynamically modulated lattice of photonic resonators, with two square sub-lattices of respective frequencies $\omega_A$ (red) and $\omega_B$ (blue). There is nearest neighbour coupling so that photons can only ‘hop’ between points via suitable modulation, acquiring a non-reciprocal phase. The phase on the horizontal bonds is $0$ (not shown), whereas the phase on the vertical bonds is linearly proportional to the column index, with a sign flip every 2nd index. This modulation choice corresponds to the Landau gauge. Image re-drawn from \cite{Fang2012}.}
	\label{fig:dynamiclattice}
\end{figure}
 
By applying a \textit{time-dependent} harmonic modulation to a photonic systems refractive index, it is possible to break this TR-symmetry and introduce a gauge field for light \cite{Fang2012}, as in Figure \ref{fig:dynamiclattice}. This process, known as dynamic modulation was first conceived by Fang et al. and later extended to a network of resonating lattices \cite{Fang2012a}. In applying the modulation, the phase of light becomes direction-dependent. This phase is strictly non-reciprocal, and consequently, possesses a broken TR-symmetry; reverting the direction of propagation will reverse the sign of phase. As far as the photon is concerned, the phase of some wave function cannot be directly measured, only differences in phase, and thus represents a gauge degree of freedom. This phase possesses the same gauge ambiguity as that of the phase acquired in an electronic gauge potential along an open path. Likewise, reverting the direction of propagation results in a change of sign. Consequently, this phase has the same properties as the electronic Aharonov-Bohm (AB) phase, so that demonstrating non-reciprocity is equivalent to showing the existence of a gauge potential for photons.  

\subsection{The electronic Aharonov-Bohm effect}

\begin{figure}[t]
	\centering
	\def\svgwidth{0.8\textwidth}
	\begin{normalsize}
		\input{figures/lasttest.pdf_tex}
	\end{normalsize}
	\caption[Aharonov-Bohm effect for charged particles]{The Aharonov-Bohm effect for electrons. An infinitely long solenoid has its magnetic flux confined entirely within its core. Surrounding the cylinder, the $\bm{B}$ field vanishes but the magnetic vector potential $\bm{A}$ remains. The potential causes the electrons to gain a negative or positive phase depending on their path through the potential.}
	\label{fig:abeffectfinal}
\end{figure}

Classically, for charged particles the AB effect is characterised by the acquisition of phase of a charged particle as it traverses a path-dependent potential where both the electric $\bm{E}$ and magnetic $\bm{B}$ fields vanish as in Figure \ref{fig:abeffectfinal} \cite{Aharonov1959}. The magnetic vector potential of the fields is non-vanishing in this region and mediates interactions with the magnetic field. In circulating around the potential, the electron acquires a positive or negative phase which can only be observed through interference of both paths. The Aharonov-Bohm effect demonstrates the importance of the gauge potential in quantum systems, and in fact is a prerequisite for the existence of such potentials \cite{Aharonov1959}. The AB effect has been widely demonstrated for electrons and other charged particles \cite{Chambers1960}, however it was only recently shown by Fang et al. (2013) that the gauge potential from non-reciprocal phases of light can be exploited to demonstrate an optical equivalent to the AB effect for light at radio frequencies \cite{Fang2013c}.
 

 \subsection{An effective magnetic field for light}
 Non-reciprocal phases of light in time-varying systems have been shown to give rise to an important phenomenon - effective magnetic fields for light. Fang et al (2012) first demonstrated that an effective magnetic field for light can arise if the phase of the light were non-reciprocal \cite{Fang2012a}. The effective magnetic field is directly linked to the idea of a gauge potential, similar to how the classical electromagnetic field $\bm{B}$ arises from the magnetic vector potential $\bm{A}$. In fact, the non-reciprocal phase shift acquired by light in a moving medium discussed in the beginning of section \ref{mome} satisfies the requirements for generating a gauge potential. This is incredibly useful from an optical standpoint, as photons, being neutral particles, do not naturally couple to magnetic fields.
 
 Fang et al. later went on to demonstrate that such effective magnetic fields can arise from the aforementioned dynamic modulation of a ring resonator lattice \cite{Fang2013}. By imposing a specific arrangement of non-reciprocal phases for light in the structure, the resulting arrangement corresponds to a uniform magnetic field for light. Despite acknowledging the possibility of the harmonic modulation to give rise to any kind of directed effective field, Fang et al. only simulated a uniform field causing the light to move in circular motion. Additionally, practical creation of a lattice with the required levels of finesse are noted to be highly difficult from an engineering standpoint.
 
 Other non-reciprocal phase induced fields also consider an array of resonators that impose a static direction dependent phase \cite{Haldane2008b}. However, all of these suffer from a similar flaw - practical implementation of the resonators is incredibly difficult, and the final system is static in nature, in contrast to dynamically modulated systems. It has been noted by Lin and Fan (2015) that the resonators themselves provide no essential role with respect to the effective field generated, beyond simplifying the theoretical treatment \cite{Lin2015a}. As a result, they have derived a 'resonator-free' system that is based instead on a purely two-dimensional network of waveguides.  
 
 Resonators and waveguides are not the only methods of inducing a non-reciprocal phase (and consequently, a gauge potential for light). Li et al. (2014) demonstrated that photon-phonon interactions within acousto-optic crystals introduces a non-reciprocal phase for light \cite{Li2014a}. From this, they demonstrated an optical Aharonov-Bohm effect at visible frequencies, one the first of such kind observed through non-reciprocal phases. The possibility of realising an effective magnetic field for photons gives rise to an important number of effects that are similar to phenomena exhibited by electrons in a magnetic fields, such as a Lorentz force \cite{Fang2012}, quantum Hall effects, and topologically protected one-way edge modes \cite{Fang2013}. While some of these effects have been noted to be possible with magneto-optics \cite{Wang2009, Lee2017a}, dynamic modulation is both more versatile and offers a greater coupling strength at optical frequencies.
 
 \subsection{Fluid moving media}
 
 The idea of exploiting non-reciprocal phase shifts to demonstrate an AB effect has been naturally extended to the simpler cases of light in moving media. Vieira et al (2014) discussed an implementation for inducing non-reciprocal phases in light by submerging a rotating cylinder in a viscous medium. The rotation of the cylinder imparts an angular \textit{biasing} to the medium, which creates a non-reciprocal phase for the light by mediating interactions between the cylinder and light in a manner similar to how the magnetic vector potential mediates interactions with the magnetic field \cite{Vieira2014a}. In this method, the effective magnetic field for the light is given by the vorticity (the local rotation of the vector field) of the medium in which it is present, again similar to how the magnetic field is the curl (or vorticity) of the magnetic potential. 
 
 Vieira's result is not the first for wavelike phenomena in rotating media. It has previously been shown by de Rosny et al. (2005) that a rotational flow can break the reciprocal transmission of ultrasound waves, along with Leonhardt's \cite{Leonhardt1999, Leonhardt2000c, Leonhardt2000a} pioneering formulations of the effective gauge potentials for light. In Leonhardt's work, not only was a possible AB effect suggested, but it was also shown that a medium with high enough viscosity and refractive index can bend and trap light similarly to a black hole, a result that has been subsequently proven in multiple experiments \cite{Genov2011}.
 
\section{Topological photonics \label{topology}}
\label{sec:topology}

Dynamically modulated resonator lattices also support one-way propagation along the edges (edge modes). However, time-harmonic modulations of more than a few resonators simultaneously is a challenging feat at optical wavelengths \cite{Tzuang2014}. Rechtsman et al. (2013) showed that it is possible to shift the modulation from the frequency to the spatial domain to observe photonic analogues of the quantum Hall effect at optical frequencies \cite{Rechtsman2013b}, corresponding to the first experimentally viable Floquet topological insulators \cite{Lindner2011}. Instead of breaking TR-symmetry in their system, the authors instead broke spatial symmetry along the $z$-axis, and identified their system as being suitably analogous to a breaking of TR-symmetry, culminating in protected edge modes. 

Most previous works have focused solely on the regime where the modulation strength is far less than the modulation frequency \cite{Lee2017a}, allowing for the application of the rotating-wave approximation (RWA), where slowly varying terms are ignored. On the other hand, there has been recent interest in light-matter interactions in the ultra-strong coupling regime \cite{Fedortchenko2016,Hirokawa2017}, where the RWA is no longer valid. In such ultra-strong coupling systems, it has been shown that topologically protected one-way edge states in dynamic modulation are less susceptible to intrinsic losses \cite{Fan2015}. In addition to this, a unique topological phase transition was found to be associated with variations in the modulation strength \cite{Yuan2015a}. Such phase transitions are not found in rotating-wave counterparts. 

\subsection{Synthetic dimensions}

Studies in topology have long hinted at rich possibilities of physics in higher dimensions, namely analogues of the quantum Hall effect in even-dimensional spaces \cite{Zhang2001}. Traditional systems in condensed matter physics however, are locked out of examining such effects. Only recently have new approaches in simulating higher dimensional topological models, so-called ‘synthetic dimensions’ have been proposed, that generate a gauge field for neutral particles by exploiting additional degrees of freedom \cite{Price2017,Saito2017,Barbarino2016}. These ideas were first proposed in ultra-cold atomic gases\cite{Stuhl2015a,Boada2012}, where the atoms possess an internal spin degree of freedom. Photons naturally possess many internal degrees of freedom: frequency, orbital angular momentum, spin angular momentum, polarization, and so on. These additional degrees of freedom can form synthetic lattice dimensions for light, leading to possible experimental analysis of higher dimensional topological photonics. Due to the long-range interactions of light in synthetic dimensions however, there is a difficulty associated with exploring quantum Hall effects \cite{acki2016,Zeng2015}.
  
Here, a basic method of achieving synthetic dimensions via dynamic modulation is demonstrated. Consider one of the most elementary photonic structures - the ubiquitous ring resonator. In the absence of group velocity dispersion (where the group velocity depends on the wavelength), the single ring supports a set of equally spaced resonant frequencies. The spacing of the modes $\Omega$ is based on the spectral range of the resonator, and is related to the round-trip time $T$ by $\Omega =  2 \pi /T$.

\begin{figure}[t]
	\centering
	\def\svgwidth{0.5\textwidth}
	\begin{normalsize}
		\input{figures/1DSynthetic.pdf_tex}
	\end{normalsize}
	\caption[Modulated ring resonator with a synthetic dimension]{A ring resonator undergoing dynamic modulation of frequency $\Omega_M$, which supports several equally spaced resonant modes. The resonator modes form a synthetic dimension along the frequency axis, and by transitioning the light between different frequencies, it is possible to examine \textit{1D} physics on the \textit{0D} ring structure. Although the light is propagating effectively in $3D$ in the ring, the single mode can be considered to be point-like $0D$.}
	\label{fig:syntheticring}
\end{figure} 

By dynamically modulating the ring structure, as in Figure \ref{fig:syntheticring}, modes on the ring undergo transitions to higher order sidebands, separated by the modulation frequency $\Omega_M$. If the modulation frequency is then chosen to be equal to the mode spacing of the resonator $\Omega$, there will be a resonant coupling of the modes that are separated by the free-spectral range of the ring (the frequency spacing naturally supported by the ring). Consequently, this system can be described by a \textit{1D} tight-binding model, despite being a \textit{0D} structure, as the spacing of the frequencies represent an additional degree of freedom for the light. This can naturally be extended to the case of a $1D$ array of ring resonators that are coupled together to form a waveguide \cite{Yuan2016c}. Each ring has a controlled phase of modulation, which gives rise to \textit{2D} physics model on a $1D$ structure. The phase of the modulation corresponds then to the hopping phase along the frequency axis.

It is possible to generate a topologically non-trivial bulk correspondence by considering boundaries in real and synthetic space. In real space, the boundary is given by the physical edges of the ring. However, in frequency space, a boundary is introduced by the group velocity dispersion (GVD) of the ring instead (how the ring affects the duration of optical pulses propagating within it). Around the zero-GVD point, most frequencies are equally spaced so the modulators induce an on-resonance coupling. On the other hand, away from the zero-GVD point, the frequencies are no longer equally spaced and thus cannot support coupling, leading to an effective \textit{boundary} in frequency space. Thus, light propagating along the system will encounter a boundary composed of the real physical size of the rings in combination with the GVD. Finally, we note that outside of applications in topological photonics discussed below, synthetic dimensions in dynamic modulation have also recently been shown to support efficient frequency manipulation of light \cite{Qin2017}.

\subsection{Weyl points}

\begin{figure}[t]
\centering
\def\svgwidth{0.8\textwidth}
\begin{normalsize}
\input{figures/3DDynamicMod.pdf_tex}
\end{normalsize}
\caption[A $3D$ synthetic resonator lattice]{An example resonator structure for examining 4D topological photonics. Dynamic coupling between rings allows for $3D$ propagation. The supported frequencies of the resonators, $\omega$, as usual provide the synthetic dimension allowing for the possibility of examining $4D$ topological physics with light.}
\label{fig:synthetic3d}
\end{figure}

In condensed matter systems, the Weyl point describes a magnetic monopole in momentum space, that is, a `source' or `sink' of Berry curvature. As such, the Weyl point is perceived as topological nodal points in 3D momentum space. Weyl points are topologically robust, in the sense that they cannot be destroyed by any perturbation that preserves translational symmetry. The Hamiltonian of a Weyl system is
$H=k_x \sigma_x+k_y \sigma_y+k_z \sigma_z+E_0 I$
where $\sigma_{x,y,z}$ are the Pauli spin matrices, and $I$ is the \textit{2x2} identity matrix, which together form a complete basis for \textit{2x2} Hermitian matrices. There has been a massive effort devoted to investigating Weyl points and their associated phenomena in electronic systems \cite{Gupta2017,Spivak2016}, and have also been found in photonic crystals \cite{Noh2017,Xiao2016}, as well as acoustic \cite{Yang2016a} and plasmonic structures \cite{Gao2016}. 

To explore a Weyl point in a planar \textit{2D} geometry, one may use a synthetic dimension to simulate the third spatial dimension, by again considering arrays of ring resonators – this time arranged in a honeycomb lattice. The size of the synthetic dimension, which corresponds to the number of modes in each individual ring can be chosen to be almost arbitrarily large without increasing the system complexity. In a recent paper, Lin et al suggested that it is possible to investigate a synthetic 3D space through dynamic modulation of \textit{2D} on-chip ring resonators \cite{Lin2017}. Each resonator supports a set of discrete modes equally spaced in resonant frequency. The synthetic dimension then originates from these discrete modes, manifesting as a synthetic periodic lattice so that the entire system can be envisaged as a 3D space. This on-chip dynamic modulation is readily feasible, as compared with previous complex electromagnetic or acoustic structures for realizing Weyl points \cite{Lu2015,Rechtsman2013}, a planar 2D approach is more flexible owing to the dynamic nature of the system, and can be achieved with the previously discussed extension of dynamic modulation in synthetic dimensions, allowing for on-chip investigation of Weyl points. 

\subsection{Towards 4D photonics}
Naturally, the quantum Hall effect was first generalized to higher dimensions for electronic systems \cite{Zhang2001}. It is a standard extension of our discussion on 3D photonic systems to then consider a 3D array of resonators in the three spatial dimensions $x$, $y$, $z$ and then extending into the synthetic frequency dimensions $\omega$ to probe elusive 4D quantum Hall effects.

A 4-dimensional model making use of 3D resonating lattices with plaquettes has only recently been proposed by Price et al., based on the authors previous works in 4D quantum Hall effects for ultra-cold atoms \cite{Price2015}. This model differs to a potential synthetic dimension with dynamic modulation in Figure \ref{fig:synthetic3d}, as their lattice model does not break TR-symmetry. Because of this, photons with positive angular momentum can backscatter into states with a negative angular momentum. However, it has been noted that such backscattering can be minimised during fabrication \cite{Hafezi2013a}. Finally, we note that achieving a 4D topological effect for light would not only be possible under a dynamic modulation regime, but also would involve a broken TR-symmetry, leading to one-way edge modes propagating on the boundaries of a 3D frequency space.
