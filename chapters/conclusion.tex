\phantomsection\addcontentsline{toc}{chapter}{Conclusion}
\chapter*{Conclusion}
In this thesis, full-wave finite difference time and frequency domain methods were developed for the investigation of non-reciprocal phase and mode conversions in a medium undergoing dynamic modulation. The frequency domain solution was shown to provide a remarkable increase in convergence time over the time domain solution through preliminary benchmarks, and is uniquely suited to simulating active nanophotonic devices. Since finite difference methods are suitable to solve a large number of differential equations (not being limited to just Maxwell's equations), the multi-frequency method is ideal for simulating devices with arbitrarily separated frequencies. The frequency domain simulations of the photonic gauge potential emerging from modulation marks the first of their kind outside the time-domain. Owing to the slow convergence time of FDTD, previous works relied on a combination of FDTD and coupled-mode theory to probe the dynamics of systems undergoing modulation. However, the frequency domain method is capable of obtaining these dynamics at a much faster speed, and is obviously appealing for rapid prototyping of devices in photonics.

The case of a moving media and a dynamically modulated waveguide were also shown to generate a non-reciprocal phase for light, leading to an invariance that is typical of gauge potentials. In particular, a photonic transition between two states was shown to correlate perfectly with electrons hopping between sites on a $1D$ tight-binding lattice under the influence of a magnetic vector potential. Consequently, the phase in the dynamic modulation scheme is directly equivalent to a gauge degree of freedom. On the macroscopic scale, a similar effect was derived for light in a rotating dielectric.

Non-reciprocal frequency conversion was demonstrated numerically in the case of a spatio-temporal refractive index modulation. However, as applying spatial modulations is significantly more difficult to engineer than temporal modulations, the gauge potential emerging from the phase of dynamic modulation was used to demonstrate a photonic AB effect that was characterised by complete non-reciprocal mode conversion. Despite the photonic AB effect using direct transitions, which are inherently reciprocal, the gauge potential resulted in a non-reciprocal mode conversion between the LR, RL, and TR paths. In comparison to the indirect transition which required a spatio-temporal modulation, the direct transition is far simpler to achieve using current modulation technology. Characteristic oscillations emerging from the direct modulation were also shown to correspond to highly phase mismatched backward propagating modes. The choice of modulation scheme for the AB interferometer forms the absolute minimum number of modulation regions that can be used to generate a gauge potential for light.

Photonic circuits are currently hampered by imperfections in the manufacturing process, so a method to bypass flaws via synthetic gauge fields would be hugely significant. Exploitation of topological effects dramatically improves the robustness of current photonic devices, including optical isolators, which currently require magneto-optic coupling effects and are largely incompatible with integrated photonics. Likewise, the topologically non-trivial properties of light could solve key limitations involving the disorder of light in waveguides and potentially improve coherence in quantum computing.  In-fact, protected edge states arising from TR-invariant topological phases would be one-way, even in reciprocal systems, making current optical isolators unnecessary. The main challenge in dynamic modulation schemes mostly lies within engineering difficulties associated with modulating coupled resonators on the micro-scale. 

Going further, the recent introduction of synthetic dimensions is potentially an important step towards realising higher-dimensional topology in photonics. The investigation of Weyl points on a 2D planar structure makes it far more experimentally viable than current architectures, and is amenable with integrated photonics in both the infrared and visible wavelengths. The idea of Weyl points in 2D systems with synthetic dimensions can be implemented in more than just dynamically modulated honeycomb lattice systems. Additionally, a 4D quantum hall effect proposed, based on a 3D coupled resonator system with an additional frequency dimension, allows for unique photon-photon interactions in 3D  systems, and could potentially realise fractional quantum hall physics for light. Exciting applications of these approaches would obviously be the observation of a 4D quantum Hall effect experimentally. Further, the edge physics associated with a 4D Hall effect would be expected to propagate along the 3D surface of a 4D boundary. Such modes would have useful applications in future optical components, paving the way towards a complete integrated photonic platform for manipulating light at the microscale. 


\phantomsection\addcontentsline{toc}{chapter}{Future work}
\chapter*{Future work}
Experimental demonstration of non-reciprocity would be ideal, and could be accomplished using the theoretical procedures derived in Chapter 2 for either moving media or dynamic modulation. In particular, the macroscopic AB effect from the rotating dielectric is promising, despite being incompatible with integrated photonics. Scaling the photonic AB effect in dynamic modulation up to the range of radio frequencies would allow for experimentally feasible investigation of the synthetic gauge potentials in modulation.

On the simulation front, more rigorous benchmarks of the time and frequency domain simulations are desirable. In particular, instead of sampling one point in the simulation space to obtain the steady-state convergence time for FDTD, all points would be sampled concurrently. Although this would dramatically increase the convergence time, the resulting benchmark would be far more rigorous. Additionally, the impact of the number of sidebands on the convergence in FDFD could be identified. 
 
Rewriting the main FDFD functions in a naturally faster language, such as \textit{C++} would net a significant increase in speed, as well as writing a customised matrix inversion solver that uses iterative techniques instead of \textit{MATLAB's} built-in UMFPACK. UMFPACK is a general purpose matrix solver, and thus it takes several steps of analysis before it decides on a suitable iteration method. Since the wave matrix is highly ill-conditioned, a preconditioning scheme could be developed before inversion is performed. Alternative iterative techniques could include the commonly used biconjugate gradient method \cite{Bank1993}, or the quasi-minimal residual method \cite{Freund1991}. Likewise, the simulation is readily extensible to three dimensions, however at this point any performance gains through the use of FDFD are largely outweighed by the exponential increase in complexity that comes with simulating extra dimensions. Regardless, recent work by Shin et al. (2012) has shown that several methods can be used to greatly increase the speed of FDFD methods by optimal choice of the PML layer type, making it amenable to \textit{3D} implementation \cite{Shin2012}. 