\chapter{Finite difference frequency domain}

The finite difference frequency domain method is a complementary numerical method of solving Maxwell's equations as an alternative to time domain simulations. 

There are alternatives, however, to obtaining full-wave solutions to Maxwell's equations. Namely, by a suitable transformation into the frequency domain, the problem of time-step based simulations instead becomes one of solving a large system of linear equations in the form

\begin{equation}
Ax = b
\end{equation}

where $A$ is the \textit{wave matrix}, $x$ is a vector that contains information on the electromagnetic fields, and $b$ is a source. Fortuitously, in two dimensions the most efficient method of solving for the field distributions is through the elementary matrix inversion,

\begin{equation}
x = A^{-1}b,
\end{equation}
 
which has the trivial solution of $x=0$ for a zero source simulation ($b=0$). In solving for $x$, one obtains the steady-state solution directly embedded in the matrix, however the frequency domain formulation can only handle one frequency at a time.


\section{The multi-frequency method (MF-FDFD)}
Here, we adapt the frequency domain method to multiple simultaneous frequencies.
Consider the frequency domain Maxwell equations;

\begin{equation}
\nabla \times \mu(\omega)^{-1} - \omega^2 \epsilon_s (\omega) \bm{E}(\omega) - \omega^2 \bm{P}(\omega) = -i \omega \bm{J}(\omega)
\end{equation}

where $\bm{E}(\omega)$ is the electric field distribution at frequency $\omega$, and $\mu(\omega)$ and $\epsilon_s(\omega)$ are the frequency dependent permeability and permittivity respectively. 

Now, we assume a region of the simulation space is altered by the standard permittivity modulation of the form

\begin{equation}
\epsilon(t) = \epsilon_s + \delta cos(\omega t + \phi).
\end{equation}

We can account for this modulation in frequency space by considering its effects on the polarisation of the material in question,

\begin{equation}
	 \delta cos(\omega t + \phi) \bm{E}.
\end{equation}

Through Euler's identity we obtain a more suggestible form

\begin{equation}
\bm{P}(\omega) = \dfrac{\delta}{2} (e^{i(\Omega t + \phi)} + e^{-i(\Omega t + \phi)}) \bm{E}(t).
\end{equation}

Using the standard inverse Fourier transform \footnote{Note that here we are using the inverse Fourier transform to move from the time to the frequency domain, although the forward Fourier transform is just as applicable, and only results in a shifted phase.},

The generation of sidebands by the modulation ensures that the time-domain electric field $\bm{E}(t)$ is the superposition of electric field components at each sideband $n$ (through the linearity of Maxwell's equations),

\begin{equation}
\tilde{\bm{E}}(t) = \Re \{\sum_n \bm{E}(\omega_n)e^{i \omega_n t}\}
\end{equation}


We implement the finite difference method to solve the above equations.

To test the validity of the simulation, we first consider a weakly modulated waveguide ($\delta = 0.001$).  A weak modulation is chosen as it can be readily approximated through coupled mode theory, providing a comparison.

To test the suitability of FDFD, we perform a frequency domain simulation of a dynamically modulated waveguide structure over several side-bands. The waveguide is identical to that of the one simulated via time-domain methods (fig X), with $\epsilon_s = 12.25$, which supports a transverse electric $\text{TE}_{00}$ mode at $w_0 = 0.129 (\frac{2\pi c}{a})$ and a  $\text{TE}_{01}$ mode at $w_1 = 0.198(\frac{2\pi c}{a})$ (see fig Y for bandstructure). For each frequency sideband, we extract the field profile amplitude and compare to the analytical results provided by coupled-mode theory. An eigenmode source is calculated a priori (see appendix Z) much like the time-domain simulations, so that the field profile can be coupled exactly into the waveguide structure. We test both the left-to-right $LR$ and right-to-left $RL$ propagation directions to ensure that parity-symmetry is broken (as is predicted by theory), and set the normalised amplitude to 1UNITS. To test time-reversal symmetry, we inject a $\text{TE}_{01}$ mode from the right-to-left direction.

%\begin{figure}[t]
%	\centering
%	\setlength{\figH}{0.5\textwidth}
%	\setlength{\figW}{1\textwidth}
%	\begin{subfigure}[t]{0.5\textwidth}
%		% This file was created by matlab2tikz.
%
%The latest updates can be retrieved from
%  http://www.mathworks.com/matlabcentral/fileexchange/22022-matlab2tikz-matlab2tikz
%where you can also make suggestions and rate matlab2tikz.
%
\definecolor{mycolor1}{rgb}{0.00000,0.44700,0.74100}%
%
\begin{tikzpicture}

\begin{axis}[%
width=0.75\figW,
height=\figH,
at={(0\figW,0\figH)},
scale only axis,
xmin=0,
xmax=30001,
xlabel style={font=\color{white!15!black}},
xlabel={nz = 155280},
y dir=reverse,
ymin=0,
ymax=30001,
axis background/.style={fill=white}
]
\addplot [color=mycolor1, draw=none, mark size=0.7pt, mark=*, mark options={solid, mycolor1}, forget plot]
  table[row sep=crcr]{%
1	1\\
1	100\\
1	9901\\
72	172\\
78	77\\
78	9978\\
181	281\\
203	103\\
209	309\\
335	235\\
335	334\\
335	435\\
483	583\\
489	389\\
541	641\\
592	492\\
592	692\\
689	789\\
720	721\\
746	846\\
849	749\\
849	848\\
849	949\\
1003	903\\
1003	1002\\
1032	1132\\
1101	1201\\
1106	1006\\
1209	1309\\
1232	1132\\
1238	1338\\
1363	1263\\
1363	1362\\
1363	1463\\
1512	1612\\
1518	1418\\
1569	1669\\
1621	1521\\
1621	1721\\
1718	1818\\
1748	1749\\
1775	1875\\
1878	1778\\
1878	1877\\
1878	1978\\
2032	1932\\
2032	2031\\
2061	2161\\
2129	2229\\
2135	2035\\
2238	2338\\
2261	2161\\
2266	2366\\
2392	2292\\
2392	2391\\
2392	2492\\
2541	2641\\
2546	2446\\
2598	2698\\
2649	2549\\
2649	2749\\
2746	2846\\
2777	2778\\
2803	2903\\
2906	2806\\
2906	2905\\
2906	3006\\
3061	2961\\
3061	3060\\
3089	3189\\
3158	3258\\
3163	3063\\
3266	3366\\
3289	3189\\
3295	3395\\
3421	3321\\
3421	3420\\
3421	3521\\
3569	3669\\
3575	3475\\
3626	3726\\
3678	3578\\
3678	3778\\
3775	3875\\
3805	3806\\
3832	3932\\
3935	3835\\
3935	3934\\
3935	4035\\
3941	13941\\
3986	13986\\
4089	3989\\
4089	4088\\
4089	14089\\
4118	4218\\
4161	14161\\
4186	4286\\
4192	4092\\
4192	14192\\
4295	4395\\
4318	4218\\
4324	4424\\
4346	14346\\
4441	14441\\
4449	4349\\
4449	4448\\
4449	4549\\
4541	14541\\
4572	14572\\
4598	4698\\
4604	4504\\
4655	4755\\
4706	4606\\
4706	4806\\
4741	14741\\
4804	4904\\
4834	4835\\
4841	14841\\
4861	4961\\
4964	4864\\
4964	4963\\
4964	5064\\
4964	14964\\
5041	15041\\
5118	5018\\
5118	5117\\
5141	15141\\
5146	5246\\
5190	15190\\
5215	5315\\
5221	5121\\
5241	15241\\
5324	5424\\
5341	15341\\
5346	5246\\
5352	5452\\
5375	15375\\
5478	5378\\
5478	5477\\
5478	5578\\
5478	15478\\
5541	15541\\
5626	5726\\
5632	5532\\
5641	15641\\
5684	5784\\
5735	5635\\
5735	5835\\
5741	15741\\
5832	5932\\
5863	5864\\
5875	15875\\
5889	5989\\
5992	5892\\
5992	5991\\
5992	6092\\
5992	15992\\
6041	16041\\
6146	6046\\
6146	6145\\
6175	6275\\
6244	6344\\
6249	6149\\
6352	6452\\
6375	6275\\
6381	6481\\
6506	6406\\
6506	6505\\
6506	6606\\
6655	6755\\
6661	6561\\
6712	6812\\
6764	6664\\
6764	6864\\
6861	6961\\
6891	6892\\
6918	7018\\
7021	6921\\
7021	7020\\
7021	7121\\
7175	7075\\
7175	7174\\
7204	7304\\
7272	7372\\
7278	7178\\
7381	7481\\
7404	7304\\
7409	7509\\
7535	7435\\
7535	7534\\
7535	7635\\
7684	7784\\
7689	7589\\
7741	7841\\
7792	7692\\
7792	7892\\
7889	7989\\
7920	7921\\
7946	8046\\
8049	7949\\
8049	8048\\
8049	8149\\
8204	8104\\
8204	8203\\
8232	8332\\
8301	8401\\
8306	8206\\
8409	8509\\
8432	8332\\
8438	8538\\
8564	8464\\
8564	8563\\
8564	8664\\
8712	8812\\
8718	8618\\
8769	8869\\
8821	8721\\
8821	8921\\
8918	9018\\
8948	8949\\
8975	9075\\
9078	8978\\
9078	9077\\
9078	9178\\
9232	9132\\
9232	9231\\
9261	9361\\
9329	9429\\
9335	9235\\
9438	9538\\
9461	9361\\
9467	9567\\
9592	9492\\
9592	9591\\
9592	9692\\
9741	9841\\
9747	9647\\
9798	9898\\
9849	9749\\
9849	9949\\
9901	1\\
9977	9978\\
10001	10100\\
10001	19901\\
10090	19990\\
10100	10001\\
10107	10106\\
10107	10207\\
10261	10161\\
10261	10260\\
10289	10389\\
10358	10458\\
10364	10264\\
10467	10567\\
10489	10389\\
10495	10595\\
10621	10521\\
10621	10620\\
10621	10721\\
10769	10869\\
10775	10675\\
10827	10927\\
10878	10778\\
10878	10978\\
10975	11075\\
11006	11007\\
11032	11132\\
11135	11035\\
11135	11134\\
11135	11235\\
11289	11189\\
11289	11288\\
11318	11418\\
11387	11487\\
11392	11292\\
11495	11595\\
11518	11418\\
11524	11624\\
11649	11549\\
11649	11648\\
11649	11749\\
11798	11898\\
11804	11704\\
11855	11955\\
11907	11807\\
11907	12007\\
12004	12104\\
12034	12035\\
12061	12161\\
12164	12064\\
12164	12163\\
12164	12264\\
12318	12218\\
12318	12317\\
12347	12447\\
12415	12515\\
12421	12321\\
12524	12624\\
12547	12447\\
12552	12652\\
12678	12578\\
12678	12677\\
12678	12778\\
12827	12927\\
12832	12732\\
12884	12984\\
12935	12835\\
12935	13035\\
13032	13132\\
13063	13064\\
13090	13190\\
13192	13092\\
13192	13191\\
13192	13292\\
13347	13247\\
13347	13346\\
13375	13475\\
13444	13544\\
13450	13350\\
13552	13652\\
13575	13475\\
13581	13681\\
13707	13607\\
13707	13706\\
13707	13807\\
13855	13955\\
13861	13761\\
13912	14012\\
13941	3941\\
13941	23941\\
13964	3964\\
13964	13864\\
13964	14064\\
13964	23964\\
14041	4041\\
14061	14161\\
14081	4081\\
14091	14092\\
14118	14218\\
14141	24141\\
14221	14121\\
14221	14220\\
14221	14321\\
14241	4241\\
14241	24241\\
14272	24272\\
14341	4341\\
14341	24341\\
14375	4375\\
14375	14275\\
14375	14374\\
14404	14504\\
14441	24441\\
14472	14572\\
14478	4478\\
14478	14378\\
14478	24478\\
14561	4561\\
14581	14681\\
14604	14504\\
14610	14710\\
14641	4641\\
14641	24641\\
14735	14635\\
14735	14734\\
14735	14835\\
14741	4741\\
14741	24741\\
14787	24787\\
14858	24858\\
14884	14984\\
14890	4890\\
14890	14790\\
14941	15041\\
14992	4992\\
14992	14892\\
14992	15092\\
14992	24992\\
15090	15190\\
15095	5095\\
15120	15121\\
15141	5141\\
15141	25141\\
15147	15247\\
15241	25241\\
15246	5246\\
15250	15150\\
15250	15249\\
15250	15350\\
15341	25341\\
15384	5384\\
15404	15304\\
15404	15403\\
15432	15532\\
15441	5441\\
15441	25441\\
15476	25476\\
15501	15601\\
15507	15407\\
15541	5541\\
15541	25541\\
15589	5589\\
15610	15710\\
15632	15532\\
15638	15738\\
15641	5641\\
15641	25641\\
15764	5764\\
15764	15664\\
15764	15763\\
15764	15864\\
15764	25764\\
15841	25841\\
15887	25887\\
15912	16012\\
15918	15818\\
15941	5941\\
15970	16070\\
16021	15921\\
16021	16121\\
16041	6041\\
16041	26041\\
16118	16218\\
16149	16150\\
16175	16275\\
16278	16178\\
16278	16277\\
16278	16378\\
16432	16332\\
16432	16431\\
16461	16561\\
16530	16630\\
16535	16435\\
16638	16738\\
16661	16561\\
16667	16767\\
16792	16692\\
16792	16791\\
16792	16892\\
16941	17041\\
16947	16847\\
16998	17098\\
17050	16950\\
17050	17150\\
17147	17247\\
17177	17178\\
17204	17304\\
17307	17207\\
17307	17306\\
17307	17407\\
17461	17361\\
17461	17460\\
17490	17590\\
17558	17658\\
17564	17464\\
17667	17767\\
17690	17590\\
17695	17795\\
17821	17721\\
17821	17820\\
17821	17921\\
17970	18070\\
17975	17875\\
18027	18127\\
18078	17978\\
18078	18178\\
18175	18275\\
18206	18207\\
18233	18333\\
18335	18235\\
18335	18334\\
18335	18435\\
18490	18390\\
18490	18489\\
18518	18618\\
18587	18687\\
18593	18493\\
18695	18795\\
18718	18618\\
18724	18824\\
18850	18750\\
18850	18849\\
18850	18950\\
18998	19098\\
19004	18904\\
19055	19155\\
19107	19007\\
19107	19207\\
19204	19304\\
19234	19235\\
19261	19361\\
19364	19264\\
19364	19363\\
19364	19464\\
19518	19418\\
19518	19517\\
19547	19647\\
19615	19715\\
19621	19521\\
19724	19824\\
19747	19647\\
19753	19853\\
19878	19778\\
19878	19877\\
19878	19978\\
19901	10001\\
19930	10030\\
20001	20100\\
20001	29901\\
20033	29933\\
20084	20184\\
20135	20035\\
20135	20235\\
20233	20333\\
20263	20264\\
20290	20390\\
20393	20293\\
20393	20392\\
20393	20493\\
20547	20447\\
20547	20546\\
20575	20675\\
20644	20744\\
20650	20550\\
20753	20853\\
20775	20675\\
20781	20881\\
20907	20807\\
20907	20906\\
20907	21007\\
21055	21155\\
21061	20961\\
21113	21213\\
21164	21064\\
21164	21264\\
21261	21361\\
21292	21293\\
21318	21418\\
21421	21321\\
21421	21420\\
21421	21521\\
21576	21476\\
21576	21575\\
21604	21704\\
21673	21773\\
21678	21578\\
21781	21881\\
21804	21704\\
21810	21910\\
21936	21836\\
21936	21935\\
21936	22036\\
22084	22184\\
22090	21990\\
22141	22241\\
22193	22093\\
22193	22293\\
22290	22390\\
22320	22321\\
22347	22447\\
22450	22350\\
22450	22449\\
22450	22550\\
22604	22504\\
22604	22603\\
22633	22733\\
22701	22801\\
22707	22607\\
22810	22910\\
22833	22733\\
22838	22938\\
22964	22864\\
22964	22963\\
22964	23064\\
23113	23213\\
23118	23018\\
23170	23270\\
23221	23121\\
23221	23321\\
23318	23418\\
23349	23350\\
23376	23476\\
23478	23378\\
23478	23477\\
23478	23578\\
23633	23533\\
23633	23632\\
23661	23761\\
23730	23830\\
23736	23636\\
23838	23938\\
23861	23761\\
23867	23967\\
23941	13941\\
23993	13993\\
23993	23893\\
23993	23992\\
23993	24093\\
24141	24241\\
24147	14147\\
24147	24047\\
24198	24298\\
24241	14241\\
24250	24150\\
24250	24350\\
24341	14341\\
24347	24447\\
24367	14367\\
24377	24378\\
24404	24504\\
24507	24407\\
24507	24506\\
24507	24607\\
24541	14541\\
24641	14641\\
24661	14661\\
24661	24561\\
24661	24660\\
24690	24790\\
24758	24858\\
24764	14764\\
24764	24664\\
24841	14841\\
24867	24967\\
24890	24790\\
24896	24996\\
24941	14941\\
25021	24921\\
25021	25020\\
25021	25121\\
25041	15041\\
25141	15141\\
25170	25270\\
25176	15176\\
25176	25076\\
25227	25327\\
25278	15278\\
25278	25178\\
25278	25378\\
25376	25476\\
25381	15381\\
25395	15395\\
25406	25407\\
25433	25533\\
25536	25436\\
25536	25535\\
25536	25636\\
25541	15541\\
25670	15670\\
25690	15690\\
25690	25590\\
25690	25689\\
25719	25819\\
25787	25887\\
25793	15793\\
25793	25693\\
25875	15875\\
25896	25996\\
25919	25819\\
25924	26024\\
25944	15944\\
26041	16041\\
26050	25950\\
26050	26049\\
26050	26150\\
26199	26299\\
26204	26104\\
26256	26356\\
26307	26207\\
26307	26407\\
26404	26504\\
26435	26436\\
26461	26561\\
26564	26464\\
26564	26563\\
26564	26664\\
26719	26619\\
26719	26718\\
26747	26847\\
26816	26916\\
26821	26721\\
26924	27024\\
26947	26847\\
26953	27053\\
27079	26979\\
27079	27078\\
27079	27179\\
27227	27327\\
27233	27133\\
27284	27384\\
27336	27236\\
27336	27436\\
27433	27533\\
27463	27464\\
27490	27590\\
27593	27493\\
27593	27592\\
27593	27693\\
27747	27647\\
27747	27746\\
27776	27876\\
27844	27944\\
27850	27750\\
27953	28053\\
27976	27876\\
27981	28081\\
28107	28007\\
28107	28106\\
28107	28207\\
28256	28356\\
28261	28161\\
28313	28413\\
28364	28264\\
28364	28464\\
28461	28561\\
28492	28493\\
28519	28619\\
28621	28521\\
28621	28620\\
28621	28721\\
28776	28676\\
28776	28775\\
28804	28904\\
28873	28973\\
28879	28779\\
28981	29081\\
29004	28904\\
29010	29110\\
29136	29036\\
29136	29135\\
29136	29236\\
29284	29384\\
29290	29190\\
29341	29441\\
29393	29293\\
29393	29493\\
29490	29590\\
29520	29521\\
29547	29647\\
29650	29550\\
29650	29649\\
29650	29750\\
29804	29704\\
29804	29803\\
29833	29933\\
29901	20001\\
29907	29807\\
29932	29933\\
30000	30000\\
};
\end{axis}
\end{tikzpicture}%
%	\end{subfigure}%
%	\begin{subfigure}[t]{0.5\textwidth}
%		% This file was created by matlab2tikz.
%
%The latest updates can be retrieved from
%  http://www.mathworks.com/matlabcentral/fileexchange/22022-matlab2tikz-matlab2tikz
%where you can also make suggestions and rate matlab2tikz.
%
\definecolor{mycolor1}{rgb}{0.00000,0.44700,0.74100}%
%
\begin{tikzpicture}

\begin{axis}[%
width=0.75\figW,
height=\figH,
at={(0\figW,0\figH)},
scale only axis,
xmin=0,
xmax=10001,
xlabel style={font=\color{white!15!black}},
xlabel={nz = 50000},
y dir=reverse,
ymin=0,
ymax=10001,
axis background/.style={fill=white}
]
\addplot [color=mycolor1, draw=none, mark size=0.7pt, mark=*, mark options={solid, mycolor1}, forget plot]
  table[row sep=crcr]{%
1	1\\
1	100\\
1	9901\\
26	25\\
26	126\\
26	9926\\
34	35\\
43	143\\
72	172\\
78	77\\
78	9978\\
95	94\\
100	1\\
112	12\\
112	111\\
112	212\\
129	29\\
146	145\\
158	58\\
163	162\\
163	263\\
194	195\\
198	98\\
198	298\\
215	214\\
215	315\\
240	241\\
249	149\\
249	349\\
262	263\\
278	378\\
283	183\\
283	282\\
283	383\\
301	201\\
318	317\\
323	423\\
331	332\\
335	235\\
335	435\\
363	263\\
369	269\\
369	368\\
369	469\\
377	378\\
386	486\\
409	309\\
415	515\\
421	321\\
421	420\\
438	437\\
455	355\\
455	454\\
455	555\\
472	372\\
489	488\\
500	401\\
506	505\\
506	606\\
537	538\\
541	441\\
541	641\\
558	557\\
558	658\\
582	583\\
592	492\\
592	692\\
605	606\\
621	721\\
626	526\\
626	625\\
626	726\\
643	543\\
661	660\\
666	766\\
674	675\\
678	578\\
678	778\\
706	606\\
712	612\\
712	711\\
712	812\\
720	721\\
729	829\\
752	652\\
758	858\\
763	663\\
763	762\\
781	780\\
798	698\\
798	797\\
798	898\\
815	715\\
832	831\\
843	743\\
849	848\\
849	949\\
880	881\\
883	783\\
883	983\\
901	901\\
918	1018\\
925	926\\
935	835\\
935	1035\\
948	949\\
963	1063\\
969	869\\
969	968\\
969	1069\\
986	886\\
1003	1002\\
1009	1109\\
1017	1018\\
1021	921\\
1021	1121\\
1049	949\\
1055	955\\
1055	1054\\
1055	1155\\
1062	1063\\
1072	1172\\
1095	995\\
1101	1201\\
1106	1006\\
1106	1105\\
1123	1122\\
1141	1041\\
1141	1140\\
1141	1241\\
1158	1058\\
1175	1174\\
1186	1086\\
1192	1191\\
1192	1292\\
1222	1223\\
1226	1126\\
1226	1326\\
1243	1242\\
1243	1343\\
1268	1269\\
1278	1178\\
1278	1378\\
1291	1292\\
1306	1406\\
1312	1212\\
1312	1311\\
1312	1412\\
1329	1229\\
1346	1345\\
1352	1452\\
1360	1361\\
1363	1263\\
1363	1463\\
1392	1292\\
1398	1298\\
1398	1397\\
1398	1498\\
1405	1406\\
1415	1515\\
1438	1338\\
1444	1544\\
1449	1349\\
1449	1448\\
1466	1465\\
1483	1383\\
1484	1483\\
1484	1584\\
1501	1401\\
1518	1517\\
1529	1429\\
1535	1534\\
1535	1635\\
1565	1566\\
1569	1469\\
1569	1669\\
1586	1585\\
1586	1686\\
1611	1612\\
1621	1521\\
1621	1721\\
1634	1635\\
1649	1749\\
1655	1555\\
1655	1654\\
1655	1755\\
1672	1572\\
1689	1688\\
1695	1795\\
1703	1704\\
1706	1606\\
1706	1806\\
1735	1635\\
1741	1641\\
1741	1740\\
1741	1841\\
1748	1749\\
1758	1858\\
1781	1681\\
1786	1886\\
1792	1692\\
1792	1791\\
1809	1808\\
1826	1726\\
1826	1825\\
1826	1926\\
1844	1744\\
1861	1860\\
1872	1772\\
1878	1877\\
1878	1978\\
1908	1909\\
1912	1812\\
1912	2012\\
1929	1928\\
1929	2029\\
1954	1955\\
1964	1864\\
1964	2064\\
1977	1978\\
1992	2092\\
1998	1898\\
1998	1997\\
1998	2098\\
2015	1915\\
2032	2031\\
2038	2138\\
2045	2046\\
2049	1949\\
2049	2149\\
2078	1978\\
2084	1984\\
2084	2083\\
2084	2184\\
2091	2092\\
2106	2206\\
2124	2024\\
2129	2229\\
2135	2035\\
2135	2134\\
2152	2151\\
2169	2069\\
2169	2168\\
2169	2269\\
2186	2086\\
2204	2203\\
2215	2115\\
2221	2220\\
2221	2321\\
2251	2252\\
2255	2155\\
2255	2355\\
2272	2271\\
2272	2372\\
2297	2298\\
2306	2206\\
2306	2406\\
2320	2321\\
2335	2435\\
2341	2241\\
2341	2340\\
2341	2441\\
2358	2258\\
2375	2374\\
2381	2481\\
2388	2389\\
2392	2292\\
2392	2492\\
2421	2321\\
2426	2326\\
2426	2425\\
2426	2526\\
2434	2435\\
2444	2544\\
2466	2366\\
2472	2572\\
2478	2378\\
2478	2477\\
2495	2494\\
2512	2412\\
2512	2511\\
2512	2612\\
2529	2429\\
2546	2545\\
2558	2458\\
2564	2563\\
2564	2664\\
2594	2595\\
2598	2498\\
2598	2698\\
2615	2614\\
2615	2715\\
2640	2641\\
2649	2549\\
2649	2749\\
2663	2664\\
2678	2778\\
2684	2584\\
2684	2683\\
2684	2784\\
2701	2601\\
2718	2717\\
2724	2824\\
2731	2732\\
2735	2635\\
2735	2835\\
2764	2664\\
2769	2669\\
2769	2768\\
2769	2869\\
2777	2778\\
2792	2892\\
2809	2709\\
2815	2915\\
2821	2721\\
2821	2820\\
2838	2837\\
2855	2755\\
2855	2854\\
2855	2955\\
2872	2772\\
2889	2888\\
2900	2801\\
2907	2906\\
2907	3007\\
2937	2938\\
2941	2841\\
2941	3041\\
2958	2957\\
2958	3058\\
2983	2984\\
2992	2892\\
2992	3092\\
3006	3007\\
3021	3121\\
3027	2927\\
3027	3026\\
3027	3127\\
3044	2944\\
3061	3060\\
3067	3167\\
3074	3075\\
3078	2978\\
3078	3178\\
3107	3007\\
3112	3012\\
3112	3111\\
3112	3212\\
3120	3121\\
3129	3229\\
3152	3052\\
3158	3258\\
3164	3064\\
3164	3163\\
3181	3180\\
3198	3098\\
3198	3197\\
3198	3298\\
3215	3115\\
3232	3231\\
3244	3144\\
3249	3248\\
3249	3349\\
3280	3281\\
3284	3184\\
3284	3384\\
3301	3301\\
3318	3418\\
3326	3327\\
3335	3235\\
3335	3435\\
3348	3349\\
3364	3464\\
3369	3269\\
3369	3368\\
3369	3469\\
3387	3287\\
3404	3403\\
3409	3509\\
3417	3418\\
3421	3321\\
3421	3521\\
3449	3349\\
3455	3355\\
3455	3454\\
3455	3555\\
3463	3464\\
3472	3572\\
3495	3395\\
3501	3601\\
3507	3407\\
3507	3506\\
3524	3523\\
3541	3441\\
3541	3540\\
3541	3641\\
3558	3458\\
3575	3574\\
3587	3487\\
3592	3591\\
3592	3692\\
3623	3624\\
3627	3527\\
3627	3727\\
3644	3643\\
3644	3744\\
3668	3669\\
3678	3578\\
3678	3778\\
3691	3692\\
3707	3807\\
3712	3612\\
3712	3711\\
3712	3812\\
3729	3629\\
3747	3746\\
3752	3852\\
3760	3761\\
3764	3664\\
3764	3864\\
3792	3692\\
3798	3698\\
3798	3797\\
3798	3898\\
3806	3807\\
3815	3915\\
3838	3738\\
3844	3944\\
3849	3749\\
3849	3848\\
3867	3866\\
3884	3784\\
3884	3883\\
3884	3984\\
3901	3801\\
3918	3917\\
3929	3829\\
3935	3934\\
3935	4035\\
3966	3967\\
3969	3869\\
3969	4069\\
3987	3986\\
3987	4087\\
4011	4012\\
4021	3921\\
4021	4121\\
4034	4035\\
4049	4149\\
4055	3955\\
4055	4054\\
4055	4155\\
4072	3972\\
4089	4088\\
4095	4195\\
4103	4104\\
4107	4007\\
4107	4207\\
4135	4035\\
4141	4041\\
4141	4140\\
4141	4241\\
4148	4149\\
4158	4258\\
4181	4081\\
4187	4287\\
4192	4092\\
4192	4191\\
4209	4208\\
4227	4127\\
4227	4226\\
4227	4327\\
4244	4144\\
4261	4260\\
4272	4172\\
4278	4277\\
4278	4378\\
4309	4310\\
4312	4212\\
4312	4412\\
4330	4329\\
4330	4430\\
4354	4355\\
4364	4264\\
4364	4464\\
4377	4378\\
4392	4492\\
4398	4298\\
4398	4397\\
4398	4498\\
4415	4315\\
4432	4431\\
4438	4538\\
4446	4447\\
4450	4350\\
4450	4550\\
4478	4378\\
4484	4384\\
4484	4483\\
4484	4584\\
4491	4492\\
4507	4607\\
4524	4424\\
4530	4630\\
4535	4435\\
4535	4534\\
4552	4551\\
4570	4470\\
4570	4569\\
4570	4670\\
4587	4487\\
4604	4603\\
4615	4515\\
4621	4620\\
4621	4721\\
4651	4652\\
4655	4555\\
4655	4755\\
4672	4671\\
4672	4772\\
4697	4698\\
4707	4607\\
4707	4807\\
4720	4721\\
4735	4835\\
4741	4641\\
4741	4740\\
4741	4841\\
4758	4658\\
4775	4774\\
4781	4881\\
4789	4790\\
4792	4692\\
4792	4892\\
4821	4721\\
4827	4727\\
4827	4826\\
4827	4927\\
4834	4835\\
4844	4944\\
4867	4767\\
4872	4972\\
4878	4778\\
4878	4877\\
4895	4894\\
4912	4812\\
4912	4911\\
4912	5012\\
4930	4830\\
4947	4946\\
4958	4858\\
4964	4963\\
4964	5064\\
4994	4995\\
4998	4898\\
4998	5098\\
5015	5014\\
5015	5115\\
5040	5041\\
5050	4950\\
5050	5150\\
5063	5064\\
5078	5178\\
5084	4984\\
5084	5083\\
5084	5184\\
5101	5001\\
5118	5117\\
5124	5224\\
5131	5132\\
5135	5035\\
5135	5235\\
5164	5064\\
5170	5070\\
5170	5169\\
5170	5270\\
5177	5178\\
5187	5287\\
5210	5110\\
5215	5315\\
5221	5121\\
5221	5220\\
5238	5237\\
5255	5155\\
5255	5254\\
5255	5355\\
5272	5172\\
5290	5289\\
5300	5201\\
5307	5306\\
5307	5407\\
5337	5338\\
5341	5241\\
5341	5441\\
5358	5357\\
5358	5458\\
5383	5384\\
5392	5292\\
5392	5492\\
5406	5407\\
5421	5521\\
5427	5327\\
5427	5426\\
5427	5527\\
5444	5344\\
5461	5460\\
5467	5567\\
5474	5475\\
5478	5378\\
5478	5578\\
5507	5407\\
5512	5412\\
5512	5511\\
5512	5612\\
5520	5521\\
5530	5630\\
5552	5452\\
5558	5658\\
5564	5464\\
5564	5563\\
5581	5580\\
5598	5498\\
5598	5597\\
5598	5698\\
5615	5515\\
5632	5631\\
5644	5544\\
5650	5649\\
5650	5750\\
5680	5681\\
5684	5584\\
5684	5784\\
5701	5701\\
5718	5818\\
5726	5727\\
5735	5635\\
5735	5835\\
5749	5750\\
5764	5864\\
5770	5670\\
5770	5769\\
5770	5870\\
5787	5687\\
5804	5803\\
5810	5910\\
5817	5818\\
5821	5721\\
5821	5921\\
5850	5750\\
5855	5755\\
5855	5854\\
5855	5955\\
5863	5864\\
5873	5973\\
5895	5795\\
5901	6001\\
5907	5807\\
5907	5906\\
5924	5923\\
5941	5841\\
5941	5940\\
5941	6041\\
5958	5858\\
5975	5974\\
5987	5887\\
5993	5992\\
5993	6093\\
6023	6024\\
6027	5927\\
6027	6127\\
6044	6043\\
6044	6144\\
6069	6070\\
6078	5978\\
6078	6178\\
6092	6093\\
6107	6207\\
6113	6013\\
6113	6112\\
6113	6213\\
6130	6030\\
6147	6146\\
6153	6253\\
6160	6161\\
6164	6064\\
6164	6264\\
6193	6093\\
6198	6098\\
6198	6197\\
6198	6298\\
6206	6207\\
6215	6315\\
6238	6138\\
6244	6344\\
6250	6150\\
6250	6249\\
6267	6266\\
6284	6184\\
6284	6283\\
6284	6384\\
6301	6201\\
6318	6317\\
6330	6230\\
6335	6334\\
6335	6435\\
6366	6367\\
6370	6270\\
6370	6470\\
6387	6386\\
6387	6487\\
6412	6413\\
6421	6321\\
6421	6521\\
6434	6435\\
6450	6550\\
6455	6355\\
6455	6454\\
6455	6555\\
6473	6373\\
6490	6489\\
6495	6595\\
6503	6504\\
6507	6407\\
6507	6607\\
6535	6435\\
6541	6441\\
6541	6540\\
6541	6641\\
6549	6550\\
6558	6658\\
6581	6481\\
6587	6687\\
6593	6493\\
6593	6592\\
6610	6609\\
6627	6527\\
6627	6626\\
6627	6727\\
6644	6544\\
6661	6660\\
6673	6573\\
6678	6677\\
6678	6778\\
6709	6710\\
6713	6613\\
6713	6813\\
6730	6729\\
6730	6830\\
6754	6755\\
6764	6664\\
6764	6864\\
6777	6778\\
6793	6893\\
6798	6698\\
6798	6797\\
6798	6898\\
6815	6715\\
6833	6832\\
6838	6938\\
6846	6847\\
6850	6750\\
6850	6950\\
6878	6778\\
6884	6784\\
6884	6883\\
6884	6984\\
6892	6893\\
6907	7007\\
6924	6824\\
6930	7030\\
6935	6835\\
6935	6934\\
6953	6952\\
6970	6870\\
6970	6969\\
6970	7070\\
6987	6887\\
7004	7003\\
7015	6915\\
7021	7020\\
7021	7121\\
7052	7053\\
7055	6955\\
7055	7155\\
7073	7072\\
7073	7173\\
7097	7098\\
7107	7007\\
7107	7207\\
7120	7121\\
7136	7236\\
7141	7041\\
7141	7140\\
7141	7241\\
7158	7058\\
7176	7175\\
7181	7281\\
7189	7190\\
7193	7093\\
7193	7293\\
7221	7121\\
7227	7127\\
7227	7226\\
7227	7327\\
7235	7236\\
7244	7344\\
7267	7167\\
7273	7373\\
7278	7178\\
7278	7277\\
7296	7295\\
7313	7213\\
7313	7312\\
7313	7413\\
7330	7230\\
7347	7346\\
7358	7258\\
7364	7363\\
7364	7464\\
7395	7396\\
7398	7298\\
7398	7498\\
7416	7415\\
7416	7516\\
7440	7441\\
7450	7350\\
7450	7550\\
7463	7464\\
7478	7578\\
7484	7384\\
7484	7483\\
7484	7584\\
7501	7401\\
7518	7517\\
7524	7624\\
7532	7533\\
7536	7436\\
7536	7636\\
7564	7464\\
7570	7470\\
7570	7569\\
7570	7670\\
7577	7578\\
7587	7687\\
7610	7510\\
7616	7716\\
7621	7521\\
7621	7620\\
7638	7637\\
7656	7556\\
7656	7655\\
7656	7756\\
7673	7573\\
7690	7689\\
7700	7601\\
7707	7706\\
7707	7807\\
7737	7738\\
7741	7641\\
7741	7841\\
7758	7757\\
7758	7858\\
7783	7784\\
7793	7693\\
7793	7893\\
7806	7807\\
7821	7921\\
7827	7727\\
7827	7826\\
7827	7927\\
7844	7744\\
7861	7860\\
7867	7967\\
7875	7876\\
7878	7778\\
7878	7978\\
7907	7807\\
7913	7813\\
7913	7912\\
7913	8013\\
7920	7921\\
7930	8030\\
7953	7853\\
7958	8058\\
7964	7864\\
7964	7963\\
7981	7980\\
7998	7898\\
7998	7997\\
7998	8098\\
8016	7916\\
8033	8032\\
8044	7944\\
8050	8049\\
8050	8150\\
8080	8081\\
8084	7984\\
8084	8184\\
8101	8101\\
8118	8218\\
8126	8127\\
8136	8036\\
8136	8236\\
8149	8150\\
8164	8264\\
8170	8070\\
8170	8169\\
8170	8270\\
8187	8087\\
8204	8203\\
8210	8310\\
8217	8218\\
8221	8121\\
8221	8321\\
8250	8150\\
8256	8156\\
8256	8255\\
8256	8356\\
8263	8264\\
8273	8373\\
8296	8196\\
8301	8401\\
8307	8207\\
8307	8306\\
8324	8323\\
8341	8241\\
8341	8340\\
8341	8441\\
8358	8258\\
8376	8375\\
8387	8287\\
8393	8392\\
8393	8493\\
8423	8424\\
8427	8327\\
8427	8527\\
8444	8443\\
8444	8544\\
8469	8470\\
8478	8378\\
8478	8578\\
8492	8493\\
8507	8607\\
8513	8413\\
8513	8512\\
8513	8613\\
8530	8430\\
8547	8546\\
8553	8653\\
8560	8561\\
8564	8464\\
8564	8664\\
8593	8493\\
8599	8499\\
8599	8598\\
8599	8699\\
8606	8607\\
8616	8716\\
8638	8538\\
8644	8744\\
8650	8550\\
8650	8649\\
8667	8666\\
8684	8584\\
8684	8683\\
8684	8784\\
8701	8601\\
8719	8718\\
8730	8630\\
8736	8735\\
8736	8836\\
8766	8767\\
8770	8670\\
8770	8870\\
8787	8786\\
8787	8887\\
8812	8813\\
8821	8721\\
8821	8921\\
8835	8836\\
8850	8950\\
8856	8756\\
8856	8855\\
8856	8956\\
8873	8773\\
8890	8889\\
8896	8996\\
8903	8904\\
8907	8807\\
8907	9007\\
8936	8836\\
8941	8841\\
8941	8940\\
8941	9041\\
8949	8950\\
8959	9059\\
8981	8881\\
8987	9087\\
8993	8893\\
8993	8992\\
9010	9009\\
9027	8927\\
9027	9026\\
9027	9127\\
9044	8944\\
9061	9060\\
9073	8973\\
9079	9078\\
9079	9179\\
9109	9110\\
9113	9013\\
9113	9213\\
9130	9129\\
9130	9230\\
9155	9156\\
9164	9064\\
9164	9264\\
9178	9179\\
9193	9293\\
9199	9099\\
9199	9198\\
9199	9299\\
9216	9116\\
9233	9232\\
9239	9339\\
9246	9247\\
9250	9150\\
9250	9350\\
9279	9179\\
9284	9184\\
9284	9283\\
9284	9384\\
9292	9293\\
9307	9407\\
9324	9224\\
9330	9430\\
9336	9236\\
9336	9335\\
9353	9352\\
9370	9270\\
9370	9369\\
9370	9470\\
9387	9287\\
9404	9403\\
9416	9316\\
9421	9420\\
9421	9521\\
9452	9453\\
9456	9356\\
9456	9556\\
9473	9472\\
9473	9573\\
9498	9499\\
9507	9407\\
9507	9607\\
9520	9521\\
9536	9636\\
9541	9441\\
9541	9540\\
9541	9641\\
9559	9459\\
9576	9575\\
9581	9681\\
9589	9590\\
9593	9493\\
9593	9693\\
9621	9521\\
9627	9527\\
9627	9626\\
9627	9727\\
9635	9636\\
9644	9744\\
9667	9567\\
9673	9773\\
9679	9579\\
9679	9678\\
9696	9695\\
9713	9613\\
9713	9712\\
9713	9813\\
9730	9630\\
9747	9746\\
9759	9659\\
9764	9763\\
9764	9864\\
9795	9796\\
9799	9699\\
9799	9899\\
9816	9815\\
9816	9916\\
9840	9841\\
9850	9750\\
9850	9950\\
9863	9864\\
9879	9979\\
9884	9784\\
9884	9883\\
9884	9984\\
9901	1\\
9901	9801\\
9919	9918\\
9932	9933\\
9935	35\\
9936	9836\\
9964	9864\\
9970	70\\
9970	9870\\
9970	9969\\
9978	9979\\
9987	87\\
10000	10000\\
};
\end{axis}
\end{tikzpicture}%
%	\end{subfigure}
%	\caption[Sparsity of the wave matrix]{Sparsity of the wave matrix \textit{A}}
%	\label{fig:sparsity}
%\end{figure}


\section{Numerical validation}

\subsection{A weakly modulated waveguide}
\begin{figure}[t]
\centering
\setlength{\figH}{1\textwidth}
\setlength{\figW}{1\textwidth}
\begin{tikzpicture}
    \begin{axis}[%
width=0.9\figW,
height=0.15\figH,
at={(0\figW,0\figH)},
scale only axis,
axis on top,
xmin=0,
xmax=10,
xlabel style={font=\color{white!15!black}},
xlabel={$\text{x (}\mu\text{m)}$},
ymin=-2.5,
ymax=2.5,
ylabel style={font=\color{white!15!black}},
ylabel={$\text{y (}\mu\text{m)}$},
axis background/.style={fill=white},
]
\filldraw[fill=blue!40!white, draw=black] (0,1.95) rectangle (60,3.05);
\filldraw[fill=blue!10!white, draw=black, opacity=1] (5.2,2.5) rectangle (24.2,3.05);
\filldraw[fill=blue!10!white, draw=black, opacity=1] (35.8,2.5) rectangle (54.8,3.05);
\draw[draw=green, line width=1mm] (1,0.2) -- (1,4.8);
\draw[line width=0.8mm, draw=green, ->] (1.2,4) -- (5,4);
\draw[line width=0.8mm, draw=green, ->] (1.2,1) -- (5,1);
\end{axis}
\end{tikzpicture}
\caption[The modulated waveguide structure]{Weakly modulated waveguide structure.}
\label{fig:weakmod}
\end{figure}


\begin{figure}
    \centering
    \setlength{\figH}{1\linewidth}
	\setlength{\figW}{1\textwidth}
	\pgfplotsset{every axis/.append style={
                    label style={font=\footnotesize},
                    tick label style={font=\footnotesize}  
                    }}
    \begin{subfigure}[t]{0.5\textwidth}
		% This file was created by matlab2tikz.
%
%The latest updates can be retrieved from
%  http://www.mathworks.com/matlabcentral/fileexchange/22022-matlab2tikz-matlab2tikz
%where you can also make suggestions and rate matlab2tikz.
%
\pgfplotsset{weakMod/.style={
	width=0.32\figW,
	height=0.15\figH,
	scale only axis,
	axis on top,
	xmin=0,
	xmax=10,
	ymin=-2.5,
	ymax=2.5,
	%xlabel style={font=\color{white!15!black}},
	%ylabel style={font=\color{white!15!black}},
	axis background/.style={fill=white},
	colormap={mymap}{[1pt] rgb(0pt)=(0,0,1); rgb(31pt)=(1,1,1); rgb(32pt)=(1,1,1); rgb(63pt)=(1,0,0)},
	colorbar,
	colorbar style={width=.03\linewidth, at={(1.08,0.075\figH)}, anchor=east}
	}
}

\begin{tikzpicture}

\begin{axis}[%
weakMod,
at={(0\figW,0.92\figH)},
point meta min=-0.153883565276573,
point meta max=0.153883565276573,
xticklabels={,,},
colorbar style={title = $V / \mu m$, width=.03\linewidth, at={(1.08,0.075\figH)}, anchor=east}
]
\addplot [forget plot] graphics [xmin=0, xmax=10, ymin=-2.5, ymax=2.5] {graphs/weakmod/nameoffile2-1.png};
\draw (0,-0.5) -- (10,-0.5);
\draw (0,0.5) -- (10,0.5);
\node at (5,1.8) {$\omega_1+2\Omega$};
\end{axis}

\begin{axis}[%
weakMod,
at={(0\figW,0.69\figH)},
point meta min=-1.37640682053953,
point meta max=1.37640682053953,
xticklabels={,,}
]
\addplot [forget plot] graphics [xmin=0, xmax=10, ymin=-2.5, ymax=2.5] {graphs/weakmod/nameoffile2-2.png};
\draw (0,-0.5) -- (10,-0.5);
\draw (0,0.5) -- (10,0.5);
\node at (5,1.8) {$\omega_1+\Omega$};
\end{axis}

\begin{axis}[%
weakMod,
at={(0\figW,0.46\figH)},
point meta min=-24.7127053691998,
point meta max=24.7127053691998,
xticklabels={,,},
ylabel={$\text{y (}\mu\text{m)}$}
]
\addplot [forget plot] graphics [xmin=0, xmax=10, ymin=-2.5, ymax=2.5] {graphs/weakmod/nameoffile2-3.png};
\draw (0,-0.5) -- (10,-0.5);
\draw (0,0.5) -- (10,0.5);
\node at (5,1.8) {$\omega_1$};
\end{axis}

\begin{axis}[%
weakMod,
at={(0\figW,0.23\figH)},
point meta min=-2.69144358712869,
point meta max=2.69144358712869,
xticklabels={,,}
]
\addplot [forget plot] graphics [xmin=0, xmax=10, ymin=-2.5, ymax=2.5] {graphs/weakmod/nameoffile2-4.png};
\draw (0,-0.5) -- (10,-0.5);
\draw (0,0.5) -- (10,0.5);
\node at (5,1.8) {$\omega_1-\Omega$};
\end{axis}

\begin{axis}[%
weakMod,
at={(0\figW,0\figH)},
point meta min=-0.251011435268562,
point meta max=0.251011435268562,
%xlabel style={font=\color{white!15!black}},
xlabel={$\text{x (}\mu\text{m)}$}
]
\addplot [forget plot] graphics [xmin=0, xmax=10, ymin=-2.5, ymax=2.5] {graphs/weakmod/nameoffile2-5.png};
\draw (0,-0.5) -- (10,-0.5);
\draw (0,0.5) -- (10,0.5);
\node at (5,1.8) {$\omega_1-2\Omega$};
\end{axis}
\node at (-1,16) {\textbf{(a)}};

\end{tikzpicture}%
    \end{subfigure}%
    \begin{subfigure}[t]{0.5\textwidth}
		% This file was created by matlab2tikz.
%
%The latest updates can be retrieved from
%  http://www.mathworks.com/matlabcentral/fileexchange/22022-matlab2tikz-matlab2tikz
%where you can also make suggestions and rate matlab2tikz.
%
\definecolor{mycolor1}{rgb}{0.00000,0.44700,0.74100}%
\definecolor{mycolor2}{rgb}{0.85000,0.32500,0.09800}%
%
\begin{tikzpicture}

\begin{axis}[%
width=0.4\figW,
height=0.15\figH,
at={(0\figW,0.86\figH)},
scale only axis,
yticklabel style={/pgf/number format/fixed},
xmin=0,
xmax=10,
ymin=0,
ymax=0.01,
axis background/.style={fill=white}
]
\addplot [color=mycolor1, forget plot]
  table[row sep=crcr]{%
0	2.33105594482646e-06\\
0.025062656641604	9.01287932868077e-07\\
0.050125313283208	9.85818098486402e-07\\
0.075187969924812	2.22160020668889e-06\\
0.100250626566416	4.06160199512851e-06\\
0.12531328320802	6.42950087937509e-06\\
0.150375939849624	1.10916214092756e-05\\
0.175438596491228	2.10439751069597e-05\\
0.200501253132832	3.54210760317291e-05\\
0.225563909774436	5.29780406878314e-05\\
0.25062656641604	7.08795739166728e-05\\
0.275689223057644	8.69794532406785e-05\\
0.300751879699248	9.99914218489389e-05\\
0.325814536340852	0.000109565720725502\\
0.350877192982456	0.000116049452212893\\
0.37593984962406	0.000120143836165418\\
0.401002506265664	0.000122620411009073\\
0.426065162907268	0.000124148882176009\\
0.451127819548872	0.000125222681870323\\
0.476190476190476	0.000126147916749036\\
0.50125313283208	0.000127067392077936\\
0.526315789473684	0.000128009668166094\\
0.551378446115288	0.000128967753098669\\
0.576441102756892	0.000129933778403758\\
0.601503759398496	0.000130899863583817\\
0.6265664160401	0.000131858166573504\\
0.651629072681704	0.000132800928978526\\
0.676691729323308	0.000133720516139098\\
0.701754385964912	0.000134609452130239\\
0.726817042606516	0.000135460449843398\\
0.75187969924812	0.000136266436293182\\
0.776942355889724	0.000137020573262347\\
0.802005012531328	0.000137716273340372\\
0.827067669172932	0.00013834721132767\\
0.852130325814536	0.00013890733087038\\
0.87719298245614	0.000139390846060993\\
0.902255639097744	0.000139792237590067\\
0.927318295739348	0.000140106242868012\\
0.952380952380952	0.000140327839360965\\
0.977443609022556	0.000140452220214622\\
1.00250626566416	0.000140474761096893\\
1.02756892230576	0.000140390977109575\\
1.05263157894737	0.000140196468654413\\
1.07769423558897	0.000139886899864048\\
1.10275689223058	0.000139457817348935\\
1.12781954887218	0.000138904607386717\\
1.15288220551378	0.00013822243286992\\
1.17794486215539	0.000137406104029872\\
1.20300751879699	0.000136449978344093\\
1.2280701754386	0.000135347878247097\\
1.2531328320802	0.000134093047860769\\
1.2781954887218	0.000132678179885692\\
1.30325814536341	0.000131095557087949\\
1.32832080200501	0.0001293373709195\\
1.35338345864662	0.000127396305960184\\
1.37844611528822	0.000125266521235278\\
1.40350877192982	0.000122945239863154\\
1.42857142857143	0.00012043533754365\\
1.45363408521303	0.00011774977441503\\
1.47869674185464	0.000114919986968045\\
1.50375939849624	0.00011689735118425\\
1.52882205513784	0.000121726198197074\\
1.55388471177945	0.000125945416088278\\
1.57894736842105	0.00012901366914319\\
1.60401002506266	0.000130183898799949\\
1.62907268170426	0.000129119878082873\\
1.65413533834586	0.000125367765914451\\
1.67919799498747	0.000118418187393541\\
1.70426065162907	0.00010764709403612\\
1.72932330827068	9.33032873045091e-05\\
1.75438596491228	9.63900619751649e-05\\
1.77944862155388	0.000110162821237525\\
1.80451127819549	0.000130418484980492\\
1.82957393483709	0.000159983186532498\\
1.8546365914787	0.000195808960613232\\
1.8796992481203	0.000236540482951088\\
1.9047619047619	0.00027956112627642\\
1.92982456140351	0.000323157866680819\\
1.95488721804511	0.000365204598420651\\
1.97994987468672	0.000404685667857003\\
2.00501253132832	0.000438717165754038\\
2.03007518796993	0.000466048108460099\\
2.05513784461153	0.000485781716335755\\
2.08020050125313	0.000497491774252237\\
2.10526315789474	0.000501354784361502\\
2.13032581453634	0.000498297105971021\\
2.15538847117795	0.000490146572364364\\
2.18045112781955	0.000479747283797727\\
2.20551378446115	0.0004709292171558\\
2.23057644110276	0.000468136437728386\\
2.25563909774436	0.000475536204729574\\
2.28070175438596	0.000508573838133546\\
2.30576441102757	0.000578824528855558\\
2.33082706766917	0.000644635317553838\\
2.35588972431078	0.000703274403488717\\
2.38095238095238	0.000752486897907784\\
2.40601503759398	0.000790679207867245\\
2.43107769423559	0.000817145173476065\\
2.45614035087719	0.000856569707501827\\
2.4812030075188	0.000886797216310967\\
2.5062656641604	0.000904513602504812\\
2.53132832080201	0.000911221038418481\\
2.55639097744361	0.000907972650749936\\
2.58145363408521	0.000896810914101952\\
2.60651629072682	0.000909095163818558\\
2.63157894736842	0.00097559916637781\\
2.65664160401003	0.00106516911730901\\
2.68170426065163	0.00117339906475096\\
2.70676691729323	0.00129389534272862\\
2.73182957393484	0.00141958915119146\\
2.75689223057644	0.0015435873059353\\
2.78195488721805	0.00165964930532967\\
2.80701754385965	0.00176247113994998\\
2.83208020050125	0.00184790060122065\\
2.85714285714286	0.00191314442684713\\
2.88220551378446	0.00195698828056314\\
2.90726817042607	0.00198002856334268\\
2.93233082706767	0.00198489595925477\\
2.95739348370927	0.00197642069804736\\
2.98245614035088	0.00196163754044314\\
3.00751879699248	0.00194945694999381\\
3.03258145363409	0.00194978408580487\\
3.05764411027569	0.00197196657922211\\
3.08270676691729	0.00202281119539654\\
3.1077694235589	0.00210487416965082\\
3.1328320802005	0.0022158529497659\\
3.15789473684211	0.00234930256909349\\
3.18295739348371	0.00249617703392651\\
3.20802005012531	0.00264645909350545\\
3.23308270676692	0.00279043805836857\\
3.25814536340852	0.00291956462772439\\
3.28320802005013	0.00302699387723154\\
3.30827067669173	0.00310794776683407\\
3.33333333333333	0.00315998612400866\\
3.35839598997494	0.00318322701102628\\
3.38345864661654	0.00318051651128588\\
3.40852130325815	0.00315750737539163\\
3.43358395989975	0.0031225557882141\\
3.45864661654135	0.00308628650386694\\
3.48370927318296	0.00306064024846753\\
3.50877192982456	0.00305728376371873\\
3.53383458646617	0.00308552425865029\\
3.55889724310777	0.00315028101921426\\
3.58395989974937	0.00325091030755287\\
3.60902255639098	0.00338139878713172\\
3.63408521303258	0.00353175785505425\\
3.65914786967419	0.00368994819977774\\
3.68421052631579	0.00384368441507989\\
3.70927318295739	0.00398180618099609\\
3.734335839599	0.00409520378563325\\
3.7593984962406	0.00417741575569467\\
3.78446115288221	0.0042250208651272\\
3.80952380952381	0.00423790197874127\\
3.83458646616541	0.00421940523275858\\
3.85964912280702	0.00417636252674028\\
3.88471177944862	0.00411888336835312\\
3.90977443609023	0.00405975682519464\\
3.93483709273183	0.00401326997915466\\
3.95989974937343	0.0039933280305192\\
3.98496240601504	0.0040110433320835\\
4.01002506265664	0.00407239661915056\\
4.03508771929825	0.00417684197740548\\
4.06015037593985	0.00431745285271002\\
4.08521303258145	0.00448247946649862\\
4.11027568922306	0.00465759818907069\\
4.13533834586466	0.00482809314380862\\
4.16040100250627	0.00498055737023629\\
4.18546365914787	0.00510405520412116\\
4.21052631578947	0.00519086387114963\\
4.23558897243108	0.00523693687523052\\
4.26065162907268	0.00524218720025869\\
4.28571428571429	0.0052106262694225\\
4.31077694235589	0.00515032961228983\\
4.33583959899749	0.00507313058079394\\
4.3609022556391	0.00499387451679455\\
4.3859649122807	0.00492903458588072\\
4.41102756892231	0.00489458101555214\\
4.43609022556391	0.00490329555835201\\
4.46115288220551	0.00496217745924617\\
4.48621553884712	0.00507087017737138\\
4.51127819548872	0.00522176568056596\\
4.53634085213033	0.0054016855372274\\
4.56140350877193	0.00559438610978223\\
4.58646616541353	0.0057830453050283\\
4.61152882205514	0.00595223615488753\\
4.63659147869674	0.00608928468040229\\
4.66165413533835	0.00618512425291711\\
4.68671679197995	0.00623480330522157\\
4.71177944862155	0.00623776280564567\\
4.73684210526316	0.00619793536580277\\
4.76190476190476	0.00612365106957959\\
4.78696741854637	0.00602726793141463\\
4.81203007518797	0.0059243818487701\\
4.83709273182957	0.00583244098729512\\
4.86215538847118	0.00576865955106506\\
4.88721804511278	0.00574737200982652\\
4.91228070175439	0.00577736777763606\\
4.93734335839599	0.00586005622226308\\
4.96240601503759	0.00598918226571152\\
4.9874686716792	0.00615218840969882\\
5.0125313283208	0.00633263451882927\\
5.03759398496241	0.006512835608889\\
5.06265664160401	0.00667610405149375\\
5.08771929824561	0.00680836841104387\\
5.11278195488722	0.00689921369358982\\
5.13784461152882	0.00694248724948097\\
5.16290726817043	0.00693660245854957\\
5.18796992481203	0.00688461664634955\\
5.21303258145363	0.00679409524218169\\
5.23809523809524	0.00667671095182401\\
5.26315789473684	0.00654746990512588\\
5.28822055137845	0.00642342627658592\\
5.31328320802005	0.006321790187222\\
5.33834586466165	0.00625751155473984\\
5.36340852130326	0.00624073870260606\\
5.38847117794486	0.0062748528299855\\
5.41353383458647	0.00635578603965757\\
5.43859649122807	0.00647289789875069\\
5.46365914786967	0.006611057994961\\
5.48872180451128	0.00675320385431907\\
5.51378446115288	0.00688270460996791\\
5.53884711779449	0.0069851780539107\\
5.56390977443609	0.00704970824293168\\
5.58897243107769	0.00706956589175394\\
5.6140350877193	0.00704256185138169\\
5.6390977443609	0.00697112631075445\\
5.66416040100251	0.00686214703909596\\
5.68922305764411	0.00672653818813961\\
5.71428571428571	0.00657845632286919\\
5.73934837092732	0.0064340507445138\\
5.76441102756892	0.00630967078536936\\
5.78947368421053	0.00621960512878739\\
5.81453634085213	0.00617370337836586\\
5.83959899749373	0.00617550408062541\\
5.86466165413534	0.00622152287021823\\
5.88972431077694	0.00630199287272465\\
5.91478696741855	0.00640278057071442\\
5.93984962406015	0.00650781763343823\\
5.96491228070175	0.0066014015995744\\
5.98997493734336	0.00666999467532613\\
6.01503759398496	0.00670343899062142\\
6.04010025062657	0.00669567072787964\\
6.06516290726817	0.00664505542404895\\
6.09022556390977	0.00655443616921541\\
6.11528822055138	0.00643092935455121\\
6.14035087719298	0.00628544076297387\\
6.16541353383459	0.00613182071478735\\
6.19047619047619	0.00598555279020764\\
6.21553884711779	0.00586191981279322\\
6.2406015037594	0.005773762453455\\
6.265664160401	0.00572922914995713\\
6.29072681704261	0.00573015955124683\\
6.31578947368421	0.00577169994871782\\
6.34085213032582	0.0058433233590717\\
6.36591478696742	0.00593086965594827\\
6.39097744360902	0.00601892222660684\\
6.41604010025063	0.00609293143563099\\
6.44110275689223	0.00614079569530174\\
6.46616541353383	0.00615387634928766\\
6.49122807017544	0.00612755183297259\\
6.51629072681704	0.00606143286901172\\
6.54135338345865	0.00595932004080523\\
6.56641604010025	0.00582892647334148\\
6.59147869674185	0.00568132877927012\\
6.61654135338346	0.00553006093660052\\
6.64160401002506	0.00538975469695613\\
6.66666666666667	0.00527429412836912\\
6.69172932330827	0.00519466235518487\\
6.71679197994987	0.00515692496770333\\
6.74185463659148	0.00516099026368992\\
6.76691729323308	0.0052006419904761\\
6.79197994987469	0.0052648537361007\\
6.81704260651629	0.00533988970443366\\
6.84210526315789	0.00541152066077952\\
6.8671679197995	0.00546686506077676\\
6.8922305764411	0.00549567070407255\\
6.91729323308271	0.00549107363896754\\
6.94235588972431	0.00544995601952891\\
6.96741854636591	0.00537301704813567\\
6.99248120300752	0.0052646249865989\\
7.01754385964912	0.00513246401303426\\
7.04260651629073	0.00498693969184507\\
7.06766917293233	0.00484027153067527\\
7.09273182957393	0.00470520411728246\\
7.11779448621554	0.00459334871743802\\
7.14285714285714	0.00451335016927559\\
7.16791979949875	0.00446930425251259\\
7.19298245614035	0.00445995671113169\\
7.21804511278195	0.00447902426986659\\
7.24310776942356	0.00451654079464245\\
7.26817042606516	0.00456074658005731\\
7.29323308270677	0.0045999581210645\\
7.31829573934837	0.00462404796940946\\
7.34335839598998	0.00462542092152954\\
7.36842105263158	0.00459954020696279\\
7.39348370927318	0.00454511143766498\\
7.41854636591479	0.00446401589936052\\
7.44360902255639	0.00436104239668221\\
7.468671679198	0.00424342284701873\\
7.4937343358396	0.00412014372133984\\
7.5187969924812	0.00400099707659422\\
7.54385964912281	0.00389696551379262\\
7.56892230576441	0.003818399802297\\
7.59398496240602	0.00377056049898826\\
7.61904761904762	0.00375214645604881\\
7.64411027568922	0.00375886691552815\\
7.66917293233083	0.00377836534510917\\
7.69423558897243	0.00380384100840521\\
7.71929824561404	0.00382569969103262\\
7.74436090225564	0.00383704058820069\\
7.76942355889724	0.00383366622303628\\
7.79448621553885	0.00381422914294992\\
7.81954887218045	0.00377959149280756\\
7.84461152882206	0.00373174625278672\\
7.86967418546366	0.00367532986717069\\
7.89473684210526	0.00361632364176862\\
7.91979949874687	0.00356124002735666\\
7.94486215538847	0.0035161548192967\\
7.96992481203008	0.00348743510061064\\
7.99498746867168	0.00347628062808456\\
8.02005012531328	0.00348199982076393\\
8.04511278195489	0.0035016179796379\\
8.07017543859649	0.00353029797720339\\
8.09523809523809	0.00356240462312989\\
8.1203007518797	0.00359260700069798\\
8.1453634085213	0.00361505297161921\\
8.17042606516291	0.00362653009861709\\
8.19548872180451	0.00362707498586391\\
8.22055137844612	0.00361678079425638\\
8.24561403508772	0.00359828898685418\\
8.27067669172932	0.00357578480748705\\
8.29573934837093	0.00355640285854321\\
8.32080200501253	0.00354572101451394\\
8.34586466165413	0.00354811424349544\\
8.37092731829574	0.00356710673404792\\
8.39598997493734	0.00360320911718777\\
8.42105263157895	0.00365300404515562\\
8.44611528822055	0.00371348452377955\\
8.47117794486216	0.00377754307586727\\
8.49624060150376	0.00383968711902125\\
8.52130325814536	0.00389487578850262\\
8.54636591478697	0.00393840836841639\\
8.57142857142857	0.0039677665061839\\
8.59649122807017	0.00398227547934529\\
8.62155388471178	0.00398319764205871\\
8.64661654135338	0.00397362587259964\\
8.67167919799499	0.0039581690380815\\
8.69674185463659	0.00394257464154704\\
8.7218045112782	0.00393267173782798\\
8.7468671679198	0.00393372219093863\\
8.7719298245614	0.00394964408050319\\
8.79699248120301	0.0039823224937534\\
8.82205513784461	0.00403133106929003\\
8.84711779448622	0.00409408837193421\\
8.87218045112782	0.00416639975128865\\
8.89724310776942	0.00424323426147012\\
8.92230576441103	0.00431955449086212\\
8.94736842105263	0.00439104909657322\\
8.97243107769424	0.00445467668972892\\
8.99749373433584	0.00450898259423882\\
9.02255639097744	0.00455418774034772\\
9.04761904761905	0.00459308451401953\\
9.07268170426065	0.00462656041265752\\
9.09774436090226	0.00465526507371267\\
9.12280701754386	0.00467971782823716\\
9.14786967418546	0.00470035370945958\\
9.17293233082707	0.00471754554055014\\
9.19799498746867	0.0047316163258144\\
9.22305764411028	0.00474284748658796\\
9.24812030075188	0.00475148520053658\\
9.27318295739348	0.00475774576985989\\
9.29824561403509	0.00476182037991059\\
9.32330827067669	0.0047638793722978\\
9.3483709273183	0.00476407606407492\\
9.3734335839599	0.00476255011683106\\
9.3984962406015	0.0047594304590336\\
9.42355889724311	0.00475483777407944\\
9.44862155388471	0.00474888657700054\\
9.47368421052632	0.00474168691119497\\
9.49874686716792	0.00473334570185936\\
9.52380952380952	0.00472396780505595\\
9.54887218045113	0.00471362747359175\\
9.57393483709273	0.00470110876284084\\
9.59899749373434	0.00468101850979356\\
9.62406015037594	0.00464195578533996\\
9.64912280701754	0.00456559384554575\\
9.67418546365915	0.00442704948796863\\
9.69924812030075	0.00419757401590522\\
9.72431077694236	0.00385082001791481\\
9.74937343358396	0.00337320968824328\\
9.77443609022556	0.00277653553046559\\
9.79949874686717	0.00210697566448238\\
9.82456140350877	0.00144168749763041\\
9.84962406015038	0.000867130180653859\\
9.87468671679198	0.000445404109844512\\
9.89974937343358	0.000189039114030387\\
9.92481203007519	6.37383173478161e-05\\
9.94987468671679	1.99535854235737e-05\\
9.9749373433584	1.19083138010841e-05\\
10	5.9947246446025e-06\\
};
\addplot [color=mycolor2, forget plot]
  table[row sep=crcr]{%
1.5	0\\
1.51	0\\
1.52	2.02337787537816e-07\\
1.53	6.07001884896979e-07\\
1.54	1.21396921427095e-06\\
1.55	2.02320509907986e-06\\
1.56	3.03466326595137e-06\\
1.57	4.24828584736569e-06\\
1.58	5.66400338496448e-06\\
1.59	7.28173483352348e-06\\
1.6	9.10138756558876e-06\\
1.61	1.11228573767761e-05\\
1.62	1.33460284917337e-05\\
1.63	1.57707735707668e-05\\
1.64	1.83969537171259e-05\\
1.65	2.12244184849553e-05\\
1.66	2.42530058879047e-05\\
1.67	2.74825424084003e-05\\
1.68	3.09128430075778e-05\\
1.69	3.45437111358743e-05\\
1.7	3.83749387442801e-05\\
1.71	4.24063062962493e-05\\
1.72	4.66375827802682e-05\\
1.73	5.10685257230813e-05\\
1.74	5.56988812035743e-05\\
1.75	6.05283838673127e-05\\
1.76	6.55567569417352e-05\\
1.77	7.07837122520029e-05\\
1.78	7.62089502374998e-05\\
1.79	8.18321599689882e-05\\
1.8	8.76530191664136e-05\\
1.81	9.36711942173621e-05\\
1.82	9.98863401961661e-05\\
1.83	0.000106298100883659\\
1.84	0.000112906108787579\\
1.85	0.000119709985163616\\
1.86	0.000126709340037108\\
1.87	0.000133903772225375\\
1.88	0.000141292869360706\\
1.89	0.000148876207913979\\
1.9	0.000156653353218922\\
1.91	0.000164623859497019\\
1.92	0.000172787269883049\\
1.93	0.000181143116451259\\
1.94	0.000189690920242178\\
1.95	0.000198430191290062\\
1.96	0.000207360428650969\\
1.97	0.000216481120431464\\
1.98	0.000225791743817957\\
1.99	0.000235291765106667\\
2	0.000244980639734206\\
2.01	0.000254857812308795\\
2.02	0.000264922716642095\\
2.03	0.000275174775781668\\
2.04	0.000285613402044044\\
2.05	0.000296237997048415\\
2.06	0.000307047951750941\\
2.07	0.000318042646479667\\
2.08	0.000329221450970056\\
2.09	0.000340583724401127\\
2.1	0.000352128815432202\\
2.11	0.00036385606224026\\
2.12	0.00037576479255789\\
2.13	0.000387854323711854\\
2.14	0.000400123962662238\\
2.15	0.000412573006042207\\
2.16	0.000425200740198357\\
2.17	0.000438006441231654\\
2.18	0.000450989375038963\\
2.19	0.000464148797355176\\
2.2	0.000477483953795912\\
2.21	0.000490994079900807\\
2.22	0.000504678401177392\\
2.23	0.000518536133145534\\
2.24	0.000532566481382471\\
2.25	0.000546768641568409\\
2.26	0.000561141799532695\\
2.27	0.000575685131300558\\
2.28	0.000590397803140422\\
2.29	0.000605278971611772\\
2.3	0.000620327783613591\\
2.31	0.000635543376433355\\
2.32	0.000650924877796576\\
2.33	0.00066647140591691\\
2.34	0.000682182069546807\\
2.35	0.000698055968028713\\
2.36	0.000714092191346815\\
2.37	0.000730289820179331\\
2.38	0.00074664792595134\\
2.39	0.000763165570888143\\
2.4	0.000779841808069164\\
2.41	0.000796675681482383\\
2.42	0.000813666226079287\\
2.43	0.000830812467830359\\
2.44	0.000848113423781083\\
2.45	0.000865568102108461\\
2.46	0.000883175502178061\\
2.47	0.000900934614601566\\
2.48	0.000918844421294838\\
2.49	0.000936903895536485\\
2.5	0.000955112002026935\\
2.51	0.000973467696948009\\
2.52	0.000991969928022991\\
2.53	0.00101061763457719\\
2.54	0.00102940974759901\\
2.55	0.00104834518980144\\
2.56	0.00106742287568417\\
2.57	0.001086641711596\\
2.58	0.00110600059579789\\
2.59	0.00112549841852638\\
2.6	0.00114513406205756\\
2.61	0.00116490640077141\\
2.62	0.00118481430121669\\
2.63	0.00120485662217626\\
2.64	0.00122503221473285\\
2.65	0.00124533992233528\\
2.66	0.00126577858086515\\
2.67	0.00128634701870395\\
2.68	0.00130704405680062\\
2.69	0.0013278685087396\\
2.7	0.00134881918080919\\
2.71	0.00136989487207049\\
2.72	0.00139109437442667\\
2.73	0.00141241647269269\\
2.74	0.00143385994466544\\
2.75	0.0014554235611943\\
2.76	0.00147710608625213\\
2.77	0.00149890627700661\\
2.78	0.00152082288389206\\
2.79	0.0015428546506816\\
2.8	0.00156500031455977\\
2.81	0.00158725860619546\\
2.82	0.00160962824981529\\
2.83	0.00163210796327738\\
2.84	0.00165469645814548\\
2.85	0.00167739243976346\\
2.86	0.0017001946073302\\
2.87	0.00172310165397485\\
2.88	0.00174611226683247\\
2.89	0.00176922512711996\\
2.9	0.00179243891021245\\
2.91	0.00181575228571997\\
2.92	0.00183916391756448\\
2.93	0.00186267246405729\\
2.94	0.00188627657797676\\
2.95	0.00190997490664637\\
2.96	0.00193376609201313\\
2.97	0.00195764877072632\\
2.98	0.00198162157421649\\
2.99	0.00200568312877489\\
3	0.00202983205563314\\
3.01	0.00205406697104321\\
3.02	0.00207838648635776\\
3.03	0.00210278920811074\\
3.04	0.00212727373809828\\
3.05	0.00215183867345994\\
3.06	0.00217648260676019\\
3.07	0.00220120412607016\\
3.08	0.00222600181504976\\
3.09	0.00225087425303\\
3.1	0.00227582001509559\\
3.11	0.00230083767216785\\
3.12	0.00232592579108786\\
3.13	0.00235108293469986\\
3.14	0.00237630766193494\\
3.15	0.00240159852789491\\
3.16	0.00242695408393654\\
3.17	0.0024523728777559\\
3.18	0.00247785345347305\\
3.19	0.0025033943517169\\
3.2	0.00252899410971035\\
3.21	0.00255465126135558\\
3.22	0.00258036433731965\\
3.23	0.00260613186512025\\
3.24	0.0026319523692117\\
3.25	0.00265782437107112\\
3.26	0.00268374638928485\\
3.27	0.00270971693963502\\
3.28	0.00273573453518636\\
3.29	0.00276179768637315\\
3.3	0.00278790490108641\\
3.31	0.0028140546847612\\
3.32	0.00284024554046419\\
3.33	0.0028664759689813\\
3.34	0.00289274446890558\\
3.35	0.00291904953672522\\
3.36	0.00294538966691172\\
3.37	0.00297176335200822\\
3.38	0.00299816908271797\\
3.39	0.00302460534799298\\
3.4	0.00305107063512273\\
3.41	0.00307756342982309\\
3.42	0.00310408221632535\\
3.43	0.00313062547746536\\
3.44	0.00315719169477279\\
3.45	0.00318377934856054\\
3.46	0.00321038691801422\\
3.47	0.00323701288128177\\
3.48	0.00326365571556318\\
3.49	0.00329031389720025\\
3.5	0.00331698590176655\\
3.51	0.00334367020415738\\
3.52	0.00337036527867985\\
3.53	0.00339706959914304\\
3.54	0.00342378163894823\\
3.55	0.0034504998711792\\
3.56	0.00347722276869261\\
3.57	0.00350394880420842\\
3.58	0.00353067645040039\\
3.59	0.0035574041799866\\
3.6	0.00358413046582007\\
3.61	0.00361085378097937\\
3.62	0.00363757259885931\\
3.63	0.00366428539326163\\
3.64	0.00369099063848577\\
3.65	0.0037176868094196\\
3.66	0.00374437238163023\\
3.67	0.0037710458314548\\
3.68	0.00379770563609129\\
3.69	0.00382435027368937\\
3.7	0.0038509782234412\\
3.71	0.00387758796567227\\
3.72	0.00390417798193222\\
3.73	0.00393074675508564\\
3.74	0.00395729276940287\\
3.75	0.00398381451065079\\
3.76	0.00401031046618356\\
3.77	0.00403677912503336\\
3.78	0.00406321897800109\\
3.79	0.00408962851774698\\
3.8	0.00411600623888128\\
3.81	0.0041423506380548\\
3.82	0.00416866021404942\\
3.83	0.00419493346786858\\
3.84	0.00422116890282769\\
3.85	0.00424736502464449\\
3.86	0.00427352034152931\\
3.87	0.00429963336427532\\
3.88	0.00432570260634862\\
3.89	0.00435172658397835\\
3.9	0.00437770381624662\\
3.91	0.00440363282517846\\
3.92	0.00442951213583156\\
3.93	0.00445534027638597\\
3.94	0.00448111577823376\\
3.95	0.00450683717606846\\
3.96	0.00453250300797447\\
3.97	0.00455811181551634\\
3.98	0.00458366214382792\\
3.99	0.00460915254170141\\
4	0.00463458156167626\\
4.01	0.00465994776012798\\
4.02	0.00468524969735677\\
4.03	0.00471048593767605\\
4.04	0.00473565504950082\\
4.05	0.00476075560543594\\
4.06	0.00478578618236415\\
4.07	0.00481074536153404\\
4.08	0.00483563172864784\\
4.09	0.00486044387394899\\
4.1	0.00488518039230963\\
4.11	0.00490983988331788\\
4.12	0.00493442095136494\\
4.13	0.00495892220573204\\
4.14	0.00498334226067722\\
4.15	0.00500767973552186\\
4.16	0.00503193325473711\\
4.17	0.0050561014480301\\
4.18	0.00508018295042992\\
4.19	0.00510417640237345\\
4.2	0.00512808044979098\\
4.21	0.00515189374419161\\
4.22	0.00517561494274846\\
4.23	0.00519924270838365\\
4.24	0.00522277570985308\\
4.25	0.00524621262183102\\
4.26	0.00526955212499439\\
4.27	0.00529279290610696\\
4.28	0.00531593365810316\\
4.29	0.00533897308017177\\
4.3	0.00536190987783936\\
4.31	0.00538474276305345\\
4.32	0.00540747045426548\\
4.33	0.0054300916765135\\
4.34	0.00545260516150463\\
4.35	0.00547500964769726\\
4.36	0.00549730388038305\\
4.37	0.00551948661176858\\
4.38	0.0055415566010568\\
4.39	0.00556351261452824\\
4.4	0.00558535342562191\\
4.41	0.00560707781501597\\
4.42	0.00562868457070806\\
4.43	0.00565017248809548\\
4.44	0.00567154037005498\\
4.45	0.00569278702702232\\
4.46	0.00571391127707158\\
4.47	0.0057349119459941\\
4.48	0.00575578786737726\\
4.49	0.00577653788268284\\
4.5	0.00579716084132517\\
4.51	0.005817655600749\\
4.52	0.005838021026507\\
4.53	0.00585825599233705\\
4.54	0.00587835938023914\\
4.55	0.00589833008055206\\
4.56	0.00591816699202975\\
4.57	0.00593786902191728\\
4.58	0.00595743508602664\\
4.59	0.00597686410881214\\
4.6	0.0059961550234455\\
4.61	0.00601530677189069\\
4.62	0.00603431830497835\\
4.63	0.00605318858248002\\
4.64	0.00607191657318193\\
4.65	0.00609050125495853\\
4.66	0.00610894161484574\\
4.67	0.00612723664911376\\
4.68	0.00614538536333968\\
4.69	0.00616338677247965\\
4.7	0.00618123990094082\\
4.71	0.0061989437826529\\
4.72	0.00621649746113939\\
4.73	0.00623389998958846\\
4.74	0.00625115043092357\\
4.75	0.00626824785787367\\
4.76	0.00628519135304312\\
4.77	0.00630198000898125\\
4.78	0.00631861292825157\\
4.79	0.00633508922350069\\
4.8	0.00635140801752684\\
4.81	0.00636756844334812\\
4.82	0.00638356964427031\\
4.83	0.00639941077395445\\
4.84	0.00641509099648401\\
4.85	0.00643060948643174\\
4.86	0.00644596542892615\\
4.87	0.0064611580197177\\
4.88	0.00647618646524458\\
4.89	0.00649104998269821\\
4.9	0.00650574780008837\\
4.91	0.00652027915630796\\
4.92	0.00653464330119747\\
4.93	0.00654883949560904\\
4.94	0.00656286701147024\\
4.95	0.0065767251318475\\
4.96	0.0065904131510091\\
4.97	0.00660393037448799\\
4.98	0.0066172761191441\\
4.99	0.00663044971322641\\
5	0.00664345049643463\\
5.01	0.00665627781998058\\
5.02	0.00666893104664919\\
5.03	0.00668140955085917\\
5.04	0.00669371271872338\\
5.05	0.00670583994810877\\
5.06	0.00671779064869613\\
5.07	0.00672956424203934\\
5.08	0.00674116016162442\\
5.09	0.00675257785292814\\
5.1	0.0067638167734764\\
5.11	0.00677487639290219\\
5.12	0.0067857561930033\\
5.13	0.00679645566779961\\
5.14	0.00680697432359016\\
5.15	0.00681731167900981\\
5.16	0.00682746726508562\\
5.17	0.00683744062529289\\
5.18	0.00684723131561092\\
5.19	0.0068568389045784\\
5.2	0.00686626297334853\\
5.21	0.00687550311574378\\
5.22	0.00688455893831039\\
5.23	0.00689343006037257\\
5.24	0.00690211611408631\\
5.25	0.00691061674449297\\
5.26	0.00691893160957259\\
5.27	0.00692706038029677\\
5.28	0.00693500274068141\\
5.29	0.00694275838783909\\
5.3	0.00695032703203114\\
5.31	0.00695770839671944\\
5.32	0.00696490221861799\\
5.33	0.00697190824774412\\
5.34	0.00697872624746948\\
5.35	0.00698535599457074\\
5.36	0.00699179727928\\
5.37	0.00699804990533498\\
5.38	0.00700411369002891\\
5.39	0.00700998846426019\\
5.4	0.00701567407258175\\
5.41	0.00702117037325023\\
5.42	0.00702647723827483\\
5.43	0.00703159455346602\\
5.44	0.00703652221848388\\
5.45	0.00704126014688634\\
5.46	0.00704580826617708\\
5.47	0.0070501665178533\\
5.48	0.00705433485745319\\
5.49	0.0070583132546032\\
5.5	0.00706210169306515\\
5.51	0.00706570017078306\\
5.52	0.00706910869992983\\
5.53	0.00707232730695368\\
5.54	0.00707535603262444\\
5.55	0.00707819493207962\\
5.56	0.00708084407487026\\
5.57	0.00708330354500674\\
5.58	0.00708557344100421\\
5.59	0.00708765387592803\\
5.6	0.00708954497743896\\
5.61	0.00709124688783818\\
5.62	0.00709275976411222\\
5.63	0.00709408377797768\\
5.64	0.00709521911592581\\
5.65	0.00709616597926703\\
5.66	0.00709692458417517\\
5.67	0.00709749516173175\\
5.68	0.00709787795796999\\
5.69	0.00709807323391883\\
5.7	0.0070980812656467\\
5.71	0.00709790234430532\\
5.72	0.00709753677617331\\
5.73	0.00709698488269972\\
5.74	0.00709624700054748\\
5.75	0.0070953234816368\\
5.76	0.00709421469318837\\
5.77	0.00709292101776664\\
5.78	0.00709144285332289\\
5.79	0.00708978061323835\\
5.8	0.00708793472636716\\
5.81	0.00708590563707933\\
5.82	0.00708369380530366\\
5.83	0.00708129970657053\\
5.84	0.00707872383205479\\
5.85	0.00707596668861843\\
5.86	0.00707302879885339\\
5.87	0.00706991070112422\\
5.88	0.00706661294961079\\
5.89	0.00706313611435089\\
5.9	0.00705948078128294\\
5.91	0.00705564755228855\\
5.92	0.00705163704523514\\
5.93	0.00704744989401858\\
5.94	0.00704308674860573\\
5.95	0.00703854827507706\\
5.96	0.00703383515566924\\
5.97	0.00702894808881769\\
5.98	0.00702388778919923\\
5.99	0.00701865498777462\\
6	0.00701325043183118\\
6.01	0.00700767488502539\\
6.02	0.00700192912742548\\
6.03	0.00699601395555408\\
6.04	0.00698993018243078\\
6.05	0.00698367863761475\\
6.06	0.0069772601672474\\
6.07	0.00697067563409496\\
6.08	0.00696392591759107\\
6.09	0.00695701191387942\\
6.1	0.00694993453585631\\
6.11	0.00694269471321321\\
6.12	0.00693529339247934\\
6.13	0.00692773153706415\\
6.14	0.00692001012729985\\
6.15	0.0069121301604838\\
6.16	0.00690409265092091\\
6.17	0.00689589862996599\\
6.18	0.006887549146066\\
6.19	0.00687904526480219\\
6.2	0.00687038806893222\\
6.21	0.0068615786584321\\
6.22	0.00685261815053807\\
6.23	0.00684350767978827\\
6.24	0.00683424839806434\\
6.25	0.00682484147463277\\
6.26	0.00681528809618612\\
6.27	0.00680558946688398\\
6.28	0.00679574680839372\\
6.29	0.00678576135993101\\
6.3	0.00677563437829999\\
6.31	0.0067653671379332\\
6.32	0.00675496093093111\\
6.33	0.00674441706710137\\
6.34	0.00673373687399755\\
6.35	0.00672292169695754\\
6.36	0.00671197289914147\\
6.37	0.00670089186156905\\
6.38	0.00668967998315647\\
6.39	0.00667833868075265\\
6.4	0.00666686938917487\\
6.41	0.00665527356124374\\
6.42	0.00664355266781743\\
6.43	0.00663170819782513\\
6.44	0.00661974165829968\\
6.45	0.00660765457440932\\
6.46	0.00659544848948847\\
6.47	0.00658312496506755\\
6.48	0.00657068558090165\\
6.49	0.00655813193499819\\
6.5	0.00654546564364317\\
6.51	0.00653268834142635\\
6.52	0.00651980168126491\\
6.53	0.00650680733442573\\
6.54	0.00649370699054613\\
6.55	0.00648050235765295\\
6.56	0.00646719516218002\\
6.57	0.00645378714898369\\
6.58	0.00644028008135651\\
6.59	0.00642667574103896\\
6.6	0.00641297592822891\\
6.61	0.00639918246158898\\
6.62	0.00638529717825148\\
6.63	0.00637132193382083\\
6.64	0.00635725860237349\\
6.65	0.00634310907645504\\
6.66	0.00632887526707443\\
6.67	0.00631455910369521\\
6.68	0.00630016253422361\\
6.69	0.00628568752499325\\
6.7	0.00627113606074641\\
6.71	0.00625651014461167\\
6.72	0.00624181179807768\\
6.73	0.006227043060963\\
6.74	0.00621220599138173\\
6.75	0.0061973026657048\\
6.76	0.00618233517851672\\
6.77	0.00616730564256755\\
6.78	0.00615221618871989\\
6.79	0.00613706896589078\\
6.8	0.00612186614098813\\
6.81	0.00610660989884155\\
6.82	0.00609130244212739\\
6.83	0.00607594599128755\\
6.84	0.00606054278444215\\
6.85	0.00604509507729543\\
6.86	0.00602960514303491\\
6.87	0.00601407527222344\\
6.88	0.00599850777268376\\
6.89	0.00598290496937553\\
6.9	0.00596726920426435\\
6.91	0.00595160283618248\\
6.92	0.00593590824068111\\
6.93	0.00592018780987376\\
6.94	0.00590444395227044\\
6.95	0.00588867909260244\\
6.96	0.00587289567163726\\
6.97	0.00585709614598341\\
6.98	0.00584128298788471\\
6.99	0.00582545868500388\\
7	0.00580962574019482\\
7.01	0.00579378667126353\\
7.02	0.00577794401071707\\
7.03	0.0057621003055003\\
7.04	0.00574625811672009\\
7.05	0.00573042001935647\\
7.06	0.00571458860196058\\
7.07	0.00569876646633878\\
7.08	0.00568295622722278\\
7.09	0.00566716051192529\\
7.1	0.00565138195998077\\
7.11	0.00563562322277111\\
7.12	0.00561988696313557\\
7.13	0.00560417585496483\\
7.14	0.00558849258277875\\
7.15	0.00557283984128727\\
7.16	0.00555722033493443\\
7.17	0.00554163677742477\\
7.18	0.00552609189123204\\
7.19	0.00551058840708974\\
7.2	0.00549512906346316\\
7.21	0.00547971660600259\\
7.22	0.00546435378697743\\
7.23	0.00544904336469075\\
7.24	0.00543378810287421\\
7.25	0.00541859077006285\\
7.26	0.00540345413894958\\
7.27	0.00538838098571915\\
7.28	0.00537337408936125\\
7.29	0.00535843623096262\\
7.3	0.00534357019297799\\
7.31	0.00532877875847954\\
7.32	0.00531406471038492\\
7.33	0.00529943083066356\\
7.34	0.00528487989952125\\
7.35	0.0052704146945629\\
7.36	0.00525603798993345\\
7.37	0.00524175255543695\\
7.38	0.0052275611556338\\
7.39	0.00521346654891628\\
7.4	0.00519947148656239\\
7.41	0.00518557871176824\\
7.42	0.00517179095865917\\
7.43	0.00515811095127968\\
7.44	0.00514454140256273\\
7.45	0.0051310850132784\\
7.46	0.0051177444709626\\
7.47	0.00510452244882599\\
7.48	0.00509142160464369\\
7.49	0.00507844457962638\\
7.5	0.00506559399727319\\
7.51	0.00505287246220712\\
7.52	0.00504028255899366\\
7.53	0.00502782685094337\\
7.54	0.00501550787889909\\
7.55	0.00500332816000879\\
7.56	0.00499129018648492\\
7.57	0.0049793964243512\\
7.58	0.0049676493121778\\
7.59	0.00495605125980625\\
7.6	0.00494460464706489\\
7.61	0.00493331182247636\\
7.62	0.00492217510195811\\
7.63	0.00491119676751753\\
7.64	0.00490037906594279\\
7.65	0.00488972420749101\\
7.66	0.0048792343645751\\
7.67	0.00486891167045092\\
7.68	0.00485875821790616\\
7.69	0.00484877605795272\\
7.7	0.0048389671985241\\
7.71	0.00482933360317967\\
7.72	0.00481987718981734\\
7.73	0.00481059982939656\\
7.74	0.0048015033446734\\
7.75	0.00479258950894949\\
7.76	0.00478386004483669\\
7.77	0.00477531662303944\\
7.78	0.0047669608611565\\
7.79	0.00475879432250407\\
7.8	0.00475081851496219\\
7.81	0.00474303488984629\\
7.82	0.0047354448408057\\
7.83	0.00472804970275112\\
7.84	0.00472085075081283\\
7.85	0.00471384919933147\\
7.86	0.00470704620088322\\
7.87	0.00470044284534118\\
7.88	0.00469404015897463\\
7.89	0.00468783910358786\\
7.9	0.00468184057570038\\
7.91	0.00467604540576977\\
7.92	0.00467045435745908\\
7.93	0.00466506812694996\\
7.94	0.00465988734230306\\
7.95	0.00465491256286699\\
7.96	0.00465014427873703\\
7.97	0.00464558291026489\\
7.98	0.00464122880762036\\
7.99	0.00463708225040613\\
8	0.00463314344732632\\
8.01	0.00462941253590985\\
8.02	0.00462588958228907\\
8.03	0.00462257458103431\\
8.04	0.0046194674550449\\
8.05	0.00461656805549695\\
8.06	0.00461387616184807\\
8.07	0.00461139148189934\\
8.08	0.00460911365191443\\
8.09	0.00460704223679582\\
8.1	0.00460517673031795\\
8.11	0.00460351655541692\\
8.12	0.00460206106453643\\
8.13	0.00460080954002934\\
8.14	0.00459976119461423\\
8.15	0.00459891517188625\\
8.16	0.00459827054688138\\
8.17	0.00459782632669319\\
8.18	0.004597581451141\\
8.19	0.00459753479348835\\
8.2	0.00459768516121053\\
8.21	0.00459803129680988\\
8.22	0.00459857187867743\\
8.23	0.00459930552199942\\
8.24	0.00460023077970717\\
8.25	0.00460134614346871\\
8.26	0.0046026500447204\\
8.27	0.00460414085573695\\
8.28	0.00460581689073796\\
8.29	0.00460767640702919\\
8.3	0.00460971760617667\\
8.31	0.00461193863521184\\
8.32	0.00461433758786567\\
8.33	0.00461691250582989\\
8.34	0.00461966138004335\\
8.35	0.00462258215200154\\
8.36	0.00462567271508726\\
8.37	0.00462893091592048\\
8.38	0.00463235455572542\\
8.39	0.00463594139171283\\
8.4	0.00463968913847563\\
8.41	0.00464359546939578\\
8.42	0.00464765801806071\\
8.43	0.00465187437968722\\
8.44	0.00465624211255112\\
8.45	0.00466075873942077\\
8.46	0.00466542174899265\\
8.47	0.00467022859732747\\
8.48	0.00467517670928475\\
8.49	0.00468026347995468\\
8.5	0.0046854862760853\\
8.51	0.00469084243750364\\
8.52	0.00469632927852935\\
8.53	0.00470194408937934\\
8.54	0.0047076841375621\\
8.55	0.00471354666926036\\
8.56	0.00471952891070091\\
8.57	0.00472562806951039\\
8.58	0.00473184133605584\\
8.59	0.00473816588476912\\
8.6	0.00474459887545404\\
8.61	0.00475113745457537\\
8.62	0.00475777875652892\\
8.63	0.00476451990489168\\
8.64	0.00477135801365152\\
8.65	0.00477829018841563\\
8.66	0.00478531352759704\\
8.67	0.00479242512357886\\
8.68	0.00479962206385545\\
8.69	0.00480690143215033\\
8.7	0.00481426030951027\\
8.71	0.0048216957753753\\
8.72	0.00482920490862433\\
8.73	0.00483678478859607\\
8.74	0.00484443249608526\\
8.75	0.00485214511431376\\
8.76	0.00485991972987672\\
8.77	0.00486775343366355\\
8.78	0.0048756433217537\\
8.79	0.00488358649628744\\
8.8	0.0048915800663115\\
8.81	0.0048996211485997\\
8.82	0.00490770686844886\\
8.83	0.00491583436044994\\
8.84	0.00492400076923469\\
8.85	0.00493220325019811\\
8.86	0.00494043897019675\\
8.87	0.00494870510822338\\
8.88	0.00495699885605808\\
8.89	0.00496531741889622\\
8.9	0.00497365801595362\\
8.91	0.00498201788104912\\
8.92	0.00499039426316516\\
8.93	0.0049987844269865\\
8.94	0.00500718565341763\\
8.95	0.0050155952400792\\
8.96	0.00502401050178388\\
8.97	0.00503242877099213\\
8.98	0.0050408473982482\\
8.99	0.00504926375259694\\
9	0.00505767522198164\\
};
\end{axis}

\begin{axis}[%
width=0.4\figW,
height=0.15\figH,
at={(0\figW,0.645\figH)},
scale only axis,
yticklabel style={/pgf/number format/fixed},
xmin=0,
xmax=10,
ymin=0,
ymax=0.1,
axis background/.style={fill=white}
]
\addplot [color=mycolor1, forget plot]
  table[row sep=crcr]{%
0	6.74317351257218e-05\\
0.025062656641604	1.85705120703021e-05\\
0.050125313283208	9.00038819074966e-05\\
0.075187969924812	0.000175071168892688\\
0.100250626566416	0.00028363634279439\\
0.12531328320802	0.000413275278926579\\
0.150375939849624	0.000751284973089269\\
0.175438596491228	0.00128982609602749\\
0.200501253132832	0.00195099498707513\\
0.225563909774436	0.00265809504335846\\
0.25062656641604	0.0033280444098044\\
0.275689223057644	0.00389796536577735\\
0.300751879699248	0.00433725448811785\\
0.325814536340852	0.00464514899113659\\
0.350877192982456	0.00484055261285347\\
0.37593984962406	0.00495091925422411\\
0.401002506265664	0.0050039153320266\\
0.426065162907268	0.0050226785987246\\
0.451127819548872	0.00502396027386954\\
0.476190476190476	0.00501808172134061\\
0.50125313283208	0.00500989527714567\\
0.526315789473684	0.00500049179434962\\
0.551378446115288	0.00498980500661439\\
0.576441102756892	0.00497773680358166\\
0.601503759398496	0.00496418349608952\\
0.6265664160401	0.00494903615546216\\
0.651629072681704	0.00493218116998542\\
0.676691729323308	0.00491350107020025\\
0.701754385964912	0.00489287568470422\\
0.726817042606516	0.00487018370026538\\
0.75187969924812	0.00484530471468725\\
0.776942355889724	0.00481812188858874\\
0.802005012531328	0.00478852532375749\\
0.827067669172932	0.00475641632179807\\
0.852130325814536	0.0047241950719809\\
0.87719298245614	0.00469258214420809\\
0.902255639097744	0.00465952260856782\\
0.927318295739348	0.00462520198506605\\
0.952380952380952	0.00458987276906351\\
0.977443609022556	0.00455386729292202\\
1.00250626566416	0.00452582432660118\\
1.02756892230576	0.00450782863565138\\
1.05263157894737	0.00449311686152178\\
1.07769423558897	0.00450369882434715\\
1.10275689223058	0.00452378926662771\\
1.12781954887218	0.0045716122765265\\
1.15288220551378	0.00463569349285962\\
1.17794486215539	0.00471531203161273\\
1.20300751879699	0.00482376766699606\\
1.2280701754386	0.00494287611635572\\
1.2531328320802	0.00507219730309384\\
1.2781954887218	0.00521098590959141\\
1.30325814536341	0.00535813928822636\\
1.32832080200501	0.00551639695811635\\
1.35338345864662	0.00568098359142618\\
1.37844611528822	0.00584563174022947\\
1.40350877192982	0.00600713133625112\\
1.42857142857143	0.00616143165545533\\
1.45363408521303	0.00630321049640837\\
1.47869674185464	0.00642495315778309\\
1.50375939849624	0.006515071532663\\
1.52882205513784	0.00672536906824165\\
1.55388471177945	0.0071533441362952\\
1.57894736842105	0.00807006734184681\\
1.60401002506266	0.00940845450836902\\
1.62907268170426	0.0107860150259409\\
1.65413533834586	0.0120909563308621\\
1.67919799498747	0.013262562044702\\
1.70426065162907	0.014234739477701\\
1.72932330827068	0.0149726985900722\\
1.75438596491228	0.0154588012990309\\
1.77944862155388	0.0156956416844791\\
1.80451127819549	0.0157095991058033\\
1.82957393483709	0.0155545140738665\\
1.8546365914787	0.0153143661543768\\
1.8796992481203	0.0151142996525134\\
1.9047619047619	0.0151325350329766\\
1.92982456140351	0.0154707211928771\\
1.95488721804511	0.0162775558267234\\
1.97994987468672	0.017483508271214\\
2.00501253132832	0.0190564034646535\\
2.03007518796993	0.0208348612729695\\
2.05513784461153	0.0227052039374947\\
2.08020050125313	0.0245584214796026\\
2.10526315789474	0.0262984235972758\\
2.13032581453634	0.0278461188366125\\
2.15538847117795	0.0291619758276248\\
2.18045112781955	0.0302017738735937\\
2.20551378446115	0.030934457649625\\
2.23057644110276	0.0313668685915589\\
2.25563909774436	0.0315319135910231\\
2.28070175438596	0.0314890341237281\\
2.30576441102757	0.0313229398751693\\
2.33082706766917	0.0311392037900564\\
2.35588972431078	0.031055086719607\\
2.38095238095238	0.0311846842828683\\
2.40601503759398	0.0316198708085116\\
2.43107769423559	0.0324120067445201\\
2.45614035087719	0.0335612665294192\\
2.4812030075188	0.0350179455787404\\
2.5062656641604	0.0366943661734142\\
2.53132832080201	0.0384817282531698\\
2.55639097744361	0.0402662501998689\\
2.58145363408521	0.0419416631869666\\
2.60651629072682	0.0434177169005629\\
2.63157894736842	0.0446255903318435\\
2.65664160401003	0.0455212532724871\\
2.68170426065163	0.0460875122462228\\
2.70676691729323	0.0463350779652674\\
2.73182957393484	0.0463026290033894\\
2.75689223057644	0.0460555153903223\\
2.78195488721805	0.045682438161493\\
2.80701754385965	0.0452892299864885\\
2.83208020050125	0.0449889661644336\\
2.85714285714286	0.0448883858093765\\
2.88220551378446	0.0450721709694407\\
2.90726817042607	0.0455885260843852\\
2.93233082706767	0.04644039488183\\
2.95739348370927	0.0475853129471905\\
2.98245614035088	0.0489437076871123\\
3.00751879699248	0.0504124997735739\\
3.03258145363409	0.0518799335070352\\
3.05764411027569	0.0532386188641394\\
3.08270676691729	0.0543954964065547\\
3.1077694235589	0.05527873457257\\
3.1328320802005	0.0558421351438785\\
3.15789473684211	0.0560676578959056\\
3.18295739348371	0.0559664587112262\\
3.20802005012531	0.0555785397579101\\
3.23308270676692	0.0549708070464188\\
3.25814536340852	0.0542330585218712\\
3.28320802005013	0.0534712649183037\\
3.30827067669173	0.0527976286223741\\
3.33333333333333	0.0523175497727934\\
3.35839598997494	0.0521148975362328\\
3.38345864661654	0.0522385038946243\\
3.40852130325815	0.0526935859422887\\
3.43358395989975	0.0534408763005055\\
3.45864661654135	0.0544037110475779\\
3.48370927318296	0.0554806621078181\\
3.50877192982456	0.0565600885625725\\
3.53383458646617	0.057533535936532\\
3.55889724310777	0.0583063521837049\\
3.58395989974937	0.0588051957688276\\
3.60902255639098	0.0589828293232237\\
3.63408521303258	0.0588207741774765\\
3.65914786967419	0.0583302718053258\\
3.68421052631579	0.0575517392818007\\
3.70927318295739	0.0565526111803045\\
3.734335839599	0.0554231809829036\\
3.7593984962406	0.054269867948385\\
3.78446115288221	0.0532054010651911\\
3.80952380952381	0.0523359665952173\\
3.83458646616541	0.0517465547228972\\
3.85964912280702	0.0514872755395969\\
3.88471177944862	0.0515643652610798\\
3.90977443609023	0.0519388896710544\\
3.93483709273183	0.0525336790305462\\
3.95989974937343	0.0532461880011374\\
3.98496240601504	0.0539635112993411\\
4.01002506265664	0.0545762301353899\\
4.03508771929825	0.0549892716396986\\
4.06015037593985	0.0551294071441783\\
4.08521303258145	0.054949838589676\\
4.11027568922306	0.0544325509775041\\
4.13533834586466	0.0535889872445899\\
4.16040100250627	0.0524593346990394\\
4.18546365914787	0.0511104000448378\\
4.21052631578947	0.0496317325602951\\
4.23558897243108	0.0481293862591383\\
4.26065162907268	0.0467166410193472\\
4.28571428571429	0.0455014118691684\\
4.31077694235589	0.0445712508534127\\
4.33583959899749	0.0439786615933839\\
4.3609022556391	0.0437309125775349\\
4.3859649122807	0.0437881360890231\\
4.41102756892231	0.0440707137447809\\
4.43609022556391	0.0444733664598258\\
4.46115288220551	0.044881422934652\\
4.48621553884712	0.0451853194014608\\
4.51127819548872	0.0452913536466704\\
4.53634085213033	0.0451285222357279\\
4.56140350877193	0.0446522074005572\\
4.58646616541353	0.0438456650698811\\
4.61152882205514	0.0427200672806893\\
4.63659147869674	0.0413135367427817\\
4.66165413533835	0.0396892789085078\\
4.68671679197995	0.0379325645935931\\
4.71177944862155	0.0361459241873209\\
4.73684210526316	0.0344415541097281\\
4.76190476190476	0.0329299126347561\\
4.78696741854637	0.0317043922369251\\
4.81203007518797	0.0308242751755998\\
4.83709273182957	0.0303011355033275\\
4.86215538847118	0.0300947864820716\\
4.88721804511278	0.0301214787782822\\
4.91228070175439	0.030271113568368\\
4.93734335839599	0.0304266587673441\\
4.96240601503759	0.030480059221582\\
4.9874686716792	0.0303424372242557\\
5.0125313283208	0.0299491622767956\\
5.03759398496241	0.0292614070573851\\
5.06265664160401	0.0282656919629681\\
5.08771929824561	0.0269724452940093\\
5.11278195488722	0.0254141814531377\\
5.13784461152882	0.0236436162789071\\
5.16290726817043	0.0217318363866752\\
5.18796992481203	0.0197663841213808\\
5.21303258145363	0.0178485996589706\\
5.23809523809524	0.0160885134717\\
5.26315789473684	0.0145940799171804\\
5.28822055137845	0.0134512875036322\\
5.31328320802005	0.012697014053062\\
5.33834586466165	0.0123127553290287\\
5.36340852130326	0.0121933216282793\\
5.38847117794486	0.012181662521356\\
5.41353383458647	0.0121507440535424\\
5.43859649122807	0.0120211962571787\\
5.46365914786967	0.0117362995407804\\
5.48872180451128	0.0112729545896646\\
5.51378446115288	0.0106677533336295\\
5.53884711779449	0.00999780079736775\\
5.56390977443609	0.00933976999904468\\
5.58897243107769	0.00875306410835869\\
5.6140350877193	0.00828034795885743\\
5.6390977443609	0.00788524475017338\\
5.66416040100251	0.00759083949133629\\
5.68922305764411	0.00736020749819774\\
5.71428571428571	0.00718946431027282\\
5.73934837092732	0.00711663930858258\\
5.76441102756892	0.00712644680373912\\
5.78947368421053	0.0072414111570899\\
5.81453634085213	0.00748220780721557\\
5.83959899749373	0.0077359652047707\\
5.86466165413534	0.00788368872237554\\
5.88972431077694	0.00792214472605729\\
5.91478696741855	0.00796659510724038\\
5.93984962406015	0.00821025764295442\\
5.96491228070175	0.00885751596237848\\
5.98997493734336	0.00995870138400683\\
6.01503759398496	0.0114157284624354\\
6.04010025062657	0.0131399253107579\\
6.06516290726817	0.0150263167329643\\
6.09022556390977	0.0169721762918357\\
6.11528822055138	0.0188842198794437\\
6.14035087719298	0.0206810984858001\\
6.16541353383459	0.0223277133015614\\
6.19047619047619	0.0237470362161825\\
6.21553884711779	0.0248934727827389\\
6.2406015037594	0.0257477022150036\\
6.265664160401	0.0263118069707085\\
6.29072681704261	0.0266114078433422\\
6.31578947368421	0.0266972091922863\\
6.34085213032582	0.0266451082119484\\
6.36591478696742	0.0265534615071863\\
6.39097744360902	0.0265356296315098\\
6.41604010025063	0.0267062951625269\\
6.44110275689223	0.0271622995059375\\
6.46616541353383	0.0279627618204811\\
6.49122807017544	0.029116151999178\\
6.51629072681704	0.0305799371077991\\
6.54135338345865	0.032271956667575\\
6.56641604010025	0.0340874273861076\\
6.59147869674185	0.0359152727661614\\
6.61654135338346	0.0376505849883459\\
6.64160401002506	0.0392029619002433\\
6.66666666666667	0.0405017742158135\\
6.69172932330827	0.0414995121141469\\
6.71679197994987	0.0421739878645878\\
6.74185463659148	0.0425297500366932\\
6.76691729323308	0.0425987036883203\\
6.79197994987469	0.0424396033194157\\
6.81704260651629	0.0421357549851931\\
6.84210526315789	0.0417899720647996\\
6.8671679197995	0.0415157767032509\\
6.8922305764411	0.041424406750184\\
6.91729323308271	0.0416087132379412\\
6.94235588972431	0.0421272277802631\\
6.96741854636591	0.0429931743835741\\
6.99248120300752	0.0441723100836515\\
7.01754385964912	0.0455901319505024\\
7.04260651629073	0.0471454013240593\\
7.06766917293233	0.0487255013741014\\
7.09273182957393	0.0502201365539619\\
7.11779448621554	0.0515318305153169\\
7.14285714285714	0.0525831978026577\\
7.16791979949875	0.0533216362216531\\
7.19298245614035	0.0537221384666543\\
7.21804511278195	0.0537886881269557\\
7.24310776942356	0.0535543905680221\\
7.26817042606516	0.0530801685125641\\
7.29323308270677	0.0524515505551487\\
7.31829573934837	0.0517728562952925\\
7.34335839598998	0.0511580916844205\\
7.36842105263158	0.0507184443064343\\
7.39348370927318	0.0505471920409937\\
7.41854636591479	0.0507047238728034\\
7.44360902255639	0.0512082747986089\\
7.468671679198	0.0520287676625586\\
7.4937343358396	0.0530963494695032\\
7.5187969924812	0.0543124474643505\\
7.54385964912281	0.0555645911211414\\
7.56892230576441	0.0567405314970848\\
7.59398496240602	0.0577396621460796\\
7.61904761904762	0.0584812243947498\\
7.64411027568922	0.0589096491629662\\
7.66917293233083	0.0589976479799488\\
7.69423558897243	0.0587475612832307\\
7.71929824561404	0.0581912157981281\\
7.74436090225564	0.0573882473408444\\
7.76942355889724	0.0564225626055711\\
7.79448621553885	0.0553964035943651\\
7.81954887218045	0.054421471317978\\
7.84461152882206	0.0536069732812392\\
7.86967418546366	0.0530454599230425\\
7.89473684210526	0.052798761771366\\
7.91979949874687	0.0528874948575749\\
7.94486215538847	0.0532874026048074\\
7.96992481203008	0.0539338264177971\\
7.99498746867168	0.0547328144359579\\
8.02005012531328	0.0555754901885613\\
8.04511278195489	0.0563522147993388\\
8.07017543859649	0.0569643024107554\\
8.09523809523809	0.0573325089633822\\
8.1203007518797	0.0574024943547238\\
8.1453634085213	0.0571478366290696\\
8.17042606516291	0.056571146441092\\
8.19548872180451	0.0557036071004766\\
8.22055137844612	0.0546029820506678\\
8.24561403508772	0.053349853076821\\
8.27067669172932	0.0520416404704716\\
8.29573934837093	0.0507839312334958\\
8.32080200501253	0.049679002452547\\
8.34586466165413	0.0488123525131211\\
8.37092731829574	0.0482394256910699\\
8.39598997493734	0.0479758463948513\\
8.42105263157895	0.0479943199139012\\
8.44611528822055	0.0482294427359187\\
8.47117794486216	0.0485889169929121\\
8.49624060150376	0.0489677878223016\\
8.52130325814536	0.0492622593977678\\
8.54636591478697	0.0493809125238903\\
8.57142857142857	0.0492526374956756\\
8.59649122807017	0.0488315800926929\\
8.62155388471178	0.0480997628714779\\
8.64661654135338	0.0470679924602155\\
8.67167919799499	0.0457754128097256\\
8.69674185463659	0.0442877301549699\\
8.7218045112782	0.0426937561213979\\
8.7468671679198	0.0410995351255694\\
8.7719298245614	0.0396191066214563\\
8.79699248120301	0.0383682261445048\\
8.82205513784461	0.0375297523280683\\
8.84711779448622	0.037044233092014\\
8.87218045112782	0.0369021101399127\\
8.89724310776942	0.0370452632659712\\
8.92230576441103	0.0373781553602326\\
8.94736842105263	0.0377849513386299\\
8.97243107769424	0.0381469638159314\\
8.99749373433584	0.0383565788387872\\
9.02255639097744	0.0383264942338786\\
9.04761904761905	0.0382458887428993\\
9.07268170426065	0.0381221363253286\\
9.09774436090226	0.0379631664456425\\
9.12280701754386	0.0378212534526078\\
9.14786967418546	0.0376975079451554\\
9.17293233082707	0.0375565814754432\\
9.19799498746867	0.0374031987864676\\
9.22305764411028	0.0372415851431048\\
9.24812030075188	0.0370754373674838\\
9.27318295739348	0.0369079291381442\\
9.29824561403509	0.0367417370751588\\
9.32330827067669	0.0366236023197283\\
9.3483709273183	0.0365118956853972\\
9.3734335839599	0.0364013186757552\\
9.3984962406015	0.0362932655545402\\
9.42355889724311	0.0361888846371135\\
9.44862155388471	0.0360891008824163\\
9.47368421052632	0.0359946386618822\\
9.49874686716792	0.0359060440128498\\
9.52380952380952	0.035823705945727\\
9.54887218045113	0.0357476854754451\\
9.57393483709273	0.0356695375588409\\
9.59899749373434	0.0355535501245126\\
9.62406015037594	0.0353247344461265\\
9.64912280701754	0.034862309883717\\
9.67418546365915	0.0340004614086755\\
9.69924812030075	0.032542163757724\\
9.72431077694236	0.0302936869273628\\
9.74937343358396	0.0271247241095803\\
9.77443609022556	0.0230478821605108\\
9.79949874686717	0.018290225048036\\
9.82456140350877	0.0133075548626569\\
9.84962406015038	0.00869319897360854\\
9.87468671679198	0.00498326841684088\\
9.89974937343358	0.00244819779455943\\
9.92481203007519	0.00100901260398001\\
9.94987468671679	0.00052958167097745\\
9.9749373433584	0.000340908991010852\\
10	0.00018733930688011\\
};
\addplot [color=mycolor2, forget plot]
  table[row sep=crcr]{%
1.5	0\\
1.51	0.000448663900524906\\
1.52	0.000897302889738628\\
1.53	0.00134589178491801\\
1.54	0.00179440540613776\\
1.55	0.00224281857768212\\
1.56	0.00269110612945622\\
1.57	0.00313924289839726\\
1.58	0.00358720372988512\\
1.59	0.00403496347915266\\
1.6	0.00448249701269537\\
1.61	0.00492977920968042\\
1.62	0.00537678496335498\\
1.63	0.00582348918245382\\
1.64	0.00626986679260592\\
1.65	0.00671589273774027\\
1.66	0.00716154198149053\\
1.67	0.00760678950859867\\
1.68	0.00805161032631738\\
1.69	0.00849597946581121\\
1.7	0.00893987198355644\\
1.71	0.00938326296273942\\
1.72	0.00982612751465352\\
1.73	0.0102684407800945\\
1.74	0.010710177930754\\
1.75	0.0111513141706118\\
1.76	0.0115918247373256\\
1.77	0.0120316849036196\\
1.78	0.0124708699786705\\
1.79	0.012909355309492\\
1.8	0.0133471162823167\\
1.81	0.0137841283239761\\
1.82	0.0142203669032783\\
1.83	0.0146558075323833\\
1.84	0.0150904257681757\\
1.85	0.0155241972136352\\
1.86	0.0159570975192044\\
1.87	0.0163891023841539\\
1.88	0.0168201875579445\\
1.89	0.0172503288415869\\
1.9	0.0176795020889985\\
1.91	0.0181076832083569\\
1.92	0.0185348481634506\\
1.93	0.0189609729750267\\
1.94	0.0193860337221352\\
1.95	0.0198100065434697\\
1.96	0.020232867638706\\
1.97	0.0206545932698354\\
1.98	0.0210751597624965\\
1.99	0.0214945435073017\\
2	0.0219127209611613\\
2.01	0.0223296686486032\\
2.02	0.0227453631630891\\
2.03	0.0231597811683266\\
2.04	0.0235728993995779\\
2.05	0.0239846946649636\\
2.06	0.0243951438467634\\
2.07	0.0248042239027117\\
2.08	0.0252119118672897\\
2.09	0.0256181848530132\\
2.1	0.0260230200517152\\
2.11	0.0264263947358249\\
2.12	0.026828286259642\\
2.13	0.027228672060606\\
2.14	0.0276275296605615\\
2.15	0.0280248366670183\\
2.16	0.0284205707744064\\
2.17	0.0288147097653274\\
2.18	0.0292072315117992\\
2.19	0.0295981139764972\\
2.2	0.0299873352139895\\
2.21	0.0303748733719673\\
2.22	0.0307607066924703\\
2.23	0.0311448135131061\\
2.24	0.0315271722682648\\
2.25	0.0319077614903286\\
2.26	0.0322865598108745\\
2.27	0.032663545961873\\
2.28	0.0330386987768801\\
2.29	0.0334119971922242\\
2.3	0.0337834202481869\\
2.31	0.0341529470901781\\
2.32	0.0345205569699053\\
2.33	0.0348862292465367\\
2.34	0.0352499433878587\\
2.35	0.0356116789714266\\
2.36	0.0359714156857099\\
2.37	0.0363291333312312\\
2.38	0.0366848118216982\\
2.39	0.0370384311851301\\
2.4	0.0373899715649776\\
2.41	0.0377394132212355\\
2.42	0.0380867365315503\\
2.43	0.0384319219923194\\
2.44	0.0387749502197852\\
2.45	0.0391158019511218\\
2.46	0.0394544580455146\\
2.47	0.0397908994852339\\
2.48	0.0401251073767007\\
2.49	0.0404570629515464\\
2.5	0.0407867475676648\\
2.51	0.0411141427102573\\
2.52	0.041439229992871\\
2.53	0.0417619911584294\\
2.54	0.0420824080802564\\
2.55	0.0424004627630922\\
2.56	0.0427161373441021\\
2.57	0.0430294140938785\\
2.58	0.0433402754174346\\
2.59	0.0436487038551908\\
2.6	0.0439546820839538\\
2.61	0.0442581929178877\\
2.62	0.0445592193094778\\
2.63	0.0448577443504858\\
2.64	0.0451537512728986\\
2.65	0.0454472234498678\\
2.66	0.045738144396642\\
2.67	0.0460264977714914\\
2.68	0.0463122673766237\\
2.69	0.0465954371590921\\
2.7	0.0468759912116959\\
2.71	0.0471539137738719\\
2.72	0.0474291892325782\\
2.73	0.0477018021231696\\
2.74	0.0479717371302645\\
2.75	0.0482389790886038\\
2.76	0.0485035129839012\\
2.77	0.0487653239536848\\
2.78	0.0490243972881305\\
2.79	0.0492807184308868\\
2.8	0.0495342729798907\\
2.81	0.0497850466881758\\
2.82	0.0500330254646703\\
2.83	0.0502781953749876\\
2.84	0.0505205426422074\\
2.85	0.0507600536476482\\
2.86	0.0509967149316308\\
2.87	0.051230513194233\\
2.88	0.0514614352960354\\
2.89	0.051689468258858\\
2.9	0.0519145992664879\\
2.91	0.0521368156653978\\
2.92	0.0523561049654557\\
2.93	0.052572454840625\\
2.94	0.0527858531296557\\
2.95	0.0529962878367664\\
2.96	0.0532037471323166\\
2.97	0.0534082193534703\\
2.98	0.0536096930048497\\
2.99	0.0538081567591799\\
3	0.054003599457924\\
3.01	0.0541960101119086\\
3.02	0.0543853779019402\\
3.03	0.0545716921794119\\
3.04	0.0547549424669002\\
3.05	0.0549351184587525\\
3.06	0.0551122100216652\\
3.07	0.0552862071952519\\
3.08	0.0554571001926016\\
3.09	0.0556248794008276\\
3.1	0.0557895353816067\\
3.11	0.0559510588717083\\
3.12	0.0561094407835138\\
3.13	0.056264672205526\\
3.14	0.0564167444028693\\
3.15	0.056565648817779\\
3.16	0.0567113770700814\\
3.17	0.0568539209576637\\
3.18	0.056993272456934\\
3.19	0.057129423723271\\
3.2	0.0572623670914641\\
3.21	0.0573920950761432\\
3.22	0.0575186003721983\\
3.23	0.0576418758551895\\
3.24	0.0577619145817461\\
3.25	0.0578787097899563\\
3.26	0.0579922548997463\\
3.27	0.058102543513249\\
3.28	0.0582095694151634\\
3.29	0.0583133265731027\\
3.3	0.0584138091379326\\
3.31	0.0585110114440998\\
3.32	0.058604928009949\\
3.33	0.058695553538031\\
3.34	0.0587828829153993\\
3.35	0.0588669112138975\\
3.36	0.0589476336904347\\
3.37	0.0590250457872527\\
3.38	0.0590991431321807\\
3.39	0.059169921538881\\
3.4	0.0592373770070838\\
3.41	0.0593015057228113\\
3.42	0.059362304058592\\
3.43	0.0594197685736639\\
3.44	0.0594738960141679\\
3.45	0.0595246833133298\\
3.46	0.0595721275916328\\
3.47	0.0596162261569789\\
3.48	0.0596569765048399\\
3.49	0.0596943763183981\\
3.5	0.059728423468676\\
3.51	0.0597591160146563\\
3.52	0.0597864522033903\\
3.53	0.0598104304700967\\
3.54	0.0598310494382492\\
3.55	0.0598483079196539\\
3.56	0.0598622049145163\\
3.57	0.0598727396114968\\
3.58	0.0598799113877573\\
3.59	0.0598837198089952\\
3.6	0.0598841646294688\\
3.61	0.0598812457920105\\
3.62	0.0598749634280307\\
3.63	0.0598653178575103\\
3.64	0.0598523095889827\\
3.65	0.059835939319506\\
3.66	0.0598162079346234\\
3.67	0.0597931165083142\\
3.68	0.0597666663029336\\
3.69	0.0597368587691419\\
3.7	0.0597036955458235\\
3.71	0.059667178459995\\
3.72	0.0596273095267033\\
3.73	0.0595840909489122\\
3.74	0.0595375251173795\\
3.75	0.059487614610523\\
3.76	0.059434362194276\\
3.77	0.0593777708219324\\
3.78	0.0593178436339815\\
3.79	0.0592545839579318\\
3.8	0.0591879953081247\\
3.81	0.0591180813855378\\
3.82	0.0590448460775772\\
3.83	0.0589682934578602\\
3.84	0.0588884277859867\\
3.85	0.0588052535073007\\
3.86	0.0587187752526411\\
3.87	0.0586289978380826\\
3.88	0.0585359262646654\\
3.89	0.0584395657181154\\
3.9	0.0583399215685534\\
3.91	0.0582369993701941\\
3.92	0.0581308048610353\\
3.93	0.0580213439625359\\
3.94	0.0579086227792843\\
3.95	0.0577926475986566\\
3.96	0.0576734248904636\\
3.97	0.0575509613065892\\
3.98	0.0574252636806165\\
3.99	0.0572963390274461\\
4	0.057164194542902\\
4.01	0.0570288376033288\\
4.02	0.0568902757651784\\
4.03	0.0567485167645862\\
4.04	0.0566035685169381\\
4.05	0.0564554391164264\\
4.06	0.0563041368355963\\
4.07	0.0561496701248828\\
4.08	0.0559920476121363\\
4.09	0.0558312781021397\\
4.1	0.0556673705761149\\
4.11	0.0555003341912191\\
4.12	0.0553301782800321\\
4.13	0.055156912350033\\
4.14	0.0549805460830672\\
4.15	0.0548010893348042\\
4.16	0.0546185521341851\\
4.17	0.0544329446828604\\
4.18	0.0542442773546185\\
4.19	0.054052560694804\\
4.2	0.053857805419727\\
4.21	0.0536600224160624\\
4.22	0.0534592227402398\\
4.23	0.0532554176178237\\
4.24	0.0530486184428844\\
4.25	0.0528388367773596\\
4.26	0.0526260843504063\\
4.27	0.0524103730577434\\
4.28	0.052191714960985\\
4.29	0.0519701222869644\\
4.3	0.051745607427049\\
4.31	0.0515181829364455\\
4.32	0.0512878615334967\\
4.33	0.0510546560989681\\
4.34	0.0508185796753265\\
4.35	0.0505796454660087\\
4.36	0.0503378668346816\\
4.37	0.0500932573044934\\
4.38	0.0498458305573153\\
4.39	0.049595600432975\\
4.4	0.0493425809284809\\
4.41	0.0490867861972377\\
4.42	0.0488282305482529\\
4.43	0.0485669284453352\\
4.44	0.0483028945062836\\
4.45	0.0480361435020681\\
4.46	0.0477666903560021\\
4.47	0.0474945501429057\\
4.48	0.0472197380882611\\
4.49	0.0469422695673588\\
4.5	0.0466621601044366\\
4.51	0.0463794253718086\\
4.52	0.0460940811889877\\
4.53	0.0458061435217985\\
4.54	0.0455156284814824\\
4.55	0.045222552323795\\
4.56	0.0449269314480946\\
4.57	0.0446287823964234\\
4.58	0.0443281218525804\\
4.59	0.0440249666411859\\
4.6	0.0437193337267391\\
4.61	0.0434112402126671\\
4.62	0.043100703340366\\
4.63	0.0427877404882353\\
4.64	0.0424723691707033\\
4.65	0.0421546070372457\\
4.66	0.0418344718713965\\
4.67	0.0415119815897513\\
4.68	0.0411871542409632\\
4.69	0.0408600080047312\\
4.7	0.0405305611907814\\
4.71	0.0401988322378407\\
4.72	0.0398648397126038\\
4.73	0.0395286023086922\\
4.74	0.039190138845607\\
4.75	0.0388494682676738\\
4.76	0.0385066096429815\\
4.77	0.0381615821623132\\
4.78	0.037814405138071\\
4.79	0.0374650980031938\\
4.8	0.0371136803100683\\
4.81	0.0367601717294328\\
4.82	0.0364045920492757\\
4.83	0.0360469611737259\\
4.84	0.0356872991219381\\
4.85	0.0353256260269703\\
4.86	0.0349619621346563\\
4.87	0.0345963278024709\\
4.88	0.0342287434983891\\
4.89	0.0338592297997398\\
4.9	0.0334878073920523\\
4.91	0.0331144970678975\\
4.92	0.0327393197257231\\
4.93	0.0323622963686827\\
4.94	0.031983448103459\\
4.95	0.0316027961390814\\
4.96	0.0312203617857384\\
4.97	0.030836166453583\\
4.98	0.0304502316515342\\
4.99	0.0300625789860719\\
5	0.0296732301600272\\
5.01	0.0292822069713663\\
5.02	0.0288895313119704\\
5.03	0.0284952251664098\\
5.04	0.0280993106107127\\
5.05	0.0277018098111296\\
5.06	0.0273027450228924\\
5.07	0.0269021385889688\\
5.08	0.0265000129388116\\
5.09	0.0260963905871041\\
5.1	0.0256912941325002\\
5.11	0.0252847462563606\\
5.12	0.024876769721484\\
5.13	0.0244673873708348\\
5.14	0.0240566221262661\\
5.15	0.0236444969872388\\
5.16	0.023231035029537\\
5.17	0.0228162594039795\\
5.18	0.0224001933351274\\
5.19	0.0219828601199889\\
5.2	0.0215642831267199\\
5.21	0.0211444857933222\\
5.22	0.0207234916263387\\
5.23	0.0203013241995452\\
5.24	0.0198780071526409\\
5.25	0.0194535641899354\\
5.26	0.0190280190790352\\
5.27	0.0186013956495276\\
5.28	0.0181737177916643\\
5.29	0.0177450094550438\\
5.3	0.0173152946472947\\
5.31	0.0168845974327586\\
5.32	0.0164529419311755\\
5.33	0.0160203523163713\\
5.34	0.015586852814949\\
5.35	0.0151524677049849\\
5.36	0.0147172213147325\\
5.37	0.0142811380213348\\
5.38	0.0138442422495488\\
5.39	0.0134065584704856\\
5.4	0.0129681112003706\\
5.41	0.0125289249993285\\
5.42	0.0120890244702021\\
5.43	0.0116484342574128\\
5.44	0.0112071790458769\\
5.45	0.0107652835599929\\
5.46	0.0103227725627234\\
5.47	0.00987967085480237\\
5.48	0.00943600327410857\\
5.49	0.00899179469526618\\
5.5	0.00854707002955485\\
5.51	0.00810185422525074\\
5.52	0.00765617226857581\\
5.53	0.0072100491855211\\
5.54	0.0067635100449506\\
5.55	0.00631657996362365\\
5.56	0.00586928411416387\\
5.57	0.00542164773768412\\
5.58	0.00497369616401169\\
5.59	0.00452545484479647\\
5.6	0.00407694940944288\\
5.61	0.00362820576365711\\
5.62	0.00317925027278717\\
5.63	0.00273011012769306\\
5.64	0.00228081414500729\\
5.65	0.00183139474828954\\
5.66	0.00138189381952503\\
5.67	0.000932385483554554\\
5.68	0.000483126083486428\\
5.69	4.48545354830961e-05\\
5.7	0.000418607764851076\\
5.71	0.000867795017905168\\
5.72	0.00131735503078073\\
5.73	0.00176695083333732\\
5.74	0.00221649837846471\\
5.75	0.00266595333053749\\
5.76	0.00311528241856961\\
5.77	0.00356445646470835\\
5.78	0.00401344809793404\\
5.79	0.0044622308501432\\
5.8	0.00491077874924374\\
5.81	0.00535906611762042\\
5.82	0.0058070674647803\\
5.83	0.00625475742684791\\
5.84	0.00670211073094244\\
5.85	0.007149102173502\\
5.86	0.00759570660678741\\
5.87	0.00804189893037143\\
5.88	0.00848765408576931\\
5.89	0.00893294705310681\\
5.9	0.00937775284914353\\
5.91	0.00982204652621862\\
5.92	0.0102658031718366\\
5.93	0.0107089979087054\\
5.94	0.0111516058950998\\
5.95	0.0115936023254603\\
5.96	0.0120349624311681\\
5.97	0.0124756614814489\\
5.98	0.0129156747843756\\
5.99	0.013354977687946\\
6	0.0137935455812168\\
6.01	0.0142313538954827\\
6.02	0.0146683781054888\\
6.03	0.0151045937306707\\
6.04	0.0155399763364142\\
6.05	0.0159745015353317\\
6.06	0.0164081449885511\\
6.07	0.0168408824070143\\
6.08	0.0172726895527827\\
6.09	0.0177035422403487\\
6.1	0.0181334163379512\\
6.11	0.0185622877688931\\
6.12	0.0189901325128614\\
6.13	0.019416926607248\\
6.14	0.0198426461484703\\
6.15	0.0202672672932921\\
6.16	0.0206907662601426\\
6.17	0.0211131193304353\\
6.18	0.021534302849884\\
6.19	0.0219542932298174\\
6.2	0.0223730669484915\\
6.21	0.022790600552399\\
6.22	0.0232068706575759\\
6.23	0.0236218539509053\\
6.24	0.0240355271914182\\
6.25	0.0244478672115903\\
6.26	0.0248588509186355\\
6.27	0.0252684552957966\\
6.28	0.0256766574036304\\
6.29	0.0260834343812908\\
6.3	0.0264887634478064\\
6.31	0.0268926219033548\\
6.32	0.0272949871305323\\
6.33	0.0276958365956191\\
6.34	0.0280951478498405\\
6.35	0.0284928985306227\\
6.36	0.0288890663628453\\
6.37	0.0292836291600875\\
6.38	0.0296765648258705\\
6.39	0.030067851354895\\
6.4	0.0304574668342728\\
6.41	0.0308453894447546\\
6.42	0.0312315974619514\\
6.43	0.0316160692575516\\
6.44	0.0319987833005325\\
6.45	0.0323797181583662\\
6.46	0.0327588524982202\\
6.47	0.0331361650881525\\
6.48	0.0335116347983014\\
6.49	0.0338852406020687\\
6.5	0.0342569615772984\\
6.51	0.0346267769074486\\
6.52	0.0349946658827578\\
6.53	0.0353606079014056\\
6.54	0.035724582470667\\
6.55	0.0360865692080607\\
6.56	0.0364465478424911\\
6.57	0.0368044982153846\\
6.58	0.037160400281819\\
6.59	0.0375142341116468\\
6.6	0.0378659798906123\\
6.61	0.038215617921462\\
6.62	0.0385631286250481\\
6.63	0.0389084925414264\\
6.64	0.0392516903309463\\
6.65	0.0395927027753352\\
6.66	0.0399315107787752\\
6.67	0.0402680953689735\\
6.68	0.0406024376982257\\
6.69	0.0409345190444722\\
6.7	0.0412643208123475\\
6.71	0.0415918245342222\\
6.72	0.0419170118712387\\
6.73	0.0422398646143381\\
6.74	0.0425603646852816\\
6.75	0.0428784941376637\\
6.76	0.0431942351579174\\
6.77	0.0435075700663134\\
6.78	0.0438184813179508\\
6.79	0.0441269515037404\\
6.8	0.0444329633513805\\
6.81	0.044736499726325\\
6.82	0.0450375436327441\\
6.83	0.0453360782144766\\
6.84	0.0456320867559748\\
6.85	0.0459255526832414\\
6.86	0.0462164595647587\\
6.87	0.0465047911124094\\
6.88	0.0467905311823897\\
6.89	0.047073663776114\\
6.9	0.0473541730411119\\
6.91	0.0476320432719169\\
6.92	0.0479072589109463\\
6.93	0.0481798045493733\\
6.94	0.0484496649279914\\
6.95	0.0487168249380686\\
6.96	0.0489812696221954\\
6.97	0.0492429841751225\\
6.98	0.0495019539445908\\
6.99	0.049758164432153\\
7	0.050011601293986\\
7.01	0.0502622503416951\\
7.02	0.0505100975431093\\
7.03	0.0507551290230677\\
7.04	0.0509973310641975\\
7.05	0.0512366901076825\\
7.06	0.0514731927540239\\
7.07	0.0517068257637906\\
7.08	0.0519375760583617\\
7.09	0.0521654307206597\\
7.1	0.0523903769958744\\
7.11	0.0526124022921778\\
7.12	0.0528314941814303\\
7.13	0.0530476403998768\\
7.14	0.0532608288488348\\
7.15	0.0534710475953721\\
7.16	0.0536782848729759\\
7.17	0.0538825290822122\\
7.18	0.0540837687913764\\
7.19	0.0542819927371338\\
7.2	0.054477189825151\\
7.21	0.0546693491307179\\
7.22	0.05485845989936\\
7.23	0.0550445115474413\\
7.24	0.0552274936627579\\
7.25	0.0554073960051208\\
7.26	0.0555842085069311\\
7.27	0.0557579212737432\\
7.28	0.0559285245848205\\
7.29	0.0560960088936793\\
7.3	0.0562603648286249\\
7.31	0.0564215831932764\\
7.32	0.0565796549670821\\
7.33	0.0567345713058257\\
7.34	0.0568863235421217\\
7.35	0.0570349031859013\\
7.36	0.0571803019248884\\
7.37	0.0573225116250658\\
7.38	0.0574615243311305\\
7.39	0.0575973322669405\\
7.4	0.0577299278359503\\
7.41	0.0578593036216365\\
7.42	0.0579854523879141\\
7.43	0.0581083670795419\\
7.44	0.0582280408225177\\
7.45	0.0583444669244643\\
7.46	0.0584576388750042\\
7.47	0.0585675503461247\\
7.48	0.0586741951925325\\
7.49	0.0587775674519985\\
7.5	0.058877661345692\\
7.51	0.0589744712785044\\
7.52	0.059067991839363\\
7.53	0.0591582178015345\\
7.54	0.0592451441229179\\
7.55	0.0593287659463272\\
7.56	0.0594090785997635\\
7.57	0.0594860775966774\\
7.58	0.0595597586362199\\
7.59	0.0596301176034842\\
7.6	0.0596971505697359\\
7.61	0.0597608537926334\\
7.62	0.0598212237164379\\
7.63	0.0598782569722124\\
7.64	0.059931950378011\\
7.65	0.059982300939057\\
7.66	0.0600293058479108\\
7.67	0.0600729624846277\\
7.68	0.0601132684169042\\
7.69	0.0601502214002151\\
7.7	0.0601838193779386\\
7.71	0.0602140604814723\\
7.72	0.0602409430303377\\
7.73	0.0602644655322741\\
7.74	0.060284626683323\\
7.75	0.0603014253679007\\
7.76	0.0603148606588613\\
7.77	0.0603249318175481\\
7.78	0.0603316382938357\\
7.79	0.0603349797261607\\
7.8	0.0603349559415416\\
7.81	0.0603315669555892\\
7.82	0.0603248129725053\\
7.83	0.0603146943850716\\
7.84	0.0603012117746275\\
7.85	0.0602843659110376\\
7.86	0.0602641577526487\\
7.87	0.0602405884462363\\
7.88	0.06021365932694\\
7.89	0.0601833719181887\\
7.9	0.0601497279316158\\
7.91	0.0601127292669627\\
7.92	0.0600723780119727\\
7.93	0.0600286764422739\\
7.94	0.0599816270212518\\
7.95	0.0599312323999111\\
7.96	0.0598774954167274\\
7.97	0.0598204190974878\\
7.98	0.0597600066551218\\
7.99	0.059696261489521\\
8	0.0596291871873483\\
8.01	0.0595587875218376\\
8.02	0.0594850664525816\\
8.03	0.0594080281253106\\
8.04	0.0593276768716596\\
8.05	0.0592440172089259\\
8.06	0.0591570538398157\\
8.07	0.0590667916521808\\
8.08	0.0589732357187446\\
8.09	0.0588763912968177\\
8.1	0.0587762638280034\\
8.11	0.0586728589378927\\
8.12	0.0585661824357488\\
8.13	0.058456240314182\\
8.14	0.0583430387488134\\
8.15	0.0582265840979292\\
8.16	0.058106882902124\\
8.17	0.0579839418839342\\
8.18	0.0578577679474617\\
8.19	0.0577283681779866\\
8.2	0.0575957498415702\\
8.21	0.0574599203846476\\
8.22	0.0573208874336107\\
8.23	0.0571786587943805\\
8.24	0.0570332424519699\\
8.25	0.0568846465700358\\
8.26	0.0567328794904218\\
8.27	0.0565779497326907\\
8.28	0.0564198659936469\\
8.29	0.0562586371468491\\
8.3	0.0560942722421129\\
8.31	0.0559267805050036\\
8.32	0.0557561713363191\\
8.33	0.0555824543115633\\
8.34	0.0554056391804088\\
8.35	0.0552257358661515\\
8.36	0.0550427544651532\\
8.37	0.0548567052462768\\
8.38	0.0546675986503102\\
8.39	0.0544754452893814\\
8.4	0.0542802559463634\\
8.41	0.0540820415742702\\
8.42	0.053880813295643\\
8.43	0.0536765824019264\\
8.44	0.0534693603528359\\
8.45	0.0532591587757157\\
8.46	0.0530459894648866\\
8.47	0.0528298643809855\\
8.48	0.0526107956502948\\
8.49	0.0523887955640629\\
8.5	0.0521638765778151\\
8.51	0.0519360513106557\\
8.52	0.0517053325445609\\
8.53	0.0514717332236622\\
8.54	0.0512352664535211\\
8.55	0.0509959455003945\\
8.56	0.0507537837904914\\
8.57	0.0505087949092201\\
8.58	0.0502609926004271\\
8.59	0.050010390765627\\
8.6	0.0497570034632228\\
8.61	0.0495008449077188\\
8.62	0.0492419294689233\\
8.63	0.0489802716711437\\
8.64	0.0487158861923725\\
8.65	0.0484487878634645\\
8.66	0.0481789916673058\\
8.67	0.0479065127379741\\
8.68	0.0476313663598907\\
8.69	0.0473535679669637\\
8.7	0.0470731331417233\\
8.71	0.0467900776144487\\
8.72	0.046504417262286\\
8.73	0.0462161681083592\\
8.74	0.0459253463208718\\
8.75	0.045631968212201\\
8.76	0.0453360502379835\\
8.77	0.0450376089961936\\
8.78	0.0447366612262126\\
8.79	0.0444332238078915\\
8.8	0.0441273137606048\\
8.81	0.0438189482422969\\
8.82	0.0435081445485204\\
8.83	0.0431949201114678\\
8.84	0.042879292498994\\
8.85	0.0425612794136322\\
8.86	0.0422408986916023\\
8.87	0.0419181683018115\\
8.88	0.0415931063448472\\
8.89	0.0412657310519637\\
8.9	0.04093606078406\\
8.91	0.0406041140306519\\
8.92	0.0402699094088358\\
8.93	0.039933465662246\\
8.94	0.0395948016600046\\
8.95	0.0392539363956651\\
8.96	0.0389108889861477\\
8.97	0.0385656786706693\\
8.98	0.0382183248096658\\
8.99	0.0378688468837076\\
9	0.0375172644924091\\
};
\end{axis}

\begin{axis}[%
width=0.4\figW,
height=0.15\figH,
at={(0\figW,0.43\figH)},
scale only axis,
xmin=0,
xmax=10,
ymin=0,
ymax=2,
ylabel style={font=\color{white!15!black}},
ylabel={$\text{Modal amplitude (W }\mu\text{m}^{\text{-1}}\text{)}$},
axis background/.style={fill=white}
]
\addplot [color=mycolor1, forget plot]
  table[row sep=crcr]{%
0	0.000122682899429508\\
0.025062656641604	0.000263579032310059\\
0.050125313283208	0.000838242130691653\\
0.075187969924812	0.00404780853583172\\
0.100250626566416	0.0147968334477311\\
0.12531328320802	0.0424087731177551\\
0.150375939849624	0.0983165636631519\\
0.175438596491228	0.189917306349692\\
0.200501253132832	0.314497809164451\\
0.225563909774436	0.458544049222592\\
0.25062656641604	0.603191971856891\\
0.275689223057644	0.731597916040012\\
0.300751879699248	0.833708862768532\\
0.325814536340852	0.90701955774666\\
0.350877192982456	0.95458779742025\\
0.37593984962406	0.982257644576392\\
0.401002506265664	0.996378002507964\\
0.426065162907268	1.00242951393513\\
0.451127819548872	1.00444163754829\\
0.476190476190476	1.00492056905285\\
0.50125313283208	1.00505783709433\\
0.526315789473684	1.00513493141161\\
0.551378446115288	1.0051537598476\\
0.576441102756892	1.00511069861563\\
0.601503759398496	1.00500314796798\\
0.6265664160401	1.00482974493801\\
0.651629072681704	1.00459057972402\\
0.676691729323308	1.00428740745903\\
0.701754385964912	1.00392384482172\\
0.726817042606516	1.00350866651699\\
0.75187969924812	1.00305415770037\\
0.776942355889724	1.00256278089423\\
0.802005012531328	1.00204640874444\\
0.827067669172932	1.00151897771866\\
0.852130325814536	1.00099626807098\\
0.87719298245614	1.00049555080328\\
0.902255639097744	1.00003508861754\\
0.927318295739348	0.999633484967188\\
0.952380952380952	0.999308885669092\\
0.977443609022556	0.999078051238895\\
1.00250626566416	0.998955334765586\\
1.02756892230576	0.999052025589532\\
1.05263157894737	0.999283669775036\\
1.07769423558897	0.999636539516037\\
1.10275689223058	1.00009068664862\\
1.12781954887218	1.00062129986661\\
1.15288220551378	1.00120046501399\\
1.17794486215539	1.00179918343616\\
1.20300751879699	1.00238946391678\\
1.2280701754386	1.00294628520575\\
1.2531328320802	1.00344923234648\\
1.2781954887218	1.00388364114748\\
1.30325814536341	1.00424836999892\\
1.32832080200501	1.00453648523418\\
1.35338345864662	1.0047478871354\\
1.37844611528822	1.0048898620005\\
1.40350877192982	1.00497234728783\\
1.42857142857143	1.00500617430131\\
1.45363408521303	1.00500137073568\\
1.47869674185464	1.00496571973916\\
1.50375939849624	1.00490371914795\\
1.52882205513784	1.00481574849751\\
1.55388471177945	1.00469809611871\\
1.57894736842105	1.00454448425831\\
1.60401002506266	1.0043477466808\\
1.62907268170426	1.00410169332695\\
1.65413533834586	1.00380287332972\\
1.67919799498747	1.00345195546112\\
1.70426065162907	1.00305449885518\\
1.72932330827068	1.00262097686364\\
1.75438596491228	1.00216602924281\\
1.77944862155388	1.0017070358658\\
1.80451127819549	1.00126221124699\\
1.82957393483709	1.00084849711875\\
1.8546365914787	1.00047956748455\\
1.8796992481203	1.00016442548611\\
1.9047619047619	0.999909813466459\\
1.92982456140351	0.999708717469445\\
1.95488721804511	0.999553171616296\\
1.97994987468672	0.99943154773177\\
2.00501253132832	0.999330537836223\\
2.03007518796993	0.99923735704914\\
2.05513784461153	0.999141833303752\\
2.08020050125313	0.999038064834647\\
2.10526315789474	0.998925389365586\\
2.13032581453634	0.998808510939417\\
2.15538847117795	0.998696756527663\\
2.18045112781955	0.998602566636105\\
2.20551378446115	0.998539442716205\\
2.23057644110276	0.998519661362834\\
2.25563909774436	0.998552107043685\\
2.28070175438596	0.998640563742611\\
2.30576441102757	0.998782741619869\\
2.33082706766917	0.998970206280794\\
2.35588972431078	0.999189241681809\\
2.38095238095238	0.999422534252128\\
2.40601503759398	0.99965143800104\\
2.43107769423559	0.999858488202098\\
2.45614035087719	1.00002978883619\\
2.4812030075188	1.00015691259048\\
2.5062656641604	1.00023801968724\\
2.53132832080201	1.00027801335502\\
2.55639097744361	1.00028768957836\\
2.58145363408521	1.00028198726123\\
2.60651629072682	1.00027758103368\\
2.63157894736842	1.00029140577087\\
2.65664160401003	1.0003342980898\\
2.68170426065163	1.00041248056139\\
2.70676691729323	1.00052519540094\\
2.73182957393484	1.00066432442172\\
2.75689223057644	1.00081520769997\\
2.78195488721805	1.00095846763434\\
2.80701754385965	1.00107258908675\\
2.83208020050125	1.00113689111863\\
2.85714285714286	1.00113446392528\\
2.88220551378446	1.0010546446271\\
2.90726817042607	1.00089466702149\\
2.93233082706767	1.00066023414982\\
2.95739348370927	1.00036491276777\\
2.98245614035088	1.00002841517791\\
3.00751879699248	0.999673993867273\\
3.03258145363409	0.999325305717488\\
3.05764411027569	0.999003185830601\\
3.08270676691729	0.998722792509059\\
3.1077694235589	0.998491539194281\\
3.1328320802005	0.998308120809851\\
3.15789473684211	0.998162785818751\\
3.18295739348371	0.998038824582867\\
3.20802005012531	0.997915067015301\\
3.23308270676692	0.997769035311376\\
3.25814536340852	0.99758030273804\\
3.28320802005013	0.99733358028559\\
3.30827067669173	0.997021092375071\\
3.33333333333333	0.996644397815405\\
3.35839598997494	0.996213308586542\\
3.38345864661654	0.99574525237194\\
3.40852130325815	0.995263479516544\\
3.43358395989975	0.99479370235185\\
3.45864661654135	0.994360706629478\\
3.48370927318296	0.993985095338045\\
3.50877192982456	0.993680630436786\\
3.53383458646617	0.99345255315703\\
3.55889724310777	0.993297121199108\\
3.58395989974937	0.993202422351804\\
3.60902255639098	0.993150336282629\\
3.63408521303258	0.993119348688659\\
3.65914786967419	0.993087800264255\\
3.68421052631579	0.993037094644202\\
3.70927318295739	0.992954401730878\\
3.734335839599	0.992834471920186\\
3.7593984962406	0.992680309711413\\
3.78446115288221	0.992502622284688\\
3.80952380952381	0.992318136266079\\
3.83458646616541	0.992147039464409\\
3.85964912280702	0.992009930960109\\
3.88471177944862	0.991924734104984\\
3.90977443609023	0.991904031432816\\
3.93483709273183	0.9919532162514\\
3.95989974937343	0.992069731402732\\
3.98496240601504	0.992243499888281\\
4.01002506265664	0.992458470550822\\
4.03508771929825	0.992695033403854\\
4.06015037593985	0.992932929817516\\
4.08521303258145	0.99315421202699\\
4.11027568922306	0.993345803887229\\
4.13533834586466	0.993501279280744\\
4.16040100250627	0.993621595187941\\
4.18546365914787	0.993714674775901\\
4.21052631578947	0.993793908924018\\
4.23558897243108	0.993875807638678\\
4.26065162907268	0.993977162266575\\
4.28571428571429	0.994112155865793\\
4.31077694235589	0.994289870059991\\
4.33583959899749	0.994512579050983\\
4.3609022556391	0.994775250983496\\
4.3859649122807	0.99506589713733\\
4.41102756892231	0.995366792671529\\
4.43609022556391	0.995657673625654\\
4.46115288220551	0.995918399488167\\
4.48621553884712	0.996131945553253\\
4.51127819548872	0.996286970983384\\
4.53634085213033	0.996379566279595\\
4.56140350877193	0.996413903624576\\
4.58646616541353	0.996401674621151\\
4.61152882205514	0.996360379544631\\
4.63659147869674	0.996310705117373\\
4.66165413533835	0.996273369125043\\
4.68671679197995	0.996265898454911\\
4.71177944862155	0.996299827426593\\
4.73684210526316	0.996378750114771\\
4.76190476190476	0.996497539529425\\
4.78696741854637	0.996642875114802\\
4.81203007518797	0.996795024116463\\
4.83709273182957	0.996930632553815\\
4.86215538847118	0.99702612744166\\
4.88721804511278	0.997061236930789\\
4.91228070175439	0.997022112724413\\
4.93734335839599	0.996903591729087\\
4.96240601503759	0.996710253041405\\
4.9874686716792	0.996456095091336\\
5.0125313283208	0.996162853122301\\
5.03759398496241	0.995857172958578\\
5.06265664160401	0.995567026349457\\
5.08771929824561	0.995317871517906\\
5.11278195488722	0.995129110846928\\
5.13784461152882	0.995011365809429\\
5.16290726817043	0.99496515867042\\
5.18796992481203	0.994980356069148\\
5.21303258145363	0.995038086027432\\
5.23809523809524	0.995113379708536\\
5.26315789473684	0.995178779906416\\
5.28822055137845	0.995208365247257\\
5.31328320802005	0.995181583974723\\
5.33834586466165	0.995086349430748\\
5.36340852130326	0.99492095320994\\
5.38847117794486	0.994694517246363\\
5.41353383458647	0.994425909908657\\
5.43859649122807	0.994141266509969\\
5.46365914786967	0.993870453044039\\
5.48872180451128	0.993642966282922\\
5.51378446115288	0.993483850782327\\
5.53884711779449	0.993410218772958\\
5.56390977443609	0.993428878404222\\
5.58897243107769	0.993535418636266\\
5.6140350877193	0.99371488760226\\
5.6390977443609	0.993943967764855\\
5.66416040100251	0.994194332161313\\
5.68922305764411	0.994436795657222\\
5.71428571428571	0.994645315815121\\
5.73934837092732	0.994800563250435\\
5.76441102756892	0.994892886649892\\
5.78947368421053	0.994923475688397\\
5.81453634085213	0.994904087986735\\
5.83959899749373	0.994855254460094\\
5.86466165413534	0.994803218485202\\
5.88972431077694	0.994776045324725\\
5.91478696741855	0.994799459930804\\
5.93984962406015	0.994893014780597\\
5.96491228070175	0.995067145889026\\
5.98997493734336	0.995321548368944\\
6.01503759398496	0.99564511016454\\
6.04010025062657	0.996017412784011\\
6.06516290726817	0.996411576765476\\
6.09022556390977	0.996798033226662\\
6.11528822055138	0.997148670736755\\
6.14035087719298	0.997440757099995\\
6.16541353383459	0.997660073164105\\
6.19047619047619	0.997802812386284\\
6.21553884711779	0.997875977368861\\
6.2406015037594	0.997896217891565\\
6.265664160401	0.997887275559222\\
6.29072681704261	0.99787639883984\\
6.31578947368421	0.997890241891476\\
6.34085213032582	0.99795083922676\\
6.36591478696742	0.998072242509441\\
6.39097744360902	0.9982583137848\\
6.41604010025063	0.998502002739153\\
6.44110275689223	0.998786218372709\\
6.46616541353383	0.999086171065411\\
6.49122807017544	0.99937284587938\\
6.51629072681704	0.999617104882142\\
6.54135338345865	0.999793828657824\\
6.56641604010025	0.999885505593881\\
6.59147869674185	0.999884759397899\\
6.61654135338346	0.999795456765427\\
6.64160401002506	0.999632236226973\\
6.66666666666667	0.999418519175763\\
6.69172932330827	0.999183276244686\\
6.71679197994987	0.998956998606703\\
6.74185463659148	0.998767439916314\\
6.76691729323308	0.99863573257414\\
6.79197994987469	0.998573433748614\\
6.81704260651629	0.998580926581659\\
6.84210526315789	0.998647407923629\\
6.8671679197995	0.998752464690072\\
6.8922305764411	0.998869011831164\\
6.91729323308271	0.998967171148131\\
6.94235588972431	0.99901854033304\\
6.96741854636591	0.99900025285473\\
6.99248120300752	0.998898265647721\\
7.01754385964912	0.998709424642929\\
7.04260651629073	0.998442030099239\\
7.06766917293233	0.998114838605002\\
7.09273182957393	0.997754734333421\\
7.11779448621554	0.997392821934729\\
7.14285714285714	0.997060708898328\\
7.16791979949875	0.996786199273523\\
7.19298245614035	0.996589675054114\\
7.21804511278195	0.99648152004972\\
7.24310776942356	0.996460936455351\\
7.26817042606516	0.996516304442546\\
7.29323308270677	0.996627014518718\\
7.31829573934837	0.996766495035491\\
7.34335839598998	0.996905994322529\\
7.36842105263158	0.997018581446831\\
7.39348370927318	0.997082812740223\\
7.41854636591479	0.99708557169243\\
7.44360902255639	0.997023715453177\\
7.468671679198	0.996904332185033\\
7.4937343358396	0.996743605724796\\
7.5187969924812	0.996564472071313\\
7.54385964912281	0.996393412046724\\
7.56892230576441	0.996256835549455\\
7.59398496240602	0.996177560382007\\
7.61904761904762	0.996171865907417\\
7.64411027568922	0.996247511632312\\
7.66917293233083	0.996402966229888\\
7.69423558897243	0.996627916281297\\
7.71929824561404	0.996904990090509\\
7.74436090225564	0.997212151186827\\
7.76942355889724	0.997525933821282\\
7.79448621553885	0.997824500251071\\
7.81954887218045	0.99809038268844\\
7.84461152882206	0.998312490809033\\
7.86967418546366	0.998487144341893\\
7.89473684210526	0.998618036603538\\
7.91979949874687	0.998715187346234\\
7.94486215538847	0.998793081043468\\
7.96992481203008	0.998868291625444\\
7.99498746867168	0.998956953378685\\
8.02005012531328	0.999072443284116\\
8.04511278195489	0.999223628348654\\
8.07017543859649	0.999413779719653\\
8.09523809523809	0.999640363147458\\
8.1203007518797	0.999895800272628\\
8.1453634085213	1.00016877806296\\
8.17042606516291	1.00044604233502\\
8.19548872180451	1.00071434123409\\
8.22055137844612	1.00096222011881\\
8.24561403508772	1.00118141002704\\
8.27067669172932	1.00136762924527\\
8.29573934837093	1.00152071775404\\
8.32080200501253	1.00164413119973\\
8.34586466165413	1.00174391818378\\
8.37092731829574	1.00182737787695\\
8.39598997493734	1.00190163412218\\
8.42105263157895	1.00197236253838\\
8.44611528822055	1.00204286996942\\
8.47117794486216	1.00211365805983\\
8.49624060150376	1.00218251661614\\
8.52130325814536	1.00224510232413\\
8.54636591478697	1.00229587922413\\
8.57142857142857	1.00232924174956\\
8.59649122807017	1.00234061745323\\
8.62155388471178	1.00232735757425\\
8.64661654135338	1.00228926635947\\
8.67167919799499	1.00222868652564\\
8.69674185463659	1.0021501368437\\
8.7218045112782	1.0020595753056\\
8.7468671679198	1.00196342488184\\
8.7719298245614	1.0018675380788\\
8.79699248120301	1.00177628496882\\
8.82205513784461	1.00169192579054\\
8.84711779448622	1.00161437766965\\
8.87218045112782	1.00154141433385\\
8.89724310776942	1.00146926005769\\
8.92230576441103	1.00139346790643\\
8.94736842105263	1.0013099200794\\
8.97243107769424	1.00121576425341\\
8.99749373433584	1.0011101098624\\
9.02255639097744	1.00099435523673\\
9.04761904761905	1.00087341416198\\
9.07268170426065	1.00074949048333\\
9.09774436090226	1.00062442132761\\
9.12280701754386	1.00049982123205\\
9.14786967418546	1.00037714225667\\
9.17293233082707	1.0002577023742\\
9.19799498746867	1.0001427010852\\
9.22305764411028	1.00003322956761\\
9.24812030075188	0.999930278224933\\
9.27318295739348	0.999834742670033\\
9.29824561403509	0.999747428415425\\
9.32330827067669	0.999669054255441\\
9.3483709273183	0.999600254279944\\
9.3734335839599	0.99954157853608\\
9.3984962406015	0.999493492486552\\
9.42355889724311	0.999456375554266\\
9.44862155388471	0.999430519160838\\
9.47368421052632	0.99941612473696\\
9.49874686716792	0.999413302192979\\
9.52380952380952	0.999422069285062\\
9.54887218045113	0.999436017905105\\
9.57393483709273	0.999169226374647\\
9.59899749373434	0.997427594818571\\
9.62406015037594	0.991704960223035\\
9.64912280701754	0.977974207654118\\
9.67418546365915	0.95075795393183\\
9.69924812030075	0.903708792441894\\
9.72431077694236	0.830978398624363\\
9.74937343358396	0.729494120737652\\
9.77443609022556	0.60173154649171\\
9.79949874686717	0.45769827920403\\
9.82456140350877	0.314182359900586\\
9.84962406015038	0.189993258039764\\
9.87468671679198	0.0986139170117935\\
9.89974937343358	0.0427655773794614\\
9.92481203007519	0.0150938533600494\\
9.94987468671679	0.00423353930772075\\
9.9749373433584	0.000925081342306211\\
10	0.0001551717969481\\
};
\addplot [color=mycolor2, forget plot]
  table[row sep=crcr]{%
1.5	1\\
1.51	1\\
1.52	0.999999595388452\\
1.53	0.999998786256345\\
1.54	0.999997572755559\\
1.55	0.999995955098833\\
1.56	0.999993933559726\\
1.57	0.99999150847257\\
1.58	0.999988680232408\\
1.59	0.999985449294924\\
1.6	0.999981816176357\\
1.61	0.999977781453409\\
1.62	0.999973345763141\\
1.63	0.999968509802855\\
1.64	0.999963274329972\\
1.65	0.999957640161889\\
1.66	0.999951608175833\\
1.67	0.999945179308703\\
1.68	0.999938354556897\\
1.69	0.999931134976134\\
1.7	0.999923521681262\\
1.71	0.999915515846051\\
1.72	0.999907118702988\\
1.73	0.999898331543047\\
1.74	0.999889155715455\\
1.75	0.999879592627451\\
1.76	0.999869643744028\\
1.77	0.999859310587666\\
1.78	0.99984859473806\\
1.79	0.999837497831829\\
1.8	0.999826021562219\\
1.81	0.999814167678802\\
1.82	0.99980193798715\\
1.83	0.999789334348515\\
1.84	0.999776358679488\\
1.85	0.999763012951649\\
1.86	0.999749299191217\\
1.87	0.999735219478674\\
1.88	0.999720775948395\\
1.89	0.999705970788257\\
1.9	0.999690806239246\\
1.91	0.999675284595045\\
1.92	0.999659408201626\\
1.93	0.999643179456821\\
1.94	0.999626600809887\\
1.95	0.999609674761066\\
1.96	0.999592403861128\\
1.97	0.999574790710915\\
1.98	0.999556837960864\\
1.99	0.999538548310534\\
2	0.99951992450811\\
2.01	0.999500969349915\\
2.02	0.999481685679897\\
2.03	0.999462076389119\\
2.04	0.999442144415234\\
2.05	0.999421892741956\\
2.06	0.999401324398522\\
2.07	0.999380442459142\\
2.08	0.999359250042444\\
2.09	0.999337750310915\\
2.1	0.999315946470326\\
2.11	0.999293841769156\\
2.12	0.999271439498007\\
2.13	0.999248742989005\\
2.14	0.999225755615208\\
2.15	0.99920248078999\\
2.16	0.999178921966432\\
2.17	0.999155082636698\\
2.18	0.999130966331405\\
2.19	0.999106576618989\\
2.2	0.999081917105063\\
2.21	0.999056991431767\\
2.22	0.999031803277118\\
2.23	0.999006356354341\\
2.24	0.99898065441121\\
2.25	0.99895470122937\\
2.26	0.998928500623659\\
2.27	0.998902056441425\\
2.28	0.998875372561834\\
2.29	0.998848452895176\\
2.3	0.998821301382166\\
2.31	0.998793921993231\\
2.32	0.99876631872781\\
2.33	0.998738495613627\\
2.34	0.99871045670598\\
2.35	0.998682206087009\\
2.36	0.998653747864972\\
2.37	0.998625086173507\\
2.38	0.998596225170898\\
2.39	0.998567169039328\\
2.4	0.99853792198414\\
2.41	0.998508488233082\\
2.42	0.998478872035558\\
2.43	0.998449077661867\\
2.44	0.998419109402448\\
2.45	0.998388971567118\\
2.46	0.9983586684843\\
2.47	0.998328204500262\\
2.48	0.998297583978341\\
2.49	0.998266811298172\\
2.5	0.99823589085491\\
2.51	0.998204827058454\\
2.52	0.998173624332668\\
2.53	0.998142287114595\\
2.54	0.998110819853677\\
2.55	0.998079227010967\\
2.56	0.998047513058346\\
2.57	0.998015682477732\\
2.58	0.997983739760292\\
2.59	0.997951689405654\\
2.6	0.997919535921112\\
2.61	0.997887283820841\\
2.62	0.997854937625099\\
2.63	0.99782250185944\\
2.64	0.99778998105392\\
2.65	0.997757379742303\\
2.66	0.997724702461273\\
2.67	0.997691953749639\\
2.68	0.997659138147548\\
2.69	0.99762626019569\\
2.7	0.997593324434511\\
2.71	0.997560335403425\\
2.72	0.997527297640026\\
2.73	0.997494215679299\\
2.74	0.997461094052841\\
2.75	0.997427937288074\\
2.76	0.997394749907464\\
2.77	0.997361536427744\\
2.78	0.997328301359132\\
2.79	0.997295049204562\\
2.8	0.997261784458908\\
2.81	0.997228511608212\\
2.82	0.997195235128923\\
2.83	0.997161959487125\\
2.84	0.997128689137781\\
2.85	0.997095428523971\\
2.86	0.997062182076142\\
2.87	0.99702895421135\\
2.88	0.996995749332517\\
2.89	0.996962571827685\\
2.9	0.996929426069277\\
2.91	0.996896316413358\\
2.92	0.996863247198906\\
2.93	0.996830222747086\\
2.94	0.996797247360522\\
2.95	0.996764325322583\\
2.96	0.996731460896666\\
2.97	0.996698658325491\\
2.98	0.996665921830396\\
2.99	0.996633255610639\\
3	0.996600663842703\\
3.01	0.996568150679612\\
3.02	0.996535720250248\\
3.03	0.996503376658676\\
3.04	0.996471123983469\\
3.05	0.996438966277054\\
3.06	0.996406907565045\\
3.07	0.996374951845595\\
3.08	0.996343103088753\\
3.09	0.996311365235823\\
3.1	0.996279742198734\\
3.11	0.996248237859413\\
3.12	0.996216856069171\\
3.13	0.996185600648089\\
3.14	0.996154475384416\\
3.15	0.996123484033972\\
3.16	0.99609263031956\\
3.17	0.996061917930384\\
3.18	0.996031350521476\\
3.19	0.996000931713132\\
3.2	0.99597066509035\\
3.21	0.995940554202285\\
3.22	0.995910602561705\\
3.23	0.995880813644455\\
3.24	0.995851190888937\\
3.25	0.995821737695591\\
3.26	0.995792457426386\\
3.27	0.99576335340432\\
3.28	0.995734428912933\\
3.29	0.995705687195821\\
3.3	0.995677131456163\\
3.31	0.995648764856262\\
3.32	0.995620590517083\\
3.33	0.995592611517813\\
3.34	0.995564830895422\\
3.35	0.995537251644234\\
3.36	0.995509876715513\\
3.37	0.995482709017053\\
3.38	0.995455751412781\\
3.39	0.995429006722363\\
3.4	0.995402477720832\\
3.41	0.995376167138213\\
3.42	0.995350077659165\\
3.43	0.995324211922633\\
3.44	0.995298572521506\\
3.45	0.99527316200229\\
3.46	0.995247982864785\\
3.47	0.995223037561779\\
3.48	0.995198328498747\\
3.49	0.995173858033563\\
3.5	0.995149628476222\\
3.51	0.995125642088571\\
3.52	0.99510190108405\\
3.53	0.995078407627448\\
3.54	0.995055163834665\\
3.55	0.995032171772482\\
3.56	0.995009433458352\\
3.57	0.99498695086019\\
3.58	0.994964725896181\\
3.59	0.994942760434594\\
3.6	0.994921056293611\\
3.61	0.994899615241164\\
3.62	0.994878438994783\\
3.63	0.994857529221456\\
3.64	0.994836887537495\\
3.65	0.994816515508423\\
3.66	0.99479641464886\\
3.67	0.994776586422428\\
3.68	0.994757032241662\\
3.69	0.994737753467936\\
3.7	0.994718751411395\\
3.71	0.994700027330902\\
3.72	0.994681582433994\\
3.73	0.994663417876847\\
3.74	0.994645534764257\\
3.75	0.994627934149623\\
3.76	0.994610617034954\\
3.77	0.994593584370871\\
3.78	0.994576837056633\\
3.79	0.994560375940163\\
3.8	0.994544201818096\\
3.81	0.994528315435827\\
3.82	0.994512717487576\\
3.83	0.99449740861646\\
3.84	0.99448238941458\\
3.85	0.994467660423113\\
3.86	0.994453222132417\\
3.87	0.994439074982151\\
3.88	0.994425219361394\\
3.89	0.994411655608787\\
3.9	0.994398384012677\\
3.91	0.994385404811273\\
3.92	0.994372718192815\\
3.93	0.99436032429575\\
3.94	0.994348223208919\\
3.95	0.994336414971754\\
3.96	0.994324899574485\\
3.97	0.994313676958358\\
3.98	0.994302747015861\\
3.99	0.994292109590957\\
4	0.994281764479338\\
4.01	0.994271711428673\\
4.02	0.994261950138875\\
4.03	0.994252480262381\\
4.04	0.994243301404427\\
4.05	0.994234413123349\\
4.06	0.994225814930882\\
4.07	0.99421750629247\\
4.08	0.994209486627592\\
4.09	0.994201755310084\\
4.1	0.994194311668484\\
4.11	0.994187154986375\\
4.12	0.994180284502743\\
4.13	0.994173699412339\\
4.14	0.994167398866054\\
4.15	0.994161381971298\\
4.16	0.994155647792392\\
4.17	0.994150195350961\\
4.18	0.994145023626344\\
4.19	0.994140131556005\\
4.2	0.994135518035954\\
4.21	0.994131181921177\\
4.22	0.994127122026071\\
4.23	0.994123337124889\\
4.24	0.994119825952192\\
4.25	0.994116587203305\\
4.26	0.994113619534784\\
4.27	0.994110921564891\\
4.28	0.994108491874068\\
4.29	0.994106329005428\\
4.3	0.994104431465245\\
4.31	0.994102797723456\\
4.32	0.994101426214165\\
4.33	0.994100315336152\\
4.34	0.994099463453397\\
4.35	0.9940988688956\\
4.36	0.994098529958711\\
4.37	0.994098444905468\\
4.38	0.994098611965934\\
4.39	0.994099029338047\\
4.4	0.99409969518817\\
4.41	0.99410060765165\\
4.42	0.994101764833376\\
4.43	0.99410316480835\\
4.44	0.994104805622256\\
4.45	0.994106685292037\\
4.46	0.994108801806475\\
4.47	0.994111153126777\\
4.48	0.994113737187163\\
4.49	0.994116551895458\\
4.5	0.994119595133693\\
4.51	0.9941228647587\\
4.52	0.99412635860272\\
4.53	0.994130074474009\\
4.54	0.99413401015745\\
4.55	0.994138163415166\\
4.56	0.994142531987137\\
4.57	0.994147113591818\\
4.58	0.994151905926767\\
4.59	0.994156906669263\\
4.6	0.994162113476937\\
4.61	0.994167523988401\\
4.62	0.99417313582388\\
4.63	0.994178946585844\\
4.64	0.994184953859644\\
4.65	0.994191155214149\\
4.66	0.994197548202383\\
4.67	0.994204130362167\\
4.68	0.994210899216756\\
4.69	0.994217852275486\\
4.7	0.994224987034409\\
4.71	0.994232300976944\\
4.72	0.994239791574515\\
4.73	0.994247456287198\\
4.74	0.994255292564364\\
4.75	0.994263297845323\\
4.76	0.99427146955997\\
4.77	0.994279805129429\\
4.78	0.994288301966693\\
4.79	0.994296957477271\\
4.8	0.994305769059831\\
4.81	0.994314734106836\\
4.82	0.994323850005193\\
4.83	0.994333114136883\\
4.84	0.994342523879607\\
4.85	0.99435207660742\\
4.86	0.994361769691364\\
4.87	0.994371600500104\\
4.88	0.99438156640056\\
4.89	0.994391664758535\\
4.9	0.99440189293934\\
4.91	0.994412248308427\\
4.92	0.994422728232001\\
4.93	0.99443333007765\\
4.94	0.994444051214954\\
4.95	0.994454889016105\\
4.96	0.994465840856518\\
4.97	0.994476904115433\\
4.98	0.994488076176529\\
4.99	0.994499354428519\\
5	0.994510736265751\\
5.01	0.994522219088799\\
5.02	0.994533800305058\\
5.03	0.994545477329327\\
5.04	0.994557247584392\\
5.05	0.994569108501604\\
5.06	0.994581057521454\\
5.07	0.994593092094139\\
5.08	0.994605209680133\\
5.09	0.994617407750738\\
5.1	0.994629683788646\\
5.11	0.994642035288486\\
5.12	0.994654459757371\\
5.13	0.994666954715433\\
5.14	0.994679517696365\\
5.15	0.99469214624794\\
5.16	0.994704837932545\\
5.17	0.994717590327689\\
5.18	0.994730401026521\\
5.19	0.994743267638334\\
5.2	0.994756187789063\\
5.21	0.994769159121783\\
5.22	0.994782179297192\\
5.23	0.994795245994094\\
5.24	0.994808356909873\\
5.25	0.994821509760964\\
5.26	0.994834702283309\\
5.27	0.994847932232818\\
5.28	0.994861197385812\\
5.29	0.994874495539466\\
5.3	0.994887824512242\\
5.31	0.994901182144317\\
5.32	0.994914566298003\\
5.33	0.994927974858156\\
5.34	0.994941405732585\\
5.35	0.994954856852447\\
5.36	0.994968326172636\\
5.37	0.99498181167217\\
5.38	0.99499531135456\\
5.39	0.995008823248178\\
5.4	0.99502234540662\\
5.41	0.995035875909053\\
5.42	0.995049412860557\\
5.43	0.995062954392465\\
5.44	0.995076498662684\\
5.45	0.99509004385602\\
5.46	0.99510358818448\\
5.47	0.995117129887583\\
5.48	0.995130667232649\\
5.49	0.995144198515081\\
5.5	0.995157722058651\\
5.51	0.99517123621576\\
5.52	0.9951847393677\\
5.53	0.995198229924906\\
5.54	0.995211706327199\\
5.55	0.995225167044017\\
5.56	0.995238610574643\\
5.57	0.995252035448418\\
5.58	0.99526544022495\\
5.59	0.995278823494313\\
5.6	0.995292183877234\\
5.61	0.995305520025275\\
5.62	0.995318830621008\\
5.63	0.995332114378169\\
5.64	0.995345370041821\\
5.65	0.995358596388492\\
5.66	0.995371792226313\\
5.67	0.995384956395145\\
5.68	0.995398087766694\\
5.69	0.995411185244623\\
5.7	0.995424247764646\\
5.71	0.995437274294622\\
5.72	0.995450263834634\\
5.73	0.995463215417062\\
5.74	0.995476128106644\\
5.75	0.995489001000531\\
5.76	0.99550183322833\\
5.77	0.995514623952138\\
5.78	0.995527372366573\\
5.79	0.995540077698785\\
5.8	0.995552739208468\\
5.81	0.995565356187855\\
5.82	0.995577927961712\\
5.83	0.995590453887315\\
5.84	0.995602933354423\\
5.85	0.99561536578524\\
5.86	0.995627750634371\\
5.87	0.99564008738876\\
5.88	0.995652375567634\\
5.89	0.995664614722424\\
5.9	0.995676804436683\\
5.91	0.995688944325999\\
5.92	0.995701034037891\\
5.93	0.995713073251701\\
5.94	0.995725061678477\\
5.95	0.995736999060849\\
5.96	0.995748885172891\\
5.97	0.995760719819979\\
5.98	0.99577250283864\\
5.99	0.995784234096388\\
6	0.99579591349156\\
6.01	0.995807540953137\\
6.02	0.995819116440555\\
6.03	0.995830639943515\\
6.04	0.995842111481779\\
6.05	0.995853531104962\\
6.06	0.995864898892309\\
6.07	0.995876214952473\\
6.08	0.995887479423277\\
6.09	0.995898692471475\\
6.1	0.995909854292498\\
6.11	0.995920965110201\\
6.12	0.995932025176592\\
6.13	0.995943034771561\\
6.14	0.9959539942026\\
6.15	0.995964903804515\\
6.16	0.995975763939127\\
6.17	0.995986574994972\\
6.18	0.995997337386993\\
6.19	0.996008051556217\\
6.2	0.996018717969435\\
6.21	0.996029337118869\\
6.22	0.996039909521835\\
6.23	0.996050435720399\\
6.24	0.996060916281021\\
6.25	0.996071351794203\\
6.26	0.996081742874121\\
6.27	0.996092090158255\\
6.28	0.996102394307013\\
6.29	0.996112656003344\\
6.3	0.996122875952354\\
6.31	0.996133054880907\\
6.32	0.996143193537223\\
6.33	0.996153292690477\\
6.34	0.99616335313038\\
6.35	0.996173375666765\\
6.36	0.996183361129161\\
6.37	0.996193310366366\\
6.38	0.996203224246015\\
6.39	0.996213103654136\\
6.4	0.99622294949471\\
6.41	0.996232762689224\\
6.42	0.996242544176211\\
6.43	0.9962522949108\\
6.44	0.996262015864248\\
6.45	0.996271708023475\\
6.46	0.996281372390593\\
6.47	0.996291009982434\\
6.48	0.996300621830065\\
6.49	0.996310208978309\\
6.5	0.99631977248526\\
6.51	0.996329313421787\\
6.52	0.996338832871048\\
6.53	0.996348331927988\\
6.54	0.99635781169884\\
6.55	0.996367273300622\\
6.56	0.996376717860631\\
6.57	0.996386146515937\\
6.58	0.996395560412864\\
6.59	0.996404960706487\\
6.6	0.996414348560103\\
6.61	0.996423725144724\\
6.62	0.996433091638549\\
6.63	0.996442449226443\\
6.64	0.996451799099414\\
6.65	0.996461142454085\\
6.66	0.996470480492166\\
6.67	0.996479814419927\\
6.68	0.996489145447665\\
6.69	0.996498474789173\\
6.7	0.996507803661209\\
6.71	0.99651713328296\\
6.72	0.996526464875508\\
6.73	0.996535799661298\\
6.74	0.9965451388636\\
6.75	0.996554483705972\\
6.76	0.996563835411729\\
6.77	0.996573195203403\\
6.78	0.996582564302209\\
6.79	0.99659194392751\\
6.8	0.996601335296281\\
6.81	0.996610739622576\\
6.82	0.99662015811699\\
6.83	0.996629591986133\\
6.84	0.996639042432093\\
6.85	0.996648510651906\\
6.86	0.996657997837027\\
6.87	0.996667505172802\\
6.88	0.99667703383794\\
6.89	0.996686585003989\\
6.9	0.996696159834814\\
6.91	0.996705759486072\\
6.92	0.996715385104697\\
6.93	0.996725037828379\\
6.94	0.996734718785054\\
6.95	0.996744429092388\\
6.96	0.996754169857268\\
6.97	0.996763942175296\\
6.98	0.996773747130287\\
6.99	0.996783585793764\\
7	0.996793459224465\\
7.01	0.996803368467844\\
7.02	0.996813314555584\\
7.03	0.996823298505106\\
7.04	0.996833321319091\\
7.05	0.996843383984992\\
7.06	0.996853487474566\\
7.07	0.996863632743398\\
7.08	0.996873820730432\\
7.09	0.996884052357513\\
7.1	0.996894328528923\\
7.11	0.99690465013093\\
7.12	0.996915018031336\\
7.13	0.996925433079034\\
7.14	0.99693589610357\\
7.15	0.996946407914703\\
7.16	0.996956969301979\\
7.17	0.996967581034308\\
7.18	0.996978243859542\\
7.19	0.996988958504065\\
7.2	0.996999725672382\\
7.21	0.997010546046718\\
7.22	0.997021420286626\\
7.23	0.997032349028591\\
7.24	0.997043332885649\\
7.25	0.997054372447009\\
7.26	0.997065468277682\\
7.27	0.997076620918116\\
7.28	0.997087830883836\\
7.29	0.997099098665092\\
7.3	0.997110424726514\\
7.31	0.997121809506778\\
7.32	0.997133253418266\\
7.33	0.997144756846747\\
7.34	0.997156320151059\\
7.35	0.997167943662797\\
7.36	0.997179627686011\\
7.37	0.997191372496908\\
7.38	0.997203178343569\\
7.39	0.997215045445661\\
7.4	0.997226973994168\\
7.41	0.997238964151126\\
7.42	0.997251016049363\\
7.43	0.997263129792247\\
7.44	0.997275305453451\\
7.45	0.997287543076711\\
7.46	0.997299842675603\\
7.47	0.997312204233326\\
7.48	0.997324627702491\\
7.49	0.997337113004919\\
7.5	0.997349660031448\\
7.51	0.997362268641747\\
7.52	0.997374938664144\\
7.53	0.997387669895449\\
7.54	0.997400462100803\\
7.55	0.997413315013523\\
7.56	0.997426228334961\\
7.57	0.997439201734368\\
7.58	0.997452234848771\\
7.59	0.997465327282857\\
7.6	0.997478478608864\\
7.61	0.997491688366483\\
7.62	0.99750495606277\\
7.63	0.997518281172062\\
7.64	0.997531663135908\\
7.65	0.997545101363006\\
7.66	0.997558595229146\\
7.67	0.997572144077168\\
7.68	0.997585747216928\\
7.69	0.997599403925265\\
7.7	0.997613113445991\\
7.71	0.997626874989876\\
7.72	0.997640687734655\\
7.73	0.997654550825032\\
7.74	0.997668463372702\\
7.75	0.997682424456377\\
7.76	0.997696433121829\\
7.77	0.997710488381928\\
7.78	0.997724589216705\\
7.79	0.997738734573412\\
7.8	0.997752923366599\\
7.81	0.997767154478197\\
7.82	0.997781426757606\\
7.83	0.997795739021804\\
7.84	0.997810090055451\\
7.85	0.997824478611013\\
7.86	0.997838903408887\\
7.87	0.997853363137543\\
7.88	0.997867856453669\\
7.89	0.997882381982326\\
7.9	0.997896938317117\\
7.91	0.997911524020355\\
7.92	0.997926137623252\\
7.93	0.997940777626108\\
7.94	0.997955442498512\\
7.95	0.997970130679553\\
7.96	0.997984840578037\\
7.97	0.997999570572715\\
7.98	0.998014319012519\\
7.99	0.998029084216809\\
8	0.998043864475622\\
8.01	0.998058658049938\\
8.02	0.998073463171949\\
8.03	0.998088278045338\\
8.04	0.998103100845566\\
8.05	0.998117929720171\\
8.06	0.998132762789068\\
8.07	0.998147598144865\\
8.08	0.99816243385318\\
8.09	0.998177267952975\\
8.1	0.998192098456887\\
8.11	0.998206923351579\\
8.12	0.998221740598088\\
8.13	0.998236548132187\\
8.14	0.998251343864755\\
8.15	0.998266125682153\\
8.16	0.998280891446605\\
8.17	0.998295638996592\\
8.18	0.998310366147252\\
8.19	0.998325070690781\\
8.2	0.998339750396851\\
8.21	0.998354403013032\\
8.22	0.998369026265214\\
8.23	0.998383617858045\\
8.24	0.998398175475375\\
8.25	0.998412696780701\\
8.26	0.998427179417623\\
8.27	0.998441621010306\\
8.28	0.998456019163952\\
8.29	0.998470371465268\\
8.3	0.998484675482952\\
8.31	0.998498928768181\\
8.32	0.998513128855102\\
8.33	0.998527273261332\\
8.34	0.998541359488463\\
8.35	0.998555385022577\\
8.36	0.998569347334759\\
8.37	0.998583243881619\\
8.38	0.998597072105824\\
8.39	0.998610829436626\\
8.4	0.998624513290407\\
8.41	0.998638121071217\\
8.42	0.998651650171323\\
8.43	0.998665097971766\\
8.44	0.998678461842917\\
8.45	0.99869173914504\\
8.46	0.998704927228858\\
8.47	0.998718023436126\\
8.48	0.998731025100207\\
8.49	0.998743929546652\\
8.5	0.998756734093783\\
8.51	0.99876943605328\\
8.52	0.998782032730776\\
8.53	0.998794521426447\\
8.54	0.998806899435616\\
8.55	0.998819164049351\\
8.56	0.998831312555071\\
8.57	0.998843342237153\\
8.58	0.998855250377546\\
8.59	0.998867034256382\\
8.6	0.998878691152593\\
8.61	0.99889021834453\\
8.62	0.998901613110585\\
8.63	0.998912872729813\\
8.64	0.998923994482557\\
8.65	0.998934975651079\\
8.66	0.998945813520186\\
8.67	0.998956505377861\\
8.68	0.998967048515896\\
8.69	0.998977440230527\\
8.7	0.998987677823067\\
8.71	0.998997758600542\\
8.72	0.999007679876332\\
8.73	0.999017438970802\\
8.74	0.999027033211947\\
8.75	0.999036459936026\\
8.76	0.999045716488205\\
8.77	0.999054800223193\\
8.78	0.999063708505885\\
8.79	0.999072438711999\\
8.8	0.999080988228718\\
8.81	0.999089354455324\\
8.82	0.999097534803843\\
8.83	0.999105526699676\\
8.84	0.999113327582241\\
8.85	0.999120934905607\\
8.86	0.999128346139129\\
8.87	0.999135558768081\\
8.88	0.99914257029429\\
8.89	0.999149378236764\\
8.9	0.999155980132324\\
8.91	0.99916237353623\\
8.92	0.999168556022804\\
8.93	0.999174525186059\\
8.94	0.999180278640313\\
8.95	0.999185814020813\\
8.96	0.999191128984346\\
8.97	0.999196221209858\\
8.98	0.999201088399058\\
8.99	0.999205728277033\\
9	0.999210138592846\\
};
\end{axis}

\begin{axis}[%
width=0.4\figW,
height=0.15\figH,
at={(0\figW,0.215\figH)},
scale only axis,
xmin=0,
xmax=10,
ymin=0,
ymax=0.2,
yticklabel style={/pgf/number format/fixed},
axis background/.style={fill=white}
]
\addplot [color=mycolor1, forget plot]
  table[row sep=crcr]{%
0	0.0001016601438951\\
0.025062656641604	1.60795577746672e-05\\
0.050125313283208	0.000108681760633028\\
0.075187969924812	0.000210980406103982\\
0.100250626566416	0.000333001451166125\\
0.12531328320802	0.000474248805061133\\
0.150375939849624	0.000909120058950467\\
0.175438596491228	0.00157900103644101\\
0.200501253132832	0.00241058528414823\\
0.225563909774436	0.00330741873410879\\
0.25062656641604	0.00416197293559212\\
0.275689223057644	0.00489112997542091\\
0.300751879699248	0.00545323336745852\\
0.325814536340852	0.00584578899991318\\
0.350877192982456	0.00609246975954079\\
0.37593984962406	0.00622853572605724\\
0.401002506265664	0.00628978623137436\\
0.426065162907268	0.00630624416623338\\
0.451127819548872	0.0062996684531127\\
0.476190476190476	0.00628347974339978\\
0.50125313283208	0.00626401451341151\\
0.526315789473684	0.00624275054890815\\
0.551378446115288	0.00621968721977046\\
0.576441102756892	0.00619480466801215\\
0.601503759398496	0.00616810100399902\\
0.6265664160401	0.00613959630478877\\
0.651629072681704	0.00610933716219489\\
0.676691729323308	0.00607740181668227\\
0.701754385964912	0.00605027120280651\\
0.726817042606516	0.00602449549797963\\
0.75187969924812	0.00599876995373009\\
0.776942355889724	0.0059734346415773\\
0.802005012531328	0.00594888318340649\\
0.827067669172932	0.00592556582729862\\
0.852130325814536	0.00591077055314746\\
0.87719298245614	0.00590981505733452\\
0.902255639097744	0.0059129658408672\\
0.927318295739348	0.00592075793725937\\
0.952380952380952	0.00594303124077444\\
0.977443609022556	0.00598239928203477\\
1.00250626566416	0.00602779546224047\\
1.02756892230576	0.0060792586484337\\
1.05263157894737	0.00613670405900232\\
1.07769423558897	0.00621141800788106\\
1.10275689223058	0.00629250914339374\\
1.12781954887218	0.00637597676526617\\
1.15288220551378	0.00646058908617629\\
1.17794486215539	0.00654487119487572\\
1.20300751879699	0.0066270786303047\\
1.2280701754386	0.0067051645563513\\
1.2531328320802	0.00677673822990729\\
1.2781954887218	0.00683901202429858\\
1.30325814536341	0.00689127418478129\\
1.32832080200501	0.00695126309356644\\
1.35338345864662	0.00699735620887025\\
1.37844611528822	0.00704517615699566\\
1.40350877192982	0.00712803453074815\\
1.42857142857143	0.00730657115274168\\
1.45363408521303	0.00761867812498051\\
1.47869674185464	0.00806330646737979\\
1.50375939849624	0.00860106560831061\\
1.52882205513784	0.00944079902991583\\
1.55388471177945	0.0104181804015152\\
1.57894736842105	0.0113526577987896\\
1.60401002506266	0.0121240542271787\\
1.62907268170426	0.0126729366741657\\
1.65413533834586	0.0129767519181916\\
1.67919799498747	0.0130486136125212\\
1.70426065162907	0.0129386208983761\\
1.72932330827068	0.0127354792149697\\
1.75438596491228	0.0125639004500726\\
1.77944862155388	0.0125820313575826\\
1.80451127819549	0.0130784884704327\\
1.82957393483709	0.0142641484644647\\
1.8546365914787	0.0160476509642303\\
1.8796992481203	0.0180728192866686\\
1.9047619047619	0.0201286275077381\\
1.92982456140351	0.0221020229720626\\
1.95488721804511	0.023896789384984\\
1.97994987468672	0.0254366721485354\\
2.00501253132832	0.0266682345805533\\
2.03007518796993	0.0275675846225961\\
2.05513784461153	0.0281545566423917\\
2.08020050125313	0.0284342379950602\\
2.10526315789474	0.0284726316791075\\
2.13032581453634	0.0283708998170459\\
2.15538847117795	0.0282606729513339\\
2.18045112781955	0.0282914337656513\\
2.20551378446115	0.0286078830296203\\
2.23057644110276	0.0293407339976987\\
2.25563909774436	0.0305147325930635\\
2.28070175438596	0.0320994394500829\\
2.30576441102757	0.034005978397566\\
2.33082706766917	0.0361072216493594\\
2.35588972431078	0.0382646680871135\\
2.38095238095238	0.0403723748290393\\
2.40601503759398	0.0422876662084348\\
2.43107769423559	0.0439186567609998\\
2.45614035087719	0.0452055138956308\\
2.4812030075188	0.04612370909025\\
2.5062656641604	0.0466859246053286\\
2.53132832080201	0.0469424746007588\\
2.55639097744361	0.0469795066559392\\
2.58145363408521	0.0469137281146922\\
2.60651629072682	0.0468821503581635\\
2.63157894736842	0.0470258695730758\\
2.65664160401003	0.0474687946365185\\
2.68170426065163	0.0482953495707097\\
2.70676691729323	0.049533752737385\\
2.73182957393484	0.0511508588477999\\
2.75689223057644	0.0530601099759928\\
2.78195488721805	0.0551388464590541\\
2.80701754385965	0.0572488051896157\\
2.83208020050125	0.0592548007656905\\
2.85714285714286	0.0610392744853372\\
2.88220551378446	0.0625125296738782\\
2.90726817042607	0.0636193975886333\\
2.93233082706767	0.0643431133987786\\
2.95739348370927	0.0647068047681686\\
2.98245614035088	0.0647725038484872\\
3.00751879699248	0.0646371311253605\\
3.03258145363409	0.0644245975308581\\
3.05764411027569	0.0642732627651733\\
3.08270676691729	0.0643188000146967\\
3.1077694235589	0.0646742166171795\\
3.1328320802005	0.0654108944758048\\
3.15789473684211	0.0665457220564363\\
3.18295739348371	0.0680383110932708\\
3.20802005012531	0.0697989889169439\\
3.23308270676692	0.0717046537686519\\
3.25814536340852	0.0736178135761729\\
3.28320802005013	0.0754046899618302\\
3.30827067669173	0.076950045716076\\
3.33333333333333	0.0781680534912028\\
3.35839598997494	0.0790094448992569\\
3.38345864661654	0.0794654024221234\\
3.40852130325815	0.0795684627182613\\
3.43358395989975	0.0793903430469033\\
3.45864661654135	0.079036266582005\\
3.48370927318296	0.0786352261972743\\
3.50877192982456	0.0783259170671436\\
3.53383458646617	0.078239002077158\\
3.55889724310777	0.0784779184529436\\
3.58395989974937	0.0791020096324312\\
3.60902255639098	0.0801162924025866\\
3.63408521303258	0.0814708057579249\\
3.65914786967419	0.0830695985001014\\
3.68421052631579	0.0847865607166596\\
3.70927318295739	0.0864839901330485\\
3.734335839599	0.0880302722744942\\
3.7593984962406	0.0893144959247352\\
3.78446115288221	0.0902572169081203\\
3.80952380952381	0.0908174117499572\\
3.83458646616541	0.0909959133173094\\
3.85964912280702	0.0908354978836232\\
3.88471177944862	0.090417519653717\\
3.90977443609023	0.089854756799032\\
3.93483709273183	0.0892801366379663\\
3.95989974937343	0.0888314553415187\\
3.98496240601504	0.0886332290023216\\
4.01002506265664	0.0887782404218204\\
4.03508771929825	0.0893125344161526\\
4.06015037593985	0.0902276405165309\\
4.08521303258145	0.0914621720154526\\
4.11027568922306	0.0929122499116605\\
4.13533834586466	0.0944478309730228\\
4.16040100250627	0.0959311009260072\\
4.18546365914787	0.0972336770303357\\
4.21052631578947	0.098250676421027\\
4.23558897243108	0.0989109237109237\\
4.26065162907268	0.0991832901288724\\
4.28571428571429	0.0990793740631683\\
4.31077694235589	0.0986526281424562\\
4.33583959899749	0.0979938154460739\\
4.3609022556391	0.0972225289329308\\
4.3859649122807	0.0964746296554699\\
4.41102756892231	0.0958860349603481\\
4.43609022556391	0.0955743732885852\\
4.46115288220551	0.0956213454718639\\
4.48621553884712	0.0960595056702504\\
4.51127819548872	0.0968668001767269\\
4.53634085213033	0.0979703451086785\\
4.56140350877193	0.0992583412808244\\
4.58646616541353	0.100597018889182\\
4.61152882205514	0.101848927526119\\
4.63659147869674	0.102889608635911\\
4.66165413533835	0.103620951225053\\
4.68671679197995	0.103980629006797\\
4.71177944862155	0.103947635712572\\
4.73684210526316	0.103544101960467\\
4.76190476190476	0.102833465448385\\
4.78696741854637	0.101914871463796\\
4.81203007518797	0.100913592414283\\
4.83709273182957	0.0999674667075569\\
4.86215538847118	0.0992100307602822\\
4.88721804511278	0.0987521501472449\\
4.91228070175439	0.0986651952887611\\
4.93734335839599	0.0989694336731041\\
4.96240601503759	0.0996306199891175\\
4.9874686716792	0.100565718768842\\
5.0125313283208	0.101656184164967\\
5.03759398496241	0.102765491970549\\
5.06265664160401	0.103757327360748\\
5.08771929824561	0.104511699806299\\
5.11278195488722	0.104937516416485\\
5.13784461152882	0.104981162198354\\
5.16290726817043	0.10463116989336\\
5.18796992481203	0.103919173385878\\
5.21303258145363	0.102917205277519\\
5.23809523809524	0.101731212553625\\
5.26315789473684	0.100490613151571\\
5.28822055137845	0.0993339934453407\\
5.31328320802005	0.0983917994109545\\
5.33834586466165	0.0977680545985446\\
5.36340852130326	0.0975244423914352\\
5.38847117794486	0.0976699161734979\\
5.41353383458647	0.0981592118438622\\
5.43859649122807	0.0989005622284983\\
5.46365914786967	0.0997701320037895\\
5.48872180451128	0.100630064690532\\
5.51378446115288	0.101346488632265\\
5.53884711779449	0.101804969099269\\
5.56390977443609	0.101922178585544\\
5.58897243107769	0.101653512814445\\
5.6140350877193	0.100996831746509\\
5.6390977443609	0.0999925513474474\\
5.66416040100251	0.098720141683477\\
5.68922305764411	0.0972908913367579\\
5.71428571428571	0.09583676786378\\
5.73934837092732	0.0944955366048836\\
5.76441102756892	0.0933931366514611\\
5.78947368421053	0.0926255649894474\\
5.81453634085213	0.0922436873968668\\
5.83959899749373	0.0922446500310789\\
5.86466165413534	0.0925723073997256\\
5.88972431077694	0.0931266114033269\\
5.91478696741855	0.0937794317615269\\
5.93984962406015	0.0943930261230012\\
5.96491228070175	0.0948376732690507\\
5.98997493734336	0.0950062195213963\\
6.01503759398496	0.0948246106186013\\
6.04010025062657	0.0942583638641794\\
6.06516290726817	0.0933152835836253\\
6.09022556390977	0.0920446826593243\\
6.11528822055138	0.0905331458743134\\
6.14035087719298	0.0888966479301089\\
6.16541353383459	0.0872688162029934\\
6.19047619047619	0.0857855186195905\\
6.21553884711779	0.0845669088732922\\
6.2406015037594	0.0836994441130101\\
6.265664160401	0.0832215865771732\\
6.29072681704261	0.0831169576596447\\
6.31578947368421	0.0833170572988356\\
6.34085213032582	0.0837128462230526\\
6.36591478696742	0.0841720009262296\\
6.39097744360902	0.0845577344536431\\
6.41604010025063	0.0847458016162289\\
6.44110275689223	0.0846378105603147\\
6.46616541353383	0.0841703197164005\\
6.49122807017544	0.0833199722567373\\
6.51629072681704	0.0821051248753968\\
6.54135338345865	0.0805842683004611\\
6.56641604010025	0.0788512241708058\\
6.59147869674185	0.0770268205970957\\
6.61654135338346	0.0752467015405658\\
6.64160401002506	0.073645365715285\\
6.66666666666667	0.0723376536559636\\
6.69172932330827	0.0714005467252441\\
6.71679197994987	0.0708595018147791\\
6.74185463659148	0.0706834016294006\\
6.76691729323308	0.0707899506697788\\
6.79197994987469	0.0710599460051543\\
6.81704260651629	0.0713562853001519\\
6.84210526315789	0.0715431392226629\\
6.8671679197995	0.0715020692208817\\
6.8922305764411	0.071143743740896\\
6.91729323308271	0.0704152904158004\\
6.94235588972431	0.0693039311513155\\
6.96741854636591	0.0678375585861847\\
6.99248120300752	0.0660825997716303\\
7.01754385964912	0.0641390787277055\\
7.04260651629073	0.0621323861953567\\
7.06766917293233	0.0602010984888896\\
7.09273182957393	0.0584805943261006\\
7.11779448621554	0.057083554259119\\
7.14285714285714	0.0560806085015179\\
7.16791979949875	0.0554863265374319\\
7.19298245614035	0.0552555489987588\\
7.21804511278195	0.0552918402947071\\
7.24310776942356	0.0554652462896414\\
7.26817042606516	0.0556336591628624\\
7.29323308270677	0.0556624182387283\\
7.31829573934837	0.0554391677907264\\
7.34335839598998	0.0548834275786631\\
7.36842105263158	0.0539517137602734\\
7.39348370927318	0.0526393938591176\\
7.41854636591479	0.0509802143033718\\
7.44360902255639	0.0490439602469301\\
7.468671679198	0.046932148053873\\
7.4937343358396	0.0447710592545353\\
7.5187969924812	0.0427009279286346\\
7.54385964912281	0.0408601132858888\\
7.56892230576441	0.0393644024940792\\
7.59398496240602	0.0382847183589178\\
7.61904761904762	0.0376302227668202\\
7.64411027568922	0.0373445312706888\\
7.66917293233083	0.0373178294824764\\
7.69423558897243	0.0374099950902683\\
7.71929824561404	0.0374757520451267\\
7.74436090225564	0.0373848045851478\\
7.76942355889724	0.0370345002059366\\
7.79448621553885	0.0363559744772762\\
7.81954887218045	0.0353158910562973\\
7.84461152882206	0.0339156629925593\\
7.86967418546366	0.032189404855922\\
7.89473684210526	0.0302012762675053\\
7.91979949874687	0.0280423778828683\\
7.94486215538847	0.0258267846218724\\
7.96992481203008	0.0236854138210459\\
7.99498746867168	0.021755186978711\\
8.02005012531328	0.0201602501207947\\
8.04511278195489	0.018984644538597\\
8.07017543859649	0.0182444368291386\\
8.09523809523809	0.0178763836583438\\
8.1203007518797	0.0177548206442681\\
8.1453634085213	0.0177279689087762\\
8.17042606516291	0.0176529025957227\\
8.19548872180451	0.0174165032867131\\
8.22055137844612	0.0169511329694343\\
8.24561403508772	0.0162513327579596\\
8.27067669172932	0.0152759162657483\\
8.29573934837093	0.0140835473521073\\
8.32080200501253	0.0127238352997998\\
8.34586466165413	0.0113117858820422\\
8.37092731829574	0.00991019813842571\\
8.39598997493734	0.00859778330332538\\
8.42105263157895	0.00741527327530656\\
8.44611528822055	0.00636566073009168\\
8.47117794486216	0.00545550794362833\\
8.49624060150376	0.00465228331889516\\
8.52130325814536	0.00396234662426513\\
8.54636591478697	0.0042369373651685\\
8.57142857142857	0.00446229564487167\\
8.59649122807017	0.00456682217423294\\
8.62155388471178	0.00459618744204186\\
8.64661654135338	0.0046567596027491\\
8.67167919799499	0.00520268616802121\\
8.69674185463659	0.00656663959344192\\
8.7218045112782	0.00819460978014019\\
8.7468671679198	0.0100282787960916\\
8.7719298245614	0.0119707089624722\\
8.79699248120301	0.013962056544836\\
8.82205513784461	0.0158586041132488\\
8.84711779448622	0.0175895423903873\\
8.87218045112782	0.0190944716663949\\
8.89724310776942	0.0203272969356765\\
8.92230576441103	0.021260170975619\\
8.94736842105263	0.0218873502750765\\
8.97243107769424	0.0222287510235953\\
8.99749373433584	0.0223827015246959\\
9.02255639097744	0.0224247339755459\\
9.04761904761905	0.0224100799351966\\
9.07268170426065	0.022348444958445\\
9.09774436090226	0.0222495395553818\\
9.12280701754386	0.0221226663358561\\
9.14786967418546	0.0219763809358014\\
9.17293233082707	0.021818282209897\\
9.19799498746867	0.0216549197773472\\
9.22305764411028	0.0214917861210879\\
9.24812030075188	0.0213333618464442\\
9.27318295739348	0.0211831907254255\\
9.29824561403509	0.0210439775572362\\
9.32330827067669	0.0209178125057598\\
9.3483709273183	0.0208562062648034\\
9.3734335839599	0.0208287525500428\\
9.3984962406015	0.0208158942217159\\
9.42355889724311	0.0208164025303595\\
9.44862155388471	0.0208759821738801\\
9.47368421052632	0.0209599513538246\\
9.49874686716792	0.0210485881775653\\
9.52380952380952	0.0211401898463095\\
9.54887218045113	0.0212331153256494\\
9.57393483709273	0.0213330793014759\\
9.59899749373434	0.0214253442769315\\
9.62406015037594	0.0214375721118134\\
9.64912280701754	0.0212940432709448\\
9.67418546365915	0.0208904684283681\\
9.69924812030075	0.020102367475211\\
9.72431077694236	0.0188066889131047\\
9.74937343358396	0.0169198394870975\\
9.77443609022556	0.014448017581371\\
9.79949874686717	0.011532203933253\\
9.82456140350877	0.0084563804608845\\
9.84962406015038	0.00559859993837568\\
9.87468671679198	0.00327717601554949\\
9.89974937343358	0.00165706879157562\\
9.92481203007519	0.000929401713406141\\
9.94987468671679	0.000673078551067262\\
9.9749373433584	0.000444209181839585\\
10	0.0002544468733784\\
};
\addplot [color=mycolor2, forget plot]
  table[row sep=crcr]{%
1.5	0\\
1.51	0.000450935233614233\\
1.52	0.000901861620311582\\
1.53	0.00135277004007353\\
1.54	0.00180365137325324\\
1.55	0.00225449650077072\\
1.56	0.00270529630430809\\
1.57	0.00315604166650466\\
1.58	0.00360672347115215\\
1.59	0.00405733260338979\\
1.6	0.00450785994989936\\
1.61	0.00495829639910029\\
1.62	0.00540863284134462\\
1.63	0.00585886016911192\\
1.64	0.00630896927720425\\
1.65	0.00675895106294089\\
1.66	0.00720879642635317\\
1.67	0.00765849627037911\\
1.68	0.00810804150105806\\
1.69	0.00855742302772519\\
1.7	0.00900663176320593\\
1.71	0.00945565862401034\\
1.72	0.0099044945305273\\
1.73	0.0103531304072187\\
1.74	0.0108015571828135\\
1.75	0.0112497657905014\\
1.76	0.011697747168127\\
1.77	0.0121454922583833\\
1.78	0.0125929920090051\\
1.79	0.0130402373729627\\
1.8	0.0134872193086547\\
1.81	0.0139339287801018\\
1.82	0.014380356757139\\
1.83	0.014826494215609\\
1.84	0.0152723321375546\\
1.85	0.0157178615114112\\
1.86	0.016163073332199\\
1.87	0.0166079586017154\\
1.88	0.0170525083287267\\
1.89	0.01749671352916\\
1.9	0.0179405652262947\\
1.91	0.0183840544509539\\
1.92	0.0188271722416954\\
1.93	0.0192699096450032\\
1.94	0.0197122577154776\\
1.95	0.020154207516026\\
1.96	0.0205957501180532\\
1.97	0.0210368766016516\\
1.98	0.0214775780557908\\
1.99	0.0219178455785075\\
2	0.0223576702770945\\
2.01	0.0227970432682901\\
2.02	0.0232359556784669\\
2.03	0.0236743986438204\\
2.04	0.0241123633105572\\
2.05	0.0245498408350833\\
2.06	0.0249868223841918\\
2.07	0.0254232991352502\\
2.08	0.0258592622763881\\
2.09	0.0262947030066838\\
2.1	0.0267296125363511\\
2.11	0.0271639820869254\\
2.12	0.0275978028914501\\
2.13	0.0280310661946624\\
2.14	0.0284637632531783\\
2.15	0.028895885335678\\
2.16	0.0293274237230907\\
2.17	0.0297583697087789\\
2.18	0.030188714598723\\
2.19	0.0306184497117044\\
2.2	0.0310475663794896\\
2.21	0.0314760559470128\\
2.22	0.031903909772559\\
2.23	0.0323311192279464\\
2.24	0.0327576756987083\\
2.25	0.0331835705842747\\
2.26	0.033608795298154\\
2.27	0.0340333412681135\\
2.28	0.0344571999363601\\
2.29	0.0348803627597208\\
2.3	0.035302821209822\\
2.31	0.0357245667732691\\
2.32	0.0361455909518255\\
2.33	0.0365658852625914\\
2.34	0.0369854412381817\\
2.35	0.0374042504269036\\
2.36	0.0378223043929347\\
2.37	0.0382395947164992\\
2.38	0.0386561129940447\\
2.39	0.0390718508384185\\
2.4	0.0394867998790428\\
2.41	0.0399009517620904\\
2.42	0.0403142981506592\\
2.43	0.0407268307249465\\
2.44	0.0411385411824232\\
2.45	0.0415494212380068\\
2.46	0.0419594626242347\\
2.47	0.0423686570914366\\
2.48	0.0427769964079064\\
2.49	0.043184472360074\\
2.5	0.0435910767526765\\
2.51	0.0439968014089283\\
2.52	0.0444016381706916\\
2.53	0.0448055788986461\\
2.54	0.045208615472458\\
2.55	0.0456107397909486\\
2.56	0.0460119437722627\\
2.57	0.0464122193540358\\
2.58	0.0468115584935618\\
2.59	0.0472099531679593\\
2.6	0.0476073953743375\\
2.61	0.0480038771299623\\
2.62	0.0483993904724211\\
2.63	0.0487939274597873\\
2.64	0.0491874801707842\\
2.65	0.0495800407049488\\
2.66	0.0499716011827945\\
2.67	0.0503621537459734\\
2.68	0.0507516905574384\\
2.69	0.0511402038016044\\
2.7	0.0515276856845087\\
2.71	0.0519141284339717\\
2.72	0.0522995242997562\\
2.73	0.0526838655537263\\
2.74	0.0530671444900064\\
2.75	0.0534493534251385\\
2.76	0.0538304846982399\\
2.77	0.0542105306711597\\
2.78	0.0545894837286353\\
2.79	0.0549673362784476\\
2.8	0.0553440807515762\\
2.81	0.0557197096023536\\
2.82	0.0560942153086192\\
2.83	0.0564675903718725\\
2.84	0.0568398273174252\\
2.85	0.0572109186945541\\
2.86	0.0575808570766513\\
2.87	0.0579496350613759\\
2.88	0.0583172452708037\\
2.89	0.0586836803515768\\
2.9	0.0590489329750524\\
2.91	0.0594129958374512\\
2.92	0.0597758616600047\\
2.93	0.0601375231891027\\
2.94	0.0604979731964392\\
2.95	0.0608572044791585\\
2.96	0.0612152098599998\\
2.97	0.0615719821874419\\
2.98	0.0619275143358471\\
2.99	0.0622817992056041\\
3	0.0626348297232705\\
3.01	0.0629865988417147\\
3.02	0.0633370995402568\\
3.03	0.0636863248248094\\
3.04	0.0640342677280175\\
3.05	0.0643809213093969\\
3.06	0.0647262786554735\\
3.07	0.065070332879921\\
3.08	0.0654130771236973\\
3.09	0.0657545045551818\\
3.1	0.0660946083703106\\
3.11	0.0664333817927121\\
3.12	0.0667708180738406\\
3.13	0.0671069104931106\\
3.14	0.0674416523580298\\
3.15	0.0677750370043311\\
3.16	0.0681070577961041\\
3.17	0.0684377081259266\\
3.18	0.0687669814149942\\
3.19	0.06909487111325\\
3.2	0.0694213706995134\\
3.21	0.0697464736816082\\
3.22	0.0700701735964899\\
3.23	0.0703924640103722\\
3.24	0.0707133385188532\\
3.25	0.0710327907470405\\
3.26	0.0713508143496756\\
3.27	0.0716674030112574\\
3.28	0.0719825504461661\\
3.29	0.0722962503987845\\
3.3	0.07260849664362\\
3.31	0.0729192829854255\\
3.32	0.0732286032593196\\
3.33	0.0735364513309054\\
3.34	0.0738428210963898\\
3.35	0.074147706482701\\
3.36	0.0744511014476057\\
3.37	0.0747529999798254\\
3.38	0.0750533960991524\\
3.39	0.0753522838565643\\
3.4	0.0756496573343383\\
3.41	0.0759455106461647\\
3.42	0.0762398379372593\\
3.43	0.0765326333844754\\
3.44	0.0768238911964151\\
3.45	0.0771136056135394\\
3.46	0.077401770908278\\
3.47	0.077688381385138\\
3.48	0.0779734313808121\\
3.49	0.0782569152642856\\
3.5	0.0785388274369436\\
3.51	0.078819162332676\\
3.52	0.0790979144179833\\
3.53	0.07937507819208\\
3.54	0.0796506481869986\\
3.55	0.0799246189676924\\
3.56	0.080196985132137\\
3.57	0.0804677413114315\\
3.58	0.0807368821698991\\
3.59	0.0810044024051866\\
3.6	0.081270296748363\\
3.61	0.0815345599640178\\
3.62	0.0817971868503579\\
3.63	0.0820581722393044\\
3.64	0.0823175109965879\\
3.65	0.0825751980218438\\
3.66	0.0828312282487059\\
3.67	0.0830855966448999\\
3.68	0.0833382982123362\\
3.69	0.0835893279872012\\
3.7	0.0838386810400481\\
3.71	0.0840863524758877\\
3.72	0.0843323374342772\\
3.73	0.0845766310894086\\
3.74	0.084819228650197\\
3.75	0.085060125360367\\
3.76	0.0852993164985394\\
3.77	0.0855367973783159\\
3.78	0.0857725633483642\\
3.79	0.0860066097925014\\
3.8	0.0862389321297771\\
3.81	0.0864695258145557\\
3.82	0.0866983863365973\\
3.83	0.0869255092211387\\
3.84	0.0871508900289728\\
3.85	0.087374524356528\\
3.86	0.0875964078359456\\
3.87	0.0878165361351578\\
3.88	0.088034904957964\\
3.89	0.0882515100441064\\
3.9	0.0884663471693448\\
3.91	0.0886794121455313\\
3.92	0.0888907008206829\\
3.93	0.0891002090790542\\
3.94	0.0893079328412095\\
3.95	0.089513868064093\\
3.96	0.0897180107410993\\
3.97	0.0899203569021425\\
3.98	0.0901209026137249\\
3.99	0.0903196439790039\\
4	0.09051657713786\\
4.01	0.0907116982669616\\
4.02	0.0909050035798311\\
4.03	0.0910964893269088\\
4.04	0.0912861517956167\\
4.05	0.0914739873104213\\
4.06	0.0916599922328954\\
4.07	0.0918441629617792\\
4.08	0.0920264959330411\\
4.09	0.0922069876199366\\
4.1	0.0923856345330676\\
4.11	0.0925624332204399\\
4.12	0.0927373802675205\\
4.13	0.0929104722972939\\
4.14	0.0930817059703176\\
4.15	0.0932510779847764\\
4.16	0.093418585076537\\
4.17	0.0935842240192003\\
4.18	0.0937479916241541\\
4.19	0.0939098847406246\\
4.2	0.0940699002557265\\
4.21	0.0942280350945134\\
4.22	0.0943842862200266\\
4.23	0.094538650633343\\
4.24	0.0946911253736231\\
4.25	0.094841707518157\\
4.26	0.0949903941824108\\
4.27	0.0951371825200709\\
4.28	0.0952820697230886\\
4.29	0.0954250530217233\\
4.3	0.0955661296845854\\
4.31	0.0957052970186774\\
4.32	0.0958425523694355\\
4.33	0.0959778931207693\\
4.34	0.0961113166951014\\
4.35	0.096242820553406\\
4.36	0.0963724021952464\\
4.37	0.0965000591588119\\
4.38	0.0966257890209543\\
4.39	0.096749589397223\\
4.4	0.0968714579418995\\
4.41	0.0969913923480314\\
4.42	0.0971093903474649\\
4.43	0.0972254497108773\\
4.44	0.0973395682478084\\
4.45	0.0974517438066909\\
4.46	0.0975619742748799\\
4.47	0.0976702575786826\\
4.48	0.0977765916833855\\
4.49	0.0978809745932828\\
4.5	0.0979834043517023\\
4.51	0.0980838790410313\\
4.52	0.0981823967827418\\
4.53	0.0982789557374146\\
4.54	0.0983735541047629\\
4.55	0.0984661901236546\\
4.56	0.0985568620721344\\
4.57	0.0986455682674452\\
4.58	0.0987323070660475\\
4.59	0.0988170768636397\\
4.6	0.0988998760951768\\
4.61	0.0989807032348877\\
4.62	0.0990595567962931\\
4.63	0.0991364353322215\\
4.64	0.0992113374348247\\
4.65	0.099284261735593\\
4.66	0.099355206905369\\
4.67	0.099424171654361\\
4.68	0.0994911547321555\\
4.69	0.0995561549277287\\
4.7	0.0996191710694578\\
4.71	0.0996802020251311\\
4.72	0.0997392467019572\\
4.73	0.099796304046574\\
4.74	0.0998513730450564\\
4.75	0.0999044527229234\\
4.76	0.0999555421451445\\
4.77	0.100004640416145\\
4.78	0.100051746679813\\
4.79	0.100096860119498\\
4.8	0.100139979958021\\
4.81	0.100181105457674\\
4.82	0.100220235920219\\
4.83	0.100257370686894\\
4.84	0.10029250913841\\
4.85	0.100325650694953\\
4.86	0.10035679481618\\
4.87	0.100385941001219\\
4.88	0.100413088788666\\
4.89	0.100438237756581\\
4.9	0.100461387522484\\
4.91	0.100482537743349\\
4.92	0.100501688115603\\
4.93	0.100518838375112\\
4.94	0.100533988297181\\
4.95	0.100547137696541\\
4.96	0.100558286427343\\
4.97	0.100567434383149\\
4.98	0.100574581496918\\
4.99	0.100579727741002\\
5	0.100582873127128\\
5.01	0.100584017706388\\
5.02	0.100583161569227\\
5.03	0.100580304845429\\
5.04	0.1005754477041\\
5.05	0.100568590353656\\
5.06	0.100559733041806\\
5.07	0.100548876055534\\
5.08	0.100536019721085\\
5.09	0.100521164403941\\
5.1	0.100504310508811\\
5.11	0.100485458479602\\
5.12	0.100464608799404\\
5.13	0.100441761990471\\
5.14	0.100416918614194\\
5.15	0.100390079271081\\
5.16	0.100361244600736\\
5.17	0.100330415281833\\
5.18	0.100297592032091\\
5.19	0.100262775608251\\
5.2	0.100225966806049\\
5.21	0.100187166460188\\
5.22	0.100146375444314\\
5.23	0.100103594670985\\
5.24	0.100058825091644\\
5.25	0.10001206769659\\
5.26	0.0999633235149457\\
5.27	0.09991259361463\\
5.28	0.099859879102324\\
5.29	0.0998051811234402\\
5.3	0.0997485008620892\\
5.31	0.0996898395410468\\
5.32	0.0996291984217193\\
5.33	0.0995665788041092\\
5.34	0.0995019820267794\\
5.35	0.0994354094668171\\
5.36	0.0993668625397972\\
5.37	0.0992963426997445\\
5.38	0.0992238514390957\\
5.39	0.0991493902886609\\
5.4	0.0990729608175832\\
5.41	0.0989945646332997\\
5.42	0.0989142033814997\\
5.43	0.098831878746084\\
5.44	0.098747592449122\\
5.45	0.0986613462508097\\
5.46	0.0985731419494256\\
5.47	0.0984829813812871\\
5.48	0.0983908664207053\\
5.49	0.09829679897994\\
5.5	0.0982007810091532\\
5.51	0.0981028144963632\\
5.52	0.0980029014673961\\
5.53	0.097901043985839\\
5.54	0.0977972441529905\\
5.55	0.0976915041078119\\
5.56	0.0975838260268771\\
5.57	0.0974742121243223\\
5.58	0.0973626646517946\\
5.59	0.0972491858984001\\
5.6	0.0971337781906519\\
5.61	0.0970164438924166\\
5.62	0.0968971854048609\\
5.63	0.096776005166397\\
5.64	0.0966529056526281\\
5.65	0.0965278893762924\\
5.66	0.0964009588872071\\
5.67	0.0962721167722117\\
5.68	0.0961413656551104\\
5.69	0.0960087081966144\\
5.7	0.0958741470942826\\
5.71	0.0957376850824631\\
5.72	0.0955993249322331\\
5.73	0.095459069451338\\
5.74	0.0953169214841309\\
5.75	0.0951728839115107\\
5.76	0.0950269596508597\\
5.77	0.094879151655981\\
5.78	0.0947294629170346\\
5.79	0.0945778964604738\\
5.8	0.0944244553489802\\
5.81	0.0942691426813986\\
5.82	0.0941119615926709\\
5.83	0.0939529152537699\\
5.84	0.0937920068716323\\
5.85	0.0936292396890909\\
5.86	0.0934646169848063\\
5.87	0.0932981420731984\\
5.88	0.0931298183043769\\
5.89	0.0929596490640712\\
5.9	0.0927876377735601\\
5.91	0.0926137878896005\\
5.92	0.0924381029043558\\
5.93	0.0922605863453238\\
5.94	0.0920812417752634\\
5.95	0.0919000727921218\\
5.96	0.0917170830289601\\
5.97	0.0915322761538789\\
5.98	0.0913456558699433\\
5.99	0.091157225915107\\
6	0.0909669900621364\\
6.01	0.0907749521185337\\
6.02	0.0905811159264591\\
6.03	0.0903854853626539\\
6.04	0.090188064338361\\
6.05	0.0899888567992467\\
6.06	0.0897878667253205\\
6.07	0.0895850981308555\\
6.08	0.0893805550643076\\
6.09	0.0891742416082342\\
6.1	0.0889661618792125\\
6.11	0.0887563200277573\\
6.12	0.088544720238238\\
6.13	0.0883313667287955\\
6.14	0.0881162637512581\\
6.15	0.0878994155910571\\
6.16	0.0876808265671419\\
6.17	0.0874605010318943\\
6.18	0.0872384433710427\\
6.19	0.0870146580035756\\
6.2	0.086789149381654\\
6.21	0.0865619219905244\\
6.22	0.0863329803484303\\
6.23	0.0861023290065235\\
6.24	0.0858699725487752\\
6.25	0.085635915591886\\
6.26	0.0854001627851959\\
6.27	0.0851627188105932\\
6.28	0.0849235883824238\\
6.29	0.0846827762473988\\
6.3	0.0844402871845029\\
6.31	0.0841961260049008\\
6.32	0.0839502975518446\\
6.33	0.0837028067005796\\
6.34	0.0834536583582502\\
6.35	0.083202857463805\\
6.36	0.0829504089879012\\
6.37	0.0826963179328093\\
6.38	0.0824405893323164\\
6.39	0.0821832282516295\\
6.4	0.0819242397872782\\
6.41	0.0816636290670167\\
6.42	0.0814014012497258\\
6.43	0.081137561525314\\
6.44	0.0808721151146179\\
6.45	0.0806050672693031\\
6.46	0.0803364232717631\\
6.47	0.0800661884350193\\
6.48	0.0797943681026193\\
6.49	0.0795209676485352\\
6.5	0.0792459924770619\\
6.51	0.0789694480227136\\
6.52	0.0786913397501215\\
6.53	0.0784116731539296\\
6.54	0.0781304537586907\\
6.55	0.077847687118762\\
6.56	0.0775633788182001\\
6.57	0.0772775344706549\\
6.58	0.0769901597192644\\
6.59	0.0767012602365477\\
6.6	0.0764108417242984\\
6.61	0.0761189099134768\\
6.62	0.0758254705641025\\
6.63	0.075530529465146\\
6.64	0.0752340924344197\\
6.65	0.0749361653184687\\
6.66	0.0746367539924613\\
6.67	0.0743358643600788\\
6.68	0.0740335023534049\\
6.69	0.0737296739328144\\
6.7	0.0734243850868623\\
6.71	0.073117641832171\\
6.72	0.0728094502133186\\
6.73	0.0724998163027258\\
6.74	0.0721887462005427\\
6.75	0.0718762460345348\\
6.76	0.071562321959969\\
6.77	0.0712469801594991\\
6.78	0.0709302268430506\\
6.79	0.0706120682477051\\
6.8	0.0702925106375845\\
6.81	0.0699715603037347\\
6.82	0.0696492235640086\\
6.83	0.069325506762949\\
6.84	0.0690004162716709\\
6.85	0.0686739584877432\\
6.86	0.0683461398350707\\
6.87	0.0680169667637743\\
6.88	0.0676864457500727\\
6.89	0.0673545832961615\\
6.9	0.0670213859300938\\
6.91	0.066686860205659\\
6.92	0.0663510127022621\\
6.93	0.066013850024802\\
6.94	0.0656753788035498\\
6.95	0.065335605694026\\
6.96	0.0649945373768785\\
6.97	0.0646521805577588\\
6.98	0.0643085419671987\\
6.99	0.0639636283604865\\
7	0.0636174465175422\\
7.01	0.0632700032427931\\
7.02	0.0629213053650485\\
7.03	0.0625713597373739\\
7.04	0.0622201732369653\\
7.05	0.0618677527650227\\
7.06	0.061514105246623\\
7.07	0.0611592376305934\\
7.08	0.0608031568893832\\
7.09	0.0604458700189363\\
7.1	0.0600873840385623\\
7.11	0.0597277059908083\\
7.12	0.0593668429413293\\
7.13	0.0590048019787586\\
7.14	0.0586415902145784\\
7.15	0.0582772147829887\\
7.16	0.0579116828407772\\
7.17	0.0575450015671878\\
7.18	0.0571771781637894\\
7.19	0.0568082198543437\\
7.2	0.0564381338846734\\
7.21	0.0560669275225291\\
7.22	0.0556946080574564\\
7.23	0.055321182800663\\
7.24	0.0549466590848842\\
7.25	0.0545710442642492\\
7.26	0.0541943457141467\\
7.27	0.0538165708310897\\
7.28	0.0534377270325805\\
7.29	0.0530578217569751\\
7.3	0.0526768624633468\\
7.31	0.0522948566313507\\
7.32	0.0519118117610864\\
7.33	0.0515277353729608\\
7.34	0.0511426350075513\\
7.35	0.0507565182254677\\
7.36	0.0503693926072138\\
7.37	0.0499812657530496\\
7.38	0.0495921452828518\\
7.39	0.0492020388359751\\
7.4	0.0488109540711124\\
7.41	0.048418898666155\\
7.42	0.0480258803180525\\
7.43	0.0476319067426721\\
7.44	0.0472369856746577\\
7.45	0.0468411248672889\\
7.46	0.0464443320923391\\
7.47	0.0460466151399337\\
7.48	0.0456479818184081\\
7.49	0.0452484399541647\\
7.5	0.0448479973915304\\
7.51	0.0444466619926131\\
7.52	0.0440444416371582\\
7.53	0.0436413442224049\\
7.54	0.0432373776629418\\
7.55	0.0428325498905625\\
7.56	0.0424268688541207\\
7.57	0.0420203425193851\\
7.58	0.041612978868894\\
7.59	0.0412047859018096\\
7.6	0.0407957716337715\\
7.61	0.0403859440967509\\
7.62	0.0399753113389039\\
7.63	0.0395638814244239\\
7.64	0.0391516624333953\\
7.65	0.0387386624616451\\
7.66	0.0383248896205957\\
7.67	0.0379103520371166\\
7.68	0.0374950578533759\\
7.69	0.0370790152266917\\
7.7	0.0366622323293835\\
7.71	0.0362447173486226\\
7.72	0.0358264784862829\\
7.73	0.0354075239587912\\
7.74	0.0349878619969774\\
7.75	0.0345675008459242\\
7.76	0.0341464487648166\\
7.77	0.0337247140267916\\
7.78	0.0333023049187872\\
7.79	0.0328792297413915\\
7.8	0.0324554968086912\\
7.81	0.0320311144481208\\
7.82	0.0316060910003104\\
7.83	0.0311804348189344\\
7.84	0.0307541542705594\\
7.85	0.0303272577344924\\
7.86	0.0298997536026287\\
7.87	0.0294716502792996\\
7.88	0.0290429561811205\\
7.89	0.028613679736838\\
7.9	0.0281838293871784\\
7.91	0.0277534135846949\\
7.92	0.0273224407936155\\
7.93	0.026890919489691\\
7.94	0.0264588581600432\\
7.95	0.0260262653030127\\
7.96	0.0255931494280076\\
7.97	0.0251595190553523\\
7.98	0.0247253827161369\\
7.99	0.0242907489520663\\
8	0.0238556263153108\\
8.01	0.0234200233683572\\
8.02	0.02298394868386\\
8.03	0.0225474108444943\\
8.04	0.0221104184428096\\
8.05	0.021672980081085\\
8.06	0.0212351043711858\\
8.07	0.0207967999344219\\
8.08	0.0203580754014089\\
8.09	0.0199189394119307\\
8.1	0.019479400614806\\
8.11	0.0190394676677584\\
8.12	0.0185991492372893\\
8.13	0.0181584539985574\\
8.14	0.0177173906352631\\
8.15	0.0172759678395396\\
8.16	0.0168341943118526\\
8.17	0.0163920787609101\\
8.18	0.0159496299035832\\
8.19	0.0155068564648416\\
8.2	0.0150637671777065\\
8.21	0.014620370783224\\
8.22	0.0141766760304642\\
8.23	0.0137326916765521\\
8.24	0.0132884264867375\\
8.25	0.0128438892345125\\
8.26	0.0123990887017912\\
8.27	0.0119540336791655\\
8.28	0.0115087329662585\\
8.29	0.0110631953722039\\
8.3	0.0106174297162878\\
8.31	0.0101714448288037\\
8.32	0.00972524955218872\\
8.33	0.00927885274253629\\
8.34	0.0088322632716201\\
8.35	0.00838549002961892\\
8.36	0.00793854192882033\\
8.37	0.00749142790871313\\
8.38	0.00704415694308664\\
8.39	0.00659673805009175\\
8.4	0.00614918030677493\\
8.41	0.00570149287054697\\
8.42	0.00525368501172904\\
8.43	0.00480576616441015\\
8.44	0.00435774600880449\\
8.45	0.00390963461039799\\
8.46	0.00346144266738102\\
8.47	0.00301318197917336\\
8.48	0.00256486640661855\\
8.49	0.00211651405300073\\
8.5	0.00166815296278696\\
8.51	0.00121983938701404\\
8.52	0.000771739688253916\\
8.53	0.000324889965085925\\
8.54	0.000134117379393911\\
8.55	0.000576990098575144\\
8.56	0.00102486361973658\\
8.57	0.00147317373903725\\
8.58	0.00192159876254749\\
8.59	0.00237005366177664\\
8.6	0.00281850186994556\\
8.61	0.00326692202484926\\
8.62	0.00371529879307083\\
8.63	0.00416361962344644\\
8.64	0.00461187339458308\\
8.65	0.00506004978157913\\
8.66	0.00550813893213873\\
8.67	0.00595613128905153\\
8.68	0.00640401748731161\\
8.69	0.00685178829169167\\
8.7	0.00729943455738187\\
8.71	0.00774694720435227\\
8.72	0.00819431720018473\\
8.73	0.0086415355482998\\
8.74	0.00908859327971689\\
8.75	0.00953548144718566\\
8.76	0.00998219112094368\\
8.77	0.0104287133856116\\
8.78	0.0108750393378976\\
8.79	0.0113211600848881\\
8.8	0.0117670667427667\\
8.81	0.0122127504358536\\
8.82	0.0126582022958848\\
8.83	0.0131034134614746\\
8.84	0.013548375077718\\
8.85	0.0139930782959029\\
8.86	0.0144375142733079\\
8.87	0.0148816741730668\\
8.88	0.0153255491640878\\
8.89	0.0157691304210155\\
8.9	0.0162124091242269\\
8.91	0.0166553764598572\\
8.92	0.0170980236198471\\
8.93	0.0175403418020103\\
8.94	0.0179823222101158\\
8.95	0.0184239560539839\\
8.96	0.0188652345495926\\
8.97	0.0193061489191935\\
8.98	0.019746690391435\\
8.99	0.020186850201492\\
9	0.0206266195912017\\
};
\end{axis}

\begin{axis}[%
width=0.4\figW,
height=0.15\figH,
at={(0\figW,0\figH)},
scale only axis,
xmin=0,
xmax=10,
xlabel style={font=\color{white!15!black}},
xlabel={$\text{x (}\mu\text{m)}$},
ymin=0,
ymax=0.02,
axis background/.style={fill=white},
legend pos=north west
]
\addplot [color=mycolor1]
  table[row sep=crcr]{%
0	1.69324677018983e-06\\
0.025062656641604	7.83006308170158e-07\\
0.050125313283208	1.12365018141638e-06\\
0.075187969924812	3.02905483621564e-06\\
0.100250626566416	7.05237662246314e-06\\
0.12531328320802	1.4360522426093e-05\\
0.150375939849624	2.61070527946276e-05\\
0.175438596491228	4.26571757301067e-05\\
0.200501253132832	6.31831356328446e-05\\
0.225563909774436	8.59537589895219e-05\\
0.25062656641604	0.000108104953997048\\
0.275689223057644	0.000127445061601481\\
0.300751879699248	0.000142953753370531\\
0.325814536340852	0.000154146418300825\\
0.350877192982456	0.000161405511249979\\
0.37593984962406	0.000165567505541847\\
0.401002506265664	0.000167534952306592\\
0.426065162907268	0.000168088503074105\\
0.451127819548872	0.000167804445166442\\
0.476190476190476	0.000167042673688775\\
0.50125313283208	0.000165975466798574\\
0.526315789473684	0.000164645766016166\\
0.551378446115288	0.00016305742552918\\
0.576441102756892	0.000161213958123539\\
0.601503759398496	0.000159119434325781\\
0.6265664160401	0.000156778491210179\\
0.651629072681704	0.000154196342821001\\
0.676691729323308	0.000151378792480607\\
0.701754385964912	0.000148332247304529\\
0.726817042606516	0.000145063735307039\\
0.75187969924812	0.000147204786922924\\
0.776942355889724	0.000153692217964686\\
0.802005012531328	0.000160072692703032\\
0.827067669172932	0.000166317297753885\\
0.852130325814536	0.000172400369539602\\
0.87719298245614	0.000178298998403369\\
0.902255639097744	0.000183992647659175\\
0.927318295739348	0.000189462867109543\\
0.952380952380952	0.000194693085672417\\
0.977443609022556	0.000199668472315678\\
1.00250626566416	0.00020437585850662\\
1.02756892230576	0.000208803718955063\\
1.05263157894737	0.000212942210713803\\
1.07769423558897	0.000216783273865693\\
1.10275689223058	0.00022032080025291\\
1.12781954887218	0.000223550880196441\\
1.15288220551378	0.000226472141168674\\
1.17794486215539	0.000229086197263934\\
1.20300751879699	0.000231398234553532\\
1.2280701754386	0.000233417869037855\\
1.2531328320802	0.000235159952665523\\
1.2781954887218	0.000236645713204893\\
1.30325814536341	0.000237904647161437\\
1.32832080200501	0.000238976487421662\\
1.35338345864662	0.000239913957639836\\
1.37844611528822	0.000240786468775938\\
1.40350877192982	0.000241685209102647\\
1.42857142857143	0.000242730419018424\\
1.45363408521303	0.000244082385965603\\
1.47869674185464	0.000245959554423062\\
1.50375939849624	0.000248672560363538\\
1.52882205513784	0.000252591377558959\\
1.55388471177945	0.000258632752267467\\
1.57894736842105	0.000267699524679319\\
1.60401002506266	0.000280274446679959\\
1.62907268170426	0.00029609870606887\\
1.65413533834586	0.000314069059493345\\
1.67919799498747	0.000332371026451301\\
1.70426065162907	0.000348746590765609\\
1.72932330827068	0.000360783202799192\\
1.75438596491228	0.000366172910723485\\
1.77944862155388	0.000362919304487359\\
1.80451127819549	0.000349512684458185\\
1.82957393483709	0.000325092813586239\\
1.8546365914787	0.000289636195579258\\
1.8796992481203	0.000244268550073367\\
1.9047619047619	0.000248837324861112\\
1.92982456140351	0.000301873863983856\\
1.95488721804511	0.000362571595575014\\
1.97994987468672	0.000427370982845299\\
2.00501253132832	0.000496677632424688\\
2.03007518796993	0.000566789110646308\\
2.05513784461153	0.000635460137926526\\
2.08020050125313	0.000700463056000303\\
2.10526315789474	0.000756170838138082\\
2.13032581453634	0.000802268997995463\\
2.15538847117795	0.000835546389456076\\
2.18045112781955	0.000853468692739431\\
2.20551378446115	0.000856083391213991\\
2.23057644110276	0.000844943203704245\\
2.25563909774436	0.000823489213148614\\
2.28070175438596	0.000797422792392981\\
2.30576441102757	0.000774823897037444\\
2.33082706766917	0.000766825866251952\\
2.35588972431078	0.00078138373097761\\
2.38095238095238	0.000822347756898894\\
2.40601503759398	0.000889245297396021\\
2.43107769423559	0.000979089582473029\\
2.45614035087719	0.0010829246261935\\
2.4812030075188	0.00119115215140442\\
2.5062656641604	0.0012957263317635\\
2.53132832080201	0.00138993034149999\\
2.55639097744361	0.00146853433246565\\
2.58145363408521	0.00152787634310518\\
2.60651629072682	0.00156595077415512\\
2.63157894736842	0.0015825185405967\\
2.65664160401003	0.00157922569773802\\
2.68170426065163	0.0015596984274492\\
2.70676691729323	0.00152955036100879\\
2.73182957393484	0.0014961816304572\\
2.75689223057644	0.00163855191069039\\
2.78195488721805	0.00181282306066498\\
2.80701754385965	0.00199963712031482\\
2.83208020050125	0.00218610054864656\\
2.85714285714286	0.00236066904535606\\
2.88220551378446	0.00251386543356965\\
2.90726817042607	0.00263874330241761\\
2.93233082706767	0.00273129113942161\\
2.95739348370927	0.00279083916578356\\
2.98245614035088	0.00282045281195445\\
3.00751879699248	0.00282722543391731\\
3.03258145363409	0.00282227972257423\\
3.05764411027569	0.00282014883718352\\
3.08270676691729	0.00283712730214789\\
3.1077694235589	0.00288840288862833\\
3.1328320802005	0.00298449914199557\\
3.15789473684211	0.00312846657951805\\
3.18295739348371	0.00331522698861743\\
3.20802005012531	0.00353323935799867\\
3.23308270676692	0.00376735640876209\\
3.25814536340852	0.00400158067469217\\
3.28320802005013	0.00422110159609837\\
3.30827067669173	0.00441361139798279\\
3.33333333333333	0.00457012699485914\\
3.35839598997494	0.00468552394388536\\
3.38345864661654	0.00475889465228041\\
3.40852130325815	0.00479375283633128\\
3.43358395989975	0.00479802784358356\\
3.45864661654135	0.00478371699359597\\
3.48370927318296	0.00476599946214946\\
3.50877192982456	0.0047616121171986\\
3.53383458646617	0.00478644237735395\\
3.55889724310777	0.00485266992318423\\
3.58395989974937	0.00496624954204912\\
3.60902255639098	0.0051256765513111\\
3.63408521303258	0.00532251680931446\\
3.65914786967419	0.00554338069189643\\
3.68421052631579	0.00577250529143313\\
3.70927318295739	0.00599417001440972\\
3.734335839599	0.00619455394239464\\
3.7593984962406	0.00636298494283514\\
3.78446115288221	0.00649268795393189\\
3.80952380952381	0.00658115477841503\\
3.83458646616541	0.00663021058704923\\
3.85964912280702	0.00664579389553124\\
3.88471177944862	0.00663741890197664\\
3.90977443609023	0.00661726468203446\\
3.93483709273183	0.00659885286063309\\
3.95989974937343	0.00659535611341496\\
3.98496240601504	0.00661772831805069\\
4.01002506265664	0.00667301426537746\\
4.03508771929825	0.00676327275040647\\
4.06015037593985	0.00688543011212696\\
4.08521303258145	0.00703209060588313\\
4.11027568922306	0.00719302505251767\\
4.13533834586466	0.00735691031640194\\
4.16040100250627	0.00751293906809888\\
4.18546365914787	0.00765207356281388\\
4.21052631578947	0.0077678686692924\\
4.23558897243108	0.00785688287478403\\
4.26065162907268	0.00791873218374192\\
4.28571428571429	0.00795584562172379\\
4.31077694235589	0.00797297082473897\\
4.33583959899749	0.00797648663607336\\
4.3609022556391	0.00797357712719081\\
4.3859649122807	0.0079713509647754\\
4.41102756892231	0.00797601010884302\\
4.43609022556391	0.00799218332271202\\
4.46115288220551	0.0080225259128181\\
4.48621553884712	0.00806764148468119\\
4.51127819548872	0.00812631406048355\\
4.53634085213033	0.00819597239719261\\
4.56140350877193	0.00827326660661366\\
4.58646616541353	0.00835463201458205\\
4.61152882205514	0.00843674209619337\\
4.63659147869674	0.00851679647695978\\
4.66165413533835	0.00859263577511881\\
4.68671679197995	0.00866271200719602\\
4.71177944862155	0.00872596680138789\\
4.73684210526316	0.00878167901282222\\
4.76190476190476	0.00882933960000973\\
4.78696741854637	0.00886859692499083\\
4.81203007518797	0.00889929315590474\\
4.83709273182957	0.00892158639652464\\
4.86215538847118	0.00893612827583229\\
4.88721804511278	0.00894424738016292\\
4.91228070175439	0.00894807844792457\\
4.93734335839599	0.00895057788708223\\
4.96240601503759	0.00895537929829104\\
4.9874686716792	0.00896646883136677\\
5.0125313283208	0.00898769802946316\\
5.03759398496241	0.00902219607289444\\
5.06265664160401	0.00907178333213277\\
5.08771929824561	0.00913650929674543\\
5.11278195488722	0.00921442789279122\\
5.13784461152882	0.00930167992569156\\
5.16290726817043	0.0093928881717052\\
5.18796992481203	0.00948180703714108\\
5.21303258145363	0.00956212611093943\\
5.23809523809524	0.00962830496003039\\
5.26315789473684	0.00967633888449036\\
5.28822055137845	0.00970435466941043\\
5.31328320802005	0.0097129691445913\\
5.33834586466165	0.00970535716299491\\
5.36340852130326	0.00968699296792002\\
5.38847117794486	0.00966505350317413\\
5.41353383458647	0.00964751214978005\\
5.43859649122807	0.00964201076821881\\
5.46365914786967	0.00965466837607673\\
5.48872180451128	0.00968904131638802\\
5.51378446115288	0.00974545802715669\\
5.53884711779449	0.00982088897832213\\
5.56390977443609	0.00990939038299855\\
5.58897243107769	0.0100030249024842\\
5.6140350877193	0.0100930669828265\\
5.6390977443609	0.0101712718861274\\
5.66416040100251	0.0102310162975009\\
5.68922305764411	0.0102681739843429\\
5.71428571428571	0.0102816439976704\\
5.73934837092732	0.0102734881775461\\
5.76441102756892	0.0102486614177601\\
5.78947368421053	0.0102143435744911\\
5.81453634085213	0.010178919310324\\
5.83959899749373	0.0101507091823035\\
5.86466165413534	0.0101366258766464\\
5.88972431077694	0.0101409890748057\\
5.91478696741855	0.0101647442566666\\
5.93984962406015	0.0102052676291452\\
5.96491228070175	0.010256808962147\\
5.98997493734336	0.010311473455848\\
6.01503759398496	0.0103605322617998\\
6.04010025062657	0.0103958130119163\\
6.06516290726817	0.010410949673798\\
6.09022556390977	0.0104023321458903\\
6.11528822055138	0.0103696573642602\\
6.14035087719298	0.0103160291307423\\
6.16541353383459	0.0102475855632564\\
6.19047619047619	0.0101726649862084\\
6.21553884711779	0.0101005700316792\\
6.2406015037594	0.0100400647889645\\
6.265664160401	0.0099978305167284\\
6.29072681704261	0.00997717528852914\\
6.31578947368421	0.00997729166064672\\
6.34085213032582	0.00999325405449161\\
6.36591478696742	0.0100167679747056\\
6.39097744360902	0.0100375002447634\\
6.41604010025063	0.0100447088322759\\
6.44110275689223	0.0100288803242484\\
6.46616541353383	0.00998314502827397\\
6.49122807017544	0.00990432279959841\\
6.51629072681704	0.00979351736684509\\
6.54135338345865	0.00965620931176221\\
6.56641604010025	0.00950180659879635\\
6.59147869674185	0.00934262019156462\\
6.61654135338346	0.00919227298634976\\
6.64160401002506	0.00906365275245585\\
6.66666666666667	0.00896668621921681\\
6.69172932330827	0.0089063844561422\\
6.71679197994987	0.0088816723151019\\
6.74185463659148	0.0088853653368567\\
6.76691729323308	0.0089053215336974\\
6.79197994987469	0.00892644422412913\\
6.81704260651629	0.00893303137681562\\
6.84210526315789	0.00891100801976234\\
6.8671679197995	0.00884975002419401\\
6.8922305764411	0.00874338361810103\\
6.91729323308271	0.00859155191866817\\
6.94235588972431	0.00839966714162226\\
6.96741854636591	0.008178636598422\\
6.99248120300752	0.0079439929431704\\
7.01754385964912	0.00771431527206954\\
7.04260651629073	0.00750886179925514\\
7.06766917293233	0.00734452698556193\\
7.09273182957393	0.00723260882427102\\
7.11779448621554	0.00717626378217067\\
7.14285714285714	0.00716958474473691\\
7.16791979949875	0.00719872477960871\\
7.19298245614035	0.00724466034648073\\
7.21804511278195	0.00728663161623047\\
7.24310776942356	0.00730533163902139\\
7.26817042606516	0.00728533434612734\\
7.29323308270677	0.00721667103872788\\
7.31829573934837	0.00709568743476831\\
7.34335839598998	0.00692534913921565\\
7.36842105263158	0.00671509310291068\\
7.39348370927318	0.00648020560844227\\
7.41854636591479	0.00624057668956966\\
7.44360902255639	0.00601858717473587\\
7.468671679198	0.00583594827627974\\
7.4937343358396	0.00570970695846209\\
7.5187969924812	0.00564835857986178\\
7.54385964912281	0.00564958481685647\\
7.56892230576441	0.00570077973207682\\
7.59398496240602	0.00578218446946916\\
7.61904761904762	0.00587122056253999\\
7.64411027568922	0.0059464816131919\\
7.66917293233083	0.0059905740138605\\
7.69423558897243	0.00599175244064647\\
7.71929824561404	0.00594464578184294\\
7.74436090225564	0.00585038265051223\\
7.76942355889724	0.00571629778178804\\
7.79448621553885	0.00555524741222561\\
7.81954887218045	0.00538442354503693\\
7.84461152882206	0.00522346859986605\\
7.86967418546366	0.00509174228216274\\
7.89473684210526	0.00500491276174639\\
7.91979949874687	0.00497163415652252\\
7.94486215538847	0.00499154532440796\\
7.96992481203008	0.00505556017649242\\
7.99498746867168	0.00514835149155434\\
8.02005012531328	0.00525190921762026\\
8.04511278195489	0.00534889184976398\\
8.07017543859649	0.00542504698873372\\
8.09523809523809	0.00547059810334316\\
8.1203007518797	0.00548080679388401\\
8.1453634085213	0.00545595507924734\\
8.17042606516291	0.00540090734057558\\
8.19548872180451	0.00532431425527959\\
8.22055137844612	0.00523745848599988\\
8.24561403508772	0.0051527385846405\\
8.27067669172932	0.0050818644325394\\
8.29573934837093	0.00503399321763697\\
8.32080200501253	0.00501420343345799\\
8.34586466165413	0.00502274920940546\\
8.37092731829574	0.00505534954511978\\
8.39598997493734	0.00510440715534228\\
8.42105263157895	0.00516074088536827\\
8.44611528822055	0.00521532873620107\\
8.47117794486216	0.00526070869527161\\
8.49624060150376	0.00529186646862084\\
8.52130325814536	0.0053066121096783\\
8.54636591478697	0.00530551308544992\\
8.57142857142857	0.00529147028898328\\
8.59649122807017	0.00526902371654791\\
8.62155388471178	0.00524348214994218\\
8.64661654135338	0.00521999314266582\\
8.67167919799499	0.00520269579859344\\
8.69674185463659	0.00519410690752319\\
8.7218045112782	0.00519485876487184\\
8.7468671679198	0.00520383019298055\\
8.7719298245614	0.00521861414656225\\
8.79699248120301	0.00523618515858388\\
8.82205513784461	0.00525359748034991\\
8.84711779448622	0.00526856306082817\\
8.87218045112782	0.00527980858123726\\
8.89724310776942	0.00528716943415074\\
8.92230576441103	0.00529143143848813\\
8.94736842105263	0.00529397391192967\\
8.97243107769424	0.0052962999625467\\
8.99749373433584	0.00529955806649012\\
9.02255639097744	0.00530415353884656\\
9.04761904761905	0.00530881347221947\\
9.07268170426065	0.00531342513614319\\
9.09774436090226	0.00531799712973293\\
9.12280701754386	0.00532259153340332\\
9.14786967418546	0.00532729011830727\\
9.17293233082707	0.00533217601215579\\
9.19799498746867	0.00533732356195577\\
9.22305764411028	0.0053427930614441\\
9.24812030075188	0.00534862852362081\\
9.27318295739348	0.00535485736178048\\
9.29824561403509	0.00536149121610015\\
9.32330827067669	0.00536852740400189\\
9.3483709273183	0.00537595064385388\\
9.3734335839599	0.00538373482737312\\
9.3984962406015	0.00539184470781103\\
9.42355889724311	0.00540023743588721\\
9.44862155388471	0.00540886391919341\\
9.47368421052632	0.0054176700083002\\
9.49874686716792	0.00542659752823289\\
9.52380952380952	0.00543558518081773\\
9.54887218045113	0.00544453487156535\\
9.57393483709273	0.00545182185142792\\
9.59899749373434	0.00545085003015344\\
9.62406015037594	0.0054278195843834\\
9.64912280701754	0.00536055474899008\\
9.67418546365915	0.00521886451732697\\
9.69924812030075	0.00496770698255355\\
9.72431077694236	0.00457470060693906\\
9.74937343358396	0.00402264279849423\\
9.77443609022556	0.00332477855953943\\
9.79949874686717	0.00253575635836115\\
9.82456140350877	0.00174757256015364\\
9.84962406015038	0.00106352873472281\\
9.87468671679198	0.000558000595922561\\
9.89974937343358	0.000246676857539484\\
9.92481203007519	9.019984822446e-05\\
9.94987468671679	2.70510675216729e-05\\
9.9749373433584	8.75512075090573e-06\\
10	4.34210791258366e-06\\
};
\addlegendentry{FDFD}

\addplot [color=mycolor2]
  table[row sep=crcr]{%
1.5	0\\
1.51	0\\
1.52	2.0236073325445e-07\\
1.53	6.07071525252697e-07\\
1.54	1.21411090490496e-06\\
1.55	2.02344660569066e-06\\
1.56	3.03503556738774e-06\\
1.57	4.248823938383e-06\\
1.58	5.66474707856319e-06\\
1.59	7.28272956278682e-06\\
1.6	9.10268518493641e-06\\
1.61	1.11245169625511e-05\\
1.62	1.33481171420391e-05\\
1.63	1.57733672044702e-05\\
1.64	1.84001378719475e-05\\
1.65	2.1228289114558e-05\\
1.66	2.42576701579026e-05\\
1.67	2.74881194912041e-05\\
1.68	3.09194648759926e-05\\
1.69	3.45515233553691e-05\\
1.7	3.83841012638461e-05\\
1.71	4.24169942377637e-05\\
1.72	4.66499872262825e-05\\
1.73	5.10828545029516e-05\\
1.74	5.5715359677851e-05\\
1.75	6.05472557103083e-05\\
1.76	6.55782849221887e-05\\
1.77	7.08081790117577e-05\\
1.78	7.62366590681156e-05\\
1.79	8.18634355862033e-05\\
1.8	8.76882084823786e-05\\
1.81	9.37106671105614e-05\\
1.82	9.99304902789486e-05\\
1.83	0.000106347346267296\\
1.84	0.000112960892844767\\
1.85	0.000119770777288349\\
1.86	0.000126776636401832\\
1.87	0.000133978096535351\\
1.88	0.000141374773605496\\
1.89	0.000148966273115981\\
1.9	0.000156752190178869\\
1.91	0.000164732109536366\\
1.92	0.000172905605583163\\
1.93	0.000181272242389349\\
1.94	0.000189831573723864\\
1.95	0.000198583143078519\\
1.96	0.000207526483692563\\
1.97	0.000216661118577808\\
1.98	0.000225986560544299\\
1.99	0.000235502312226536\\
2	0.000245207866110245\\
2.01	0.000255102704559696\\
2.02	0.000265186299845563\\
2.03	0.000275458114173334\\
2.04	0.000285917599712256\\
2.05	0.000296564198624827\\
2.06	0.000307397343096823\\
2.07	0.000318416455367865\\
2.08	0.000329620947762523\\
2.09	0.000341010222721951\\
2.1	0.000352583672836057\\
2.11	0.000364340680876203\\
2.12	0.000376280619828438\\
2.13	0.000388402852927251\\
2.14	0.000400706733689861\\
2.15	0.000413191605951015\\
2.16	0.000425856803898326\\
2.17	0.000438701652108118\\
2.18	0.000451725465581797\\
2.19	0.000464927549782732\\
2.2	0.000478307200673657\\
2.21	0.000491863704754582\\
2.22	0.000505596339101214\\
2.23	0.000519504371403889\\
2.24	0.000533587060007008\\
2.25	0.00054784365394898\\
2.26	0.000562273393002664\\
2.27	0.000576875507716314\\
2.28	0.000591649219455021\\
2.29	0.00060659374044265\\
2.3	0.000621708273804273\\
2.31	0.000636992013609088\\
2.32	0.000652444144913836\\
2.33	0.000668063843806696\\
2.34	0.000683850277451668\\
2.35	0.000699802604133443\\
2.36	0.000715919973302741\\
2.37	0.00073220152562214\\
2.38	0.00074864639301237\\
2.39	0.000765253698699086\\
2.4	0.000782022557260106\\
2.41	0.00079895207467312\\
2.42	0.000816041348363865\\
2.43	0.000833289467254759\\
2.44	0.000850695511813998\\
2.45	0.000868258554105109\\
2.46	0.00088597765783696\\
2.47	0.000903851878414219\\
2.48	0.000921880262988266\\
2.49	0.000940061850508551\\
2.5	0.000958395671774394\\
2.51	0.00097688074948723\\
2.52	0.000995516098303291\\
2.53	0.00101430072488672\\
2.54	0.00103323362796315\\
2.55	0.00105231379837362\\
2.56	0.00107154021912907\\
2.57	0.00109091186546513\\
2.58	0.00111042770489735\\
2.59	0.00113008669727695\\
2.6	0.00114988779484684\\
2.61	0.00116982994229818\\
2.62	0.00118991207682725\\
2.63	0.00121013312819283\\
2.64	0.00123049201877385\\
2.65	0.00125098766362761\\
2.66	0.00127161897054823\\
2.67	0.00129238484012561\\
2.68	0.00131328416580477\\
2.69	0.00133431583394548\\
2.7	0.00135547872388244\\
2.71	0.00137677170798569\\
2.72	0.00139819365172147\\
2.73	0.00141974341371351\\
2.74	0.00144141984580454\\
2.75	0.00146322179311834\\
2.76	0.00148514809412206\\
2.77	0.0015071975806889\\
2.78	0.00152936907816122\\
2.79	0.00155166140541393\\
2.8	0.00157407337491827\\
2.81	0.00159660379280596\\
2.82	0.00161925145893365\\
2.83	0.00164201516694776\\
2.84	0.00166489370434965\\
2.85	0.00168788585256107\\
2.86	0.00171099038699004\\
2.87	0.00173420607709701\\
2.88	0.00175753168646133\\
2.89	0.00178096597284807\\
2.9	0.00180450768827516\\
2.91	0.00182815557908084\\
2.92	0.00185190838599143\\
2.93	0.00187576484418942\\
2.94	0.0018997236833818\\
2.95	0.0019237836278688\\
2.96	0.00194794339661286\\
2.97	0.00197220170330789\\
2.98	0.00199655725644888\\
2.99	0.00202100875940171\\
3	0.00204555491047338\\
3.01	0.00207019440298233\\
3.02	0.00209492592532926\\
3.03	0.00211974816106803\\
3.04	0.00214465978897694\\
3.05	0.00216965948313026\\
3.06	0.00219474591297003\\
3.07	0.00221991774337803\\
3.08	0.00224517363474819\\
3.09	0.00227051224305904\\
3.1	0.00229593221994658\\
3.11	0.00232143221277731\\
3.12	0.00234701086472149\\
3.13	0.00237266681482672\\
3.14	0.00239839869809165\\
3.15	0.00242420514554002\\
3.16	0.00245008478429482\\
3.17	0.0024760362376528\\
3.18	0.00250205812515908\\
3.19	0.00252814906268203\\
3.2	0.0025543076624884\\
3.21	0.00258053253331855\\
3.22	0.00260682228046201\\
3.23	0.00263317550583314\\
3.24	0.00265959080804707\\
3.25	0.00268606678249572\\
3.26	0.00271260202142415\\
3.27	0.002739195114007\\
3.28	0.00276584464642509\\
3.29	0.00279254920194231\\
3.3	0.00281930736098259\\
3.31	0.00284611770120703\\
3.32	0.0028729787975913\\
3.33	0.00289988922250307\\
3.34	0.00292684754577969\\
3.35	0.002953852334806\\
3.36	0.00298090215459228\\
3.37	0.00300799556785234\\
3.38	0.0030351311350818\\
3.39	0.00306230741463644\\
3.4	0.00308952296281071\\
3.41	0.00311677633391645\\
3.42	0.00314406608036159\\
3.43	0.00317139075272907\\
3.44	0.00319874889985591\\
3.45	0.00322613906891229\\
3.46	0.0032535598054808\\
3.47	0.00328100965363584\\
3.48	0.00330848715602305\\
3.49	0.00333599085393886\\
3.5	0.00336351928741016\\
3.51	0.00339107099527407\\
3.52	0.00341864451525772\\
3.53	0.00344623838405822\\
3.54	0.00347385113742266\\
3.55	0.00350148131022816\\
3.56	0.00352912743656205\\
3.57	0.00355678804980205\\
3.58	0.00358446168269662\\
3.59	0.00361214686744525\\
3.6	0.00363984213577887\\
3.61	0.00366754601904035\\
3.62	0.00369525704826495\\
3.63	0.00372297375426094\\
3.64	0.00375069466769011\\
3.65	0.00377841831914847\\
3.66	0.00380614323924691\\
3.67	0.00383386795869186\\
3.68	0.00386159100836604\\
3.69	0.00388931091940922\\
3.7	0.00391702622329894\\
3.71	0.00394473545193135\\
3.72	0.00397243713770195\\
3.73	0.0040001298135864\\
3.74	0.00402781201322136\\
3.75	0.00405548227098524\\
3.76	0.00408313912207903\\
3.77	0.00411078110260708\\
3.78	0.00413840674965787\\
3.79	0.00416601460138481\\
3.8	0.00419360319708692\\
3.81	0.00422117107728961\\
3.82	0.00424871678382536\\
3.83	0.00427623885991437\\
3.84	0.00430373585024517\\
3.85	0.00433120630105529\\
3.86	0.00435864876021171\\
3.87	0.00438606177729146\\
3.88	0.00441344390366199\\
3.89	0.00444079369256165\\
3.9	0.00446810969917999\\
3.91	0.00449539048073805\\
3.92	0.00452263459656862\\
3.93	0.00454984060819636\\
3.94	0.0045770070794179\\
3.95	0.00460413257638184\\
3.96	0.00463121566766872\\
3.97	0.00465825492437083\\
3.98	0.00468524892017201\\
3.99	0.0047121962314273\\
4	0.00473909543724254\\
4.01	0.00476594511955388\\
4.02	0.00479274386320712\\
4.03	0.00481949025603706\\
4.04	0.00484618288894665\\
4.05	0.00487282035598605\\
4.06	0.00489940125443162\\
4.07	0.00492592418486476\\
4.08	0.00495238775125059\\
4.09	0.00497879056101662\\
4.1	0.00500513122513116\\
4.11	0.0050314083581817\\
4.12	0.0050576205784531\\
4.13	0.00508376650800569\\
4.14	0.00510984477275316\\
4.15	0.0051358540025404\\
4.16	0.00516179283122109\\
4.17	0.00518765989673524\\
4.18	0.0052134538411865\\
4.19	0.00523917331091934\\
4.2	0.00526481695659612\\
4.21	0.00529038343327389\\
4.22	0.00531587140048112\\
4.23	0.00534127952229423\\
4.24	0.00536660646741391\\
4.25	0.00539185090924136\\
4.26	0.00541701152595422\\
4.27	0.00544208700058244\\
4.28	0.0054670760210839\\
4.29	0.00549197728041987\\
4.3	0.00551678947663023\\
4.31	0.00554151131290859\\
4.32	0.0055661414976771\\
4.33	0.00559067874466117\\
4.34	0.00561512177296393\\
4.35	0.00563946930714045\\
4.36	0.00566372007727184\\
4.37	0.0056878728190391\\
4.38	0.00571192627379672\\
4.39	0.0057358791886461\\
4.4	0.00575973031650878\\
4.41	0.00578347841619938\\
4.42	0.00580712225249838\\
4.43	0.00583066059622463\\
4.44	0.00585409222430766\\
4.45	0.00587741591985972\\
4.46	0.00590063047224764\\
4.47	0.00592373467716441\\
4.48	0.00594672733670054\\
4.49	0.00596960725941517\\
4.5	0.00599237326040695\\
4.51	0.00601502416138463\\
4.52	0.00603755879073747\\
4.53	0.00605997598360536\\
4.54	0.00608227458194866\\
4.55	0.00610445343461784\\
4.56	0.00612651139742283\\
4.57	0.00614844733320209\\
4.58	0.00617026011189149\\
4.59	0.00619194861059285\\
4.6	0.00621351171364221\\
4.61	0.00623494831267792\\
4.62	0.00625625730670839\\
4.63	0.00627743760217953\\
4.64	0.00629848811304202\\
4.65	0.00631940776081822\\
4.66	0.00634019547466884\\
4.67	0.00636085019145931\\
4.68	0.00638137085582588\\
4.69	0.00640175642024142\\
4.7	0.00642200584508099\\
4.71	0.00644211809868701\\
4.72	0.00646209215743427\\
4.73	0.00648192700579456\\
4.74	0.00650162163640104\\
4.75	0.00652117505011233\\
4.76	0.00654058625607625\\
4.77	0.00655985427179334\\
4.78	0.00657897812318003\\
4.79	0.00659795684463151\\
4.8	0.00661678947908432\\
4.81	0.00663547507807865\\
4.82	0.00665401270182028\\
4.83	0.00667240141924226\\
4.84	0.0066906403080663\\
4.85	0.00670872845486382\\
4.86	0.00672666495511667\\
4.87	0.00674444891327762\\
4.88	0.00676207944283043\\
4.89	0.00677955566634973\\
4.9	0.00679687671556051\\
4.91	0.00681404173139728\\
4.92	0.006831049864063\\
4.93	0.00684790027308761\\
4.94	0.00686459212738632\\
4.95	0.00688112460531751\\
4.96	0.00689749689474036\\
4.97	0.00691370819307217\\
4.98	0.00692975770734532\\
4.99	0.00694564465426395\\
5	0.00696136826026034\\
5.01	0.00697692776155086\\
5.02	0.00699232240419177\\
5.03	0.00700755144413455\\
5.04	0.00702261414728102\\
5.05	0.00703750978953806\\
5.06	0.00705223765687203\\
5.07	0.00706679704536295\\
5.08	0.00708118726125822\\
5.09	0.00709540762102613\\
5.1	0.00710945745140902\\
5.11	0.00712333608947611\\
5.12	0.00713704288267599\\
5.13	0.00715057718888885\\
5.14	0.00716393837647836\\
5.15	0.00717712582434321\\
5.16	0.00719013892196835\\
5.17	0.00720297706947593\\
5.18	0.00721563967767593\\
5.19	0.00722812616811639\\
5.2	0.00724043597313344\\
5.21	0.00725256853590098\\
5.22	0.00726452331047994\\
5.23	0.00727629976186743\\
5.24	0.00728789736604536\\
5.25	0.00729931561002894\\
5.26	0.00731055399191473\\
5.27	0.00732161202092843\\
5.28	0.00733248921747242\\
5.29	0.00734318511317288\\
5.3	0.00735369925092669\\
5.31	0.00736403118494802\\
5.32	0.00737418048081457\\
5.33	0.00738414671551354\\
5.34	0.00739392947748731\\
5.35	0.00740352836667881\\
5.36	0.00741294299457658\\
5.37	0.00742217298425958\\
5.38	0.00743121797044164\\
5.39	0.00744007759951567\\
5.4	0.00744875152959758\\
5.41	0.00745723943056991\\
5.42	0.00746554098412511\\
5.43	0.00747365588380865\\
5.44	0.0074815838350618\\
5.45	0.00748932455526405\\
5.46	0.00749687777377542\\
5.47	0.00750424323197834\\
5.48	0.00751142068331935\\
5.49	0.00751840989335052\\
5.5	0.00752521063977056\\
5.51	0.00753182271246571\\
5.52	0.00753824591355036\\
5.53	0.0075444800574074\\
5.54	0.00755052497072832\\
5.55	0.00755638049255308\\
5.56	0.00756204647430969\\
5.57	0.00756752277985359\\
5.58	0.00757280928550672\\
5.59	0.00757790588009644\\
5.6	0.00758281246499415\\
5.61	0.00758752895415366\\
5.62	0.00759205527414942\\
5.63	0.00759639136421443\\
5.64	0.00760053717627797\\
5.65	0.00760449267500309\\
5.66	0.00760825783782392\\
5.67	0.00761183265498275\\
5.68	0.00761521712956685\\
5.69	0.0076184112775452\\
5.7	0.00762141512780488\\
5.71	0.00762422872218739\\
5.72	0.00762685211552467\\
5.73	0.00762928537567502\\
5.74	0.0076315285835588\\
5.75	0.00763358183319387\\
5.76	0.00763544523173103\\
5.77	0.0076371188994891\\
5.78	0.00763860296998997\\
5.79	0.00763989758999338\\
5.8	0.00764100291953163\\
5.81	0.00764191913194407\\
5.82	0.00764264641391148\\
5.83	0.00764318496549026\\
5.84	0.00764353500014652\\
5.85	0.00764369674479001\\
5.86	0.00764367043980791\\
5.87	0.0076434563390985\\
5.88	0.00764305471010473\\
5.89	0.00764246583384759\\
5.9	0.00764169000495949\\
5.91	0.00764072753171738\\
5.92	0.0076395787360759\\
5.93	0.00763824395370034\\
5.94	0.00763672353399951\\
5.95	0.00763501784015859\\
5.96	0.00763312724917179\\
5.97	0.00763105215187499\\
5.98	0.00762879295297828\\
5.99	0.00762635007109842\\
6	0.00762372393879127\\
6.01	0.00762091500258406\\
6.02	0.00761792372300773\\
6.03	0.00761475057462907\\
6.04	0.00761139604608291\\
6.05	0.00760786064010425\\
6.06	0.00760414487356027\\
6.07	0.00760024927748235\\
6.08	0.00759617439709811\\
6.09	0.0075919207918633\\
6.1	0.00758748903549372\\
6.11	0.00758287971599713\\
6.12	0.00757809343570511\\
6.13	0.00757313081130488\\
6.14	0.00756799247387117\\
6.15	0.007562679068898\\
6.16	0.00755719125633048\\
6.17	0.00755152971059664\\
6.18	0.00754569512063918\\
6.19	0.00753968818994726\\
6.2	0.00753350963658829\\
6.21	0.00752716019323971\\
6.22	0.00752064060722076\\
6.23	0.00751395164052428\\
6.24	0.00750709406984849\\
6.25	0.00750006868662876\\
6.26	0.00749287629706943\\
6.27	0.00748551772217562\\
6.28	0.007477993797785\\
6.29	0.00747030537459961\\
6.3	0.00746245331821767\\
6.31	0.00745443850916541\\
6.32	0.00744626184292882\\
6.33	0.00743792422998551\\
6.34	0.00742942659583649\\
6.35	0.00742076988103794\\
6.36	0.007411955041233\\
6.37	0.00740298304718352\\
6.38	0.00739385488480178\\
6.39	0.00738457155518219\\
6.4	0.00737513407463299\\
6.41	0.00736554347470782\\
6.42	0.00735580080223731\\
6.43	0.00734590711936064\\
6.44	0.00733586350355689\\
6.45	0.0073256710476765\\
6.46	0.0073153308599725\\
6.47	0.00730484406413169\\
6.48	0.00729421179930572\\
6.49	0.00728343522014202\\
6.5	0.00727251549681459\\
6.51	0.00726145381505468\\
6.52	0.00725025137618119\\
6.53	0.00723890939713101\\
6.54	0.00722742911048907\\
6.55	0.00721581176451815\\
6.56	0.00720405862318849\\
6.57	0.00719217096620714\\
6.58	0.00718015008904687\\
6.59	0.007167997302975\\
6.6	0.00715571393508162\\
6.61	0.00714330132830763\\
6.62	0.00713076084147226\\
6.63	0.00711809384930021\\
6.64	0.00710530174244825\\
6.65	0.00709238592753142\\
6.66	0.00707934782714854\\
6.67	0.00706618887990729\\
6.68	0.00705291054044855\\
6.69	0.00703951427947009\\
6.7	0.00702600158374962\\
6.71	0.00701237395616701\\
6.72	0.00699863291572572\\
6.73	0.00698477999757336\\
6.74	0.00697081675302138\\
6.75	0.00695674474956373\\
6.76	0.00694256557089452\\
6.77	0.0069282808169246\\
6.78	0.00691389210379697\\
6.79	0.00689940106390095\\
6.8	0.00688480934588511\\
6.81	0.00687011861466875\\
6.82	0.00685533055145206\\
6.83	0.00684044685372459\\
6.84	0.00682546923527232\\
6.85	0.00681039942618286\\
6.86	0.00679523917284898\\
6.87	0.00677999023797022\\
6.88	0.00676465440055255\\
6.89	0.00674923345590597\\
6.9	0.00673372921563994\\
6.91	0.00671814350765646\\
6.92	0.00670247817614091\\
6.93	0.00668673508155019\\
6.94	0.00667091610059844\\
6.95	0.0066550231262398\\
6.96	0.00663905806764848\\
6.97	0.00662302285019566\\
6.98	0.00660691941542337\\
6.99	0.00659074972101499\\
7	0.00657451574076233\\
7.01	0.00655821946452911\\
7.02	0.00654186289821078\\
7.03	0.00652544806369025\\
7.04	0.00650897699878976\\
7.05	0.00649245175721829\\
7.06	0.00647587440851471\\
7.07	0.00645924703798626\\
7.08	0.00644257174664218\\
7.09	0.00642585065112245\\
7.1	0.00640908588362129\\
7.11	0.00639227959180525\\
7.12	0.00637543393872578\\
7.13	0.00635855110272586\\
7.14	0.00634163327734076\\
7.15	0.00632468267119238\\
7.16	0.00630770150787721\\
7.17	0.00629069202584752\\
7.18	0.00627365647828556\\
7.19	0.00625659713297063\\
7.2	0.00623951627213857\\
7.21	0.00622241619233361\\
7.22	0.00620529920425221\\
7.23	0.00618816763257862\\
7.24	0.00617102381581191\\
7.25	0.00615387010608429\\
7.26	0.00613670886897018\\
7.27	0.00611954248328607\\
7.28	0.00610237334088058\\
7.29	0.00608520384641467\\
7.3	0.00606803641713153\\
7.31	0.00605087348261592\\
7.32	0.00603371748454268\\
7.33	0.00601657087641402\\
7.34	0.0059994361232854\\
7.35	0.00598231570147954\\
7.36	0.00596521209828838\\
7.37	0.0059481278116626\\
7.38	0.00593106534988841\\
7.39	0.00591402723125128\\
7.4	0.00589701598368634\\
7.41	0.00588003414441502\\
7.42	0.0058630842595678\\
7.43	0.00584616888379259\\
7.44	0.0058292905798485\\
7.45	0.00581245191818473\\
7.46	0.0057956554765042\\
7.47	0.00577890383931173\\
7.48	0.00576219959744636\\
7.49	0.00574554534759764\\
7.5	0.00572894369180552\\
7.51	0.00571239723694368\\
7.52	0.00569590859418593\\
7.53	0.00567948037845552\\
7.54	0.00566311520785711\\
7.55	0.00564681570309117\\
7.56	0.00563058448685055\\
7.57	0.00561442418319915\\
7.58	0.00559833741693245\\
7.59	0.00558232681291967\\
7.6	0.00556639499542761\\
7.61	0.00555054458742583\\
7.62	0.00553477820987328\\
7.63	0.00551909848098611\\
7.64	0.00550350801548683\\
7.65	0.0054880094238346\\
7.66	0.00547260531143676\\
7.67	0.00545729827784159\\
7.68	0.00544209091591244\\
7.69	0.00542698581098319\\
7.7	0.00541198553999529\\
7.71	0.00539709267061646\\
7.72	0.00538230976034128\\
7.73	0.0053676393555739\\
7.74	0.00535308399069316\\
7.75	0.00533864618710031\\
7.76	0.00532432845224988\\
7.77	0.00531013327866386\\
7.78	0.00529606314292988\\
7.79	0.00528212050468355\\
7.8	0.00526830780557577\\
7.81	0.00525462746822538\\
7.82	0.00524108189515792\\
7.83	0.00522767346773091\\
7.84	0.00521440454504665\\
7.85	0.00520127746285307\\
7.86	0.00518829453243349\\
7.87	0.0051754580394862\\
7.88	0.00516277024299462\\
7.89	0.00515023337408911\\
7.9	0.00513784963490134\\
7.91	0.00512562119741226\\
7.92	0.00511355020229481\\
7.93	0.00510163875775236\\
7.94	0.00508988893835426\\
7.95	0.00507830278386949\\
7.96	0.00506688229809981\\
7.97	0.00505562944771367\\
7.98	0.00504454616108222\\
7.99	0.00503363432711874\\
8	0.00502289579412297\\
8.01	0.00501233236863176\\
8.02	0.0050019458142775\\
8.03	0.00499173785065579\\
8.04	0.00498171015220399\\
8.05	0.00497186434709203\\
8.06	0.00496220201612717\\
8.07	0.00495272469167433\\
8.08	0.00494343385659332\\
8.09	0.00493433094319495\\
8.1	0.00492541733221726\\
8.11	0.00491669435182377\\
8.12	0.0049081632766251\\
8.13	0.00489982532672574\\
8.14	0.00489168166679748\\
8.15	0.00488373340518098\\
8.16	0.00487598159301725\\
8.17	0.00486842722341023\\
8.18	0.00486107123062228\\
8.19	0.00485391448930385\\
8.2	0.00484695781375874\\
8.21	0.00484020195724644\\
8.22	0.00483364761132276\\
8.23	0.00482729540522007\\
8.24	0.00482114590526841\\
8.25	0.00481519961435849\\
8.26	0.00480945697144789\\
8.27	0.0048039183511113\\
8.28	0.00479858406313581\\
8.29	0.00479345435216218\\
8.3	0.0047885293973729\\
8.31	0.00478380931222765\\
8.32	0.00477929414424697\\
8.33	0.00477498387484454\\
8.34	0.00477087841920868\\
8.35	0.00476697762623337\\
8.36	0.00476328127849906\\
8.37	0.00475978909230357\\
8.38	0.00475650071774309\\
8.39	0.00475341573884335\\
8.4	0.00475053367374086\\
8.41	0.00474785397491404\\
8.42	0.00474537602946399\\
8.43	0.00474309915944452\\
8.44	0.00474102262224105\\
8.45	0.00473914561099774\\
8.46	0.00473746725509229\\
8.47	0.00473598662065771\\
8.48	0.00473470271115021\\
8.49	0.00473361446796231\\
8.5	0.00473272077108036\\
8.51	0.00473202043978519\\
8.52	0.00473151223339501\\
8.53	0.00473119485204927\\
8.54	0.00473106693753223\\
8.55	0.00473112707413499\\
8.56	0.00473137378955455\\
8.57	0.00473180555582858\\
8.58	0.00473242079030434\\
8.59	0.00473321785664029\\
8.6	0.00473419506583885\\
8.61	0.00473535067730864\\
8.62	0.00473668289995478\\
8.63	0.00473818989329534\\
8.64	0.00473986976860249\\
8.65	0.00474172059006664\\
8.66	0.00474374037598169\\
8.67	0.00474592709995001\\
8.68	0.00474827869210512\\
8.69	0.00475079304035059\\
8.7	0.00475346799161338\\
8.71	0.00475630135310989\\
8.72	0.00475929089362316\\
8.73	0.00476243434478938\\
8.74	0.0047657294023923\\
8.75	0.00476917372766367\\
8.76	0.00477276494858839\\
8.77	0.00477650066121251\\
8.78	0.0047803784309529\\
8.79	0.00478439579390673\\
8.8	0.00478855025815964\\
8.81	0.00479283930509094\\
8.82	0.00479726039067467\\
8.83	0.00480181094677503\\
8.84	0.00480648838243503\\
8.85	0.004811290085157\\
8.86	0.00481621342217397\\
8.87	0.00482125574171061\\
8.88	0.00482641437423271\\
8.89	0.00483168663368425\\
8.9	0.00483706981871098\\
8.91	0.00484256121386962\\
8.92	0.00484815809082187\\
8.93	0.00485385770951234\\
8.94	0.00485965731932971\\
8.95	0.00486555416025026\\
8.96	0.00487154546396339\\
8.97	0.00487762845497817\\
8.98	0.00488380035171073\\
8.99	0.00489005836755164\\
9	0.00489639971191312\\
};
\addlegendentry{CMT}
\end{axis}
\node at (-1,15) {\textbf{b)}};
\end{tikzpicture}%
    \end{subfigure}
    \caption{\textbf{a)} The modal field profiles of the transverse electric field $E_z$ propagating along the length of the waveguide structure. Each graph represents a sideband of the original frequency $\omega_0$ separated by integer multiples of the modulation frequency $n \Omega$. \textbf{b)} The corresponding amplitudes of each mode sideband along the structure as calculated theoretically (orange) and through the simulation (blue).}
    \label{fig:weakMod}
\end{figure}

\begin{figure}[t]
\centering
\setlength{\figH}{0.4\textwidth}
\setlength{\figW}{0.7\textwidth}
% This file was created by matlab2tikz.
%
%The latest updates can be retrieved from
%  http://www.mathworks.com/matlabcentral/fileexchange/22022-matlab2tikz-matlab2tikz
%where you can also make suggestions and rate matlab2tikz.
%
\definecolor{mycolor1}{rgb}{0.00000,0.44700,0.74100}%
%
\begin{tikzpicture}

\begin{axis}[%
width=0.8\figW,
height=0.25\figH,
at={(0\figW,0\figH)},
scale only axis,
xmin=-3,
xmax=3,
xtick={-3,-2,-1,0,1,2,3},
yminorticks=false,
xlabel={Sideband (n)},
ymode=log,
ymin=0.00000001,
ymax=10,
grid=both,
ylabel = {Maximum $|E_z (\omega_n)|^2$},
%ylabel={$\text{Maximum $|$E}_\text{z}\text{(}\omega{}_\text{n}\text{)$|$}$},
axis background/.style={fill=white}
]
\addplot[only marks, mark=*, mark options={}, mark size=3.000pt, draw=black, fill=mycolor1] table[row sep=crcr]{%
x	y\\
-3	0.000000058939162\\
-2	0.000049831160601\\
-1	0.003480773455234\\
0	1.010334172885610\\
1	0.011021244259482\\
2	0.000107038587645\\
3	0.000002341788501\\
};
\end{axis}
\end{tikzpicture}%
\caption[Maximum field amplitude at each sideband]{The maximum field amplitude calculated for each sideband $n$ on a log scale. The field intensity decreases exponentially with $n$. Note that the amplitude of the sidebands is not symmetric about $n=0$.}
\label{fig:sideamp}
\end{figure}

There is excellent agreement between \textit{FDFD} and \textit{CMT} for predicting the modal amplitude. However, \textit{FDFD} shows minor oscillations along the length of the waveguide. Performing a standard Fourier transform on the amplitudes, the oscillation is found to be at a frequency of $\Pi_1 = 2 \pi / (k_1 + k_2)$.

Visual inspection alone supports this argument: in figure \ref{fig:weakmod}, it is possible to observe a second mode generation \textit{before} the region of modulation itself. This suggests that the sideband generated is also backwards-propagating.

The maximum $|E_z|$ field at each sideband $\omega_n$ is shown in figure \ref{fig:sideamp}. At the initial band $\omega_0$, the field amplitude is normalised to unity. The fields exponentially decrease with $n$. However, the higher sidebands decrease at a slower rate than that of the negative sidebands. This is attributed to the band structure of the waveguide itself - it will always be possible for light t
\section{Non-reciprocal propagation}



\section{Complete optical isolation through photonic transitions}

As a more practical example, we consider a microcavity ring resonator of ... Below, the results of the frequency domain simulation are shown. Such simulations represent the first of their kind in the frequency domain. Indeed, owing to the fast convergence time of \textit{FDFD}, it is amenable to rapid prototyping of photonic structures. Previously, \textit{FDTD} simulations required several hours to solve .. etc

As a more practical example, we consider a ring resonator of inner radius $2.9 \mu m$ and outer radius $3.18 \mu m$. The refractive index of the resonator has an imaginary component at $n=4-5e^{-9}i$, chosen such that radiative losses lead to critical coupling in the waveguide. When the length of the ring is modulated, the generated sidebands are no longer critically coupled allowing for non-reciprocal propagation through the system.  The ring is dynamically modulated at a frequency of $10e^9 \frac{rad}{s}$ and modulation strength of $\delta = 0.1 \epsilon_0$, both chosen to be achievable with current modulation technology (see section Z for feasibility). As before, the fundamental mode is calculated a priori and coupled exactly into the input waveguide. We obtain analytical results through standard coupled mode theory (see appendix Z), and compare the contrast ratio. 


\section{Phase matched multi-frequency (PM-FDFD)}
The MF-FDFD technique is quite flexible in it's applications, however, Shi's implementation can only consider transitions along the same wave-vector on the dispersion relation. Here, I will show that by accounting for a permittivity modulation of the form

\begin{equation}
	\epsilon(x,t)=\delta \cos{\Omega t + \Gamma x}
\end{equation}

where $\Omega$ is as usual the modulation frequency, and $\Gamma$ is defined as the difference in wave-vectors (EXPLAIN BETTER).

then it becomes possible to extend MF-FDFD to simulate \textit{indirect} interband photonic transitions, which allow for a change of both mode and wavevector, a technique that will be referred to as phase-matched finite difference frequency domain (\textit{PM-FDFD)}. The derivation largely follows Shi's original formulation, however now requiring the use of the $2D$ Fourier transform to account for the changing frequency profile along the length of the modulated structure;



\section{Comparisons with FDFD}

The difficulty in drawing benchmark comparisons between $FDTD$ and $FDFD$ lie in the inherently different techniques both methods use to approach a full-wave solution. $FDTD$ operates by effectively `brute-forcing' a solution over each time-step, whereas $FDFD$ generates and then solves a wave matrix, independent of any meaningful time-step. Naturally, the solution obtained by \textit{FDFD} is dependent not only on the spatial step chosen in the simulation, but also in the number of side-bands calculated. In figure ZZ, we show the maximum modal amplitude for each side-band, noting that the rapid drop-off in amplitude is exponential in nature. Thus, only several frequency side-bands are required for a full-wave solution. 

To compare the speeds of \textit{TD} and \textit{FD} methods, we calculate the time until a steady-state for time domain, and compare it to the time obtained before a frequency domain solution. This is because \textit{FD} always yields the steady-state solution, whereas \textit{TD} can simulate field propagation indefinitely. To determine the time until the steady state fields are obtained, a single probe is placed at the far end of the simulation region, just within the boundaries of the PML layer. The probe collects a single value of the transverse electric field at every time-step. When the steady state solution is reached, the field will have no transient response. 


\subsection{Impact of PML layer width on solution convergence}

\subsection{Accelerated solution through the correct choice of PML}

\subsection{Ill-conditioned wave matrices in FDFD}
Suppose in \textit{FDFD} that we want to solve the general wave matrix equation

\begin{equation}
x = A^{-1}b.
\end{equation}

Now suppose that $b$ is changed a small amount and the matrix is solved again, 

\begin{equation}
x + \Delta x = A^{-1} (b + \Delta b).
\end{equation}

If a small change in $b$ produces a large change in $x$, the matrix $A$ is said to be `ill-conditioned'. The degree to which a matrix is conditioned is characterised by the condition number $\kappa$.