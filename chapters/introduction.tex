\chapter{Introduction}
Reciprocity is a fundamental principle governing transmission in physical systems. It ensures that the transfer of some physical quantity, such as light, motion, or even electrical charge is identical between any two points regardless of geometric asymmetries in the intervening space \cite{Coulais2017}. Physical intuition suggests that when reversing the position of a source and detector in some wave-like experiment, the observed signal will not change \cite{Deak2012}. This principle holds for many physical systems, and in-fact many systems rely upon it - the detection of earthquakes \cite{Buehler2016}, operation of interferometers, and even the identification of cracks in concrete \cite{Scalerandi2012c, Scalerandi2012} exploit reciprocity to some degree. However, reciprocal transmission is not always a desirable trait in physical systems. There is growing demand for the ability to isolate a region of space to allow wave transmission in one direction, while blocking it in the other. Significant research has gone into searching for such techniques to induce non-reciprocity. Formally, a system is said to be non-reciprocal if it is asymmetric under time-reversal (\textit{TR}) and parity-transformation (\textit{PT}) symmetry. That is to say, a non-reciprocal system is one in which the laws governing it change when the direction of time and spatial locations is reversed ($t \rightarrow -t$, $\bm{r} \rightarrow -\bm{r}$) \cite{Caloz2016}. This is achieved by the application of an external force, such as a mechanical velocity or magnetic field which impose a preferential direction, or \textit{bias}, on the system. 

Reciprocity is not a property specific to light, but also one of numerous phenomena involving the transmission of wavelike phenomena - including acoustics and structural analysis. Both acoustic and mechanical non-reciprocity are active fields of research \cite{Coulais2017}. Such acoustic non-reciprocity forces sound to travel in one direction only, and has been borne from ideas originally formulated in electromagnetic non-reciprocity. Devices that exploit non-reciprocal wave transmission may lead to solutions to long-standing problems across many branches of physics - such as energy conversion, signal processing, and energy harvesting.

\newpage

For electronic systems, breaking reciprocity relies on the existence of a magnetic field. Photons, however, are uncharged spin-1 Bosons and as a result there exists no naturally occurring gauge potential through which to control light. Manipulating the propagation of light at the nano-scale has been a long-standing hurdle in the development of integrated photonics \cite{Hua2016a,Huang2016,Bi2013,Doerr2013,Doerr2014,El-Ganainy2013,Bi2011}. With the development of artificial photonic structures such as photonic crystals and meta-materials \cite{Joannopoulos1997}, on-chip integrated photonics has been realised as not just a possibility, but increasingly more simple to achieve.

\section{Reciprocity in electromagnetism}

\begin{figure}
	\centering
	\begin{subfigure}{0.5\textwidth}
		\centering
		\begin{tikzpicture}[scale=2]
\pgfmathsetmacro{\lH}{1}
\pgfmathsetmacro{\lR}{2}
\pgfmathsetmacro{\sA}{asin(\lH/\lR)}
\pgfmathsetmacro{\base}{1}
\pgfmathsetmacro{\xshi}{\base/2}

\draw[yshift = -2cm, xshift = -\xshi cm,name path=lens, fill=mode1, opacity=0.2] (0, \lH cm)
arc[start angle = -\sA, delta angle = 2*\sA, radius = \lR cm] --
+(\base cm, 0)
arc[start angle = 180 - \sA, delta angle = 2*\sA, radius = \lR cm]
-- cycle;
\fill[fill = black] (-1cm, 0) coordinate (F) circle[radius = 0cm];

\begin{scope}[decoration = {
	markings,
	mark = at position 0.1 with {\arrow[scale=1.5]{stealth}},
	mark = at position 0.75 with {\arrow[scale=1.5]{stealth}}
}
]
\foreach \y  in {0.25, 0, -0.25}{
	\draw[postaction = decorate] (-1.5cm, \y cm) -- (0, \y cm)
	coordinate (A) 
	-- ($(F)!3!(A)$) ;      
	%extend lines along the path command here;
}
\end{scope}
\node at (-1.5cm,1) {\textit{A}};
\node at (1.5cm,1) {\textit{B}};
\end{tikzpicture}
		\caption{}
	\end{subfigure}%
	\begin{subfigure}{0.5\textwidth}
		\centering
		\input{reciprocity2.tex}
		\caption{}
	\end{subfigure}%
	\caption[Time-reversed path in a double concave lens]{Reciprocity in a double concave lens for light rays travelling from \textbf{(a)} A to B, and \textbf{(b)} B to A. Since reciprocity holds for the system, light travelling along its time reversed path (B to A) will return back to its original state. This is true for \textit{any} scattering event, not just the divergence of rays through a lens.}
	\label{fig:reciprocity}
\end{figure}

In electromagnetism reciprocity is known under the guise of \textit{Lorentz reciprocity}, which mathematically expresses the idea of parity and time-reversal (PT) symmetry for an electromagnetic wave. In general terms, Lorentz reciprocity demands that if light can travel from some region \textit{A} to another region \textit{B}, then the reverse also holds (Fig \ref{fig:reciprocity}). Consider for example the random scattering of a laser in a medium. If it were possible to selectively take all the scattered wavelets and reverse their direction, then reciprocity would ensure that the they would recombine back into a coherent beam that emerges at the original angle of incidence \cite{Wiersma2013a}. In other words, if the propagation of a light wave were to be reversed then the properties of its transmission would remain unaffected \cite{Potton2004a}. It is evident that this is a direct consequence of the more generalised form of reciprocity as described above, as reciprocity requires that waves which propagate along time-reversed paths exhibit identical properties of transmission, regardless of how complicated such a path might be \cite{Buehler2016}.

Although reciprocal propagation of light is sometimes a desirable facet of optical devices, being able to dynamically constrain and manipulate the direction light can travel has tremendously useful applications, ranging from the reduction of noise in optical experiments \cite{Khanikaev2014a}, to exhibiting novel quantum effects of light \cite{Yilmaz2015a}. The most commonly employed method of breaking reciprocity in photonics is through the magnetic biasing of a system (where the magnetic field $\bm{B}$ is a pseudo-vector, making it odd under time-reversal \cite{Wang2006a}). In optics, magnetically biased techniques are crucial to the function of optical isolators, non-reciprocal devices that permit forward propagation of light while restricting counter-propagation \cite{Jalas2013a}, effectively constituting \textit{diodes} for light. Such commercially available isolators make use of the magneto-optic Faraday effect, where the plane of polarisation of a light source is rotated along the path of its trajectory. However, practical implementation of the Faraday effect in integrated photonics remains largely infeasible, owing to the incompatibility of binding magneto-optic materials to standard metal-oxide semiconductors \cite{Feng2011a} as well as the difficulty in scaling to optical frequencies \cite{Monticone2017}. Additionally, since the rotation of the plane of polarisation is dependent on the path length the light travels through, any optical device constructed with this approach must be large enough to accommodate a sufficient rotation. These restrictions impose a fundamental limit on both the size and potential applications of non-reciprocal devices that make use of a magneto-optic biasing. Despite this, there is good reason why magneto-optics has remained the de-facto standard of non-reciprocal devices; magneto-optically active materials are inexpensive, widely available, and simple to work with, making them attractive from an engineering standpoint in devices where size limitations are not strictly enforced.

Small-scale compatibility of non-reciprocal optics forms a primary hurdle in the realisation of fully-fledged photonic circuits. This has motivated a significant amount of interest in searching for alternative methods for inducing non-reciprocity without bulk magneto-optics \cite{Fang2012a, Rabla, Shoji2014,Li2014a}. Being able to isolate signals from one another is crucial to integrated photonics, which are currently hampered by significant noise and backscattering, leading to coherent interference in experiments and noise-sensitive designs. Nature however, provides startlingly few options to break reciprocity for light. As of yet there are only three known classes of techniques that are non-reciprocal. These includes the aforementioned magneto-optic biasing, non-linearities of materials, and the use of media that possess a \textit{time-dependence}.

\subsection{Relation to time-reversal symmetry}
In most literature, reciprocity and time-reversal symmetry are often used synonymously. Despite this being the case for a majority of reciprocal processes, there are several situations where a breaking of TR-symmetry does not imply a reciprocity violation. Lorentz reciprocity occurs when a wave satisfies a symmetry property that connects a scattering process with the reversed one. However, this is a generalisation of the principle of TR-symmetry, and the two are not always interchangeable in the realm of optics. This is because Lorentz reciprocity relates input and output waves in pairs, irrespective of the presence (or lack thereof) of other waves in the system. For optical systems reciprocity can be applied even when absorption in a material takes place - a process that is TR asymmetric, but still reciprocal \cite{Potton2004a}.

\section{Non-reciprocity in time-dependent systems}
\label{mome}

Consider a glass of water with light incident upon its surface. Now, assume that the water is dispersionless, and to a large degree incompressible. As the light passes through the water, it will gain an overall phase shift while its other properties will remain unchanged. But what if the water is then set into motion? It is found that the light develops an interference structure that is dependent on the velocity of the medium \cite{Zalesny2001a}. Even coherent light - lasers, will slightly bend in their trajectory whilst passing through the water. The amount of dragging the light experiences is directly proportional to the velocity and refractive index of the medium, a phenomenon that has been well known since Fizeau's experiment (1851) on the speed of light in water. Most importantly however, the symmetry of light transmission in the water is non-reciprocal: a slight phase shift is acquired when the light propagates with the water flow. On the other hand, light propagating in the reverse direction will acquire a negative phase shift. This idea of a medium that possesses some kind of time-dependent bias forms an important class of non-reciprocal devices. 

Lorentz reciprocity is derived on the grounds of time-independent media and time-harmonic fields \cite{Caloz2016}. It is apparent then that an optical medium that exhibits some kind of time-dependence can not be constrained by the reciprocity theorem. A moving medium will induce a non-reciprocal phase shift for light. This can be intuitively understood due to the flow vector $\bm{v}$ of the water being odd under time-reversal, akin to the pseudo-vector in magneto-optically active materials. The phase shifts obtained by the light are not directly observable. Similarly to how voltage can only be measured as a potential difference, the phase of light can only be measured relative to the phase of another electromagnetic wave. This measurement is exclusively made by examining the interference pattern of the light \cite{Tzuang2014a}.

\section{Outline}
This thesis will focus primarily on non-reciprocity in time-dependent media, and discuss how such non-reciprocal phases give rise to a synthetic gauge potential for light. The introduction of a gauge potential gives rise to a photonic analogue of the Aharonov-Bohm effect which is demonstrated through full-wave finite difference methods. 

In Chapter \ref{chapter:synthetic}, a review of recent developments in the field of synthetic gauge fields is presented. The review explores how non-reciprocal phases give rise to a photonic gauge field, and how suitable tailoring of the phases culminates in effective magnetic fields. These effective magnetic fields for light are vital in the nascent field of topological photonics. 

Chapter \ref{chapter:theory} presents a theoretical derivation of the effective gauge potential arising from a moving medium - and how such motion is equivalent to a modulation of the permittivity of a waveguide. It is then shown how a photonic analogue of the Aharonov-Bohm effect can be demonstrated with a rotating dielectric medium and a modulated waveguide. 

In Chapter \ref{chapter:method}, the finite-difference method for electromagnetic simulations is discussed and implemented for the purpose of demonstrating non-reciprocal phases in modulated waveguides. Both time and frequency domain simulations are considered.

Finally, in Chapter \ref{chapter:results}, non-reciprocal mode conversion is numerically validated and demonstrated, as well as a photonic Aharonov-Bohm effect for the purposes of optical isolation. Discussions of the feasibility of such devices is considered, and both implementations of numerical methods are then compared.