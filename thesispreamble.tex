
\usepackage{bm}
\usepackage{geometry}
% The UOW default dimensions are: 
 %\geometry{a4paper,inner=4.0cm, outer=2cm, top=3cm, bottom=2cm}

% These aren't especially pleasing to look at. Without changing the dimensions
% of the textblock you can use:
  \geometry{a4paper,inner=3cm, outer=3cm, top=3cm, bottom=3cm}
 \pdfpagewidth=\paperwidth 
  \pdfpageheight=\paperheight
  % This acts as a failsafe to ensure things aren't stretched or moved when it's finally printed as a PDF.

%\usepackage[parfill]{parskip} 
% Activate to begin paragraphs with an empty (return) line, comment out the indent below if you chose the return line option.

\setlength{\parindent}{1em}  % Sets the length of the paragraph indent. Current setup has a an indent. Disable this if you activate the return line above.

% Double or one and a half spacing.
\usepackage{setspace}
  \onehalfspacing

  
\usepackage{graphicx}
  \DeclareGraphicsRule{.tif}{png}{.png}{`convert #1 `dirname #1`/`basename #1 .tif`.png}
 \graphicspath{{figures/}}
% Graphics. Remove me and you won't have any figures, and that would be very boring.

\usepackage[usenames,dvipsnames,svgnames,table]{xcolor}
% Adds the ability to make coloured text and lines throughout the document. See documentation for xcolor.

%-------------------- Tables, figures and captions
\usepackage[font={small},labelfont={bf},margin=4ex]{caption}
% Makes bold labeled and smaller font captions. Must be loaded before the longtable package to avoid conflicts! 

\usepackage{longtable}
% Long tables (more than one page). Different headers and footers for beginning and end pages, etc.

\usepackage{afterpage}
% Make a longtable start on the next clear page, but fills the previous one with text first (no random gaps in the text-from long tables anymore! Man, the day I discovered this...)

\usepackage{booktabs}
% Nice looking tables and lines in tables

\usepackage{multirow}
% Entries in tables over multiple rows

\usepackage{lscape}
% Pages in landscape

\usepackage{pdflscape}
% Landscape pages also rotated in the pdf

\usepackage{wrapfig}
% Allows figures that don't take up the entire width of the page, wrapping the text around the figure

%\usepackage[position=top,singlelinecheck=false,captionskip=4pt]{subfig} 
% Multiple figures in an individual figure. Fig. 1 a, b, c, etc. each with, or without, it's own individual caption, and with a global caption for all sub figures.

%-------------------- Special symbols and fonts
\usepackage{amssymb}
% Maths symbols

%-------------------- Document sections, headers, footers, and bibliography
\usepackage{fancyhdr}
% for creating different headers and footers

%-------------------- Bibliography


%\usepackage{tikz}
%\usetikzlibrary{shapes,arrows}



\usepackage{etoolbox}

\usepackage{mathptmx}
\usepackage{amsfonts}
\usepackage{latexsym}
\usepackage{amsmath}
\usepackage[utf8]{inputenc}
\DeclareUnicodeCharacter{0327}{,}%
\usepackage{float}
\usepackage{tensor}
\usepackage{appendix}
%\usepackage{subfigure}
\usepackage{listings}
\usepackage{braket}
\usepackage{tikz}
\usetikzlibrary{arrows,calc,positioning}
\usepackage{pgfplots}
\usepgfplotslibrary{fillbetween}
\usepgfplotslibrary{external}
\usepgfplotslibrary{groupplots}
\usetikzlibrary{patterns}
\usetikzlibrary{external}
\usetikzlibrary{decorations.markings,intersections,calc}
\tikzexternalize[prefix=./OutputTikz/] 
\pgfplotsset{every axis/.append style={
                    label style={font=\small},
                    tick label style={font=\small}  
                    }}
\pgfplotsset{compat=newest} 
\pgfplotsset{plot coordinates/math parser=false}
\newlength\figH
\newlength\figW
\setlength{\figH}{4cm}
\setlength{\figW}{\linewidth}
\definecolor{eps}{RGB}{44,127,184}%
\definecolor{mod_eps}{RGB}{127,205,187}%
\definecolor{src}{RGB}{34,177,76}%

\definecolor{epsilon}{RGB}{63,72,204}
\definecolor{modup}{RGB}{0,162,232}
\definecolor{moddown}{RGB}{153,217,234}

\definecolor{mode1}{RGB}{23,46,124}%
\definecolor{mode2}{RGB}{187,47,46}%
\definecolor{mode3}{RGB}{86,152,52}

\definecolor{gblue}{RGB}{16,116,188}


\usepackage[backend=biber,style=ieee,url=false,doi=false,isbn=false,eprint=false]{biblatex} 
\renewbibmacro{in:}{}
\AtEveryBibitem{%
  \clearlist{language}%
}
\AtEveryBibitem{\clearfield{month}}

\renewcommand{\thefootnote}{\arabic{footnote}}

\usepackage{matlab-prettifier}
%-------------------- Hyperlinks in your document.
\usepackage[unicode=true,breaklinks=true]{hyperref}
% The hyperref package allows you to have clickable links in your pdf. It also allows you to have the mailto link associated with your name. It can be  a bit finicky, so load it last.

%-------------------- Command renewals, New commands etc.
\renewcommand{\thefootnote}{\arabic{footnote}}
