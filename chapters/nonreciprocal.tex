\chapter{Simulation results and discussion}
\label{chapter:results}

In this chapter, the results of applying the MF-FDFD method are presented to demonstrate non-reciprocity in modulated waveguides. `$a$' is used regularly in the units of frequency, and is a normalisation constant which allows the simulation to be scaled to any size, a common procedure in electromagnetic simulations. For all results presented here, $a = 1  \mskip3mu \mu m$.

\section{Numerical validation of FDFD}

As a preliminary verification of the MF-FDFD implementation, an identical simulation to the time-domain one of Section \ref{sec:demonstration} is calculated over one sideband. The waveguide is again of width $1.1 \mskip3mu \mu m$ and length $23  \mskip3mu \mu m$, with a modulation of $\epsilon'(t) = 0.1 \cos (\omega t)$ applied between $2$ to $21  \mskip3mu \mu m$. The upper and lower halves of the waveguide are modulated with a $\pi$ phase difference, and the waveguide is injected with a modal source at $1  \mskip3mu \mu m$ of frequency $0.129  \mskip3mu (2 \pi c/a)$. The amplitude of the modes are extracted along the length of the modulation region, and compared to the results of FDTD. As an additional benefit to the FDFD method, the field evolution of the mode can be visualised completely, as in Figures \ref{sfig:mode1} and \ref{sfig:mode2}. 

There is excellent agreement between FDFD and FDTD as seen in Figure \ref{sfig:coherence}. Both simulations predict the characteristic minor oscillations during modulations, however FDTD predicts that power is lost at a faster rate. This is likely because the FDFD simulation will only operate over the number of sidebands specified. Since small amounts of power are naturally lost to higher order sidebands, the frequency domain simulation can not completely predict the modal variation. However, it is worth pointing out that both simulations perfectly agree on the convergence of the coherence length at $19 \mskip3mu  \mu m$. 

\begin{figure}[t]
	\centering
	\begin{subfigure}{\textwidth}
		\centering
		\setlength{\figH}{0.3\textwidth}
		\setlength{\figW}{\textwidth}
		% This file was created by matlab2tikz.
%
%The latest updates can be retrieved from
%  http://www.mathworks.com/matlabcentral/fileexchange/22022-matlab2tikz-matlab2tikz
%where you can also make suggestions and rate matlab2tikz.
%
\definecolor{mycolor1}{rgb}{0.00000,0.44700,0.74100}%
\definecolor{mycolor2}{rgb}{0.85000,0.32500,0.09800}%
\definecolor{mycolor3}{rgb}{0.92900,0.69400,0.12500}%
\definecolor{mycolor4}{rgb}{0.49400,0.18400,0.55600}%
%
\begin{tikzpicture}

\begin{axis}[%
width=0.89\figW,
height=\figH,
at={(0\figW,0\figH)},
scale only axis,
xmin=0,
xmax=20,
ymin=0,
clip=false,
ymax=1,
xlabel = {Modulated length ($\mu m$)},
ylabel = {Photon flux ($n$)},
axis background/.style={fill=white}
]
\addplot [color=mode1]
  table[row sep=crcr]{%
0.0500000000000007	0.997209339000001\\
0.600000000000001	0.996082311999999\\
1.4	0.999022263000001\\
1.65	0.993382490999998\\
1.9	0.982718813000002\\
2.3	0.964655487000002\\
2.55	0.958933380000001\\
2.9	0.952184463000002\\
3.1	0.942576759000001\\
3.3	0.927887047999999\\
3.6	0.905261375999999\\
3.75	0.898179482\\
3.95	0.894331988000001\\
4.25	0.891038763000001\\
4.4	0.884646372999999\\
4.55	0.873285725999999\\
4.75	0.852737034\\
4.95	0.832707403000001\\
5.1	0.822307978000001\\
5.25	0.817025767000001\\
5.7	0.808020533000001\\
5.85	0.796413183999999\\
6	0.778965206999999\\
6.35	0.734063726999999\\
6.45	0.725368186000001\\
6.6	0.717762148999999\\
6.8	0.714571905\\
7	0.710239075000001\\
7.1	0.704527122000002\\
7.2	0.695582756\\
7.35	0.676607331\\
7.7	0.626013035\\
7.8	0.615969687\\
7.9	0.609305496000001\\
8.05	0.60488509\\
8.35	0.601126816000001\\
8.45	0.596091709\\
8.55	0.587603563000002\\
8.65	0.575691364000001\\
8.8	0.553178839000001\\
9	0.522013173000001\\
9.1	0.509348663000001\\
9.2	0.500006169999999\\
9.3	0.494200418999998\\
9.45	0.490807847999999\\
9.7	0.487941278000001\\
9.8	0.483361936000001\\
9.9	0.475333524\\
10	0.463777976999999\\
10.15	0.441435343999999\\
10.4	0.402711426\\
10.5	0.391020555000001\\
10.6	0.382831853999999\\
10.7	0.378095650999999\\
10.85	0.375642477\\
11.05	0.3735295\\
11.15	0.369619484000001\\
11.25	0.362489979999999\\
11.35	0.351964510999998\\
11.5	0.331161032000001\\
11.75	0.294155885999999\\
11.85	0.282716205\\
11.95	0.274574935\\
12.05	0.269782465999999\\
12.2	0.267346573000001\\
12.45	0.265035599000001\\
12.55	0.260834346999999\\
12.65	0.253599511000001\\
12.8	0.237392255\\
13.2	0.187501166000001\\
13.3	0.179857078000001\\
13.45	0.173855179\\
13.65	0.172357041000001\\
13.85	0.169724951999999\\
14	0.162292416\\
14.15	0.149391302000002\\
14.6	0.104563253999999\\
14.75	0.0974418640000003\\
14.9	0.09499396\\
15.25	0.0918480200000005\\
15.4	0.0853413819999993\\
15.6	0.0710626949999984\\
15.9	0.0485382899999998\\
16.05	0.0417482520000014\\
16.25	0.0384152650000011\\
16.7	0.0351825370000007\\
16.9	0.0281486870000016\\
17.35	0.0103134300000001\\
17.6	0.00701842200000158\\
19	0.00148695800000098\\
};
\addlegendentry{$\ket{1}$};
\addplot [color=mode2]
  table[row sep=crcr]{%
0.0500000000000007	0.00130598300000173\\
1	0.00400867200000121\\
1.25	0.00901523799999993\\
1.85	0.0263906570000003\\
2.45	0.032637703999999\\
2.65	0.0439684269999994\\
3.05	0.0701613660000007\\
3.2	0.0747393179999989\\
3.4	0.0759868879999992\\
3.6	0.0771289639999999\\
3.75	0.0821341589999989\\
3.9	0.0923892029999998\\
4.1	0.112325324\\
4.3	0.132503894999999\\
4.45	0.142997595000001\\
4.6	0.147944964000001\\
4.85	0.148831626\\
5	0.151200215999999\\
5.1	0.155822466\\
5.2	0.163467695000001\\
5.35	0.180056466\\
5.7	0.223845708999999\\
5.8	0.232024059\\
5.9	0.237034020999999\\
6.05	0.239504298\\
6.35	0.241410404\\
6.45	0.246267885000002\\
6.55	0.254644155000001\\
6.7	0.273296892000001\\
7.05	0.323916465\\
7.15	0.333628284\\
7.25	0.339698582\\
7.4	0.342845347000001\\
7.7	0.344257661\\
7.8	0.349064426999998\\
7.9	0.357711713\\
8	0.370116940999999\\
8.15	0.393714546999998\\
8.35	0.426048259000002\\
8.45	0.438825869999999\\
8.55	0.447869961999999\\
8.65	0.453002192\\
8.8	0.454976944999999\\
9.05	0.455989771999999\\
9.15	0.46050056\\
9.25	0.468945652999999\\
9.35	0.481335282\\
9.5	0.505348065\\
9.7	0.538942721000002\\
9.8	0.552489726000001\\
9.9	0.562260167000002\\
10	0.567977389999999\\
10.15	0.570399415000001\\
10.4	0.570791991\\
10.5	0.574658502999998\\
10.6	0.582342869000001\\
10.7	0.593951037\\
10.85	0.616955474000001\\
11.1	0.656995724000002\\
11.2	0.668787980000001\\
11.3	0.676677401999999\\
11.4	0.680713769\\
11.55	0.681555320000001\\
11.75	0.681265320000001\\
11.85	0.684219528\\
11.95	0.690689294999999\\
12.05	0.700869254000001\\
12.2	0.721617654999999\\
12.45	0.758751946\\
12.55	0.769948553999999\\
12.65	0.777569054000001\\
12.75	0.781569362999999\\
12.9	0.782457425\\
13.15	0.782154120000001\\
13.25	0.785593393999999\\
13.35	0.792298928000001\\
13.5	0.808085910999999\\
13.9	0.857002188999999\\
14	0.863983598000001\\
14.1	0.867771946000001\\
14.25	0.868751656000001\\
14.55	0.868512811999999\\
14.7	0.874832677000001\\
14.85	0.886968615000001\\
15.25	0.926425729999998\\
15.4	0.934265283999999\\
15.55	0.936641204000001\\
16	0.937225667\\
16.15	0.943829580999999\\
16.4	0.961348467000001\\
16.6	0.974242917000002\\
16.75	0.979810169\\
16.9	0.981439937000001\\
17.45	0.981089626999999\\
17.7	0.989015279\\
18	0.998277784999999\\
18.2	1\\
18.5	0.997101053000002\\
18.9	0.993978749\\
19	0.994044237000001\\
};
\addlegendentry{$\ket{2}$}
\addplot [color=mode1, dashed]
  table[row sep=crcr]{%
0.0500000000000007	0.986694052000001\\
0.199999999999999	0.995338852\\
0.350000000000001	0.999720845999999\\
0.5	0.998736347000001\\
0.649999999999999	0.992905961000002\\
1.1	0.970586748999999\\
1.25	0.968817562000002\\
1.65	0.968736516\\
1.8	0.96266893\\
1.95	0.951704654\\
2.3	0.921421076000001\\
2.45	0.914489442000001\\
2.6	0.913358597999999\\
3.05	0.918425266\\
3.2	0.911459430000001\\
3.35	0.898777691999999\\
3.65	0.870176603000001\\
3.8	0.862215405000001\\
3.95	0.860582315999999\\
4.35	0.863610502\\
4.45	0.859005038999999\\
4.55	0.850809978000001\\
4.7	0.832837961999999\\
5	0.792670651000002\\
5.1	0.783428644000001\\
5.2	0.777721363000001\\
5.35	0.775212203999999\\
5.7	0.775125292999999\\
5.8	0.769786952\\
5.9	0.760574705\\
6.05	0.740514132000001\\
6.35	0.695389793\\
6.45	0.684781181000002\\
6.55	0.678039108\\
6.65	0.675047397\\
6.8	0.675445301\\
7	0.676491379000002\\
7.1	0.67344808\\
7.2	0.666571595000001\\
7.3	0.655718989\\
7.45	0.633696021999999\\
7.7	0.594953963000002\\
7.8	0.583650304999999\\
7.9	0.576265646\\
8	0.572737298\\
8.15	0.572549955\\
8.35	0.573068835000001\\
8.45	0.56976654\\
8.55	0.562569932999999\\
8.65	0.551291265\\
8.8	0.528351226000002\\
9.05	0.487278348\\
9.15	0.474875100999999\\
9.25	0.46642327\\
9.35	0.461965199000002\\
9.5	0.460841005999999\\
9.7	0.46120307\\
9.8	0.458263487\\
9.9	0.451640051999998\\
10	0.441052800000001\\
10.15	0.419146350999998\\
10.4	0.378993314999999\\
10.5	0.366502645000001\\
10.6	0.357714438999999\\
10.7	0.352761213000001\\
10.85	0.350919177000002\\
11.1	0.350448699000001\\
11.2	0.346637943000001\\
11.3	0.339333867000001\\
11.4	0.328559826999999\\
11.55	0.307720994\\
11.75	0.278821315999998\\
11.85	0.267281336\\
11.95	0.259008083000001\\
12.05	0.254180084000001\\
12.2	0.252071075\\
12.45	0.251483852\\
12.6	0.245501526999998\\
12.7	0.237664842000001\\
12.85	0.220868841000001\\
13.15	0.183089106000001\\
13.3	0.170449332\\
13.45	0.164485076999998\\
13.65	0.163598780000001\\
13.85	0.162269086999999\\
14	0.15603861\\
14.15	0.144254965999998\\
14.6	0.100622414\\
14.75	0.0933171520000009\\
14.9	0.0907820719999997\\
15.3	0.0875060310000002\\
15.45	0.0807486439999998\\
15.65	0.0669081769999984\\
15.9	0.049559704\\
16.05	0.0429908810000015\\
16.25	0.0396613419999987\\
16.7	0.0365461459999992\\
16.9	0.0297755170000009\\
17.35	0.0121282299999983\\
17.55	0.00921560100000107\\
19	0.00205285599999883\\
};
%\addlegendentry{$\ket{1}$ TD};
\addplot [color=mode2, dashed]
  table[row sep=crcr]{%
0.0500000000000007	0.00221802900000156\\
0.850000000000001	0.00582627400000035\\
1.1	0.0097343770000009\\
1.4	0.0198939749999987\\
1.7	0.029623492999999\\
1.95	0.0329643090000005\\
2.3	0.03675286\\
2.45	0.0422475300000009\\
2.65	0.0544176450000009\\
3	0.0775024649999985\\
3.15	0.0828504069999987\\
3.35	0.0854133419999989\\
3.6	0.0886016000000005\\
3.75	0.0947910790000002\\
3.9	0.10543096\\
4.15	0.129462504999999\\
4.35	0.146831486\\
4.5	0.154842831\\
4.65	0.158284257999998\\
4.95	0.162379112\\
5.1	0.170050675999999\\
5.25	0.183598153999998\\
5.45	0.207654961999999\\
5.65	0.230807343999999\\
5.8	0.243064097000001\\
5.95	0.249822298000002\\
6.4	0.261662499\\
6.5	0.269430125\\
6.65	0.286602654999999\\
7.05	0.340965801999999\\
7.15	0.349817611999999\\
7.3	0.357632011\\
7.75	0.371192294\\
7.85	0.379247058000001\\
8	0.396694839999999\\
8.25	0.433824799\\
8.4	0.453948307000001\\
8.5	0.463998901\\
8.6	0.470805353999999\\
8.75	0.475859165999999\\
9.05	0.482985608\\
9.15	0.489356490999999\\
9.25	0.498796601999999\\
9.4	0.517851310000001\\
9.75	0.566569400999999\\
9.9	0.580676912000001\\
10.05	0.588715038\\
10.3	0.594643509000001\\
10.45	0.600281448\\
10.6	0.61165759\\
10.75	0.629107645000001\\
11.1	0.675062894\\
11.25	0.688151403999999\\
11.4	0.695901053\\
11.85	0.712871225000001\\
12	0.724777358000001\\
12.2	0.746461385\\
12.45	0.774416689999999\\
12.6	0.786430815999999\\
12.75	0.793687932000001\\
13.15	0.807606872000001\\
13.3	0.817792232999999\\
13.6	0.844549375\\
13.8	0.860627362999999\\
14	0.871852792999999\\
14.25	0.880085314999999\\
14.5	0.888553403\\
14.7	0.900148941000001\\
15.1	0.927127182\\
15.3	0.934342208\\
15.9	0.950310096999999\\
16.5	0.972822297\\
16.75	0.976590496\\
17.1	0.981624049000001\\
17.55	0.991271328\\
17.8	0.98955943\\
18.05	0.988140959999999\\
18.3	0.991819562\\
18.7	0.999900176000001\\
18.85	0.999009899000001\\
19	0.99430156\\
};
\node at (-1, 1) {\textbf{(a)}};
%\addlegendentry{$\ket{2}$ TD};
\end{axis}
\end{tikzpicture}%
		\phantomsubcaption%
		\label{sfig:coherence}%
	\end{subfigure}
	\setlength{\figH}{0.3\textwidth}
	\setlength{\figW}{0.4\textwidth}
	\begin{subfigure}{0.5\textwidth}
		\centering
		% This file was created by matlab2tikz.
%
%The latest updates can be retrieved from
%  http://www.mathworks.com/matlabcentral/fileexchange/22022-matlab2tikz-matlab2tikz
%where you can also make suggestions and rate matlab2tikz.
%
\begin{tikzpicture}

\begin{axis}[%
width=\figW,
height=0.5\figH,
at={(0\figW,0\figH)},
scale only axis,
point meta min=-4.99371933183411,
point meta max=4.99371933183411,
title = {},
axis on top,
xmin=0,
xmax=23,
clip=false,
xlabel={$\text{x (}\mu\text{m)}$},
ymin=-2.5,
ymax=2.5,
ylabel={$\text{y (}\mu\text{m)}$},
axis background/.style={fill=white},
colormap={mymap}{[1pt] rgb(0pt)=(0,0,1); rgb(31pt)=(1,1,1); rgb(32pt)=(1,1,1); rgb(63pt)=(1,0,0)}
]
\addplot [forget plot] graphics [xmin=0, xmax=23, ymin=-2.5, ymax=2.5] {graphs/fdfd/reciprocal/mode1-1.png};
\draw (0,0.55) -- (23,0.55);
\draw (0,-0.55) -- (23,-0.55);
\node at (2,1.8) {$\ket{1}$};
\node at (-1,3.3) {\textbf{(b)}};
\end{axis}
\end{tikzpicture}%
		\phantomsubcaption%
		\label{sfig:mode1}%
	\end{subfigure}%
	\begin{subfigure}{0.5\textwidth}
		\centering
		% This file was created by matlab2tikz.
%
%The latest updates can be retrieved from
%  http://www.mathworks.com/matlabcentral/fileexchange/22022-matlab2tikz-matlab2tikz
%where you can also make suggestions and rate matlab2tikz.
%
\begin{tikzpicture}

\begin{axis}[%
width=\figW,
height=0.5\figH,
at={(0\figW,0\figH)},
scale only axis,
point meta min=-4.99371933183411,
point meta max=4.99371933183411,
%title = {$\ket{2}$},
axis on top,
xmin=0,
clip=false,
xmax=23,
xlabel style={font=\color{white!15!black}},
xlabel={$\text{x (}\mu\text{m)}$},
ymin=-2.5,
ymax=2.5,
ylabel style={font=\color{white!15!black}},
axis background/.style={fill=white},
colormap={mymap}{[1pt] rgb(0pt)=(0,0,1); rgb(31pt)=(1,1,1); rgb(32pt)=(1,1,1); rgb(63pt)=(1,0,0)},
colorbar,
colorbar style={width=.02\linewidth, at={(1.05,0.25\figH)}, anchor=east}
]
\addplot [forget plot] graphics [xmin=0, xmax=23, ymin=-2.5, ymax=2.5] {graphs/fdfd/reciprocal/mode2-1.png};
\draw (0,0.55) -- (23,0.55);
\draw (0,-0.55) -- (23,-0.55);
\node at (2,1.8) {$\ket{2}$};
\node at (-1,3.3) {\textbf{(c)}};
\end{axis}
\end{tikzpicture}%
		\phantomsubcaption%
		\label{sfig:mode2}%
	\end{subfigure}
	\caption[Comparison of mode transition for FDTD and FDFD]{\textbf{(a)} Amplitude of both modes $\ket{1}$ and $\ket{2}$ for the time and frequency domain simulations, with the time domain in dashed lines. \textbf{(b)}Mode evolution of $\ket{1}$ as it propagates through the waveguide, and likewise for \textbf{(c)} $\ket{2}$. Note how $\ket{1}$ completely vanishes after reaching the coherence length.}
	\label{fig:coherencelength}
\end{figure} 


\subsection{Direct validation through coupled-mode theory}

The finite difference method simulates modes over any number of given sidebands. However, the implementation of the time domain method provided here is incapable of extracting these sidebands. Thus, to independently verify the frequency domain implementation, a simulation of a dynamically modulated waveguide over several sidebands is performed and compared to coupled-mode theory. Since the previously shown coupled mode-theory assumes only transitions between two modes (Equation \ref{eqn:theorymode}), a differential equation that describes transitions between $n$ modes is derived in Appendix \ref{app:cmt}. This allows for the comparison of the MF-FDFD implementation against a semi-analytic result.

The simulation considers a slab waveguide of relative permittivity $\epsilon= 4$ surrounded by vacuum in a region $10 \times 4  \mskip3mu \mu m$ discretised to $\Delta = 0.04 \mskip3mu  \mu m$. The waveguide is $10  \mskip3mu \mu m$ long and $0.75  \mskip3mu \mu m$ wide, which leads to a choice of an even mode of $\omega_1 = 0.667  \mskip3mu (2\pi c/a)$ at $k_1 = 0.841 \mskip3mu  (2\pi /a)$ and an odd mode at $\omega_2 = 0.645 \mskip3mu  (2\pi c/a)$ and $k_2 = 1.097 \mskip3mu  (2\pi /a)$. A permittivity modulation of $0.1 \epsilon_0 \cos(\Omega t)$ is applied between $1.5$ and $9 \mskip3mu  \mu m$, where the top and bottom halves of the waveguide are modulated with a $\pi$ phase difference to maximise coupling. This weak choice of modulation is insufficient to transition a mode, however coupled mode theory becomes exact in the limit of small modulations. Thus, it is perfectly suited to analysing the response of FDFD. The simulation is truncated to 5 frequency sidebands for $\omega_n = \omega_1 + n \Omega$ where $n \in [-4,4]$, and a continuous wave input of the first mode is excited at location $1 \mskip3mu  \mu m$ using the standard solver, whose power is normalised to $1  \mskip3mu W \mu m^{-1}$. For each frequency sideband the amplitude of the mode is extracted and compared to the semi-analytic results provided by the coupled-mode theory equation \ref{eqn:cmttheory}.

The maximum $|E_z|^2$ field at each sideband $\omega_n$ is first extracted and shown in figure \ref{fig:sideamp}. At the initial band $\omega_1$, the field amplitude is normalised to unity. The fields exponentially decrease with $|n|$. However, the higher sidebands decrease at a slower rate than that of the negative sidebands. Importantly, this means that the impact of reducing the sideband count in the simulation is minimal for a weak modulation. Since the final fields are the sum of each field component at all the sidebands calculated, any missing sidebands will lead to a slight inaccuracy in the final result. Here, the combined second sideband frequencies have a maximum amplitude of less than $0.0001 \% $ of the initial mode.

\begin{figure}[t]
	\centering
	\setlength{\figH}{\textwidth}
	\setlength{\figW}{\textwidth}
	% This file was created by matlab2tikz.
%
%The latest updates can be retrieved from
%  http://www.mathworks.com/matlabcentral/fileexchange/22022-matlab2tikz-matlab2tikz
%where you can also make suggestions and rate matlab2tikz.
%
\definecolor{mycolor1}{rgb}{0.00000,0.44700,0.74100}%
%
\begin{tikzpicture}

\begin{axis}[%
width=0.8\figW,
height=0.25\figH,
at={(0\figW,0\figH)},
scale only axis,
xmin=-3,
xmax=3,
xtick={-3,-2,-1,0,1,2,3},
yminorticks=false,
xlabel={Sideband (n)},
ymode=log,
ymin=0.00000001,
ymax=10,
grid=both,
ylabel = {Maximum $|E_z (\omega_n)|^2$},
%ylabel={$\text{Maximum $|$E}_\text{z}\text{(}\omega{}_\text{n}\text{)$|$}$},
axis background/.style={fill=white}
]
\addplot[only marks, mark=*, mark options={}, mark size=3.000pt, draw=black, fill=mycolor1] table[row sep=crcr]{%
x	y\\
-3	0.000000058939162\\
-2	0.000049831160601\\
-1	0.003480773455234\\
0	1.010334172885610\\
1	0.011021244259482\\
2	0.000107038587645\\
3	0.000002341788501\\
};
\end{axis}
\end{tikzpicture}%
	\caption[Maximum $|E_z(\omega_n)|$ field amplitude at each sideband $n$.]{The maximum field amplitude calculated for each sideband $n$ on a log scale. The field intensity decreases exponentially with $n$. Note that the amplitude of the sidebands is not symmetric about $n=0$.}
	\label{fig:sideamp}
\end{figure}

The results of the simulation for the modal profile and corresponding amplitudes in each mode are shown in Figure \ref{fig:weakMod}. There is an excellent agreement between CMT and FDFD in Figure \ref{sfig:cmtcompare}. However, FDFD shows minor oscillations along the length of the waveguide that CMT can not predict, identical to the oscillations in previous simulations with only temporal modulations. By performing a standard Fourier transform on the modal amplitude at frequency sideband $\omega_1+\Omega$, the minor oscillation is found to be at a period of $T=0.075  \mskip3mu \mu m$. This period corresponds exactly to the generation of backward propagating modes, $0.075  \mskip3mu \mu m = 1/(k_1+k_2)$.


\begin{figure}[h!]
	\centering
	\setlength{\figH}{1\linewidth}
	\setlength{\figW}{1\textwidth}
	\begin{subfigure}[t]{0.5\textwidth}
		% This file was created by matlab2tikz.
%
%The latest updates can be retrieved from
%  http://www.mathworks.com/matlabcentral/fileexchange/22022-matlab2tikz-matlab2tikz
%where you can also make suggestions and rate matlab2tikz.
%
\pgfplotsset{weakMod/.style={
	width=0.32\figW,
	height=0.15\figH,
	scale only axis,
	axis on top,
	xmin=0,
	xmax=10,
	ymin=-2.5,
	ymax=2.5,
	%xlabel style={font=\color{white!15!black}},
	%ylabel style={font=\color{white!15!black}},
	axis background/.style={fill=white},
	colormap={mymap}{[1pt] rgb(0pt)=(0,0,1); rgb(31pt)=(1,1,1); rgb(32pt)=(1,1,1); rgb(63pt)=(1,0,0)},
	colorbar,
	colorbar style={width=.03\linewidth, at={(1.08,0.075\figH)}, anchor=east}
	}
}

\begin{tikzpicture}

\begin{axis}[%
weakMod,
at={(0\figW,0.92\figH)},
point meta min=-0.153883565276573,
point meta max=0.153883565276573,
xticklabels={,,},
colorbar style={title = $V / \mu m$, width=.03\linewidth, at={(1.08,0.075\figH)}, anchor=east}
]
\addplot [forget plot] graphics [xmin=0, xmax=10, ymin=-2.5, ymax=2.5] {graphs/weakmod/nameoffile2-1.png};
\draw (0,-0.5) -- (10,-0.5);
\draw (0,0.5) -- (10,0.5);
\node at (5,1.8) {$\omega_1+2\Omega$};
\end{axis}

\begin{axis}[%
weakMod,
at={(0\figW,0.69\figH)},
point meta min=-1.37640682053953,
point meta max=1.37640682053953,
xticklabels={,,}
]
\addplot [forget plot] graphics [xmin=0, xmax=10, ymin=-2.5, ymax=2.5] {graphs/weakmod/nameoffile2-2.png};
\draw (0,-0.5) -- (10,-0.5);
\draw (0,0.5) -- (10,0.5);
\node at (5,1.8) {$\omega_1+\Omega$};
\end{axis}

\begin{axis}[%
weakMod,
at={(0\figW,0.46\figH)},
point meta min=-24.7127053691998,
point meta max=24.7127053691998,
xticklabels={,,},
ylabel={$\text{y (}\mu\text{m)}$}
]
\addplot [forget plot] graphics [xmin=0, xmax=10, ymin=-2.5, ymax=2.5] {graphs/weakmod/nameoffile2-3.png};
\draw (0,-0.5) -- (10,-0.5);
\draw (0,0.5) -- (10,0.5);
\node at (5,1.8) {$\omega_1$};
\end{axis}

\begin{axis}[%
weakMod,
at={(0\figW,0.23\figH)},
point meta min=-2.69144358712869,
point meta max=2.69144358712869,
xticklabels={,,}
]
\addplot [forget plot] graphics [xmin=0, xmax=10, ymin=-2.5, ymax=2.5] {graphs/weakmod/nameoffile2-4.png};
\draw (0,-0.5) -- (10,-0.5);
\draw (0,0.5) -- (10,0.5);
\node at (5,1.8) {$\omega_1-\Omega$};
\end{axis}

\begin{axis}[%
weakMod,
at={(0\figW,0\figH)},
point meta min=-0.251011435268562,
point meta max=0.251011435268562,
%xlabel style={font=\color{white!15!black}},
xlabel={$\text{x (}\mu\text{m)}$}
]
\addplot [forget plot] graphics [xmin=0, xmax=10, ymin=-2.5, ymax=2.5] {graphs/weakmod/nameoffile2-5.png};
\draw (0,-0.5) -- (10,-0.5);
\draw (0,0.5) -- (10,0.5);
\node at (5,1.8) {$\omega_1-2\Omega$};
\end{axis}
\node at (-1,16) {\textbf{(a)}};

\end{tikzpicture}%
		\phantomsubcaption
		\label{sfig:weakmod}
	\end{subfigure}%
	\begin{subfigure}[t]{0.5\textwidth}
		% This file was created by matlab2tikz.
%
%The latest updates can be retrieved from
%  http://www.mathworks.com/matlabcentral/fileexchange/22022-matlab2tikz-matlab2tikz
%where you can also make suggestions and rate matlab2tikz.
%
\definecolor{mycolor1}{rgb}{0.00000,0.44700,0.74100}%
\definecolor{mycolor2}{rgb}{0.85000,0.32500,0.09800}%
%
\begin{tikzpicture}

\begin{axis}[%
width=0.4\figW,
height=0.15\figH,
at={(0\figW,0.92\figH)},
scale only axis,
xmin=0,
xmax=10,
ymin=0,
ymax=0.00006,
axis background/.style={fill=white}
]
\addplot [color=mode1, forget plot, line width = 0.2mm]
  table[row sep=crcr]{%
0	5.43387557172537e-12\\
2.33082706766917	4.15554692878573e-07\\
2.58145363408521	8.04269815191105e-07\\
2.65664160401002	1.13458524886312e-06\\
2.78195488721805	2.75443581720936e-06\\
2.88220551378446	3.82980313062831e-06\\
2.98245614035088	3.84802184072441e-06\\
3.08270676691729	4.09176513294085e-06\\
3.15789473684211	5.51922256164517e-06\\
3.30827067669173	9.65933932128848e-06\\
3.35839598997494	1.01329342037104e-05\\
3.43358395989975	9.7503546498956e-06\\
3.50877192982456	9.34698401167111e-06\\
3.55889724310777	9.92427050050537e-06\\
3.60902255639098	1.14338577574813e-05\\
3.7593984962406	1.74508023960129e-05\\
3.80952380952381	1.79598131815339e-05\\
3.85964912280702	1.74420039549261e-05\\
3.95989974937343	1.5946668758815e-05\\
4.01002506265664	1.65844142241411e-05\\
4.06015037593985	1.86403991353501e-05\\
4.21052631578947	2.6945067729045e-05\\
4.26065162907268	2.74805266418099e-05\\
4.31077694235589	2.65258951159808e-05\\
4.3859649122807	2.42953819480363e-05\\
4.43609022556391	2.40423073325502e-05\\
4.46115288220551	2.46232051370754e-05\\
4.48621553884712	2.57137243551142e-05\\
4.53634085213033	2.91782066437207e-05\\
4.61152882205514	3.54291152433461e-05\\
4.63659147869674	3.70793879191922e-05\\
4.66165413533835	3.82557620248747e-05\\
4.68671679197995	3.88727722544502e-05\\
4.71177944862155	3.89096848198989e-05\\
4.76190476190476	3.74991024223448e-05\\
4.83709273182957	3.40173678701206e-05\\
4.86215538847118	3.32774330153995e-05\\
4.88721804511278	3.30322850192744e-05\\
4.91228070175439	3.33779784380539e-05\\
4.93734335839599	3.43402589280117e-05\\
4.9874686716792	3.78494222292858e-05\\
5.06265664160401	4.45703653060292e-05\\
5.08771929824561	4.63538804211794e-05\\
5.11278195488722	4.75991495889616e-05\\
5.13784461152882	4.81981292086431e-05\\
5.16290726817043	4.81164536676459e-05\\
5.18796992481203	4.73979463677665e-05\\
5.23809523809524	4.45784691347484e-05\\
5.28822055137845	4.12604051298615e-05\\
5.31328320802005	3.99650311706523e-05\\
5.33834586466165	3.91564508568365e-05\\
5.36340852130326	3.89468195542975e-05\\
5.38847117794486	3.93737780388648e-05\\
5.41353383458647	4.03960161818873e-05\\
5.46365914786968	4.37060878120121e-05\\
5.51378446115288	4.73716227489263e-05\\
5.53884711779449	4.8792712444623e-05\\
5.56390977443609	4.96983863111922e-05\\
5.58897243107769	4.99787618970515e-05\\
5.6140350877193	4.95976774299578e-05\\
5.6390977443609	4.85966020402628e-05\\
5.68922305764411	4.52463159970051e-05\\
5.73934837092732	4.13970089834237e-05\\
5.78947368421053	3.86834879577691e-05\\
5.81453634085213	3.81146134031951e-05\\
5.83959899749373	3.81368506499058e-05\\
5.88972431077694	3.97151141680041e-05\\
5.98997493734336	4.44888289692358e-05\\
6.01503759398496	4.4936094301562e-05\\
6.04010025062657	4.48320064965202e-05\\
6.06516290726817	4.41567615894201e-05\\
6.11528822055138	4.13568523640606e-05\\
6.2155388471178	3.43621038911834e-05\\
6.2406015037594	3.33363328692826e-05\\
6.265664160401	3.28240666522817e-05\\
6.29072681704261	3.28347284828112e-05\\
6.34085213032582	3.41444278788572e-05\\
6.41604010025063	3.71238134793117e-05\\
6.46616541353383	3.78701941219362e-05\\
6.49122807017544	3.75468914661781e-05\\
6.54135338345865	3.55134953480274e-05\\
6.66666666666667	2.78181785517972e-05\\
6.71679197994987	2.65938751233818e-05\\
6.76691729323308	2.70466771130629e-05\\
6.8922305764411	3.02023964877662e-05\\
6.94235588972431	2.97020206154741e-05\\
6.99248120300752	2.77162762500893e-05\\
7.11779448621554	2.1098852439394e-05\\
7.16791979949875	1.99746805016332e-05\\
7.21804511278195	2.00616584091762e-05\\
7.34335839598998	2.13945187006459e-05\\
7.39348370927318	2.06580379806809e-05\\
7.44360902255639	1.90186907857992e-05\\
7.54385964912281	1.51863402155783e-05\\
7.59398496240602	1.42171264769786e-05\\
7.64411027568922	1.41290804887007e-05\\
7.76942355889724	1.46969967094179e-05\\
7.84461152882206	1.39259300944161e-05\\
7.96992481203008	1.2162203580246e-05\\
8.04511278195489	1.22613284752049e-05\\
8.19548872180451	1.31556729527915e-05\\
8.39598997493734	1.29831159423333e-05\\
8.62155388471178	1.58658634550335e-05\\
8.7719298245614	1.55996883623999e-05\\
8.84711779448622	1.67615595980664e-05\\
9.04761904761905	2.10964253533064e-05\\
9.17293233082707	2.22552359279149e-05\\
9.32330827067669	2.26945466739181e-05\\
9.52380952380952	2.23158718224425e-05\\
9.62406015037594	2.15477535139286e-05\\
9.64912280701754	2.08446471621215e-05\\
9.67418546365915	1.95987671691711e-05\\
9.69924812030075	1.76196276182594e-05\\
9.72431077694236	1.48288148107412e-05\\
9.79949874686717	4.43934645133481e-06\\
9.82456140350877	2.07846284006052e-06\\
9.84962406015038	7.5191475090719e-07\\
9.89974937343358	3.57357858860041e-08\\
10	3.59374752179065e-11\\
};
\addplot [color=mode2, dashed, forget plot, line width = 0.5mm]
  table[row sep=crcr]{%
1.5	0\\
2.331	4.55758724982047e-07\\
2.655	1.59956065992617e-06\\
2.923	3.4474900711956e-06\\
3.168	6.04500525369644e-06\\
3.407	9.49516199177936e-06\\
3.654	1.39891369101974e-05\\
3.931	1.99746349487384e-05\\
4.352	3.01312148938138e-05\\
4.742	3.92109647950889e-05\\
4.985	4.39479584990465e-05\\
5.194	4.71192896878136e-05\\
5.388	4.91524537000743e-05\\
5.574	5.01925575964179e-05\\
5.759	5.0313296824811e-05\\
5.948	4.9516196257926e-05\\
6.147	4.77524459796541e-05\\
6.368	4.4856263265558e-05\\
6.64	4.03262877739508e-05\\
7.442	2.63712804393634e-05\\
7.676	2.35983383820582e-05\\
7.894	2.19240859902214e-05\\
8.11	2.11825016283029e-05\\
8.335	2.13363968892821e-05\\
8.586	2.24494528513475e-05\\
8.916	2.49038214494846e-05\\
9.001	2.56222738137524e-05\\
};
\end{axis}

\begin{axis}[%
width=0.4\figW,
height=0.15\figH,
at={(0\figW,0.69\figH)},
scale only axis,
xmin=0,
xmax=10,
ymin=0,
ymax=0.004,
axis background/.style={fill=white}
]
\addplot [color=mode1, forget plot, line width = 0.2mm]
  table[row sep=crcr]{%
0	4.54703830143899e-09\\
0.275689223057643	1.51941339936457e-05\\
0.451127819548873	2.52401768339183e-05\\
1.45363408521303	3.97304625625594e-05\\
1.57894736842105	6.51259869020038e-05\\
1.67919799498747	0.000175895551990379\\
1.75438596491228	0.000238974537602132\\
1.80451127819549	0.000246791504064703\\
1.92982456140351	0.000239343214227361\\
1.97994987468672	0.000305673061468781\\
2.03007518796993	0.000434091444263984\\
2.18045112781955	0.000912147145111675\\
2.23057644110276	0.000983880445240004\\
2.28070175438597	0.000991559270044462\\
2.35588972431078	0.000964418411161461\\
2.38095238095238	0.000972484533821927\\
2.40601503759398	0.000999816229947825\\
2.43107769423559	0.00105053818120737\\
2.45614035087719	0.00112635861105836\\
2.4812030075188	0.00122625651255603\\
2.53132832080201	0.00148084340935029\\
2.60651629072682	0.00188509814085691\\
2.63157894736842	0.00199144331246615\\
2.65664160401002	0.00207218449949842\\
2.68170426065163	0.0021240587850464\\
2.70676691729323	0.00214693945004818\\
2.73182957393484	0.00214393345262565\\
2.78195488721805	0.00208688515637867\\
2.83208020050125	0.00202400707654427\\
2.85714285714286	0.00201496718057115\\
2.88220551378446	0.00203150059589774\\
2.90726817042606	0.00207831371054645\\
2.93233082706767	0.00215671027678077\\
2.95739348370927	0.0022643620082814\\
3.00751879699248	0.0025414201334204\\
3.05764411027569	0.00283435053856174\\
3.08270676691729	0.00295887002931572\\
3.1077694235589	0.00305573849594509\\
3.1328320802005	0.00311834405742673\\
3.15789473684211	0.00314358226193256\\
3.18295739348371	0.00313224450067473\\
3.20802005012531	0.00308897408162245\\
3.25814536340852	0.00294122463663626\\
3.30827067669173	0.00278758958814684\\
3.33333333333333	0.00273712601422815\\
3.35839598997494	0.00271596254521178\\
3.38345864661654	0.00272886128914784\\
3.40852130325815	0.0027766139994565\\
3.43358395989975	0.00285592725976613\\
3.48370927318296	0.00307810386792262\\
3.53383458646617	0.0033101077573594\\
3.55889724310777	0.00339963070497085\\
3.58395989974937	0.00345805104941022\\
3.60902255639098	0.003478974154973\\
3.63408521303258	0.00345988347483761\\
3.65914786967419	0.0034024206088823\\
3.68421052631579	0.00331220269436017\\
3.734335839599	0.00307172899026398\\
3.78446115288221	0.00283081470250757\\
3.80952380952381	0.00273905339945557\\
3.83458646616541	0.00267770592568972\\
3.85964912280702	0.00265093954249096\\
3.88471177944862	0.00265888376477719\\
3.90977443609023	0.00269764826026275\\
3.95989974937343	0.00283515653665312\\
4.01002506265664	0.00297856489579118\\
4.03508771929825	0.00302381999546419\\
4.06015037593985	0.00303925153206919\\
4.08521303258145	0.00301948476103142\\
4.11027568922306	0.00296290260591903\\
4.13533834586466	0.00287177955390128\\
4.16040100250627	0.0027519817970667\\
4.28571428571429	0.00207037848208813\\
4.31077694235589	0.00198659640263799\\
4.33583959899749	0.00193412267554471\\
4.3609022556391	0.0019123927148641\\
4.3859649122807	0.0019174008621512\\
4.43609022556391	0.00197788032427049\\
4.48621553884712	0.0020417130894117\\
4.51127819548872	0.00205130671514731\\
4.53634085213033	0.00203658351918001\\
4.56140350877193	0.00199381962574208\\
4.58646616541354	0.00192244234541938\\
4.61152882205514	0.00182500414846665\\
4.66165413533835	0.00157523886027811\\
4.71177944862155	0.00130652783535545\\
4.73684210526316	0.00118622064949392\\
4.76190476190476	0.00108437914613191\\
4.78696741854637	0.00100516848711329\\
4.81203007518797	0.000950135940101404\\
4.83709273182957	0.000918158812790892\\
4.88721804511278	0.000907303483790756\\
4.96240601503759	0.000929034010150431\\
5.0125313283208	0.000896952321081557\\
5.06265664160401	0.00079894934214586\\
5.11278195488722	0.000645880618932893\\
5.18796992481203	0.000390709941234135\\
5.23809523809524	0.000258840265729532\\
5.28822055137845	0.000180937135505488\\
5.33834586466165	0.000151603943791656\\
5.51378446115288	0.000113800961187849\\
5.6390977443609	6.21770847697434e-05\\
5.78947368421053	5.24380355457765e-05\\
5.96491228070176	7.84555890245286e-05\\
6.01503759398496	0.000130318856328415\\
6.06516290726817	0.000225790194559039\\
6.14035087719298	0.000427707834578683\\
6.19047619047619	0.000563921729051842\\
6.2406015037594	0.000662944169352642\\
6.29072681704261	0.000708167027404727\\
6.36591478696742	0.000705086318014025\\
6.41604010025063	0.000713226201307648\\
6.44110275689223	0.000737790514449443\\
6.46616541353383	0.000781916048628872\\
6.49122807017544	0.000847750307238826\\
6.51629072681704	0.000935132553516738\\
6.56641604010025	0.00116195270580377\\
6.64160401002506	0.00153687222175236\\
6.66666666666667	0.00164039371462898\\
6.69172932330827	0.00172220950571145\\
6.71679197994987	0.00177864525240246\\
6.74185463659148	0.00180877963818382\\
6.76691729323308	0.00181464955592503\\
6.81704260651629	0.0017754218481727\\
6.8671679197995	0.00172355971527338\\
6.8922305764411	0.0017159814746055\\
6.91729323308271	0.00173128501731767\\
6.94235588972431	0.00177470332045004\\
6.96741854636591	0.00184841304357697\\
6.99248120300752	0.00195119297812596\\
7.04260651629073	0.00222268886600574\\
7.09273182957394	0.00252206211549932\\
7.11779448621554	0.00265552955625914\\
7.14285714285714	0.00276499269115327\\
7.16791979949875	0.00284319688935497\\
7.19298245614035	0.00288606816143044\\
7.21804511278195	0.00289322297041927\\
7.24310776942356	0.00286807274911283\\
7.29323308270677	0.00275116515563845\\
7.34335839598998	0.0026171503447916\\
7.36842105263158	0.0025723605928647\\
7.39348370927318	0.00255501862322838\\
7.41854636591479	0.0025709690230169\\
7.44360902255639	0.0026222874078492\\
7.468671679198	0.00270699266448382\\
7.4937343358396	0.00281922232698761\\
7.59398496240602	0.00333386858474327\\
7.61904761904762	0.00342005360670861\\
7.64411027568922	0.00347034676450342\\
7.66917293233083	0.00348072246716669\\
7.69423558897243	0.00345127595672778\\
7.71929824561404	0.00338621759606461\\
7.76942355889724	0.0031835055709788\\
7.81954887218045	0.00296169654041378\\
7.84461152882206	0.00287370758437611\\
7.86967418546366	0.00281382081844761\\
7.89473684210526	0.00278770924458982\\
7.91979949874687	0.00279708711230953\\
7.94486215538847	0.00283954727636626\\
7.99498746867168	0.00299568097608116\\
8.04511278195489	0.00317557211279151\\
8.07017543859649	0.00324493174914409\\
8.09523809523809	0.00328701658403574\\
8.1203007518797	0.00329504635814359\\
8.1453634085213	0.00326587523138322\\
8.17042606516291	0.00320029460965898\\
8.19548872180451	0.00310289184400503\\
8.24561403508772	0.00284620682331926\\
8.29573934837093	0.00257900767152819\\
8.32080200501253	0.00246800328467955\\
8.34586466165413	0.00238264575786573\\
8.37092731829574	0.00232704219100377\\
8.39598997493734	0.00230168183730228\\
8.42105263157895	0.0023034547439984\\
8.47117794486216	0.0023608828545445\\
8.52130325814536	0.00242677020097304\\
8.54636591478697	0.00243847452169277\\
8.57142857142857	0.00242582230027999\\
8.59649122807017	0.00238452321434934\\
8.62155388471178	0.00231358718829178\\
8.64661654135338	0.00221539591423436\\
8.69674185463659	0.0019614030422801\\
8.7468671679198	0.0016891717875378\\
8.7719298245614	0.00156967360948279\\
8.79699248120301	0.0014721207774766\\
8.82205513784461	0.00140848230980595\\
8.84711779448622	0.00137227520537486\\
8.87218045112782	0.0013617657327778\\
8.92230576441103	0.00139712649813362\\
8.99749373433584	0.00147122714021641\\
9.07268170426065	0.00145329727800636\\
9.3734335839599	0.0013250560013347\\
9.62406015037594	0.00124783686368879\\
9.64912280701754	0.0012153806504287\\
9.67418546365915	0.00115603137600218\\
9.69924812030075	0.00105899242203478\\
9.72431077694236	0.000917707467653628\\
9.74937343358396	0.000735750658021672\\
9.79949874686717	0.000334532332308513\\
9.82456140350877	0.000177091016421826\\
9.84962406015038	7.55717083951168e-05\\
9.87468671679198	2.48329641134859e-05\\
9.92481203007519	1.01810643471367e-06\\
10	3.50960167594394e-08\\
};
\addplot [color=mode2, dashed, forget plot, line width = 0.5mm]
  table[row sep=crcr]{%
1.5	0\\
1.595	1.81366181895015e-05\\
1.691	7.2936201254592e-05\\
1.789	0.000165516505362007\\
1.89	0.000297562146313268\\
1.996	0.000472849597157321\\
2.11	0.000698318376350926\\
2.238	0.00098909740936648\\
2.393	0.00137959209563299\\
2.924	0.00274999714226887\\
3.049	0.00301558292379589\\
3.161	0.0032174491599779\\
3.266	0.00337042199205051\\
3.366	0.00347990785575902\\
3.463	0.00354998377682136\\
3.559	0.00358283398596981\\
3.654	0.00357891808303989\\
3.75	0.00353823144911836\\
3.847	0.00346047486528356\\
3.947	0.00334348473430524\\
4.051	0.00318495524405904\\
4.162	0.00297859634470399\\
4.284	0.00271422107006458\\
4.426	0.00236842757769828\\
4.628	0.00183571919372127\\
4.908	0.00110120284994331\\
5.05	0.000767131146311328\\
5.171	0.000518486779633065\\
5.281	0.000328580024241631\\
5.385	0.000185550441884175\\
5.484	8.56532052431191e-05\\
5.581	2.42630140174782e-05\\
5.676	4.35594124326144e-07\\
5.771	1.30447211894591e-05\\
5.867	6.25689713160682e-05\\
5.964	0.000149153793218559\\
6.064	0.000274976345489009\\
6.169	0.00044400753160545\\
6.281	0.000661384115757002\\
6.405	0.000939483638061844\\
6.552	0.0013073519201825\\
6.783	0.00192797253868626\\
7.012	0.00253092624570783\\
7.153	0.00286536885477418\\
7.273	0.00311414323738113\\
7.383	0.00330593168093252\\
7.486	0.00344931267000881\\
7.585	0.00355087854623015\\
7.682	0.00361375554927079\\
7.777	0.00363890196027938\\
7.872	0.00362748182843831\\
7.968	0.00357903341738108\\
8.065	0.00349341411933501\\
8.165	0.00336846619642373\\
8.27	0.00320022883478543\\
8.382	0.0029835658798536\\
8.506	0.00270610342892397\\
8.652	0.0023413915868602\\
8.879	0.0017321512279338\\
9.001	0.0014044343992623\\
};
\end{axis}

\begin{axis}[%
width=0.4\figW,
height=0.15\figH,
at={(0\figW,0.46\figH)},
ylabel = {Modal amplitude ($W / \mu m$)},
scale only axis,
xmin=0,
xmax=10,
ymin=0,
ymax=2,
axis background/.style={fill=white}
]
\addplot [color=mode1, forget plot, line width = 0.2mm]
  table[row sep=crcr]{%
0	1.50510945928772e-08\\
0.125313283208021	0.00179850403735315\\
0.150375939849624	0.00966614669053101\\
0.175438596491228	0.0360685832511223\\
0.200501253132833	0.0989088719692397\\
0.225563909774436	0.210262645077451\\
0.325814536340852	0.822684478134947\\
0.350877192982455	0.911237862983644\\
0.375939849624061	0.964830080328762\\
0.401002506265664	0.992769123881761\\
0.451127819548873	1.0089030032407\\
0.676691729323307	1.00859319678078\\
1.15288220551378	1.00240237114422\\
1.47869674185464	1.00995609785085\\
2.13032581453634	0.997618441525015\\
2.73182957393484	1.00132909017038\\
3.1077694235589	0.996985353842565\\
4.28571428571429	0.988258978440136\\
5.08771929824561	0.990657665362935\\
5.73934837092732	0.989628160643383\\
7.4937343358396	0.993497815553269\\
7.86967418546366	0.99697657741603\\
9.12280701754386	1.00099989228537\\
9.59899749373434	0.994861806905559\\
9.62406015037594	0.983478728130972\\
9.64912280701754	0.956433550836699\\
9.67418546365915	0.903940686964638\\
9.69924812030075	0.816689581536787\\
9.74937343358396	0.532161672190799\\
9.79949874686717	0.209487714786331\\
9.82456140350877	0.0987105552727012\\
9.84962406015038	0.0360974381005637\\
9.87468671679198	0.00972470462840924\\
9.92481203007519	0.000227824409254396\\
10	2.40782860316813e-08\\
};
\addplot [color=mode2, dashed, forget plot, line width = 0.5mm]
  table[row sep=crcr]{%
1.5	1\\
3.394	0.990789631812531\\
5.321	0.989720107682176\\
9.001	0.998149177678249\\
};
\end{axis}

\begin{axis}[%
width=0.4\figW,
height=0.15\figH,
at={(0\figW,0.23\figH)},
scale only axis,
xmin=0,
xmax=10,
ymin=0,
ymax=0.015,
axis background/.style={fill=white}
]
\addplot [color=mode1, forget plot, line width = 0.2mm]
  table[row sep=crcr]{%
0	1.0334785471855e-08\\
1.52882205513784	8.91286863229368e-05\\
1.72932330827068	0.000162192430835262\\
1.82957393483709	0.000203465931416957\\
1.92982456140351	0.000488499419457611\\
2.03007518796993	0.000759971721924657\\
2.13032581453634	0.000804907956428735\\
2.23057644110276	0.00086087867152429\\
2.30576441102757	0.00115640656677485\\
2.4812030075188	0.00212739654024219\\
2.55639097744361	0.00220707404563569\\
2.65664160401002	0.00225328646424394\\
2.70676691729323	0.00245359266024892\\
2.75689223057644	0.00281537527066433\\
2.88220551378446	0.00390781636622783\\
2.93233082706767	0.00414003624184822\\
2.98245614035088	0.00419547725480207\\
3.1077694235589	0.00418275429504611\\
3.15789473684211	0.0044283331240127\\
3.20802005012531	0.00487189885382833\\
3.30827067669173	0.00592130953570624\\
3.35839598997494	0.00624249238328822\\
3.40852130325815	0.00633114025934667\\
3.55889724310777	0.00615878368470746\\
3.60902255639098	0.00641862030833629\\
3.65914786967419	0.00690055819496749\\
3.7593984962406	0.00797707918228951\\
3.80952380952381	0.00824780227696209\\
3.85964912280702	0.00825108767576488\\
4.01002506265664	0.00788157597239447\\
4.06015037593985	0.00814102711318121\\
4.11027568922306	0.00863268618364721\\
4.18546365914787	0.00945438794883913\\
4.23558897243108	0.00978337082934821\\
4.28571428571429	0.00981672236475006\\
4.33583959899749	0.00960278786567947\\
4.41102756892231	0.00919413170041672\\
4.46115288220551	0.00914344170984904\\
4.51127819548872	0.00938317697647761\\
4.56140350877193	0.00985221831382077\\
4.63659147869674	0.0105862715652503\\
4.68671679197995	0.0108119712086499\\
4.73684210526316	0.0107213810508\\
4.78696741854637	0.010386641025482\\
4.86215538847118	0.00984263020345644\\
4.91228070175439	0.00973482076136989\\
4.96240601503759	0.00992626043941591\\
5.11278195488722	0.0110118823516601\\
5.16290726817043	0.0109476817132528\\
5.21303258145363	0.0105919511421355\\
5.31328320802005	0.00968094619132565\\
5.36340852130326	0.00951101686375999\\
5.41353383458647	0.009635230869808\\
5.53884711779449	0.0103642517333036\\
5.58897243107769	0.0103334366675174\\
5.6390977443609	0.00999851032497112\\
5.76441102756892	0.00872227797359848\\
5.81453634085213	0.0085088978645711\\
5.86466165413534	0.00856963209730921\\
5.98997493734336	0.00902618174774794\\
6.04010025062657	0.00888463915835125\\
6.09022556390977	0.00847222360585498\\
6.2155388471178	0.00715156207638401\\
6.265664160401	0.0069258324724224\\
6.31578947368421	0.00694173203693715\\
6.41604010025063	0.00718185089157686\\
6.46616541353383	0.00708464272116061\\
6.51629072681704	0.00674125153080496\\
6.69172932330827	0.00509803807266351\\
6.74185463659148	0.00499614326590248\\
6.8922305764411	0.00506143227346989\\
6.94235588972431	0.00480303487302614\\
7.01754385964912	0.00411382142003802\\
7.09273182957394	0.00341997991273324\\
7.14285714285714	0.00314503464990068\\
7.19298245614035	0.00305317569515395\\
7.34335839598998	0.00301219062278157\\
7.39348370927318	0.00277090578585515\\
7.59398496240602	0.00146571965982112\\
7.66917293233083	0.00139262039728294\\
7.76942355889724	0.0013715542055035\\
7.84461152882206	0.00115027219622554\\
8.02005012531328	0.000406435684933371\\
8.09523809523809	0.000319565092700813\\
8.29573934837093	0.000198346306019559\\
8.47117794486216	2.97625669229973e-05\\
8.69674185463659	4.31207555493529e-05\\
8.82205513784461	0.000251495324420148\\
8.94736842105263	0.00047905610206378\\
9.09774436090226	0.000495042010426161\\
9.74937343358396	0.000286280968269992\\
9.84962406015038	3.13443212700548e-05\\
9.9749373433584	1.9732179801224e-07\\
10	6.47432116807067e-08\\
};
\addplot [color=mode2, dashed, forget plot, line width = 0.5mm]
  table[row sep=crcr]{%
1.5	0\\
1.712	9.11138449914972e-05\\
1.928	0.000367915050102141\\
2.152	0.00083993555118056\\
2.391	0.00152977430997048\\
2.657	0.00248527792823161\\
2.985	0.00385668749730428\\
3.95	0.00801175078471417\\
4.214	0.00888958138227025\\
4.452	0.00949985758846061\\
4.675	0.00989045262878108\\
4.891	0.0100868217521022\\
5.104	0.0100981044041841\\
5.319	0.00992565027084957\\
5.539	0.0095648240833377\\
5.77	0.00900059830803457\\
6.02	0.00820359909393531\\
6.308	0.00709607212055197\\
6.704	0.00537229021816188\\
7.321	0.00269044846926647\\
7.611	0.00162748913806077\\
7.861	0.00089133776752881\\
8.091	0.000394990404432249\\
8.311	0.000102561248818134\\
8.525	3.01898049670513e-07\\
8.738	8.09388707008196e-05\\
8.954	0.000345839594766417\\
9.001	0.000427107166496299\\
};
\end{axis}

\begin{axis}[%
width=0.4\figW,
height=0.15\figH,
at={(0\figW,0\figH)},
scale only axis,
xlabel={x ($\mu m$)},
xmin=0,
xmax=10,
ymin=0,
ymax=0.00015,
axis background/.style={fill=white},
legend style={legend cell align=left, align=left, draw=white!15!black},
legend pos=north west
]
\addplot [color=mode1, line width = 0.2mm]
  table[row sep=crcr]{%
0	2.867039938792e-12\\
2.40601503759398	7.90757198387837e-07\\
2.80701754385965	3.99854861221627e-06\\
2.95739348370927	7.78878324858567e-06\\
3.15789473684211	9.78730313860865e-06\\
3.23308270676692	1.41929743104896e-05\\
3.33333333333333	2.08860607493477e-05\\
3.38345864661654	2.26470783122323e-05\\
3.45864661654135	2.28839482740995e-05\\
3.53383458646617	2.29100306317065e-05\\
3.58395989974937	2.46636345142548e-05\\
3.63408521303258	2.83291851861378e-05\\
3.78446115288221	4.21549968674384e-05\\
3.83458646616541	4.39596924284302e-05\\
3.90977443609023	4.37881918724514e-05\\
3.98496240601504	4.37943280910957e-05\\
4.03508771929825	4.57418582957558e-05\\
4.08521303258145	4.94502982899547e-05\\
4.21052631578947	6.03397836638209e-05\\
4.26065162907268	6.27063193974209e-05\\
4.33583959899749	6.36243390559343e-05\\
4.43609022556391	6.38749942645234e-05\\
4.51127819548872	6.60369802094607e-05\\
4.63659147869674	7.25358222304351e-05\\
4.73684210526316	7.71178862848387e-05\\
4.83709273182957	7.95947038305656e-05\\
5.06265664160401	8.22972528258248e-05\\
5.13784461152882	8.65212494396417e-05\\
5.23809523809524	9.27042564029534e-05\\
5.28822055137845	9.41744995497373e-05\\
5.36340852130326	9.38378327610678e-05\\
5.43859649122807	9.29683716552887e-05\\
5.48872180451128	9.3877521630148e-05\\
5.53884711779449	9.64498603241992e-05\\
5.66416040100251	0.000104673694480084\\
5.71428571428572	0.000105712203295028\\
5.78947368421053	0.000104332814657937\\
5.86466165413534	0.000102751184163807\\
5.91478696741855	0.000103322025802655\\
6.09022556390977	0.000108208514072672\\
6.14035087719298	0.000106420457026246\\
6.265664160401	9.99566150419184e-05\\
6.31578947368421	9.95463488813186e-05\\
6.44110275689223	0.000100578440557442\\
6.49122807017544	9.80956101184205e-05\\
6.54135338345865	9.32423782717962e-05\\
6.64160401002506	8.21498012175681e-05\\
6.69172932330827	7.93236840799239e-05\\
6.74185463659148	7.89497171691522e-05\\
6.81704260651629	7.97990495797762e-05\\
6.8671679197995	7.83180754915236e-05\\
6.91729323308271	7.381476437196e-05\\
6.96741854636591	6.68900966083186e-05\\
7.04260651629073	5.6383005519578e-05\\
7.09273182957394	5.23106304051169e-05\\
7.14285714285714	5.1402945411283e-05\\
7.26817042606516	5.30760965347099e-05\\
7.31829573934837	5.03487801726266e-05\\
7.36842105263158	4.5092475380315e-05\\
7.44360902255639	3.62233915804921e-05\\
7.4937343358396	3.26007535509376e-05\\
7.54385964912281	3.19178086023442e-05\\
7.61904761904762	3.44712308937289e-05\\
7.66917293233083	3.58869770149539e-05\\
7.71929824561404	3.53388134719523e-05\\
7.76942355889724	3.26760603304166e-05\\
7.86967418546366	2.59258394681439e-05\\
7.91979949874687	2.47171461857931e-05\\
7.96992481203008	2.55586886979131e-05\\
8.09523809523809	2.99274436077468e-05\\
8.1453634085213	2.97674458273889e-05\\
8.24561403508772	2.65507149208588e-05\\
8.32080200501253	2.51422360726394e-05\\
8.39598997493734	2.60549724071524e-05\\
8.52130325814536	2.81601320821778e-05\\
8.64661654135338	2.72483284096126e-05\\
8.7719298245614	2.72339336113703e-05\\
8.99749373433584	2.80853156997551e-05\\
9.64912280701754	2.87355472163853e-05\\
9.67418546365915	2.72365468507729e-05\\
9.69924812030075	2.4678112664489e-05\\
9.72431077694236	2.09278856431183e-05\\
9.79949874686717	6.43006030820459e-06\\
9.82456140350877	3.05400985212145e-06\\
9.84962406015038	1.131093370077e-06\\
9.89974937343358	6.08494712395213e-08\\
10	1.88542514933943e-11\\
};
\addlegendentry{FDFD}

\addplot [color=mode2, dashed, line width = 0.5mm]
  table[row sep=crcr]{%
1.5	0\\
2.494	9.13235737698415e-07\\
2.892	3.22639673022707e-06\\
3.231	7.00985234658447e-06\\
3.557	1.24830570076284e-05\\
3.906	2.02189273199593e-05\\
4.383	3.28281976500477e-05\\
4.947	4.73778926952662e-05\\
5.255	5.34289222180462e-05\\
5.523	5.68879617812712e-05\\
5.777	5.83428735279057e-05\\
6.03	5.79529625355235e-05\\
6.295	5.5685320724308e-05\\
6.592	5.12575586029129e-05\\
6.991	4.33129694119572e-05\\
7.672	2.96686321306083e-05\\
7.996	2.52192394754047e-05\\
8.297	2.29184988604914e-05\\
8.605	2.24182578651977e-05\\
8.959	2.37473373569941e-05\\
9.001	2.40038747154614e-05\\
};
\addlegendentry{CMT}

\end{axis}
\node at (-1,16) {\textbf{(b)}};
\end{tikzpicture}%
		\phantomsubcaption
		\label{sfig:cmtcompare}
	\end{subfigure}
	\caption[Electric field profiles for the weakly modulated waveguide]{\textbf{(a)} The modal field profiles of the transverse electric field $E_z$ propagating along the length of the waveguide structure. Each graph represents a sideband of the original frequency $\omega_1$ separated by integer multiples of the modulation frequency $n \Omega$. \textbf{(b)} The corresponding modal amplitudes of each sideband along the structure as calculated theoretically via CMT (dashed) and through the FDFD simulation (solid). }
	\label{fig:weakMod}
\end{figure}

\newpage

\section{Non-reciprocal mode conversion through indirect transitions}
\label{nonrec}
\begin{figure}[b!]	
	\centering
	\setlength{\figH}{\textwidth}
	\setlength{\figW}{\textwidth}
	% This file was created by matlab2tikz.
%
%The latest updates can be retrieved from
%  http://www.mathworks.com/matlabcentral/fileexchange/22022-matlab2tikz-matlab2tikz
%where you can also make suggestions and rate matlab2tikz.
%
%
\begin{tikzpicture}

\begin{axis}[%
width=0.8\figW,
height=0.4\figH,
at={(0\figW,0\figH)},
scale only axis,
xmin=-3,
xmax=3,
xlabel style={font=\color{white!15!black}},
xlabel={Wavevector $k$ ($2 \pi /a$)},
ymin=0,
ymax=1,
ylabel style={font=\color{white!15!black}},
ylabel={Frequency $\omega$ ($2\pi c/a$)},
axis background/.style={fill=white},
legend pos=south east
]
%\draw[dashed] (-3,0.6468) -- (3,0.6468);
%\draw[dashed] (-3,0.8879) -- (3,0.8879);
%\node at (2.8,0.767) {$\Omega$};
%\draw[dashed] (1.836,0.6468) -- (1.836,0);
%\draw[dashed] (1.367,0.8879) -- (1.367,0);
\draw[->] (1.836,0.6468) -- (1.367,0.8879);
\node[label={90:{$\ket{1}$}},circle,fill=mode1,inner sep=2pt] at (1.836,0.6468) {};
\node[label={90:{$\ket{2}$}},circle,fill=mode2,inner sep=2pt] at (1.367,0.8879) {};
%\node[label={180:{}},circle,fill=mode3,inner sep=2pt] at (-2.305,0.8879) {};
\addplot [color=mode1, forget plot, line width=1.0pt]
table[row sep=crcr]{%
	-0	0\\
	-0.0425483776636622	0.04\\
	-0.112747396858924	0.0920000000000001\\
	-0.138703575229707	0.108\\
	-0.242619225661566	0.16\\
	-0.530251013303014	0.268\\
	-0.68277422737238	0.317\\
	-1.13395473723024	0.451\\
	-1.40367460079584	0.527\\
	-3.00095872332965	0.965\\
};
\addlegendentry{Solver};
\addplot [color=mode1, forget plot, line width=1.0pt]
table[row sep=crcr]{%
	-0	0.649\\
	-0.326077818628705	0.656\\
	-0.605487195406852	0.672\\
	-0.681620546537656	0.68\\
	-0.720361604419702	0.706\\
	-0.780889623877291	0.731\\
	-0.822707823290201	0.745\\
	-0.979328562522052	0.791\\
	-1.40286353275794	0.896\\
	-1.58333534884432	0.938\\
	-1.73350139501897	0.973\\
	-1.85486453299797	1.001\\
};
\addplot [color=mode1, forget plot, line width=1.0pt]
  table[row sep=crcr]{%
0.00856527091199988	0.0089999999999999\\
0.0355740339675572	0.0329999999999999\\
0.0773642883861596	0.0680000000000001\\
0.108965071776361	0.0899999999999999\\
0.216446056096282	0.148\\
0.266469085354788	0.17\\
0.44579687604775	0.239\\
0.779460193245421	0.347\\
0.90346776025157	0.384\\
2.38185947042077	0.796\\
3.00095872332965	0.965\\
};
\addplot [color=mode1, forget plot, line width=1.0pt]
  table[row sep=crcr]{%
0	0.649\\
0.241290268767007	0.653\\
0.432426865263543	0.661\\
0.681620546537656	0.68\\
0.70634705751922	0.698\\
0.812185645324833	0.742\\
0.983207321695099	0.792\\
1.48370503209761	0.915\\
1.85486453299797	1.001\\
};
\addplot [name path=A, color=black, line width=1.0pt]
  table[row sep=crcr]{%
0.00996678000000006	0.00996678000000006\\
1.00664	1.00664\\
};
\addplot [color=mode3, dashed, line width=1.0pt]
  table[row sep=crcr]{%
0.00996678000000006	0.00783098999999998\\
0.149502	0.108661\\
0.199336	0.137065\\
0.259136	0.166038\\
0.33887	0.199306\\
0.448505	0.239874\\
0.598007	0.290189\\
0.797342	0.352394\\
1.07641	0.434535\\
1.48505	0.549875\\
3	0.965172\\
};
\addlegendentry{MPB};
\addplot [color=mode3, dashed, line width=1.0pt, forget plot]
  table[row sep=crcr]{%
0.00996678000000006	0.248465\\
0.0398670999999999	0.251399\\
0.0797342000000001	0.260564\\
0.159468	0.294355\\
0.209302	0.32355\\
0.269103	0.364287\\
0.33887	0.417217\\
0.657807	0.673008\\
0.697674	0.695525\\
0.747508	0.71813\\
0.807309	0.740237\\
0.89701	0.768316\\
1.02658	0.803978\\
1.22591	0.85408\\
1.85382	1.00166\\
};
\addplot [name path=B, line width=1.0pt, color=black]
  table[row sep=crcr]{%
0	0\\
-1.00664	1.00664\\
};

\addplot [color=mode3, dashed, forget plot, line width=1.0pt]
  table[row sep=crcr]{%
-0	0\\
-0.119601	0.0893928000000002\\
-0.199336	0.137065\\
-0.289037	0.179054\\
-0.418605	0.229214\\
-0.61794	0.296612\\
-0.92691	0.391046\\
-1.42525	0.533211\\
-2.51163	0.831749\\
-3	0.965172\\
};
\addplot [color=mode3, dashed, forget plot, line width=1.0pt]
  table[row sep=crcr]{%
-0	0.248268\\
-0.0498339000000001	0.253143\\
-0.0996678	0.267229\\
-0.159468	0.294355\\
-0.229236	0.336534\\
-0.318937	0.401633\\
-0.458472	0.515573\\
-0.627907	0.653101\\
-0.687708	0.690319\\
-0.757475	0.722111\\
-0.857143	0.756346\\
-1.02658	0.803978\\
-1.33555	0.880479\\
-1.85382	1.00166\\
};
\end{axis}



\end{tikzpicture}%
	\caption[Dispersion for a waveguide of width $0.22  \mskip3mu \mu m$ and relative permittivity $12.25$.]{Dispersion relation for the waveguide of width $0.22  \mskip3mu \mu m$ and relative permittivity $\epsilon=12.25$, compared between the analytic solver and the open-source \textit{MPB}.}
	\label{fig:bandfinal}
\end{figure} 

Having extensively validated the implementation of the frequency domain method, non-reciprocal mode transition is now demonstrated for a spatio-temporal direct permittivity modulation. The simulation is chosen to be a thin slab dielectric waveguide of width $0.22  \mskip3mu \mu m$ and permittivity $\epsilon = 12.25$, corresponding to that of silicon. This particular waveguide is chosen as it possesses highly parallel bands in the dispersion relation (Fig \ref{fig:bandfinal}). Being parallel, there will almost always be a phase-matched pair of modes in the forward ($+k$) direction. So that for any even mode $\ket{1}$, a single modulation profile will ensure a transition to a corresponding odd mode $\ket{2}$ over a wide range of frequencies. This is ideal from an optical isolation point of view, which requires an ideally infinite frequency operation range. Likewise, in the backward propagating direction ($-k$), there is no single phase-matched pair of modes for a transition. Thus, a single modulation between the even and odd modes on this choice of waveguide provides highly \textit{broadband} optical isolation.

The simulation is run for an even and odd mode pair at frequencies $\omega_1 = 0.647 \mskip3mu  (2 \pi c/a)$ and $\omega_2 = 0.8879 \mskip3mu  (2\pi c/a)$ respectively, chosen from the calculated dispersion relation of Figure \ref{fig:bandfinal}. These modes have corresponding wavevectors $k_1 = 1.189 \mskip3mu  (2 \pi /a)$ and $k_2 = 0.912  \mskip3mu (2 \pi/a)$. A permittivity modulation of the form $\epsilon(x,t)' = \delta \cos (\Omega t + \Lambda x)$ is applied between $2.5  \mskip3mu \mu m$ and $7.5  \mskip3mu \mu m$ ($L_c = 5  \mskip3mu \mu m$) where $\Omega = \omega_2-\omega_1$ is the modulation frequency and $\Lambda = k_2 - k_1$. The upper and lower halves of the waveguide have a $\pi$ phase difference. The coherence length was determined by extending the modulation over a wider range and sampling the mode amplitudes as usual, with a strong modulation of $\delta = 1$. A modal source is excited at $1  \mskip3mu \mu m$, and frequencies are extracted at positions $D_1 = 1.5 \mskip3mu  \mu m$ and $D_2 = 9 \mskip3mu  \mu m$ over 5 sidebands at a spatial discretisation of $\Delta = 0.04  \mskip3mu \mu m$.

%% This file was created by matlab2tikz.
%
%The latest updates can be retrieved from
%  http://www.mathworks.com/matlabcentral/fileexchange/22022-matlab2tikz-matlab2tikz
%where you can also make suggestions and rate matlab2tikz.
%
\definecolor{mycolor1}{rgb}{0.00000,0.44700,0.74100}%
\definecolor{mycolor2}{rgb}{0.85000,0.32500,0.09800}%
\definecolor{mycolor3}{rgb}{0.92900,0.69400,0.12500}%
\definecolor{mycolor4}{rgb}{0.49400,0.18400,0.55600}%
%
\begin{tikzpicture}

\begin{axis}[%
width=0.4\figW,
height=0.5\figH,
at={(0\figW,0\figH)},
scale only axis,
xmin=0,
xmax=5,
ymin=0,
ymax=1,
xlabel = {Modulated length ($\mu m$)},
ylabel = {Photon flux ($n$)},
axis background/.style={fill=white},
axis x line*=bottom,
axis y line*=left
]
\addplot [color=mode1]
  table[row sep=crcr]{%
0.00999999999999979	1\\
0.13	0.992\\
0.2	0.988\\
0.29	0.975\\
0.35	0.967\\
0.55	0.943\\
0.7	0.917\\
0.77	0.906\\
1.08	0.84\\
1.15	0.823\\
1.31	0.787\\
1.49	0.737\\
1.58	0.715\\
2.86	0.335\\
2.93	0.317\\
3.07	0.279\\
3.15	0.259\\
3.26	0.23\\
3.39	0.2\\
3.44	0.189\\
3.59	0.155\\
3.64	0.144\\
3.9	0.0961999999999996\\
4.11	0.0646000000000004\\
4.35	0.0351999999999997\\
4.59	0.0152999999999999\\
4.8	0.00460999999999956\\
5.01	0.00107000000000035\\
};
\addplot [color=mode2]
  table[row sep=crcr]{%
0.00999999999999979	0.00076699999999974\\
0.17	0.00628000000000029\\
0.49	0.0303000000000004\\
0.66	0.0549999999999997\\
0.8	0.0716000000000001\\
0.9	0.0978000000000003\\
0.96	0.113\\
1.05	0.127\\
1.13	0.139\\
1.19	0.155\\
1.25	0.177\\
1.32	0.202\\
1.55	0.251\\
1.73	0.32\\
1.81	0.332\\
1.86	0.341\\
1.92	0.363\\
1.96	0.382\\
2.01	0.408\\
2.09	0.436\\
2.15	0.445\\
2.19	0.449\\
2.26	0.467\\
2.31	0.491\\
2.43	0.549\\
2.51	0.561\\
2.6	0.573\\
2.65	0.592\\
2.81	0.668\\
2.92	0.675\\
2.95	0.678\\
2.99	0.689\\
3.04	0.712\\
3.12	0.756\\
3.17	0.773\\
3.24	0.776\\
3.3	0.774\\
3.35	0.783\\
3.42	0.815\\
3.48	0.847\\
3.53	0.863\\
3.58	0.865\\
3.69	0.858\\
3.72	0.864\\
3.76	0.879\\
3.86	0.926\\
3.9	0.935000000000001\\
3.96	0.931\\
4.05	0.917\\
4.09	0.921\\
4.14	0.939\\
4.21	0.97\\
4.24	0.978\\
4.27	0.981\\
4.3	0.978\\
4.36	0.962\\
4.41	0.951\\
4.45	0.951\\
4.5	0.965\\
4.54	0.981\\
4.59	0.997\\
4.64	0.999\\
4.7	0.983\\
4.77	0.959\\
4.81	0.955\\
4.84	0.958\\
4.88	0.969\\
4.93	0.985\\
4.96	0.991\\
4.99	0.991\\
5.01	0.988\\
};
\addplot [color=mode1, dashed]
  table[row sep=crcr]{%
0.00999999999999979	0.999990209\\
0.22	0.99526858\\
0.43	0.982005207\\
0.64	0.96042884\\
0.86	0.929316078\\
1.09	0.888113089\\
1.33	0.836575682\\
1.6	0.769602103\\
1.9	0.686271286\\
2.29	0.568622403\\
3.2	0.290741175\\
3.51	0.207120884\\
3.78	0.143143349\\
4.02	0.094757081\\
4.25	0.0569368749999999\\
4.47	0.0293268749999998\\
4.68	0.0112758949999998\\
4.89	0.00165378700000041\\
5.01	9.7910900000997e-06\\
};
\addplot [color=mode2, dashed]
  table[row sep=crcr]{%
0.00999999999999979	9.7910900000997e-06\\
0.22	0.00473141999999971\\
0.43	0.0179947929999997\\
0.64	0.0395711600000004\\
0.86	0.0706839219999997\\
1.09	0.111886911\\
1.33	0.163424318\\
1.6	0.230397897\\
1.9	0.313728714\\
2.29	0.431377597\\
3.2	0.709258825\\
3.51	0.792879116\\
3.78	0.856856651\\
4.02	0.905242919\\
4.25	0.943063125\\
4.47	0.970673125\\
4.68	0.988724105\\
4.89	0.998346213\\
5.01	0.999990209\\
};
\end{axis}
\end{tikzpicture}%
Figure \ref{fig:bandyu} shows the results of the simulation for the forward, backward, and time-reversed simulations. Only the total combined field of all sidebands is shown ($\sum E_z(\omega_n)$), as opposed to each individual sideband. The system exhibits a strongly non-reciprocal response - on the left to right path, the mode completely converts from $\ket{1}$ to $\ket{2}$. However, on the right to left direction the modulation does not cause a transition since there is no corresponding point on the dispersion relation for the mode to `jump'. Similarly, on the time-reversed path the modulation does not transition the odd mode to the second even mode (the third band on the dispersion relation). These non-reciprocal mode conversions can be removed using standard modal filters, allowing for complete isolation \cite{Lee2003,Jiao2005}.

\begin{figure}[t]
	\centering
	\setlength{\figH}{0.4\textwidth}
	\setlength{\figW}{\textwidth}
	\begin{subfigure}[t]{0.5\textwidth}
		% This file was created by matlab2tikz.
%
%The latest updates can be retrieved from
%  http://www.mathworks.com/matlabcentral/fileexchange/22022-matlab2tikz-matlab2tikz
%where you can also make suggestions and rate matlab2tikz.
%
\begin{tikzpicture}

\begin{axis}[%
width=0.35\figW,
height=0.2\figH,
at={(0\figW,0.75\figW)},
scale only axis,
point meta min=-1,
point meta max=1,
axis on top,
clip=false,
xmin=0,
xmax=10,
ymin=0,
ymax=1,
xlabel = {x ($\mu m$)},
ytick={0,0.5,1},
yticklabels={$0$,$0.5$, $1$},
ylabel={$\text{y (}\mu\text{m)}$},
axis background/.style={fill=white},
colormap={mymap}{[1pt] rgb(0pt)=(0,0,1); rgb(31pt)=(1,1,1); rgb(32pt)=(1,1,1); rgb(63pt)=(1,0,0)},
colorbar,
colorbar style={width=.02\linewidth, at={(1.05,0.1\figH)}, anchor=east,ytick={-1,1}},
colorbar style={title={$V / \mu m$}}
]
\addplot graphics [xmin=0, xmax=10, ymin=0, ymax=1] {graphs/fdtd/phasematch/LR/field-1.png};
\node at (-1,2) {\textbf{(a)}};
\node at (2,1.2) {$\ket{1} \rightarrow$};
\node at (8,1.2) {$\rightarrow \ket{2}$};
\end{axis}

\begin{axis}[%
width=0.35\figW,
height=0.2\figH,
at={(0\figW,0.5\figW)},
scale only axis,
point meta min=-1,
point meta max=1,
axis on top,
xmin=0,
clip=false,
xmax=10,
ymin=0,
ymax=1,
xlabel = {x ($\mu m$)},
ytick={0,0.5,1},
yticklabels={$0$,$0.5$, $1$},
ylabel={$\text{y (}\mu\text{m)}$},
axis background/.style={fill=white},
colormap={mymap}{[1pt] rgb(0pt)=(0,0,1); rgb(31pt)=(1,1,1); rgb(32pt)=(1,1,1); rgb(63pt)=(1,0,0)},
colorbar,
colorbar style={width=.02\linewidth, at={(1.05,0.1\figH)}, anchor=east,ytick={-1,1}}
]
\addplot [forget plot] graphics [xmin=0, xmax=10, ymin=0, ymax=1] {graphs/fdtd/phasematch/RL/fieldRL-1.png};
\node at (-1,2) {\textbf{(b)}};
\node at (2,1.2) {$\bra{1} \leftarrow$};
\node at (8,1.2) {$\leftarrow \bra{1}$};
\end{axis}

\begin{axis}[%
width=0.35\figW,
height=0.2\figH,
at={(0\figW,0.25\figW)},
scale only axis,
point meta min=-1,
point meta max=1,
axis on top,
xmin=0,
xmax=10,
ymin=0,
ymax=1,
ytick={0,0.5,1},
clip=false,
xlabel = {x ($\mu m$)},
yticklabels={$0$,$0.5$, $1$},
ylabel={$\text{y (}\mu\text{m)}$},
axis background/.style={fill=white},
colormap={mymap}{[1pt] rgb(0pt)=(0,0,1); rgb(31pt)=(1,1,1); rgb(32pt)=(1,1,1); rgb(63pt)=(1,0,0)},
colorbar,
colorbar style={width=.02\linewidth, at={(1.05,0.1\figH)}, anchor=east,ytick={-1,1}}
]
\addplot [forget plot] graphics [xmin=0, xmax=10, ymin=0, ymax=1] {graphs/fdtd/phasematch/TR/fieldTR-1.png};
\node at (-1,2) {\textbf{(c)}};
\node at (2,1.2) {$\bra{2} \leftarrow$};
\node at (8,1.2) {$\leftarrow \bra{2}$};
\end{axis}
\end{tikzpicture}%
	\end{subfigure}%
	\begin{subfigure}[t]{0.5\textwidth}
		% This file was created by matlab2tikz.
%
%The latest updates can be retrieved from
%  http://www.mathworks.com/matlabcentral/fileexchange/22022-matlab2tikz-matlab2tikz
%where you can also make suggestions and rate matlab2tikz.
%
\begin{tikzpicture}

\begin{axis}[%
width=0.4\figW,
height=0.5\figH,
at={(0\figW,1.2\figH)},
scale only axis,
xmin=0.3,
xmax=1,
xtick={0,0.6468,0.8879,1},
xticklabels={$0$,$0.6468$, $0.8879$,$1$},
ymin=0,
ymax=1,
axis background/.style={fill=white},
legend pos = north west
]
\addplot [color=mode2]
  table[row sep=crcr]{%
0	1.13081867070264e-06\\
0.831674911128847	0.00228169467746064\\
0.838881336867126	0.00202377499498518\\
0.842618002064753	0.0021823789198463\\
0.845820857948432	0.000498850577067556\\
0.849824427803032	0.00520157028131374\\
0.853827997657632	0.000158448678008183\\
0.858365376826178	0.00870591815290878\\
0.862102042023805	5.71739905264046e-05\\
0.865304897907484	0.0133645158166602\\
0.866639421192351	0.0173069295692947\\
0.869041563105111	0.00685264541172503\\
0.870642991046951	5.45500426432088e-05\\
0.872244418988791	0.00968327963340587\\
0.87544727487247	0.048612317596354\\
0.87704870281431	0.0344557748770749\\
0.879183940070097	0.000143863495585039\\
0.88025155869799	0.0162957894625595\\
0.882119891296803	0.170638303761119\\
0.887991793750216	1\\
0.889059412378109	0.94000608315541\\
0.893062982232709	0.255338855812756\\
0.896532742773362	4.8698814648418e-05\\
0.898134170715202	0.0209729769593403\\
0.900269407970989	0.0458723635961076\\
0.902404645226775	0.0252393913145816\\
0.905340596453482	9.65076844801072e-05\\
0.909344166308081	0.0156763040359307\\
0.914148450133601	0.000139489728146369\\
0.918152019988201	0.00780486864007757\\
0.92295630381372	0.000160502704093402\\
0.927226778325293	0.00448871207940704\\
0.932031062150813	0.00033301513591999\\
0.936835345976332	0.00232446948495424\\
0.941639629801852	0.000858887074389303\\
0.946177008970398	0.00117400414562741\\
0.950981292795918	0.00101797533071468\\
0.962458193045771	0.00106641777035499\\
0.969397714127077	0.000883365111032708\\
1.00196008227782	0.000118551983291804\\
1.02357935949266	0.000217825320773857\\
1.05854386955616	0.000104712082806824\\
1.15596406935142	4.98964502413379e-05\\
1.3342563802096	7.9476366254827e-06\\
};
\addlegendentry{$D_2$};
\addplot [color=mode1]
  table[row sep=crcr]{%
0	1.39644351637713e-08\\
0.609076427213103	0.00233545672979218\\
0.616282852951382	0.00430043014561177\\
0.620553327462955	0.000248637961077769\\
0.623756183346635	0.00794154826945515\\
0.626959039230315	7.47971153032267e-05\\
0.630428799770968	0.0162317707623461\\
0.632830941683728	0.00204526366888014\\
0.634699274282541	0.00886138610265252\\
0.637368320852274	0.0457755204955326\\
0.638969748794114	0.0193771513115848\\
0.640037367422007	0.000188198916072357\\
0.641104986049901	0.0271105460388243\\
0.642973318648714	0.289533327408285\\
0.64670998384634	1\\
0.647777602474233	0.934370010222459\\
0.653382600270673	0.00025932622168523\\
0.654717123555539	0.0216863788474451\\
0.656318551497379	0.0470796939868294\\
0.657653074782246	0.0325194712216197\\
0.660055216695006	3.53592225057486e-05\\
0.663258072578686	0.0169734429174946\\
0.666994737776312	8.14290493287295e-05\\
0.669930689003018	0.00862561475450141\\
0.673667354200645	4.6983461483352e-05\\
0.676870210084324	0.00514223972653949\\
0.680606875281951	0.000200359717916987\\
0.683809731165631	0.00342867981165518\\
0.687546396363257	0.000316680510486389\\
0.69101615690391	0.00206216482641941\\
0.69501972675851	0.000905361107496283\\
0.699557105927056	0.000132641242323706\\
0.706496627008362	3.5452938580427e-05\\
0.713436148089669	9.90346817442145e-07\\
0.720642573827948	4.63588478374355e-05\\
0.730251141478987	0.000719353650030063\\
0.741461137071867	0.000186432024472216\\
0.758276130461185	0.000260226636216387\\
0.782297549588784	0.00020597772171671\\
1.3342563802096	1.2617425668715e-06\\
};
\addlegendentry{$D_1$};
\end{axis}

\begin{axis}[%
width=0.4\figW,
height=0.5\figH,
at={(0\figW,0.6\figH)},
scale only axis,
xmin=0.3,
xmax=1,
xtick={0,0.6468,0.8879,1},
xticklabels={$0$,$0.6468$, $0.8879$,$1$},
ymin=0,
ymax=1,
axis background/.style={fill=white},
ylabel={Photon flux $n$}
]
\addplot [color=mode2]
table[row sep=crcr]{%
	0	2.23814667652533e-05\\
	0.599467859562064	0.00433628591280777\\
	0.607208094614289	0.00142683721257098\\
	0.611478569125863	0.00804573548853171\\
	0.616015948294409	2.38495126281268e-05\\
	0.621620946090848	0.0165221829549806\\
	0.626425229916368	0.000103599094986251\\
	0.630695704427941	0.0391146642895677\\
	0.632030227712808	0.0463381738836528\\
	0.634966178939514	0.0150116981517305\\
	0.636567606881354	0.000103582077880748\\
	0.638702844137141	0.0650888340040292\\
	0.64350712796266	0.709716332149051\\
	0.64670998384634	1\\
	0.64831141178818	0.928598166077518\\
	0.657119265468299	5.24822337082398e-05\\
	0.661389739979872	0.0480230679381304\\
	0.665660214491445	0.00758763999999168\\
	0.667528547090259	6.15704802147121e-05\\
	0.672065926258805	0.0163909949062666\\
	0.677937828712218	0.000108656655533057\\
	0.682742112537738	0.00847027912289966\\
	0.688347110334177	9.69307907228156e-05\\
	0.693685203473643	0.00469925536614135\\
	0.699290201270083	0.000427018840210458\\
	0.706763531665336	0.000799353937359193\\
	0.73692375790332	0.000616286557307388\\
	1.04172887616684	6.40938787921375e-05\\
	1.3342563802096	8.46451827918315e-06\\
};
\addplot [color=mode1]
table[row sep=crcr]{%
	0	1.56938971469511e-05\\
	0.601069287503903	0.00451669176492464\\
	0.60800880858521	0.000851375047454139\\
	0.612279283096783	0.00805966375870093\\
	0.617083566922302	9.43746109351995e-05\\
	0.622421660061768	0.0163816251332891\\
	0.626959039230315	4.73278638151164e-05\\
	0.630962609084915	0.0368452763636986\\
	0.632564037026754	0.0462407726858916\\
	0.636834511538327	2.87090439536897e-05\\
	0.638969748794114	0.0661715327694608\\
	0.643774032619634	0.733609777486815\\
	0.64670998384634	1\\
	0.64831141178818	0.926810610112689\\
	0.656852360811326	8.02636681402902e-05\\
	0.661122835322899	0.0475272659011989\\
	0.666727833119339	8.47098536960189e-06\\
	0.671532116944858	0.0167160835620319\\
	0.676870210084324	1.7232287029989e-05\\
	0.681674493909844	0.008387982526882\\
	0.687279491706284	0.000175526652539837\\
	0.692350680188777	0.0048707240118413\\
	0.697688773328243	0.000322623719081649\\
	0.704895199066522	0.000921248344013526\\
	0.73532232996148	0.000254606514388689\\
	0.772955886594717	0.000574892630722301\\
	0.819664201565048	0.000238753269072856\\
	1.06735172323628	5.2218821191552e-06\\
	1.3342563802096	1.2676653494248e-05\\
};
\end{axis}


\begin{axis}[%
width=0.4\figW,
height=0.5\figH,
at={(0\figW,0\figH)},
scale only axis,
xlabel = {Frequency $\omega$ ($2 \pi c/a$)},
xmin=0.3,
xmax=1,
xtick={0,0.6468,0.8879,1},
xticklabels={$0$,$0.6468$, $0.8879$,$1$},
ymin=0,
ymax=1,
axis background/.style={fill=white}
]
\addplot [color=mode2]
table[row sep=crcr]{%
	0	1.22034047884689e-05\\
	0.8311411018149	0.0023013266385894\\
	0.835411576326473	0.00175360932962421\\
	0.839415146181073	0.000965341281125021\\
	0.843685620692646	0.00486553930555145\\
	0.848489904518166	0.000250045011197297\\
	0.853561093000659	0.00825574609007673\\
	0.858098472169205	4.97169960933519e-05\\
	0.861301328052885	0.0109829985996603\\
	0.863436565308671	0.0163494616861442\\
	0.866372516535378	0.00457764465273192\\
	0.867973944477217	1.94703586926526e-05\\
	0.870109181733004	0.013154486929305\\
	0.873578942273657	0.0470453003425026\\
	0.87544727487247	0.0315310399247428\\
	0.87784941678523	2.28975304077395e-05\\
	0.878917035413123	0.0124357545233098\\
	0.880785368011937	0.12400183153239\\
	0.885856556494429	0.869270322219562\\
	0.887991793750216	1\\
	0.889059412378109	0.957556042194106\\
	0.892262268261789	0.513878092808121\\
	0.897600361401255	0.00108842262365672\\
	0.899201789343095	0.0122024441442494\\
	0.902137740569802	0.047153720606177\\
	0.904272977825588	0.0315377211631087\\
	0.908009643023215	1.99441444377335e-05\\
	0.912547022191761	0.0164419581043247\\
	0.916016782732414	0.0041292680012357\\
	0.918152019988201	5.37346492168744e-05\\
	0.92295630381372	0.00824687137239732\\
	0.92856130161016	0.000227034194552944\\
	0.933632490092653	0.00460027207861069\\
	0.938970583232119	0.000392405550539987\\
	0.944308676371585	0.00264766067464928\\
	0.949646769511052	0.000683950981152037\\
	0.954717957993545	0.00165690344143354\\
	0.960056051133011	0.000724562177670807\\
	0.965394144272477	0.000892302126201283\\
	0.97126604672589	0.0010733450416256\\
	0.986212707516396	0.000294720956906636\\
	1.00943341267307	0.000131287278812842\\
	1.02411316880661	0.00048471950436868\\
	1.04066125753895	0.000233818595714919\\
	1.06841934186418	6.61720048844572e-06\\
	1.10391766124163	0.000200101820605481\\
	1.14662240635736	3.20710461947371e-05\\
	1.3342563802096	4.61615482860722e-05\\
};
\end{axis}
\end{tikzpicture}%
	\end{subfigure}
	\caption[Non-reciprocal frequency conversion in a strongly modulated waveguide]{Phase matched non-reciprocal mode conversion in a modulated structure for \textbf{a)} LR, \textbf{b)} LR, and \textbf{c)} TR directions. On the left are the $E_z$ fields, and on the right are the corresponding frequencies for detectors $D_1$ and $D_2$. Note that in all the Fourier graphs, results are shown for both detectors. However, in the case of the $RL$ and $TR$ directions, the input and output are identical.}
	\label{fig:bandyu}
\end{figure} 


\section{Optical isolation through a photonic AB effect}
\label{sec:opticAB}

\begin{figure}[t!]
	\centering
	\setlength{\figH}{1\textwidth}
	\setlength{\figW}{1\textwidth}
	
\begin{tikzpicture}
    \begin{axis}[%
width=0.9\figW,
height=0.15\figH,
at={(0\figW,0\figH)},
scale only axis,
axis on top,
xmin=0,
xmax=60,
xlabel={$\text{x (}\mu\text{m)}$},
ymin=0,
ymax=5,
ylabel={$\text{y (}\mu\text{m)}$},
axis background/.style={fill=white},
]
\fill[fill=eps] (0,1.95) rectangle (60,3.05);
\fill[fill=eps] (24.2,1.5) rectangle (35.8,3.5);
\fill[fill=modup] (5.2,2.5) rectangle (24.2,3.05);
\fill[fill=moddown] (35.8,2.5) rectangle (54.8,3.05);
\draw (0,1.95) -- (24.2,1.95) -- (24.2,1.5) -- (35.8,1.5) -- (35.8,1.95) -- (60,1.95);
\draw (0,3.05) -- (24.2,3.05) -- (24.2,3.5) -- (35.8,3.5) -- (35.8,3.05) -- (60,3.05);
\draw[draw=src, line width=0.5mm] (1.5,0.8) -- (1.5,4.2);
\draw[line width=0.5mm, draw=src, ->] (2,4) -- (5,4);
\draw[line width=0.5mm, draw=src, ->] (2,1) -- (5,1);
\draw[dashed, draw=red, line width=0.7mm] (5.2,2.5) -- (5.2,3.05) -- (24.2,3.05) -- (24.2,2.5) -- (5.2,2.5);
\draw[dashed, draw=red, line width=0.7mm] (35.8,2.5) -- (35.8,3.05) -- (54.8,3.05) -- (54.8,2.5) -- (35.8,2.5);
\node at (15,4) {$\phi(x_1)$};
\node at (45,4) {$\phi(x_2)$};
\node at (30,4) {$T_f$};
\end{axis}
\end{tikzpicture}%
	\caption[Waveguide for demonstrating a photonic AB effect]{The modulated waveguide structure for demonstrating a photonic Aharonov-Bohm effect. The dark blue region is a material of unity permeability and relative permittivity $\epsilon = 12.25$. The light blue sections enclosed in dashed red lines are regions where the permittivity is modulated by an amount $0.1 \epsilon \cos(\omega t + \phi)$, with $\phi(x_1)$ on the left and $\phi(x_2)$ on the right. The green line indicates the region over which the modal source is injected, in this case, from left to right (LR). The central waveguide imparts a mode-dependent phase on light passing through it, described by the transfer matrix $T_f$.}%
	\label{fig:abcavity}
\end{figure}

The difficulty in applying spatio-temporal modulations for the purpose of optical isolation is that the spatial dependence of the modulation is highly difficult to engineer. Currently, such modulations are created by simultaneously controlling the permittivity of hundreds of discrete points in a waveguide at different phases. However, without the wavevector shift the mode transition is purely reciprocal, as modes propagating in the reverse direction are still phase-matched (the transition on the dispersion relation is completely vertical, and so the modulation does not discriminate between left and right propagating modes). Here, based on the work of Fang et al. \cite{Fang2012}, a broadband tunable optical isolator is constructed and simulated in the frequency domain using only the gauge potential that emerges from the phase of \textit{temporal} dynamic modulations, as discussed in Section \ref{sec:gauge}. The choice of a direct transition marks the ideal configuration for a modulation, as only two separate regions need to be modulated with no spatial dependence.

Consider first the structure of Figure \ref{fig:abcavity}, which consists of a single narrow waveguide connected to a central waveguide region. The narrow waveguide supports an even and odd mode $\ket{1}$ and $\ket{2}$ respectively. A modulation of the form
\begin{equation}
\epsilon(t)' = \delta \cos (\Omega t + \phi(x_1))
\end{equation}
is applied on the left side of the waveguide, and another modulation 
\begin{equation}
\epsilon(t)' = \delta \cos (\Omega t + \phi(x_2))
\end{equation}
on the right side to induce reciprocal transitions between the modes. The central waveguide region can be chosen to impart a mode-dependent phase on light passing through it, so there are effectively two pathways through which light can propagate. The transfer matrix for modes passing through the central waveguide is then 

\begin{equation}
T_f =
\begin{bmatrix}
e^{i \phi(P_1)} & 0 \\
0 & e^{i \phi(P_2)} 
\end{bmatrix}.
\end{equation}

In the second path ($P_2$), a photon of mode $\ket{1}$ is injected from the left and undergoes a transition to state $\ket{2}$, gaining an additional $\phi(x_1)$ phase. Upon passing through the central region, the photon acquires an additional phase $\phi(P_2)$, and undergoes a transition back to $\ket{1}$ in the second modulated region, gaining a final $-\phi(x_2)$ phase. In the first path ($P_1$), the initial mode is no longer converted and only acquires a $\phi(P_1)$ phase from the central waveguide region.

Thus, the total phase difference between the two pathways is then 

\begin{equation}
\Delta \phi_{LR} = \phi(P_2) - \phi(P_1) + \phi(x_1) - \phi(x_2).
\end{equation}
Likewise, in the right to left direction the total phase difference is instead
\begin{equation}
\Delta \phi_{RL} =  \phi(P_2) - \phi(P_1) + \phi(x_2) - \phi(x_1),
\end{equation}
which is identical to the discussion of the photonic Aharonov-Bohm effect of Section \ref{sec:opticAB}. In this case, the two modulated regions play the role of the separate interferometer arms, with the path-dependent phase of the central waveguide being equivalent to the phase acquired by an electron passing through a magnetic vector potential $\bm{A}$. 

Because of this, despite the nature of the direct modulation itself being reciprocal, the gauge potential can be designed to generate non-reciprocal phases. In particular, non-reciprocal frequency conversion can be demonstrated without requiring a spatial modulation profile by choosing the phases so that $\phi(x_2)-\phi(x_1) = \pi/2$, and $\phi(P_2) - \phi(P_1) = \pi/2$. To allow for interference between modes, the modulated length is chosen as $L = 0.5 L_c$, so that only half of the mode will transition (50\% of the modal power will transition to the other mode). From these choices, the transfer matrix for modes propagating right to left is then

\begin{equation}
T_{LR} = T(\phi_2) T_f T(\phi_1) = e^{-i(2kL - \phi(P_1))} 
\begin{bmatrix}
0 & i \\
-1 & 0
\end{bmatrix},
\end{equation}

On the other hand, in the right to left direction the transfer matrix is instead given by

\begin{equation}
T_{RL} = T(\phi_1)T_f T(\phi_2) =  e^{-i(2kL - \phi(P_1))}
\begin{bmatrix}
1 & 0 \\
0 & i
\end{bmatrix}.
\end{equation}

From this, an even mode of state $\ket{1}$ injected from the left of the waveguide will completely convert to an odd mode $\ket{2}$, whereas the same even mode injected from the right will remain an even mode.

\begin{figure}[t]
	\centering
	\setlength{\figH}{\textwidth}
	\setlength{\figW}{\textwidth}
	% This file was created by matlab2tikz.
%
%The latest updates can be retrieved from
%  http://www.mathworks.com/matlabcentral/fileexchange/22022-matlab2tikz-matlab2tikz
%where you can also make suggestions and rate matlab2tikz.
%
\definecolor{mycolor1}{rgb}{0.00000,0.44700,0.74100}%
\definecolor{mycolor2}{rgb}{0.00000,0.44706,0.74118}%
\definecolor{mycolor3}{rgb}{0.46600,0.67400,0.18800}%
\definecolor{mycolor4}{rgb}{0.00000,0.49804,0.00000}%
%
\begin{tikzpicture}

\begin{axis}[%
width=0.8\figW,
height=0.4\figH,
at={(0\figW,0\figH)},
scale only axis,
xmin=0.001,
yticklabel style={/pgf/number format/fixed},
xticklabel style={/pgf/number format/fixed},
xmax=0.5,
xlabel={Wavevector $\text{k (2}\pi\text{/a)}$},
ymin=0,
ymax=0.25,
ylabel={Frequency $\omega\text{ (2}\pi\text{c/a)}$},
axis background/.style={fill=white},
legend pos=south east
]
\addplot [color=mycolor1, line width=1.0pt]
  table[row sep=crcr]{%
0.00106370944159151	0.001\\
0.0173164838778657	0.015\\
0.0339791004711106	0.025\\
0.0485238451323132	0.032\\
0.0840513505460331	0.046\\
0.161161500553998	0.0710000000000001\\
0.322500034364579	0.117\\
0.501279538435555	0.166\\
};
\addlegendentry{Solver};
\addplot [color=mycolor2, line width=1.0pt, forget plot]
  table[row sep=crcr]{%
0	0.129\\
0.060088932026341	0.131\\
0.129014261816538	0.135\\
0.137389857852266	0.137\\
0.140026562489592	0.139\\
0.158721291577674	0.147\\
0.187634882659275	0.156\\
0.230105681117631	0.167\\
0.292652360394282	0.182\\
0.374312220807439	0.201\\
0.503433266613461	0.231\\
};
\addplot [color=mycolor3, dashed, line width=1.0pt]
  table[row sep=crcr]{%
0	0\\
0.0348259	0.0222165\\
0.0557214	0.0333214000000001\\
0.0800995	0.0439132\\
0.111443	0.0552305\\
0.163682	0.071727\\
0.254229	0.0979683\\
0.431841	0.147035\\
0.501493	0.166061\\
};
\addplot [color=mycolor4, dashed, line width=1.0pt]
  table[row sep=crcr]{%
0	0.093062\\
0.0139303	0.0938928\\
0.0313433	0.0971598\\
0.0487562	0.102545\\
0.0731343	0.11255\\
0.132338	0.138462\\
0.160199	0.14803\\
0.198507	0.158972\\
0.264677	0.17542\\
0.445771	0.217557\\
0.501493	0.230556\\
};
\addlegendentry{MPB};
\addplot [color=black, line width=1.0pt]
  table[row sep=crcr]{%
0	0\\
0.250746	0.250746\\
};

\node[label={180:{$\ket{1}$}},circle,fill=mode1,inner sep=2pt] at (0.366,0.129) {};
\node[label={180:{$\ket{2}$}},circle,fill=mode2,inner sep=2pt] at (0.366,0.199) {};
\draw[line width=0.2mm, draw=black, ->] (0.366,0.135) -- (0.366,0.193);
\node[label={180:{$\Omega$}}] at (0.366,0.164) {};
\end{axis}
\end{tikzpicture}%
	\caption[Dispersion relation of a waveguide of width $1.1 \mu m$ and relative permittivity $12.25$.]{Dispersion relation of a waveguide with width $1.1 \mu m$ and relative permittivity $12.25$. Green dashes indicate dispersions obtained through \textit{MPB}, whereas blue lines indicate those calculated analytically using the purpose-written solver. The red line is the direct transition between modes $\ket{1}$ and $\ket{2}$. Since the modulation has no spatial dependence, the modulation does not discriminate between forward and backward propagating modes.}
	\label{fig:banddiagram}
\end{figure}


This is implemented in the frequency domain simulation by choosing the narrow waveguide to be of width $1.1  \mskip3mu \mu m$ extending from $0$ to $60 \mskip3mu  \mu m$, and the central waveguide region to be $2.2  \mskip3mu \mu m$ wide and $11.6 \mskip3mu  \mu m$ long, with relative permittivity $\epsilon = 12.25$. For the narrow waveguide, this choice corresponds to the dispersion relation of Figure \ref{fig:banddiagram}. The direct transition is induced between an even mode $\ket{1}$ chosen at $\omega_1 = 0.129 \mskip3mu (2 \pi c/a)$ and odd mode $\ket{2}$ at $\omega_2 = 0.199  \mskip3mu (2 \pi c/a)$, which share the same wavevector $k_{1,2} = 0.380 \mskip3mu  (2 \pi / a)$. To satisfy the modulation condition for the above transfer matrix, the waveguide is modulated on the left and right by $0.1 \epsilon \cos (\Omega t + \phi)$, with the phase of modulation set at $0$ and $\pi/2$ respectively ($\phi(x_2) - \phi(x_1) = \pi/2$). The choice of width for the central waveguide ($2 \mskip3mu \mu m$) ensures that even and odd modes acquire a $\pi/2$ phase difference upon passing through ($\phi(P_2) - \phi(P_1) = \pi/2$). The length of the modulated regions is half the coherence length ($19  \mskip3mu  \mu m$), so that a mode injected from the left will undergo a 50\% transition to the second mode, and vice versa. To test the non-reciprocal conversion, a modal source is excited at $1  \mskip3mu \mu m$ for the left to right direction, and $59 \mskip3mu  \mu m$ for the right to left and time-reversed directions. The simulation is performed over 3 sidebands at a spatial discretisation of $\Delta = 0.04 \mskip3mu  \mu m$, however only the first set of sidebands are shown.

\begin{figure}[t!]
	\centering
	\setlength{\figH}{1\textwidth}
	\setlength{\figW}{1\textwidth}
	% This file was created by matlab2tikz.
%
%The latest updates can be retrieved from
%  http://www.mathworks.com/matlabcentral/fileexchange/22022-matlab2tikz-matlab2tikz
%where you can also make suggestions and rate matlab2tikz.
%
\pgfplotsset{fangFields/.style={
	width=0.8\figW,
height=0.08\figH,
	scale only axis,
	axis on top,
	title style = {at={(0.4\figH,0.06\figH)}},
	xmin=0,
	xmax=60,
	ymin=0,
	clip=false,
	ymax=5,
	axis background/.style={fill=white},
	colormap={mymap}{[1pt] rgb(0pt)=(0,0,1); rgb(31pt)=(1,1,1); rgb(32pt)=(1,1,1); rgb(63pt)=(1,0,0)},
	colorbar,
	colorbar style={width=.01\linewidth, at={(1.02,0.04\figH)}, anchor=east}	
	}
}

\begin{tikzpicture}
\begin{axis}[%
fangFields,
at={(0\figW,0.95\figH)},
title = {Total field},
point meta min=-23,
point meta max=23,
xticklabels={,,},
colorbar style={title={$V / \mu m$}, width=.01\linewidth, at={(1.02,0.04\figH)}, anchor=east},
ylabel={$\text{y (}\mu\text{m)}$}
]
\addplot [forget plot] graphics [xmin=0, xmax=60, ymin=0, ymax=5] {graphs/fangcavity/LR/totalfield.png};
\draw (0,1.95) -- (24.2,1.95) -- (24.2,1.5) -- (35.8,1.5) -- (35.8,1.95) -- (60,1.95);
\draw (0,3.05) -- (24.2,3.05) -- (24.2,3.5) -- (35.8,3.5) -- (35.8,3.05) -- (60,3.05);
\node at (55,6) {$\longrightarrow \ket{2}$};
\node at (5,6) {$\ket{1} \longrightarrow$};
\node at (-2,7) {\textbf{(a)}};
\end{axis}

\begin{axis}[%
fangFields,
at={(0\figW,0.8\figH)},
title = {$\omega_1 - \Omega$},
point meta min=-0.11845616858408,
point meta max=0.11845616858408,
xticklabels={,,},
ylabel={$\text{y (}\mu\text{m)}$}
]
\addplot [forget plot] graphics [xmin=0, xmax=60, ymin=0, ymax=5] {graphs/fangcavity/LR/nameoffile2-1.png};
\draw (0,1.95) -- (24.2,1.95) -- (24.2,1.5) -- (35.8,1.5) -- (35.8,1.95) -- (60,1.95);
\draw (0,3.05) -- (24.2,3.05) -- (24.2,3.5) -- (35.8,3.5) -- (35.8,3.05) -- (60,3.05);
\end{axis}

\begin{axis}[%
fangFields,
at={(0\figW,0.65\figH)},
title = {$\omega_1$},
point meta min=-6.64719566790506,
point meta max=6.64719566790506,
xticklabels={,,},
ylabel={$\text{y (}\mu\text{m)}$}
]
\addplot [forget plot] graphics [xmin=0, xmax=60, ymin=0, ymax=5] {graphs/fangcavity/LR/nameoffile2-2.png};
\draw (0,1.95) -- (24.2,1.95) -- (24.2,1.5) -- (35.8,1.5) -- (35.8,1.95) -- (60,1.95);
\draw (0,3.05) -- (24.2,3.05) -- (24.2,3.5) -- (35.8,3.5) -- (35.8,3.05) -- (60,3.05);
\end{axis}

\begin{axis}[%
fangFields,
at={(0\figW,0.5\figH)},
title = {$\omega_1 + \Omega$},
point meta min=-10.2001641787316,
point meta max=10.2001641787316,
xlabel={$\text{x (}\mu\text{m)}$},
ylabel={$\text{y (}\mu\text{m)}$}
]
\addplot [forget plot] graphics [xmin=0, xmax=60, ymin=0, ymax=5] {graphs/fangcavity/LR/nameoffile2-3.png};
\draw (0,1.95) -- (24.2,1.95) -- (24.2,1.5) -- (35.8,1.5) -- (35.8,1.95) -- (60,1.95);
\draw (0,3.05) -- (24.2,3.05) -- (24.2,3.5) -- (35.8,3.5) -- (35.8,3.05) -- (60,3.05);
\node at (-2,-3) {\textbf{(b)}};
\end{axis}

% This file was created by matlab2tikz.
%
%The latest updates can be retrieved from
%  http://www.mathworks.com/matlabcentral/fileexchange/22022-matlab2tikz-matlab2tikz
%where you can also make suggestions and rate matlab2tikz.
%
%\definecolor{mycolor1}{rgb}{0.00000,0.44700,0.74100}%
%\definecolor{mycolor2}{rgb}{0.85000,0.32500,0.09800}%
%\definecolor{mycolor3}{rgb}{0.92900,0.69400,0.12500}%
%
\definecolor{mycolor1}{RGB}{0,128,0}%
\definecolor{mycolor2}{RGB}{128,0,0}%
\definecolor{mycolor3}{RGB}{0,0,128}%
%\begin{tikzpicture}

\begin{axis}[%
width=0.8\figW,
height=0.4\figH,
at={(0\figW,0\figH)},
scale only axis,
legend,
xlabel={$\text{x (}\mu\text{m)}$},
ylabel={$\text{Modal amplitude} (W/\mu m)$},
xmin=0,
xmax=60,
ymin=0,
ymax=0.8,
axis background/.style={fill=white}
]
\addplot [color=mode2]
  table[row sep=crcr]{%
0	2.16924151672515e-05\\
0.100041684035013	0.00177432994685489\\
0.125052105043771	0.00482537314767484\\
0.150062526052523	0.0107552712016386\\
0.175072947061274	0.0201677225744774\\
0.225093789078784	0.0468289341874737\\
0.275114631096287	0.0731266675521027\\
0.300125052105045	0.0827778339171985\\
0.325135473113797	0.0896166941604193\\
0.350145894122548	0.0939767029181908\\
0.400166736140058	0.0977803033876583\\
0.500208420175071	0.0990575471716966\\
1.3505627344727	0.101747452510068\\
2.0508545227178	0.0999552435717348\\
2.52605252188412	0.0994947084469331\\
3.32638599416423	0.101299129310398\\
4.02667778240934	0.0989424575743314\\
4.42684451854939	0.0984154258896766\\
5.32721967486453	0.103601871621372\\
5.60233430596082	0.114460386416361\\
5.67736556898708	0.113750784135917\\
5.75239683201334	0.109948388902289\\
5.8274280950396	0.104246102425762\\
5.95248020008337	0.102476296843612\\
6.05252188411838	0.0977396976386231\\
6.1525635681534	0.089347598286956\\
6.32763651521467	0.0693704006092162\\
6.35264693622343	0.0710235497335603\\
6.62776156731972	0.133311285649505\\
6.70279283034598	0.144580973397566\\
6.77782409337224	0.151727893297029\\
6.8528553563985	0.15439101494502\\
6.92788661942476	0.152439408420051\\
6.97790746144227	0.149852978794925\\
7.05293872446853	0.156345883027178\\
7.12796998749479	0.157934671164718\\
7.1779908295123	0.156165185440472\\
7.22801167152981	0.151853222309953\\
7.27803251354731	0.144972340197818\\
7.35306377657357	0.130043363864104\\
7.42809503959983	0.109979974389368\\
7.50312630262609	0.0853432806216858\\
7.62817840766986	0.0400334217499463\\
7.70320967069613	0.0606680313931989\\
7.8782826177574	0.11051996847948\\
7.97832430179241	0.135030733754149\\
8.05335556481867	0.152900067379349\\
8.12838682784493	0.174959632356902\\
8.17840766986244	0.186570054473684\\
8.22842851187995	0.195454269695162\\
8.27844935389746	0.20145444652654\\
8.32847019591497	0.20446446743874\\
8.37849103793247	0.204430864114016\\
8.42851187994998	0.201352821142144\\
8.47853272196749	0.195281224440919\\
8.52855356398499	0.186316804984948\\
8.5785744060025	0.174607498366662\\
8.65360566902876	0.15232790792323\\
8.72863693205503	0.125129762160988\\
8.80366819508129	0.0940223633098043\\
8.97874114214256	0.0177566767445825\\
9.00375156315131	0.0155877396127764\\
9.02876198416007	0.0238326683026528\\
9.07878282617757	0.0456764611652147\\
9.27886619424761	0.138358471211816\\
9.35389745727387	0.169117527066348\\
9.42892872030013	0.195796583174094\\
9.47894956231763	0.210751028304536\\
9.52897040433514	0.223051138135752\\
9.57899124635265	0.232449199534976\\
9.62901208837015	0.238742612293052\\
9.67903293038766	0.241779355170621\\
9.72905377240517	0.2414628138594\\
9.77907461442268	0.237755985303117\\
9.82909545644019	0.230685224054163\\
9.87911629845769	0.220343948199542\\
9.9291371404752	0.206897192397648\\
9.97915798249271	0.190588854360072\\
10.1792413505627	0.116261620764512\\
10.3042934556065	0.064522145803501\\
10.3293038766153	0.0616150805804097\\
10.354314297624	0.0638270964165386\\
10.3793247186328	0.0692992195971058\\
10.4293455606503	0.0870639364839221\\
10.5043768236765	0.12162727850302\\
10.6294289287203	0.180717093084183\\
10.7044601917466	0.211626430961616\\
10.7544810337641	0.229274251346844\\
10.8045018757816	0.244170561635414\\
10.8545227177991	0.256073241895876\\
10.9045435598166	0.264806798596105\\
10.9545644018341	0.270257215345438\\
11.1296373488954	0.279936760405413\\
11.1796581909129	0.27748493549732\\
11.2296790329304	0.271399926403831\\
11.2796998749479	0.261751861409998\\
11.3297207169654	0.248693981282536\\
11.3797415589829	0.232475596395958\\
11.4547728220092	0.203073973827593\\
11.5548145060442	0.158033021037433\\
11.6298457690705	0.12475009348455\\
11.679866611088	0.107126661740296\\
11.7048770320967	0.101099201469843\\
11.7298874531055	0.097662419280617\\
11.7548978741142	0.0973312647564981\\
11.779908295123	0.100066752806811\\
11.8049187161317	0.105644406781458\\
11.8299291371405	0.113714382544742\\
11.9049604001667	0.14684395023987\\
12.1050437682368	0.248262764354564\\
12.1550646102543	0.269598243907751\\
12.2050854522718	0.288004747583933\\
12.2551062942893	0.303085772929613\\
12.3051271363068	0.314543475936482\\
12.3551479783243	0.322168858942256\\
12.4051688203418	0.325836806798876\\
12.4551896623593	0.325503592111268\\
12.5052105043768	0.321205779281712\\
12.5552313463943	0.313060246738203\\
12.6052521884118	0.301265664768358\\
12.6552730304293	0.286106447124126\\
12.7052938724469	0.26796118261997\\
12.7803251354731	0.23625237529383\\
12.9804085035431	0.146470994055846\\
13.0304293455606	0.131577806163925\\
13.0554397665694	0.12699691653139\\
13.0804501875782	0.125430501502407\\
13.1054606085869	0.12636173291142\\
13.1304710295957	0.129801910806037\\
13.1554814506044	0.135601993131552\\
13.2055022926219	0.15289069260627\\
13.2555231346394	0.175420065818017\\
13.3555648186745	0.227082368238364\\
13.4556065027095	0.277662662661584\\
13.5306377657357	0.310359532731695\\
13.5806586077532	0.328509204526455\\
13.6306794497707	0.343220441045432\\
13.6807002917882	0.354160447958968\\
13.7307211338058	0.361089779042203\\
13.7807419758233	0.363860877143921\\
13.8307628178408	0.362419136422268\\
13.8807836598583	0.356805776807711\\
13.9308045018758	0.347162567012958\\
13.9808253438933	0.333739158082096\\
14.0308461859108	0.31690468889181\\
14.1058774489371	0.28641228210013\\
14.3059608170071	0.196600268935718\\
14.3559816590246	0.180672146037445\\
14.3809920800334	0.175199720339549\\
14.4060025010421	0.171745462174236\\
14.4310129220509	0.170490474779733\\
14.4560233430596	0.171509042854332\\
14.4810337640684	0.17475530155356\\
14.5310546060859	0.187220605297796\\
14.5810754481034	0.205845931701965\\
14.6561067111296	0.240290234917708\\
14.7811588161734	0.300401332883233\\
14.8561900791997	0.331860039377794\\
14.9062109212172	0.349443206589669\\
14.9562317632347	0.363790379112871\\
15.0062526052522	0.374590322332566\\
15.0562734472697	0.384065918716232\\
15.1062942892872	0.393639895212196\\
15.1563151313047	0.399045856667115\\
15.2063359733222	0.400151754869661\\
15.2563568153397	0.396923938376581\\
15.3063776573572	0.389433324190051\\
15.3563984993747	0.377864407493377\\
15.4064193413923	0.362528300449526\\
15.4564401834098	0.343881996964825\\
15.531471446436	0.311146136630207\\
15.6815339724885	0.241830991349232\\
15.731554814506	0.223637595989402\\
15.7815756565236	0.211565703900419\\
15.8065860775323	0.208488248195202\\
15.8315964985411	0.207623713896922\\
15.8566069195498	0.209037617757936\\
15.8816173405586	0.212689546521581\\
15.9066277615673	0.21844047508533\\
15.9566486035848	0.235324691607659\\
16.0066694456023	0.257527380231913\\
16.0817007086286	0.296026590521826\\
16.1817423926636	0.34805359044055\\
16.2567736556899	0.38236277041667\\
16.3067944977074	0.40147960901804\\
16.3568153397249	0.416951404443324\\
16.4068361817424	0.428395115600253\\
16.4568570237599	0.435555996656504\\
16.5068778657774	0.438297383209047\\
16.5568987077949	0.43659471463311\\
16.6069195498124	0.430532678203718\\
16.6569403918299	0.420305139298691\\
16.7069612338474	0.406218193749794\\
16.7569820758649	0.388697335678934\\
16.8320133388912	0.357238924905445\\
17.0070862859525	0.27783057528584\\
17.05710712797	0.259142372145703\\
17.1071279699875	0.245020129816282\\
17.157148812005	0.237403947079144\\
17.1821592330138	0.236502111655199\\
17.2071696540225	0.237662724921996\\
17.2321800750313	0.240880445742285\\
17.2822009170488	0.253080370814693\\
17.3322217590663	0.271716042909112\\
17.3822426010838	0.294896853735985\\
17.4822842851188	0.347471849911734\\
17.5573155481451	0.386406232727261\\
17.6323468111713	0.420757100965204\\
17.6823676531888	0.439800902400897\\
17.7323884952063	0.455088849131101\\
17.7824093372238	0.466209594704132\\
17.8324301792413	0.472884659891491\\
17.8824510212589	0.474962888093295\\
17.9324718632764	0.472418554248108\\
17.9824927052939	0.46535241641589\\
18.0325135473114	0.453995631773132\\
18.0825343893289	0.438717045005077\\
18.1325552313464	0.420034891696773\\
18.2075864943727	0.387180244860659\\
18.3576490204252	0.317054163731413\\
18.4076698624427	0.297654458554675\\
18.4576907044602	0.283165195215844\\
18.5077115464777	0.275215907686515\\
18.5327219674865	0.274045976619909\\
18.5577323884952	0.274820381104043\\
18.6077532305127	0.282029408074123\\
18.6577740725302	0.295917026334415\\
18.7077949145477	0.314912909372609\\
18.782826177574	0.349108216219484\\
18.9078782826178	0.408525170425499\\
18.982909545644	0.439419193292181\\
19.1079616506878	0.483179969181599\\
19.1579824927053	0.496398913535373\\
19.2080033347228	0.505226200745909\\
19.2580241767403	0.509432263819754\\
19.3080450187578	0.508915980036264\\
19.3580658607753	0.503707534907271\\
19.4080867027928	0.493974397885154\\
19.4581075448103	0.480030729946357\\
19.5081283868278	0.462351021624826\\
19.5831596498541	0.430311595501109\\
19.7582325969154	0.348541446867479\\
19.8082534389329	0.33017179811187\\
19.8582742809504	0.317343285511733\\
19.8832847019591	0.31351516537282\\
19.9082951229679	0.311605401150857\\
19.9333055439767	0.311689393142984\\
19.9583159649854	0.31377695476818\\
20.0083368070029	0.323676478387995\\
20.0583576490204	0.340194636397982\\
20.1083784910379	0.36165160547516\\
20.1834097540642	0.399084398181138\\
20.308461859108	0.462236374315815\\
20.3584827011255	0.484222046397136\\
20.408503543143	0.50304052787537\\
20.4585243851605	0.51809795381206\\
20.508545227178	0.528966935493024\\
20.5585660691955	0.535370797091545\\
20.608586911213	0.537172732239888\\
20.6586077532305	0.534368928224339\\
20.708628595248	0.527085041891496\\
20.7586494372655	0.515575829037161\\
20.808670279283	0.500228120122124\\
20.908711963318	0.463187745094153\\
20.9837432263443	0.431178705273929\\
21.1087953313881	0.377132500504111\\
21.1588161734056	0.359580689222888\\
21.2088370154231	0.346872976687422\\
21.2588578574406	0.340496890130829\\
21.2838682784494	0.340004798970106\\
21.3088786994581	0.341384716238707\\
21.3588995414756	0.349649883446347\\
21.4089203834931	0.364559131859288\\
21.4589412255106	0.384755284541562\\
21.5339724885369	0.421336002563386\\
21.6840350145894	0.497905181771003\\
21.7340558566069	0.519893564250538\\
21.7840766986244	0.538615019799785\\
21.8340975406419	0.55345207087931\\
21.8841183826594	0.563954394456538\\
21.9341392246769	0.56982869690917\\
21.9841600666945	0.570932335948299\\
22.034180908712	0.567270163338691\\
22.0842017507295	0.558994233422354\\
22.134222592747	0.546406307702625\\
22.1842434347645	0.529963334759799\\
22.234264276782	0.510286144718521\\
22.3092955398083	0.476499759112912\\
22.4093372238433	0.43147839324169\\
22.4843684868695	0.400099657409925\\
22.534389328887	0.382787811957677\\
22.5593997498958	0.376184351204863\\
22.6094205919133	0.369194167117776\\
22.6594414339308	0.369136553467769\\
22.7094622759483	0.375997884515968\\
22.7594831179658	0.389047200655327\\
22.8095039599833	0.4070325378691\\
22.8845352230096	0.439974713151727\\
23.1596498541059	0.568670581365602\\
23.2096706961234	0.585696025700031\\
23.2596915381409	0.598499377701252\\
23.3097123801584	0.606718654996406\\
23.3597332221759	0.61014335924412\\
23.4097540641934	0.608712995122197\\
23.4597749062109	0.602517987627834\\
23.5097957482284	0.591802626698531\\
23.5598165902459	0.576969747113061\\
23.6098374322634	0.558586734299276\\
23.6848686952897	0.526018725068759\\
23.859941642351	0.445218195030975\\
23.9099624843685	0.426635975750258\\
23.959983326386	0.41235098772637\\
24.0100041684035	0.403526244811594\\
24.060025010421	0.40085775260372\\
24.1100458524385	0.403459763540006\\
24.160066694456	0.411400911667272\\
24.2350979574823	0.430386245883632\\
24.2851187994998	0.443970330459052\\
24.3351396415173	0.452904210344741\\
24.3851604835348	0.454088194014808\\
24.4101709045436	0.451475787476362\\
24.4351813255523	0.446585318192632\\
24.4852021675698	0.430995569224507\\
24.5352230095873	0.408162238497951\\
24.5852438516048	0.380206398641313\\
24.6852855356398	0.321515827566458\\
24.7102959566486	0.309147696005979\\
24.7353063776574	0.299024701418595\\
24.7603167986661	0.29153202480299\\
24.7853272196749	0.287128817874972\\
24.8103376406836	0.286103752571336\\
24.8353480616924	0.288602668731912\\
24.8603584827011	0.294524473305941\\
24.8853689037099	0.303410087379099\\
24.9353897457274	0.327908772864831\\
25.0104210087536	0.373095102783871\\
25.0604418507712	0.402347718910931\\
25.1104626927887	0.427131020479088\\
25.1604835348062	0.44529233871549\\
25.1854939558149	0.451431118971179\\
25.2105043768237	0.455451749422259\\
25.2355147978324	0.457287725428088\\
25.2605252188412	0.456912040552417\\
25.2855356398499	0.454336675848786\\
25.3105460608587	0.449647654381508\\
25.3355564818674	0.442914632984099\\
25.385577323885	0.423851421539219\\
25.4355981659025	0.398720624360323\\
25.585660691955	0.314808231847493\\
25.6106711129637	0.304714833139279\\
25.6356815339725	0.297203335699237\\
25.6606919549812	0.292763037112991\\
25.68570237599	0.291649592606987\\
25.7107127969987	0.293864392516696\\
25.7357232180075	0.299254130091342\\
25.7607336390163	0.307475267398971\\
25.8107544810338	0.330414683685291\\
25.88578574406	0.372979265606261\\
25.9358065860775	0.400713867856041\\
25.985827428095	0.424278801915591\\
26.0358482701125	0.441527390510984\\
26.0608586911213	0.447322311855075\\
26.0858691121301	0.451072055141623\\
26.1108795331388	0.452708181223954\\
26.1358899541476	0.452201422176714\\
26.1609003751563	0.449561128635708\\
26.1859107961651	0.444835240387967\\
26.2359316381826	0.42956858856769\\
26.2859524802001	0.407633751389199\\
26.3359733222176	0.380816767833302\\
26.4360150062526	0.324154900537152\\
26.4860358482701	0.302122686838651\\
26.5110462692789	0.294577773527244\\
26.5360566902876	0.289951402772836\\
26.5610671112964	0.288519959613062\\
26.5860775323051	0.290361787791511\\
26.6110879533139	0.29534335335947\\
26.6360983743226	0.303141853116365\\
26.6861192163401	0.325263604441297\\
26.7611504793664	0.366513896374016\\
26.8111713213839	0.393553974323133\\
26.8611921634014	0.416632876361732\\
26.9112130054189	0.434270084792587\\
26.9362234264277	0.441278740410219\\
26.9612338474364	0.44624857057758\\
26.9862442684452	0.449090469345819\\
27.0112546894539	0.449754116319056\\
27.0362651104627	0.448227370068437\\
27.0612755314714	0.444536216451617\\
27.0862859524802	0.438745266598218\\
27.1363067944977	0.421322619862032\\
27.1863276365152	0.397305858568807\\
27.2613588995415	0.353754895001217\\
27.311379741559	0.324517128637837\\
27.3363901625677	0.311403310394759\\
27.3614005835765	0.301273379355486\\
27.3864110045852	0.294325360515749\\
27.411421425594	0.290220330431808\\
27.4364318466028	0.289210180655331\\
27.4614422676115	0.291356823334901\\
27.4864526886203	0.296521786339156\\
27.511463109629	0.304389140019218\\
27.5614839516465	0.326373876099737\\
27.6365152146728	0.36704585638703\\
27.6865360566903	0.393648473995945\\
27.7615673197165	0.429043694152483\\
27.8115881617341	0.44583289949172\\
27.8365985827428	0.451227441272266\\
27.8616090037516	0.454482367416425\\
27.8866194247603	0.455542741926173\\
27.9116298457691	0.454392429534444\\
27.9366402667778	0.451053922392951\\
27.9616506877866	0.445588708784911\\
28.0116715298041	0.428725252848217\\
28.0616923718216	0.405124478154505\\
28.1117132138391	0.376857804933508\\
28.1867444768654	0.332385426097964\\
28.2367653188829	0.307301646147053\\
28.2617757398916	0.297884166099578\\
28.2867861609004	0.291283560905391\\
28.3117965819091	0.287910700369039\\
28.3368070029179	0.287987965389902\\
28.3618174239266	0.291510150369973\\
28.3868278449354	0.298245432116282\\
28.4118382659441	0.308043647082087\\
28.4618591079617	0.333008670196165\\
28.6119216340142	0.418216401942615\\
28.6619424760317	0.43855457653784\\
28.6869528970404	0.445994905680124\\
28.7119633180492	0.451416860933392\\
28.7369737390579	0.454723901895647\\
28.7619841600667	0.455859938534537\\
28.7869945810754	0.454808512628702\\
28.8120050020842	0.451592541392053\\
28.837015423093	0.446274605409322\\
28.8870362651105	0.42978710177934\\
28.937057107128	0.406727188851598\\
29.0120883701542	0.364574230575414\\
29.0621092121717	0.336015368625077\\
29.1121300541892	0.311890326887095\\
29.137140475198	0.302917865610482\\
29.1621508962068	0.296700065597548\\
29.1871613172155	0.29361725374941\\
29.2121717382243	0.293867791853408\\
29.237182159233	0.297434571030124\\
29.2621925802418	0.304088055020713\\
29.2872030012505	0.313424652525697\\
29.337223843268	0.338070997700505\\
29.4622759483118	0.408775264045545\\
29.5122967903293	0.431444333469784\\
29.5623176323468	0.447422132652335\\
29.5873280533556	0.452482257320746\\
29.6123384743643	0.455452349916335\\
29.6373488953731	0.456280339867824\\
29.6623593163818	0.454954005435582\\
29.6873697373906	0.4515007749593\\
29.7123801583993	0.445988058928364\\
29.7624010004168	0.429259520963001\\
29.8124218424343	0.406154701692202\\
29.8874531054606	0.364538829684925\\
29.9374739474781	0.336795549900955\\
29.9874947894956	0.313808211080904\\
30.0125052105044	0.305483812060388\\
30.0375156315131	0.299925531765922\\
30.0625260525219	0.297473529155212\\
30.0875364735306	0.298285412375613\\
30.1125468945394	0.302310142748276\\
30.1375573155481	0.309296751085355\\
30.1625677365569	0.318834442873779\\
30.2125885785744	0.343462543575761\\
30.3376406836182	0.412717646954697\\
30.3876615256357	0.434568233081237\\
30.4376823676532	0.449664947595139\\
30.4626927886619	0.454274411012207\\
30.4877032096707	0.456792737029076\\
30.5127136306795	0.457171385379198\\
30.5377240516882	0.455401204664156\\
30.562734472697	0.451512292315677\\
30.5877448937057	0.445574375362206\\
30.6377657357232	0.428034680895273\\
30.6877865777407	0.404182079047395\\
30.762817840767	0.361515356528315\\
30.8128386827845	0.333263111663953\\
30.862859524802	0.30994558387242\\
30.8878699458108	0.301541805203073\\
30.9128803668195	0.295968504154963\\
30.9378907878283	0.293565835629018\\
30.962901208837	0.294485085481789\\
30.9879116298458	0.29866280914036\\
31.0129220508545	0.305831725814365\\
31.0379324718633	0.31556402191697\\
31.0879533138808	0.340562491640739\\
31.2130054189246	0.410297604883382\\
31.2630262609421	0.432162804107101\\
31.3130471029596	0.447227144622204\\
31.3380575239683	0.451807077466171\\
31.3630679449771	0.454288189909747\\
31.3880783659858	0.454622077926075\\
31.4130887869946	0.452799054064947\\
31.4380992080033	0.448847968788293\\
31.4631096290121	0.442836556949047\\
31.5131304710296	0.425104067534157\\
31.5631513130471	0.40097021252538\\
31.6131721550646	0.375042914560687\\
31.7132138390996	0.319547321848681\\
31.7632346811171	0.297708074706883\\
31.7882451021259	0.290181989306923\\
31.8132555231346	0.28656308997666\\
31.8382659441434	0.28704860750058\\
31.8632763651521	0.290867964990213\\
31.8882867861609	0.297765574133216\\
31.9132972071697	0.307357180460663\\
31.9633180491872	0.332340300126383\\
32.0883701542309	0.402951537615976\\
32.1383909962484	0.425380636421387\\
32.1884118382659	0.441084204064914\\
32.2134222592747	0.446004195215743\\
32.2384326802835	0.44883422244957\\
32.2634431012922	0.449521627080188\\
32.288453522301	0.448051773510343\\
32.3384743643185	0.440085655687604\\
32.388495206336	0.426069686607477\\
32.4385160483535	0.405290913641714\\
32.488536890371	0.379605129360314\\
32.588578574406	0.325068942657346\\
32.6385994164235	0.304036744695054\\
32.6636098374323	0.296978806488262\\
32.688620258441	0.292830129657482\\
32.7136306794498	0.291866188846264\\
32.7386411004585	0.294165112734504\\
32.7636515214673	0.299594661993922\\
32.788661942476	0.307834949577135\\
32.8386827844935	0.330836812798808\\
32.9137140475198	0.373430625000481\\
32.9637348895373	0.401363741393496\\
33.0137557315548	0.425361080446194\\
33.0637765735723	0.443162857746422\\
33.0887869945811	0.449256612594283\\
33.1137974155898	0.453309426868188\\
33.1388078365986	0.455244676393157\\
33.1638182576073	0.455024964612953\\
33.1888286786161	0.452651764176132\\
33.2138390996248	0.448165579539776\\
33.2388495206336	0.4416466397562\\
33.2888703626511	0.423038169172131\\
33.3388912046686	0.398325806097482\\
33.4889537307211	0.315085253463494\\
33.5139641517299	0.305026225974657\\
33.5389745727386	0.297531145930137\\
33.5639849937474	0.293052190485938\\
33.5889954147561	0.291881733407024\\
33.6140058357649	0.294104478309855\\
33.6390162567737	0.299582634084018\\
33.6640266777824	0.307980118085055\\
33.7140475197999	0.331528353490313\\
33.7890787828262	0.375176312295871\\
33.8390996248437	0.403776158406174\\
33.8891204668612	0.42824913646907\\
33.9391413088787	0.446378212530909\\
33.9641517298875	0.452582040309011\\
33.9891621508962	0.456709337786414\\
34.014172571905	0.458684460072668\\
34.0391829929137	0.458471583778511\\
34.0641934139225	0.456168272517843\\
34.0892038349312	0.451730626993665\\
34.11421425594	0.445236689003856\\
34.1642350979575	0.42662112586877\\
34.214255939975	0.401851081392529\\
34.3643184660275	0.318189925597522\\
34.3893288870363	0.308086100118381\\
34.414339308045	0.300507500135389\\
34.4393497290538	0.295900637108268\\
34.4643601500625	0.294555462910132\\
34.4893705710713	0.296558588791243\\
34.51438099208	0.301895229959086\\
34.5393914130888	0.310171367609556\\
34.5894122551063	0.333480691554072\\
34.6644435181326	0.376739836506204\\
34.7144643601501	0.405214770447053\\
34.7644852021676	0.429587534299422\\
34.8145060441851	0.447505690443172\\
34.8395164651938	0.453559049433586\\
34.8645268862026	0.457504491658838\\
34.8895373072113	0.459265494914099\\
34.9145477282201	0.458805400971109\\
34.9395581492288	0.456126932106059\\
34.9645685702376	0.451272255496924\\
34.9895789912464	0.444469530263454\\
35.0395998332639	0.425227616900045\\
35.0896206752814	0.399697489323913\\
35.2396832013339	0.313436300480184\\
35.2646936223426	0.302600882807219\\
35.2897040433514	0.294214418720003\\
35.3147144643602	0.288735576914249\\
35.3397248853689	0.286471197851562\\
35.3647353063777	0.287535800958331\\
35.3897457273864	0.292271307718927\\
35.4147561483952	0.299970230787139\\
35.4397665694039	0.310717922544242\\
35.4897874114214	0.338064326673802\\
35.6148395164652	0.414660533684582\\
35.6648603584827	0.439847701964091\\
35.7148812005002	0.459600972008538\\
35.7649020425177	0.47425776310164\\
35.8149228845352	0.485112052510367\\
35.8899541475615	0.496286155573678\\
35.9649854105877	0.503430937920641\\
36.040016673614	0.507041167991183\\
36.1650687786578	0.508579524353216\\
36.2150896206753	0.510045316070602\\
36.3901625677366	0.516499355867339\\
36.8653605669029	0.53197950355797\\
37.1654856190079	0.54209041606731\\
37.3405585660692	0.544295405878422\\
37.5406419341392	0.542671210503855\\
38.1158816173406	0.534308104353791\\
38.2659441433931	0.536060174194049\\
38.5910796165069	0.555339709113873\\
38.7661525635682	0.562042638960101\\
38.9662359316382	0.565737326425413\\
39.9916631929971	0.573914896128322\\
40.0666944560233	0.574043237578444\\
40.2917882451021	0.582484038374893\\
41.2671946644435	0.609995853476775\\
41.4422676115048	0.611172787382301\\
41.6423509795748	0.608717379993536\\
42.1175489787411	0.599354214167981\\
42.4426844518549	0.614894693359517\\
42.6927886619425	0.625522499000361\\
42.8928720300125	0.630056159033209\\
43.0929553980825	0.630775679438329\\
43.9933305543977	0.627904416207087\\
44.2434347644852	0.638673030978524\\
44.4935389745727	0.6454305222614\\
45.0437682367653	0.6544139420348\\
45.3689037098791	0.657753235030263\\
45.5939974989579	0.65613213724054\\
45.8691121300542	0.649848132586811\\
45.994164235098	0.648123900395078\\
46.4443518132555	0.663603964724693\\
46.7444768653606	0.672982397287903\\
46.9445602334306	0.675864956500391\\
47.1696540225094	0.675183335931301\\
47.5698207586494	0.668857885164201\\
47.8699458107545	0.665063244996524\\
48.320133388912	0.68247252584225\\
48.5702375989996	0.687701952220657\\
48.8703626511046	0.689840214108628\\
49.3205502292622	0.689099218000557\\
49.6206752813672	0.685066387458328\\
49.8957899124635	0.681288836209035\\
50.4210087536474	0.694995705881219\\
50.7711546477699	0.701086879687082\\
51.0212588578574	0.701626338160395\\
51.271363067945	0.698456258393229\\
51.7215506461025	0.688162708233847\\
51.7715714881201	0.688267831483429\\
52.3718215923301	0.705239327902987\\
52.6469362234264	0.708699243646898\\
52.9220508545227	0.708329587610443\\
53.2471863276365	0.703986514654105\\
53.6223426427678	0.695150400454459\\
53.7473947478116	0.692695153366984\\
54.2225927469779	0.704352381677474\\
54.5477282200917	0.708481776729244\\
54.822842851188	0.70852616349579\\
55.122967903293	0.704834507353233\\
55.4731137974156	0.696715763021693\\
55.6731971654856	0.692025353768031\\
56.1734055856607	0.704138393642566\\
56.4735306377657	0.70791548757223\\
56.7736556898708	0.707912189658153\\
57.0737807419758	0.704110324986551\\
57.4489370571071	0.695307861742712\\
57.6240100041684	0.691484527843869\\
58.1242184243435	0.703679859278367\\
58.4243434764485	0.707490678453127\\
58.7244685285536	0.707505584839858\\
59.0245935806586	0.70370623274367\\
59.3997498957899	0.694889950574684\\
59.6248436848687	0.687351240184341\\
59.6498541058775	0.678851872754237\\
59.6748645268862	0.66131091238767\\
59.699874947895	0.630409119758582\\
59.7248853689037	0.58208070942004\\
59.7498957899125	0.51401120661545\\
59.7749062109212	0.427506025108435\\
59.8249270529387	0.229317574038404\\
59.8499374739475	0.141582252537219\\
59.8749478949562	0.0755012925976786\\
59.899958315965	0.0338812566064419\\
59.9249687369737	0.0124680967985711\\
59.9499791579825	0.00367341869250737\\
59.9749895789912	0.000849081312637168\\
60	0.000151740102459996\\
};
\addlegendentry{$\omega_1 + \Omega$};
\addplot [color=mode1]
  table[row sep=crcr]{%
0	4.69675297978256e-05\\
0.0750312630262613	0.00203747438293078\\
0.150062526052523	0.014662296392487\\
0.175072947061274	0.0372045438228596\\
0.200083368070032	0.0834448580514859\\
0.225093789078784	0.154123142888871\\
0.275114631096287	0.320050289847849\\
0.300125052105045	0.39051996351656\\
0.325135473113797	0.443665628991965\\
0.350145894122548	0.479336638242394\\
0.375156315131306	0.500656372415229\\
0.400166736140058	0.511921373097707\\
0.425177157148809	0.517126905856053\\
0.475197999166319	0.520386733031046\\
0.775323051271364	0.529416438908562\\
0.875364735306377	0.52930431158439\\
0.950395998332638	0.52665407824211\\
1.00041684035015	0.523175553777705\\
1.20050020842017	0.487596875697761\\
1.25052105043768	0.482153244654818\\
1.30054189245519	0.479406248866148\\
1.3505627344727	0.479658158419802\\
1.40058357649021	0.482949300075724\\
1.45060441850771	0.489064551332916\\
1.52563568153397	0.502558920571182\\
1.62567736556899	0.525310460264606\\
1.75072947061275	0.553528137689426\\
1.82576073363902	0.566673909545585\\
1.90079199666528	0.575127377739555\\
1.95081283868279	0.577688686083803\\
2.00083368070029	0.577622421203905\\
2.0508545227178	0.574908457838767\\
2.12588578574406	0.566100354291294\\
2.20091704877032	0.552330301461772\\
2.30095873280533	0.528591750624123\\
2.45102125885786	0.491034880523138\\
2.52605252188412	0.4765333898645\\
2.57607336390163	0.470015256355303\\
2.62609420591913	0.466605246545953\\
2.67611504793664	0.46657945566605\\
2.72613588995415	0.469967784537523\\
2.77615673197165	0.476549898249878\\
2.85118799499791	0.491401882545091\\
2.95122967903293	0.517079613924878\\
3.10129220508545	0.556531449137012\\
3.17632346811171	0.572049701738109\\
3.25135473113797	0.58268022590714\\
3.30137557315548	0.586575728558323\\
3.35139641517299	0.587738551712746\\
3.4014172571905	0.586134271699358\\
3.451438099208	0.581836476139962\\
3.52646936223427	0.570773224319986\\
3.60150062526053	0.555114086845151\\
3.70154230929554	0.529844280868701\\
3.82659441433931	0.498295587821332\\
3.90162567736557	0.48365707809451\\
3.95164651938308	0.477066128060024\\
4.00166736140059	0.473569448944616\\
4.05168820341809	0.473417033979274\\
4.1017090454356	0.476615584811796\\
4.15172988745311	0.482927562927095\\
4.22676115047937	0.497204313573839\\
4.32680283451438	0.52179875209363\\
4.45185493955815	0.553446398302192\\
4.52688620258441	0.56906855614227\\
4.60191746561067	0.580155689590072\\
4.65193830762818	0.584486306081132\\
4.70195914964569	0.586143017528933\\
4.75197999166319	0.585052712334893\\
4.8020008336807	0.581249281488745\\
4.87703209670696	0.570794549821223\\
4.95206335973322	0.555502458834106\\
5.05210504376824	0.53019297247743\\
5.17715714881201	0.497537834394414\\
5.25218841183827	0.481695005156283\\
5.30220925385577	0.474154106787431\\
5.35223009587328	0.469655937237093\\
5.40225093789079	0.4685140751807\\
5.45227177990829	0.470796196138387\\
5.5022926219258	0.476314844391148\\
5.57732388495207	0.489701072775333\\
5.65235514797833	0.507319473829028\\
5.85243851604835	0.5565860935817\\
5.92746977907461	0.569843199194011\\
6.00250104210087	0.577962442859032\\
6.05252188411838	0.580138884630344\\
6.10254272613589	0.579612953283345\\
6.1525635681534	0.576413035660352\\
6.22759483117966	0.566926822146854\\
6.30262609420592	0.552663892620167\\
6.40266777824093	0.528751102534528\\
6.55273030429345	0.49220274128885\\
6.62776156731972	0.478658538440513\\
6.67778240933723	0.472825854165158\\
6.72780325135473	0.470046304806736\\
6.77782409337224	0.470511186840199\\
6.82784493538975	0.474174163348088\\
6.87786577740725	0.48075797679892\\
6.95289704043351	0.495039259027962\\
7.05293872446853	0.518950707610891\\
7.1779908295123	0.54892771780964\\
7.25302209253856	0.563303976772815\\
7.32805335556482	0.573141670323665\\
7.37807419758233	0.576709612892984\\
7.42809503959983	0.577708874308115\\
7.47811588161734	0.576106408044467\\
7.5531471446436	0.569009368835303\\
7.62817840766986	0.556945371006861\\
7.70320967069613	0.541108488107305\\
7.92830345977491	0.489046096863383\\
8.00333472280117	0.477384546201165\\
8.05335556481867	0.472878542157787\\
8.10337640683618	0.471368196742873\\
8.15339724885369	0.472945329238961\\
8.2034180908712	0.477477599422315\\
8.27844935389746	0.489028593466031\\
8.35348061692372	0.504691336615089\\
8.55356398499375	0.54971206246249\\
8.62859524802001	0.562126845217861\\
8.70362651104627	0.569872659500639\\
8.75364735306378	0.572055489496144\\
8.80366819508129	0.571731838205253\\
8.87869945810755	0.566600190321211\\
8.95373072113381	0.556349784571012\\
9.02876198416007	0.541958053154971\\
9.15381408920383	0.51298862163987\\
9.25385577323885	0.490536877640416\\
9.32888703626511	0.477462514229629\\
9.37890787828262	0.471607133487417\\
9.42892872030013	0.46851023556539\\
9.47894956231763	0.468368066948258\\
9.52897040433514	0.471164304175652\\
9.57899124635265	0.476672335167031\\
9.65402250937891	0.48908783117907\\
9.75406419341392	0.510349274074827\\
9.87911629845769	0.537405298643797\\
9.95414756148395	0.55045071119126\\
10.0291788245102	0.559350214095637\\
10.1042100875365	0.563244942846424\\
10.1792413505627	0.56176945659027\\
10.254272613589	0.555035266803763\\
10.3293038766153	0.543669662601147\\
10.4293455606503	0.523290941270496\\
10.6044185077116	0.48465948796165\\
10.6794497707378	0.472183660319139\\
10.7294706127553	0.466622519744455\\
10.7794914547728	0.463707876313585\\
10.8295122967903	0.463621459007364\\
10.8795331388078	0.466343690813687\\
10.9545644018341	0.475167958055671\\
11.0295956648604	0.488367212235985\\
11.3047102959567	0.5427086796156\\
11.3797415589829	0.55130672939319\\
11.4547728220092	0.555073244558194\\
11.5298040850354	0.553641984706779\\
11.6048353480617	0.547135288998227\\
11.679866611088	0.536158233509326\\
11.779908295123	0.516535446699855\\
11.979991663193	0.474884613631382\\
12.0550229262193	0.464306939235655\\
12.1050437682368	0.460142412960167\\
12.1550646102543	0.458619486663508\\
12.2050854522718	0.459819252469039\\
12.280116715298	0.466427485673123\\
12.3551479783243	0.477763188164509\\
12.4551896623593	0.497273586881164\\
12.6052521884118	0.526435140153779\\
12.6802834514381	0.537203073242729\\
12.7553147144644	0.54377985332669\\
12.8303459774906	0.545492816292082\\
12.9053772405169	0.542146747897021\\
12.9804085035431	0.53402583582718\\
13.0554397665694	0.521884405628342\\
13.1554814506044	0.501584193918951\\
13.3055439766569	0.470570861325186\\
13.3805752396832	0.458895287909073\\
13.4556065027095	0.452013785342288\\
13.505627344727	0.450593804938954\\
13.5556481867445	0.451811273919532\\
13.6306794497707	0.458267369551969\\
13.705710712797	0.469236670308725\\
13.8307628178408	0.492883064829336\\
13.9558149228845	0.515571322391708\\
14.0308461859108	0.525556787755534\\
14.1058774489371	0.531424258454592\\
14.1809087119633	0.532563294226058\\
14.2559399749896	0.528856098877341\\
14.3309712380158	0.520576139142882\\
14.4060025010421	0.508486522900952\\
14.5310546060859	0.483277024430656\\
14.6561067111296	0.458507325456452\\
14.7311379741559	0.447422255349736\\
14.8061692371822	0.44109534422175\\
14.8561900791997	0.439982207455827\\
14.9062109212172	0.441430547037953\\
14.9812421842434	0.448060112330808\\
15.0562734472697	0.45898512700041\\
15.1813255523135	0.482234578027366\\
15.3063776573572	0.504261625936749\\
15.3814089203835	0.513839622765182\\
15.4564401834098	0.519357787472259\\
15.531471446436	0.520232532970368\\
15.6065027094623	0.516332658204696\\
15.6815339724885	0.507980066795078\\
15.7565652355148	0.495939814356191\\
15.8816173405586	0.471097327285193\\
16.0066694456023	0.446977950847639\\
16.0817007086286	0.436291540234762\\
16.1567319716549	0.43030601377086\\
16.2067528136724	0.429363476708097\\
16.2817840766986	0.432600664611385\\
16.3568153397249	0.440801289962998\\
16.4568570237599	0.457075480763073\\
16.6569403918299	0.492249157138119\\
16.7319716548562	0.501311282192077\\
16.8070029178825	0.506328559362338\\
16.8820341809087	0.506747596941239\\
16.957065443935	0.502463057151175\\
17.0320967069612	0.493818543690601\\
17.1071279699875	0.481596682479442\\
17.2321800750313	0.456686051992754\\
17.357232180075	0.432800620672936\\
17.4322634431013	0.422422656942778\\
17.5072947061276	0.416795940818339\\
17.5573155481451	0.416086154726607\\
17.6323468111713	0.419603267937788\\
17.7073780741976	0.427936466319004\\
17.8074197582326	0.444125111042993\\
17.9824927052939	0.474391751299606\\
18.0575239683201	0.483864183321998\\
18.1325552313464	0.489516631509311\\
18.2075864943727	0.490707277897336\\
18.2826177573989	0.487289384488044\\
18.3576490204252	0.479469548708096\\
18.4326802834514	0.467915364653798\\
18.5327219674865	0.448637982187478\\
18.682784493539	0.418979256405116\\
18.7578157565652	0.407691078749288\\
18.8328470195915	0.400899461082219\\
18.882867861609	0.399360671354508\\
18.9328887036265	0.400319736102439\\
19.0079199666528	0.406153303901952\\
19.082951229679	0.416277289065491\\
19.2080033347228	0.438385262969113\\
19.3330554397666	0.459610958908598\\
19.4080867027928	0.468921761940031\\
19.4831179658191	0.474336937012964\\
19.5581492288454	0.475254971634676\\
19.6331804918716	0.471506974773746\\
19.7082117548979	0.463358500664427\\
19.7832430179241	0.451503911574676\\
19.9082951229679	0.426683052769896\\
20.0333472280117	0.402057007171557\\
20.1083784910379	0.39087239109783\\
20.1834097540642	0.384276828993066\\
20.2334305960817	0.382910364888851\\
20.2834514380992	0.384060978965046\\
20.3584827011255	0.390181241893643\\
20.4335139641517	0.400538386538528\\
20.5585660691955	0.42263231941287\\
20.6836181742393	0.443572610786575\\
20.7586494372655	0.452623972960055\\
20.8336807002918	0.457718307119087\\
20.908711963318	0.458283136681438\\
20.9837432263443	0.454180043122342\\
21.0587744893706	0.445705795318474\\
21.1338057523968	0.433586932963486\\
21.2588578574406	0.408553807165575\\
21.3839099624844	0.384061211169737\\
21.4589412255106	0.373154195125565\\
21.5339724885369	0.367011364140595\\
21.5839933305544	0.366013022481461\\
21.6340141725719	0.367548805655204\\
21.7090454355982	0.374196694265628\\
21.7840766986244	0.384914319851625\\
22.0591913297207	0.430243172000708\\
22.134222592747	0.437121042668124\\
22.2092538557732	0.439635320143168\\
22.2842851187995	0.437443953491538\\
22.3593163818258	0.430644757652431\\
22.434347644852	0.419679913919431\\
22.534389328887	0.400241944709201\\
22.7344726969571	0.358561264083455\\
22.8095039599833	0.347761601753582\\
22.8595248020008	0.343448409438011\\
22.9095456440183	0.341815829050056\\
22.9595664860358	0.342965869355879\\
23.0095873280534	0.346788718708034\\
23.0846185910796	0.356822860220802\\
23.1846602751146	0.375676010112272\\
23.3597332221759	0.410651031894872\\
23.4347644852022	0.421860786942041\\
23.5097957482284	0.429042875396391\\
23.5848270112547	0.431516718578855\\
23.659858274281	0.429006896050161\\
23.7348895373072	0.421625596367718\\
23.8099208003335	0.409804164502106\\
23.8849520633597	0.394340024443707\\
23.9849937473947	0.369977359786574\\
24.160066694456	0.326357161557034\\
24.2350979574823	0.311007852213635\\
24.3101292205086	0.29933501997192\\
24.3851604835348	0.291854822483948\\
24.4601917465611	0.288483391401641\\
24.5602334305961	0.288743027498079\\
24.8353480616924	0.294023273607131\\
24.9353897457274	0.290942011661016\\
25.0354314297624	0.284388410933104\\
25.3605669028762	0.258663604118595\\
25.4355981659025	0.257642002894919\\
25.5106294289287	0.26016033213191\\
25.585660691955	0.266162319946908\\
25.68570237599	0.278219249164799\\
25.9107961650688	0.307617504212516\\
25.985827428095	0.313919444753189\\
26.0608586911213	0.317262238022991\\
26.1358899541476	0.31745699133667\\
26.2109212171738	0.314672917059319\\
26.3109629012088	0.307345645454433\\
26.5860775323051	0.284087395036188\\
26.6611087953314	0.28197123936436\\
26.7361400583576	0.282827708418303\\
26.8361817423927	0.288184934809188\\
26.9862442684452	0.301516830280555\\
27.111296373489	0.311926546675878\\
27.211338057524	0.31653476228341\\
27.2863693205502	0.31682999747931\\
27.3614005835765	0.314238556192237\\
27.4614422676115	0.306591525029859\\
27.5864943726553	0.292477354703017\\
27.7365568987078	0.275509498222824\\
27.8115881617341	0.270009526514215\\
27.8866194247603	0.267796722449106\\
27.9616506877866	0.269263701081314\\
28.0366819508128	0.273970842883202\\
28.1617340558566	0.286419414476633\\
28.3117965819091	0.301738417464193\\
28.4118382659441	0.308184775180493\\
28.4868695289704	0.309955811348075\\
28.5619007919967	0.308804992734814\\
28.6619424760317	0.303199606732449\\
28.7869945810754	0.291620388588193\\
28.9620675281367	0.275265487513366\\
29.037098791163	0.271380878783226\\
29.1121300541892	0.27067318803806\\
29.1871613172155	0.273319393657587\\
29.2872030012505	0.281384548928081\\
29.5873280533556	0.311636376384484\\
29.6623593163818	0.314870145538499\\
29.7373905794081	0.315170903270456\\
29.8124218424343	0.31255467233489\\
29.9124635264694	0.305354227908673\\
30.2125885785744	0.278923587180984\\
30.2876198416007	0.276797737455944\\
30.3626511046269	0.277762707483788\\
30.4626927886619	0.283453909657091\\
30.5877448937057	0.29506506689394\\
30.7378074197582	0.308766549890876\\
30.8378491037932	0.314001314692426\\
30.9128803668195	0.314710437403797\\
30.9879116298458	0.312505892527504\\
31.0879533138808	0.305342950737654\\
31.2130054189246	0.291868443646592\\
31.3630679449771	0.275952485855065\\
31.4380992080033	0.271075809718262\\
31.5131304710296	0.269546992326539\\
31.5881617340559	0.271640162841258\\
31.6882034180909	0.279235953606239\\
31.8632763651521	0.298611153902023\\
31.9883284701959	0.310120505770037\\
32.0883701542309	0.314648587685291\\
32.1634014172572	0.314630445936935\\
32.2634431012922	0.310156486215305\\
32.3634847853272	0.301505830193996\\
32.6385994164235	0.274625324806046\\
32.7136306794498	0.271698095558001\\
32.788661942476	0.272119272205543\\
32.8636932055023	0.275753908637689\\
32.9637348895373	0.284405744258216\\
33.1888286786161	0.305906641256712\\
33.2888703626511	0.310635186505358\\
33.3639016256774	0.31104257642189\\
33.4639433097124	0.307536434875814\\
33.5889954147561	0.29829487425274\\
33.7890787828262	0.282611791003724\\
33.8891204668612	0.279557894557023\\
33.9641517298875	0.280651218093858\\
34.0641934139225	0.286432376598391\\
34.1892455189662	0.2982727295364\\
34.3393080450188	0.312308177402485\\
34.4393497290538	0.317508031755573\\
34.51438099208	0.317943718899265\\
34.5894122551063	0.31501498816074\\
34.6644435181326	0.308808448992593\\
34.7644852021676	0.296293081661254\\
35.0395998332639	0.257477548038871\\
35.1146310962901	0.252775670556417\\
35.1896623593164	0.252735835183827\\
35.2646936223426	0.257411764748717\\
35.3397248853689	0.26622787656212\\
35.4397665694039	0.282663643814857\\
35.739891621509	0.336202354467368\\
35.839933305544	0.34898294622613\\
35.939974989579	0.358148019128436\\
36.040016673614	0.363919837060742\\
36.1650687786578	0.367097225913177\\
36.3151313047103	0.366587699046853\\
36.5652355147978	0.360844460080749\\
36.8653605669029	0.354702398100756\\
37.4406002501042	0.345549973593712\\
37.7407253022093	0.334693970054836\\
38.0158399333055	0.32574232148469\\
38.2159233013756	0.322878987414953\\
38.8912046686119	0.317567706365757\\
39.5414756148395	0.302024939234919\\
39.8916215089621	0.302218396818091\\
40.1167152980409	0.300457160077407\\
40.3418090871196	0.294779148305857\\
40.8420175072947	0.279912920905787\\
41.0921217173822	0.278152007746499\\
41.4422676115048	0.276150355341024\\
41.6673614005836	0.270518926733075\\
42.2426010837849	0.252906641144847\\
42.4927052938724	0.251807123735112\\
42.8178407669862	0.250401367655719\\
43.0179241350563	0.245818581189532\\
43.6431846602751	0.227936343112177\\
43.9182992913714	0.227394553557197\\
44.1934139224677	0.225766748407061\\
44.3934972905377	0.22084857022044\\
45.0187578157566	0.201434618663491\\
45.2938724468529	0.200450266633517\\
45.5689870779491	0.198476000740023\\
45.7690704460192	0.193330514599943\\
46.3693205502293	0.173978136139198\\
46.594414339308	0.172994474024833\\
46.9195498124218	0.171824653792235\\
47.1196331804919	0.167325581353964\\
47.4447686536057	0.155249661377162\\
47.6698624426845	0.148800760722118\\
47.8699458107545	0.146791962112012\\
48.3701542309296	0.143969464596381\\
48.5702375989996	0.137939177148553\\
49.0704460191747	0.12043148754335\\
49.2705293872447	0.118507358435657\\
49.745727386411	0.116140567833845\\
49.945810754481	0.110542430193682\\
50.5210504376824	0.0914202969385158\\
50.7461442267612	0.0903716756456987\\
51.0962901208837	0.088914566092626\\
51.2963734889537	0.0842052960331046\\
51.9216340141726	0.0660979478346562\\
52.1717382242601	0.0659218663976162\\
52.4718632763652	0.0653999556875178\\
52.6969570654439	0.0610522912035449\\
53.0721133805752	0.0486216871456406\\
53.2721967486453	0.0441001666123455\\
53.4472696957065	0.0436194180026135\\
53.6973739057941	0.0470688408154132\\
54.0475197999166	0.051389794999551\\
54.2976240100042	0.0504082743174905\\
54.5977490621092	0.0452421802844682\\
55.3480616923718	0.0301332467094753\\
55.6231763234681	0.0320131570056077\\
56.1233847436432	0.0340917633586812\\
56.6736140058358	0.032543318667706\\
57.4739474781159	0.0297905953077162\\
58.7494789495623	0.0309718174464493\\
59.6748645268862	0.0279748547087308\\
59.7248853689037	0.0231266913330117\\
59.8499374739475	0.00394784313498064\\
59.899958315965	0.00149884556923041\\
};
\addlegendentry{$\omega_1$};
\addplot [color=mode3]
  table[row sep=crcr]{%
0	1.93533878700691e-07\\
5.75239683201334	0.00189966583372581\\
7.15298040850355	0.00329485981568922\\
8.55356398499375	0.00187382504186928\\
10.2042517715715	0.00211720105225766\\
11.8299291371405	0.00297493769918589\\
14.1809087119633	0.00178622677617568\\
15.7565652355148	0.00301831181861445\\
18.5077115464777	0.00203403743309849\\
20.0083368070029	0.00259004375094207\\
22.434347644852	0.00156839555094024\\
24.0350145894123	0.00251004892046325\\
25.1604835348062	0.00244579784320109\\
26.6861192163401	0.00241562177268406\\
28.1867444768654	0.00234969461135393\\
29.7373905794081	0.00255221024752927\\
31.3630679449771	0.00240407125908604\\
32.9137140475198	0.00252422297410959\\
34.51438099208	0.00245503840076111\\
36.0900375156315	0.00243781718930336\\
37.8157565652355	0.00313689879859425\\
39.4664443518133	0.00348250933375738\\
41.7173822426011	0.00279063989784589\\
43.7932471863276	0.00323436368918095\\
46.0441850771155	0.0028485758109511\\
48.3701542309296	0.00282215078657799\\
51.4964568570238	0.00302468841288572\\
59.8249270529387	0.000713963086397484\\
60	7.48335516220777e-07\\
};
\addlegendentry{$\omega_1 - \Omega$};
\end{axis}
%\end{tikzpicture}%
\end{tikzpicture}%
	\caption[Modal field profiles at each sideband $n$ for the left-to-right direction]{Propagation in the forward (RL) direction for the photonic AB effect. \textbf{(a)} $E_z$ field profiles of the total field, and its separate components in the first sidebands. \textbf{(b)} Corresponding modal amplitudes of each sideband propagating through the modulated region. Mode $\ket{1}$ is completely converted to $\ket{2}$.}
	\label{fig:LRFang}
\end{figure}

\begin{figure}[t!]
	\centering
	\setlength{\figH}{1\textwidth}
	\setlength{\figW}{1\textwidth}   
	% This file was created by matlab2tikz.
%
%The latest updates can be retrieved from
%  http://www.mathworks.com/matlabcentral/fileexchange/22022-matlab2tikz-matlab2tikz
%where you can also make suggestions and rate matlab2tikz.
%
\pgfplotsset{fangFields/.style={
	width=0.8\figW,
height=0.08\figH,
	scale only axis,
	axis on top,
	title style = {at={(0.4\figH,0.06\figH)}},
	xmin=0,
	xmax=60,
	ymin=0,
	ymax=5,
	clip=false,
	axis background/.style={fill=white},
	colormap={mymap}{[1pt] rgb(0pt)=(0,0,1); rgb(31pt)=(1,1,1); rgb(32pt)=(1,1,1); rgb(63pt)=(1,0,0)},
	colorbar,
	colorbar style={width=.01\linewidth, at={(1.02,0.04\figH)}, anchor=east}	
	}
}

\begin{tikzpicture}
\begin{axis}[%
fangFields,
at={(0\figW,0.95\figH)},
title = {Total fields},
point meta min=-19.1125620409379,
point meta max=19.1125620409379,
xticklabels={,,},
colorbar style={title = {$V / \mu m$}, width=.01\linewidth, at={(1.02,0.04\figH)}, anchor=east},
ylabel={$\text{y (}\mu\text{m)}$}
]
\addplot [forget plot] graphics [xmin=0, xmax=60, ymin=0, ymax=5] {graphs/fangcavity/RL/totalfield.png};
\draw (0,1.95) -- (24.2,1.95) -- (24.2,1.5) -- (35.8,1.5) -- (35.8,1.95) -- (60,1.95);
\draw (0,3.05) -- (24.2,3.05) -- (24.2,3.5) -- (35.8,3.5) -- (35.8,3.05) -- (60,3.05);
\node at (55,6) {$ \longleftarrow \bra{1}$};
\node at (5,6) {$\bra{1} \longleftarrow $};
\node at (-2,7) {\textbf{(a)}};
\end{axis}

\begin{axis}[%
fangFields,
at={(0\figW,0.8\figH)},
title = {$\omega_1 - \Omega$},
title style = {at={(0.4\figH,0.06\figH)}},
point meta min=-0.24239253622059,
point meta max=0.24239253622059,
xticklabels={,,},
ylabel={$\text{y (}\mu\text{m)}$}
]
\addplot [forget plot] graphics [xmin=0, xmax=60, ymin=0, ymax=5] {graphs/fangcavity/RL/fields-1.png};
\draw (0,1.95) -- (24.2,1.95) -- (24.2,1.5) -- (35.8,1.5) -- (35.8,1.95) -- (60,1.95);
\draw (0,3.05) -- (24.2,3.05) -- (24.2,3.5) -- (35.8,3.5) -- (35.8,3.05) -- (60,3.05);
\end{axis}

\begin{axis}[%
fangFields,
at={(0\figW,0.65\figH)},
title = {$\omega_1$},
point meta min=-14.5540242292171,
point meta max=14.5540242292171,
xticklabels={,,},
ylabel={$\text{y (}\mu\text{m)}$}
]
\addplot [forget plot] graphics [xmin=0, xmax=60, ymin=0, ymax=5] {graphs/fangcavity/RL/fields-2.png};
\draw (0,1.95) -- (24.2,1.95) -- (24.2,1.5) -- (35.8,1.5) -- (35.8,1.95) -- (60,1.95);
\draw (0,3.05) -- (24.2,3.05) -- (24.2,3.5) -- (35.8,3.5) -- (35.8,3.05) -- (60,3.05);
\end{axis}

\begin{axis}[%
fangFields,
width=0.8\figW,
height=0.08\figH,
at={(0\figW,0.5\figH)},
title = {$\omega_1 + \Omega$},
point meta min=-13.8291082848291,
point meta max=13.8291082848291,
xlabel={$\text{x (}\mu\text{m)}$},
ylabel={$\text{y (}\mu\text{m)}$}
]
\addplot [forget plot] graphics [xmin=0, xmax=60, ymin=0, ymax=5] {graphs/fangcavity/RL/fields-3.png};
\draw (0,1.95) -- (24.2,1.95) -- (24.2,1.5) -- (35.8,1.5) -- (35.8,1.95) -- (60,1.95);
\draw (0,3.05) -- (24.2,3.05) -- (24.2,3.5) -- (35.8,3.5) -- (35.8,3.05) -- (60,3.05);
\node at (-2,-3) {\textbf{(b)}};
\end{axis}

% This file was created by matlab2tikz.
%
%The latest updates can be retrieved from
%  http://www.mathworks.com/matlabcentral/fileexchange/22022-matlab2tikz-matlab2tikz
%where you can also make suggestions and rate matlab2tikz.
%
\definecolor{mycolor1}{rgb}{0.00000,0.44700,0.74100}%
\definecolor{mycolor2}{rgb}{0.85000,0.32500,0.09800}%
\definecolor{mycolor3}{rgb}{0.92900,0.69400,0.12500}%
%
\begin{axis}[%
width=0.8\figW,
height=0.4\figH,
at={(0\figW,0\figH)},
scale only axis,
xmin=0,
xmax=60,
ymin=0,
ymax=1.4,
xlabel = {x ($\mu m$)},
ylabel = {Modal amplitude ($W / \mu m$)},
axis background/.style={fill=white},
legend style={legend cell align=left, align=left, draw=white!15!black}
]
\addplot [color=mode1]
  table[row sep=crcr]{%
0	7.51619029415451e-06\\
0.200166805671394	0.00379037548862016\\
0.250208507089241	0.0103601153182282\\
0.300250208507087	0.0231981085613455\\
0.350291909924934	0.0437211994866402\\
0.450375312760634	0.10283820788451\\
0.550458715596328	0.163001450910691\\
0.600500417014182	0.18581025676545\\
0.650542118432028	0.20242126113466\\
0.750625521267722	0.220002134869326\\
0.900750625521269	0.225944525166042\\
1.40116763969975	0.22483554386806\\
2.35195996663887	0.219527655919862\\
3.00250208507089	0.22476780270425\\
3.60300250208507	0.219009513596532\\
4.1534612176814	0.213890144274416\\
5.40450375312761	0.235305435729153\\
5.70475396163469	0.229285938999453\\
5.95496246872393	0.224124501842226\\
6.10508757297748	0.230969959065362\\
6.30525437864888	0.251684596496411\\
6.50542118432027	0.272397536117424\\
6.65554628857381	0.277817982275955\\
6.80567139282736	0.273050794410679\\
7.0558798999166	0.261220534402\\
7.20600500417014	0.267535904975972\\
7.35613010842368	0.286094405731575\\
7.60633861551293	0.318725432360985\\
7.75646371976647	0.325622571921961\\
7.90658882402002	0.321645900471353\\
8.20683903252711	0.307727985506261\\
8.35696413678065	0.310281568738077\\
8.70725604670559	0.332592705907345\\
9.00750625521268	0.345237433032345\\
9.45788156797331	0.353653170266654\\
9.90825688073394	0.355653334455702\\
10.0083402835696	0.364679439856367\\
10.5087572977481	0.423492067463556\\
10.6588824020017	0.429407397829756\\
10.9090909090909	0.423915529439718\\
11.0592160133445	0.424417566730099\\
11.209341117598	0.43642034810437\\
11.4095079232694	0.466761528807211\\
11.6096747289408	0.495010747486667\\
11.7597998331943	0.505105368575471\\
11.9599666388657	0.504288185752898\\
12.210175145955	0.500071503182568\\
12.4103419516264	0.507423290042148\\
13.0608840700584	0.544651632514707\\
13.4612176814012	0.556727299106747\\
13.861551292744	0.558470625781382\\
14.2118432026689	0.597966433988304\\
14.4120100083403	0.615888347359217\\
14.5621351125938	0.63096089920041\\
14.7122602168474	0.635995457809962\\
15.162635529608	0.637094561896923\\
15.3127606338616	0.654205825225603\\
15.6630525437865	0.704333334989677\\
15.81317764804	0.713654827992222\\
16.0133444537114	0.713661895165444\\
16.3135946622185	0.713013977343905\\
16.5638031693078	0.724878669393952\\
17.2143452877398	0.758946478744448\\
17.5646371976647	0.768706835310908\\
17.8648874061718	0.771050390927911\\
18.3653044203503	0.815466107797768\\
18.465387823186	0.826199846687366\\
18.6155129274395	0.841602248122278\\
18.7656380316931	0.847000003323359\\
19.2160133444537	0.848853023965553\\
19.3661384487073	0.864680855721943\\
19.7164303586322	0.909942581170348\\
19.8665554628857	0.918128768536\\
20.0667222685571	0.91817860067524\\
20.3669724770642	0.917578638515771\\
20.8673894912427	0.934320737047734\\
21.1175979983319	0.946600942269129\\
21.2677231025855	0.958665337128288\\
21.417848206839	0.96096187398296\\
21.6680567139283	0.962401342948368\\
21.8682235195997	0.962799652401614\\
22.418682235196	1.00155206507281\\
22.7189324437031	1.03540832669466\\
22.8690575479566	1.04008319587251\\
23.2193494578816	1.04477880455485\\
23.3694745621351	1.05827679861182\\
23.7197664720601	1.09874577114226\\
23.8198498748957	1.10185723330824\\
23.9199332777314	1.0974175828938\\
24.0200166805671	1.08342888280995\\
24.1201000834028	1.0560965277534\\
24.2201834862385	1.02052299499949\\
24.2702251876564	0.991482708443833\\
24.3202668890742	0.956445443942805\\
24.3703085904921	0.908273305174724\\
24.4703919933278	0.788427220458296\\
24.5704753961635	0.674149775049109\\
24.6205170975813	0.641337315016969\\
24.6705587989992	0.632616246927363\\
24.720600500417	0.650706302748169\\
24.7706422018349	0.691824691015697\\
24.8707256046706	0.809494443931349\\
24.9207673060884	0.868785099566772\\
24.9708090075063	0.919427217249449\\
25.0208507089241	0.958443805437632\\
25.070892410342	0.980993956078393\\
25.1209341117598	0.985513973689898\\
25.1709758131776	0.971798874580756\\
25.2210175145955	0.940986050474812\\
25.2710592160133	0.895610661761872\\
25.4712260216847	0.675533072761858\\
25.5212677231026	0.651024484423836\\
25.5713094245204	0.652177324906667\\
25.6213511259383	0.678746085987342\\
25.6713928273561	0.726009493927641\\
25.8215179316097	0.900441103084681\\
25.8715596330275	0.945001900725714\\
25.9216013344454	0.974855042049533\\
25.9716430358632	0.987624605373888\\
26.0216847372811	0.982329665679913\\
26.0717264386989	0.959334804299296\\
26.1217681401168	0.920371160633351\\
26.1718098415346	0.868630275633699\\
26.3219349457882	0.693930509144607\\
26.371976647206	0.656409766385586\\
26.4220183486239	0.643305303850148\\
26.4720600500417	0.65778698881833\\
26.5221017514596	0.696488675755973\\
26.6221851542952	0.812995137087022\\
26.6722268557131	0.873075721511206\\
26.7222685571309	0.925001934443635\\
26.7723102585488	0.96398153112694\\
26.8223519599666	0.986832121214924\\
26.8723936613845	0.991819936863251\\
26.9224353628023	0.978569266459473\\
26.9724770642202	0.948043738422214\\
27.022518765638	0.902597976493148\\
27.1226021684737	0.784131985007519\\
27.1726438698916	0.724105212524641\\
27.2226855713094	0.675033449908973\\
27.2727272727273	0.646130141816457\\
27.3227689741451	0.643805983830511\\
27.372810675563	0.668514868001139\\
27.4228523769808	0.71448402778897\\
27.5729774812344	0.890220249017702\\
27.6230191826522	0.936115101785042\\
27.6730608840701	0.967369245623615\\
27.7231025854879	0.981433447666888\\
27.7731442869058	0.977170963380424\\
27.8231859883236	0.954797654838039\\
27.8732276897415	0.915896165444444\\
27.9232693911593	0.863507708094858\\
28.0733944954128	0.681534161420828\\
28.1234361968307	0.640176003165841\\
28.1734778982485	0.624086093572892\\
28.2235195996664	0.639742245618748\\
28.2735613010842	0.680577820449315\\
28.3736447039199	0.801861763773744\\
28.4236864053378	0.863869416161499\\
28.4737281067556	0.917269543074958\\
28.5237698081735	0.957289465363047\\
28.5738115095913	0.980806100894839\\
28.6238532110092	0.986164608402262\\
28.673894912427	0.973081184304839\\
28.7239366138449	0.942624120341847\\
28.7739783152627	0.897268745552651\\
28.8740617180984	0.779641601766087\\
28.9241034195163	0.720708611107369\\
28.9741451209341	0.673374990703337\\
29.024186822352	0.646823204939146\\
29.0742285237698	0.647162274230837\\
29.1242702251877	0.674384393042907\\
29.1743119266055	0.72238218785801\\
29.3244370308591	0.901300598916244\\
29.3744787322769	0.947740181713264\\
29.4245204336947	0.979516024657485\\
29.4745621351126	0.994200993935721\\
29.5246038365304	0.990773152527517\\
29.5746455379483	0.969559137352782\\
29.6246872393661	0.93224747620264\\
29.674728940784	0.881971055120125\\
29.8248540450375	0.709303962076177\\
29.8748957464554	0.670975698963893\\
29.9249374478732	0.656069194938993\\
29.9749791492911	0.668022368601385\\
30.0250208507089	0.70401919989439\\
30.0750625521268	0.756491015880385\\
30.1751459549625	0.875188197823697\\
30.2251876563803	0.926446918046132\\
30.2752293577982	0.96522290539324\\
30.325271059216	0.988253639493138\\
30.3753127606339	0.993735509656446\\
30.4253544620517	0.981232329456056\\
30.4753961634696	0.951652824039577\\
30.5254378648874	0.907296620945765\\
30.6255212677231	0.791160680496375\\
30.675562969141	0.732154025030283\\
30.7256046705588	0.683770475944286\\
30.7756463719766	0.655015187674728\\
30.8256880733945	0.652212532953854\\
30.8757297748123	0.675982402608021\\
30.9257714762302	0.720891626872472\\
31.0758965804837	0.895096470706683\\
31.1259382819016	0.941258840392806\\
31.1759799833194	0.973138821677743\\
31.2260216847373	0.988139379698659\\
31.2760633861551	0.985068247804996\\
31.326105087573	0.964077652391005\\
31.3761467889908	0.926674571726359\\
31.4261884904087	0.875804015316852\\
31.5763135946622	0.696852391553534\\
31.6263552960801	0.655120852908105\\
31.6763969974979	0.637008854049689\\
31.7264386989158	0.646764939057938\\
31.7764804003336	0.681994997187445\\
31.8265221017515	0.735002446495599\\
31.9266055045872	0.857104863150383\\
31.976647206005	0.910553513159321\\
32.0266889074229	0.951527946293091\\
32.0767306088407	0.976586404450934\\
32.1267723102586	0.983759888427151\\
32.1768140116764	0.972445503342605\\
32.2268557130942	0.943375437649685\\
32.2768974145121	0.898659893451537\\
32.3769808173478	0.778445711178229\\
32.4270225187656	0.716300420914969\\
32.4770642201835	0.665046167740499\\
32.5271059216013	0.634015946181265\\
32.5771476230192	0.630312617619666\\
32.627189324437	0.654844095881565\\
32.6772310258549	0.701792137091012\\
32.8273561301084	0.883873201177373\\
32.8773978315263	0.932139939620974\\
32.9274395329441	0.965675268038702\\
32.977481234362	0.981870616116673\\
33.0275229357798	0.979550898849077\\
33.0775646371977	0.958906533436334\\
33.1276063386155	0.92150271250938\\
33.1776480400334	0.870366709028389\\
33.3277731442869	0.690452119806906\\
33.3778148457048	0.649080698060239\\
33.4278565471226	0.632023973261525\\
33.4778982485404	0.643455625198136\\
33.5279399499583	0.680646983923126\\
33.5779816513761	0.735537743870985\\
33.6780650542118	0.860474059892397\\
33.7281067556297	0.914837622072781\\
33.7781484570475	0.956464593755825\\
33.8281901584654	0.981995571763015\\
33.8782318598832	0.989556527419701\\
33.9282735613011	0.978652074184375\\
33.9783152627189	0.950135976941233\\
34.0283569641368	0.90625469121602\\
34.1284403669725	0.789125622897231\\
34.1784820683903	0.728644770062154\\
34.2285237698082	0.678299977992793\\
34.278565471226	0.64764033288435\\
34.3286071726439	0.643442246673565\\
34.3786488740617	0.66603478541537\\
34.4286905754796	0.710982377813757\\
34.628857381151	0.936326782492799\\
34.6788990825688	0.969604966669166\\
34.7289407839867	0.9858358003086\\
34.7789824854045	0.983750353915745\\
34.8290241868224	0.963427529969259\\
34.8790658882402	0.926298373455708\\
34.929107589658	0.875226165038555\\
35.0792326939116	0.693971860453239\\
35.1292743953294	0.651655176039341\\
35.1793160967473	0.632482524768861\\
35.2293577981651	0.640926312581229\\
35.279399499583	0.674845091031649\\
35.3294412010008	0.726666392969712\\
35.4795663052544	0.900788742854857\\
35.5296080066722	0.942538282449604\\
35.5796497080901	0.967876227401099\\
35.6296914095079	0.974450810947005\\
35.6797331109258	0.96577995539193\\
35.8798999165972	0.870138851170218\\
35.929941618015	0.868202636998426\\
35.9799833194329	0.877920394473797\\
36.0300250208507	0.899096106246382\\
36.0800667222686	0.93120254494589\\
36.1301084236864	0.972424200768366\\
36.2301918265221	1.07157551569595\\
36.3302752293578	1.17422548176243\\
36.4303586321935	1.26108467931033\\
36.4804003336113	1.29407174796548\\
36.5304420350292	1.31839047510635\\
36.580483736447	1.33317556031317\\
36.6305254378649	1.33787600444352\\
36.6805671392827	1.33224490321312\\
36.7306088407006	1.31633389208083\\
36.7806505421184	1.29049273954153\\
36.8306922435363	1.25537480779456\\
36.8807339449541	1.21194945676792\\
36.9808173477898	1.10576671553439\\
37.1309424520434	0.931886317839954\\
37.1809841534612	0.882051843477846\\
37.2310258548791	0.840910853413149\\
37.2810675562969	0.812006552706855\\
37.3311092577148	0.798149048449602\\
37.3811509591326	0.800737339086403\\
37.4311926605505	0.819350570729121\\
37.4812343619683	0.851865636419561\\
37.5312760633862	0.895025444514268\\
37.6814011676397	1.05208140530552\\
37.7814845704754	1.14887051961207\\
37.8315262718932	1.18806290793083\\
37.8815679733111	1.21911582990713\\
37.9316096747289	1.24099835646513\\
37.9816513761468	1.25301922847829\\
38.0316930775646	1.25480526241744\\
38.0817347789825	1.2462877074314\\
38.1317764804003	1.2276956769385\\
38.1818181818182	1.19955633945947\\
38.231859883236	1.16270222095802\\
38.3319432860717	1.0678073002875\\
38.5321100917431	0.849123757728101\\
38.582151793161	0.804863774869389\\
38.6321934945788	0.771543936628873\\
38.6822351959967	0.752105446927274\\
38.7322768974145	0.748292748800949\\
38.7823185988324	0.760150825689053\\
38.8323603002502	0.785994471366656\\
38.8824020016681	0.822876822349834\\
38.9824854045038	0.91564211605256\\
39.1326105087573	1.06833918184795\\
39.1826522101751	1.1098101377232\\
39.232693911593	1.14311319095918\\
39.2827356130108	1.16720276457171\\
39.3327773144287	1.18136210864686\\
39.3828190158465	1.18518163133432\\
39.4328607172644	1.17854665920254\\
39.4829024186822	1.16163305548611\\
39.5329441201001	1.13491010202435\\
39.5829858215179	1.09915107971041\\
39.6830692243536	1.00526700908505\\
39.9332777314429	0.738360638525116\\
39.9833194328607	0.704466228550494\\
40.0333611342786	0.685371139032448\\
40.0834028356964	0.683076011562576\\
40.1334445371143	0.697569117950295\\
40.1834862385321	0.726814369238205\\
40.23352793995	0.76739512023407\\
40.3336113427857	0.866954963490301\\
40.4336947456214	0.968139656555522\\
40.533778148457	1.05081490856861\\
40.5838198498749	1.08102974652988\\
40.6338615512927	1.10237797750186\\
40.6839032527106	1.11417828310618\\
40.7339449541284	1.11604003447898\\
40.7839866555463	1.10784535191014\\
40.8340283569641	1.08973890689695\\
40.884070058382	1.06212479475184\\
40.9841534612177	0.986777724250587\\
41.0842368640534	0.891063190156075\\
41.2844036697248	0.675447422095068\\
41.3344453711426	0.634457242291859\\
41.3844870725605	0.606456476010315\\
41.4345287739783	0.594697962852941\\
41.4845704753962	0.600592273652346\\
41.534612176814	0.623149593065754\\
41.5846538782319	0.659354850976769\\
41.6847372810676	0.756463665308239\\
41.8348623853211	0.909571687185334\\
41.8849040867389	0.95207824184827\\
41.9349457881568	0.987445455021508\\
41.9849874895746	1.01454842894288\\
42.0350291909925	1.03258625755011\\
42.0850708924103	1.04104873978643\\
42.1351125938282	1.03969442240921\\
42.185154295246	1.02853717471901\\
42.2351959966639	1.00783971487943\\
42.2852376980817	0.978113744529445\\
42.3352793994996	0.940127595334943\\
42.4353628023353	0.843848838949903\\
42.6355296080067	0.621709287305762\\
42.6855713094245	0.576409574478845\\
42.7356130108424	0.542779799403597\\
42.7856547122602	0.524434763863574\\
42.8356964136781	0.523413619180708\\
42.8857381150959	0.539319532641407\\
42.9357798165138	0.569448342746753\\
42.9858215179316	0.610805393534342\\
43.185988323603	0.824841523491187\\
43.2360300250209	0.868967333111264\\
43.2860717264387	0.9058300080953\\
43.3361134278565	0.9343715020077\\
43.3861551292744	0.953830851821401\\
43.4361968306922	0.963713634214137\\
43.4862385321101	0.963773944931653\\
43.5362802335279	0.954005044502331\\
43.5863219349458	0.934636678754472\\
43.6363636363636	0.906138676979474\\
43.6864053377815	0.869231969465218\\
43.7864887406172	0.774474655239523\\
43.9866555462886	0.552604067592391\\
44.0366972477064	0.507588723433265\\
44.0867389491243	0.47517592808277\\
44.1367806505421	0.459485609825002\\
44.18682235196	0.462610900055815\\
44.2368640533778	0.483558243484552\\
44.2869057547957	0.518707218025675\\
44.3869891576314	0.612904432148845\\
44.4870725604671	0.712656591264562\\
44.5871559633027	0.796906797193792\\
44.6371976647206	0.829167847913453\\
44.6872393661384	0.853481166963242\\
44.7372810675563	0.869188537123321\\
44.7873227689741	0.875880391068677\\
44.837364470392	0.87337165305776\\
44.9374478732277	0.845305340506904\\
44.9874895746455	0.825531229229\\
45.0375312760634	0.797289094515605\\
45.0875729774812	0.761220050841246\\
45.1876563803169	0.669433656305692\\
45.3878231859883	0.45524845698948\\
45.4378648874062	0.412533831545076\\
45.487906588824	0.383301350065764\\
45.5379482902419	0.372140403991381\\
45.5879899916597	0.38090899272801\\
45.6380316930776	0.407550002061775\\
45.6880733944954	0.447302039104805\\
45.838198498749	0.596502559281227\\
45.9382819015847	0.687838216112318\\
45.9883236030025	0.725083467637603\\
46.0383653044204	0.755167388697622\\
46.0884070058382	0.777316855934991\\
46.138448707256	0.791007465570488\\
46.1884904086739	0.795934752617676\\
46.2385321100917	0.791995774990873\\
46.2885738115096	0.779277232752165\\
46.3386155129274	0.758048241392316\\
46.3886572143453	0.728757357440983\\
46.4386989157631	0.692034859939717\\
46.5387823185988	0.599799951344004\\
46.7889908256881	0.333506559596479\\
46.8390325271059	0.298055104781383\\
46.8890742285238	0.280189358490148\\
46.9391159299416	0.283297514214375\\
46.9891576313595	0.311409626776694\\
47.0892410341952	0.410455498664284\\
47.2393661384487	0.565394489098615\\
47.3394495412844	0.645883152899685\\
47.3894912427022	0.675679733785074\\
47.4395329441201	0.697575921756012\\
47.4895746455379	0.711125892247196\\
47.5396163469558	0.716066277162057\\
47.5896580483736	0.712307373059346\\
47.6396997497915	0.69992847404329\\
47.6897414512093	0.679175978522814\\
47.7397831526272	0.650464267791122\\
47.789824854045	0.61438073209878\\
47.8899082568807	0.523402001166943\\
48.14011676397	0.260467343882844\\
48.1901584653878	0.229216820185385\\
48.2402001668057	0.220418738432834\\
48.2902418682235	0.236111497409667\\
48.3402835696414	0.269717631010394\\
48.4403669724771	0.361156907621513\\
48.5404503753128	0.455122333165889\\
48.6405337781485	0.533635007185644\\
48.6905754795663	0.563842520251988\\
48.7406171809842	0.587097023851257\\
48.790658882402	0.602976348405562\\
48.8407005838199	0.611232038636238\\
48.8907422852377	0.611770137764886\\
49.0408673894912	0.596094943576695\\
49.0909090909091	0.579304643159766\\
49.1409507923269	0.555293772191369\\
49.1909924937448	0.524464207870835\\
49.2910758965805	0.44466632965576\\
49.3911592994162	0.346276356618269\\
49.5412844036697	0.191321690255798\\
49.5913261050876	0.153931547064303\\
49.6413678065054	0.138924077328952\\
49.6914095079233	0.152609816702693\\
49.791492910759	0.245451290222178\\
49.8915763135947	0.336281315373796\\
50.0417014178482	0.449371762028086\\
50.0917431192661	0.478356175703603\\
50.1417848206839	0.500837942984234\\
50.1918265221017	0.51648067722887\\
50.2418682235196	0.525090817676343\\
50.2919099249375	0.526606565224377\\
50.3419516263553	0.521089502414625\\
50.3919933277731	0.508716933091257\\
50.442035029191	0.489773980993043\\
50.4920767306088	0.464645013300185\\
50.5921601334445	0.39780576070514\\
50.6922435362802	0.312888300569767\\
50.8423686405338	0.164236808150669\\
50.9424520433695	0.0637724162456337\\
50.9924937447873	0.0383173321681198\\
51.2427022518766	0.279792471343896\\
51.3427856547123	0.355726661643594\\
51.442869057548	0.410595615257698\\
51.4929107589658	0.429078666631568\\
51.5429524603837	0.44122043587673\\
51.5929941618015	0.446866423734114\\
51.6430358632193	0.445966646282507\\
51.6930775646372	0.438576930222922\\
51.743119266055	0.424858388991858\\
51.7931609674729	0.405075080911303\\
52.1434528773978	0.222125051587831\\
52.3436196830692	0.0961769300234749\\
52.3936613844871	0.115643264733379\\
52.4937447873228	0.195297003374513\\
52.5938281901585	0.26217477268009\\
52.6939115929942	0.31005466540293\\
52.743953294412	0.326134670382537\\
52.7939949958299	0.336797588728494\\
52.8940783986656	0.343225427349488\\
52.9941618015012	0.33442900378158\\
53.094245204337	0.344025595245462\\
53.1943286071726	0.335361416155799\\
53.2944120100083	0.309322452698211\\
53.394495412844	0.268343012987799\\
53.5446205170976	0.188741405093694\\
53.5946622185154	0.161077478547888\\
53.6447039199333	0.154264071160718\\
53.8949124270225	0.219464091745365\\
53.9949958298582	0.233627407076362\\
54.1451209341118	0.240132705297114\\
54.2952460383653	0.242806000980828\\
54.395329441201	0.244501365394171\\
54.5454545454545	0.232290156311244\\
54.7456213511259	0.214671857303813\\
54.9457881567973	0.215716146940899\\
55.7964970809008	0.22188530868484\\
56.5971643035863	0.216600307605241\\
56.8974145120934	0.217303121122896\\
57.7481234361968	0.223596971562024\\
58.4987489574646	0.218159443649022\\
58.8490408673895	0.21787522507887\\
59.2493744787323	0.220124140112702\\
59.349457881568	0.211780617829888\\
59.3994995829858	0.201701461227742\\
59.4495412844037	0.185917043657199\\
59.4995829858215	0.16373574667503\\
59.5996663886572	0.10391212455314\\
59.6997497914929	0.0442523631249827\\
59.7497914929108	0.0234689020727572\\
59.7998331943286	0.0104745610962311\\
59.8999165971643	0.00112148232856413\\
60	4.45548702074916e-05\\
};
\addlegendentry{$\omega_1+\Omega$}

\addplot [color=mode2]
  table[row sep=crcr]{%
0	3.45045436702662e-05\\
0.200166805671394	0.00237412030661233\\
0.250208507089241	0.0104637743404083\\
0.300250208507087	0.0352602558046726\\
0.350291909924934	0.0923281083697916\\
0.400333611342788	0.194136927796009\\
0.450375312760634	0.33924124659967\\
0.550458715596328	0.678486101605031\\
0.600500417014182	0.82358809492127\\
0.650542118432028	0.933096014614854\\
0.700583819849875	1.00629560271609\\
0.750625521267722	1.04949124647134\\
0.800667222685568	1.07157253938584\\
0.850708924103422	1.08090731849324\\
0.950792326939116	1.08438677853061\\
3.95329441201001	1.08397726865935\\
7.95663052543787	1.0761382646399\\
10.208507089241	1.06083779782121\\
10.6588824020017	1.05544573415797\\
11.3094245204337	1.04652551428335\\
11.9599666388657	1.04121308567235\\
12.7105921601334	1.0293005123318\\
13.2610508757298	1.02503876809016\\
14.0116763969975	1.00969431868931\\
14.5621351125938	1.00544355453479\\
15.4128440366973	0.985763594263055\\
15.9132610508757	0.98060480693804\\
16.814011676397	0.958186548949293\\
17.2643869891576	0.953706338155087\\
18.1651376146789	0.930005437041267\\
18.5654712260217	0.92703332300983\\
18.9658048373645	0.910516077523454\\
19.3160967472894	0.899970040713633\\
20.0667222685571	0.888647746570513\\
20.767306088407	0.859981147672833\\
21.3678065054212	0.850962891768219\\
22.0683903252711	0.822826346927094\\
22.8690575479566	0.820198112610143\\
23.3694745621351	0.807316580589152\\
23.7698081734779	0.800140319749261\\
23.9199332777314	0.786571420811796\\
24.070058381985	0.760375209479683\\
24.1701417848207	0.734339176378271\\
24.3202668890742	0.682848223317222\\
24.6205170975813	0.566991755985924\\
24.720600500417	0.542866833447157\\
24.8206839032527	0.534846573379802\\
24.9207673060884	0.543205245653631\\
25.0208507089241	0.564347569822658\\
25.3211009174312	0.641859578495442\\
25.4211843202669	0.655066970693184\\
25.5212677231026	0.657802971068996\\
25.6213511259383	0.650933511709013\\
25.8215179316097	0.621093053942623\\
25.9716430358632	0.602607421419677\\
26.0717264386989	0.599549733715534\\
26.1718098415346	0.605462004589505\\
26.371976647206	0.632449805863331\\
26.5221017514596	0.64918668197425\\
26.6221851542952	0.651968367863056\\
26.7222685571309	0.646422153924398\\
26.8723936613845	0.625292492666638\\
27.0725604670559	0.591341366787013\\
27.1726438698916	0.582288113344191\\
27.2727272727273	0.583045947797515\\
27.372810675563	0.593666529427438\\
27.5729774812344	0.630946583484494\\
27.7231025854879	0.653862886461845\\
27.8231859883236	0.65938711679965\\
27.9232693911593	0.655247021448382\\
28.023352793995	0.642172860434883\\
28.3736447039199	0.579138204963733\\
28.4737281067556	0.575511320520626\\
28.5738115095913	0.582575168459954\\
28.7239366138449	0.607493448692324\\
28.9241034195163	0.642318525955183\\
29.024186822352	0.650739541679833\\
29.1242702251877	0.650331181972405\\
29.2243536280234	0.641527548741095\\
29.6246872393661	0.589717160709554\\
29.7247706422018	0.59286093306104\\
29.8248540450375	0.604806811222055\\
30.1751459549625	0.657817436501247\\
30.2752293577982	0.659648403800915\\
30.3753127606339	0.652534075398378\\
30.5254378648874	0.628416667853273\\
30.7256046705588	0.591598050316939\\
30.8256880733945	0.581928918141116\\
30.9257714762302	0.582316235847557\\
31.0258548790659	0.592389875837299\\
31.3761467889908	0.646323412929597\\
31.4762301918265	0.649417393534222\\
31.5763135946622	0.643281794757421\\
31.7264386989158	0.619999761377954\\
31.9266055045872	0.583250256254452\\
32.0266889074229	0.573988752014124\\
32.1267723102586	0.575067639881077\\
32.2268557130942	0.586368259330662\\
32.4270225187656	0.625350953949841\\
32.5771476230192	0.649815929953739\\
32.6772310258549	0.656859806860687\\
32.7773144286906	0.654981129889919\\
32.9274395329441	0.638182957200435\\
33.1776480400334	0.603409135763009\\
33.2777314428691	0.599668861565831\\
33.3778148457048	0.605174279791257\\
33.5279399499583	0.6267909733157\\
33.7281067556297	0.657301105742683\\
33.8281901584654	0.663630344501179\\
33.9282735613011	0.660685430317571\\
34.0283569641368	0.648061381288699\\
34.1784820683903	0.615348797397303\\
34.3786488740617	0.568264008279719\\
34.4787322768974	0.554617811946031\\
34.5788156797331	0.552034196984032\\
34.6788990825688	0.559731156315635\\
35.0792326939116	0.611835982681995\\
35.2293577981651	0.611527359469093\\
35.4295246038365	0.603936729920676\\
35.5296080066722	0.608415861314548\\
35.6296914095079	0.624887276502228\\
35.7297748123436	0.65520635452053\\
35.8298582151793	0.697514106015717\\
36.1301084236864	0.842313646959006\\
36.2301918265221	0.876262221065637\\
36.3302752293578	0.895270509284302\\
36.4303586321935	0.89737588689291\\
36.5304420350292	0.882878334140834\\
36.6305254378649	0.854503705338026\\
36.930775646372	0.748585229533418\\
37.0308590492077	0.733394645437762\\
37.0809007506255	0.733151059426241\\
37.1809841534612	0.747640677237001\\
37.2810675562969	0.778436874974417\\
37.581317764804	0.890307103181648\\
37.6814011676397	0.910286347754017\\
37.7814845704754	0.914804226998591\\
37.8815679733111	0.903799509395981\\
37.9816513761468	0.879885515740924\\
38.2819015846539	0.792218657717164\\
38.3819849874896	0.782471193920266\\
38.4820683903253	0.790804255217843\\
38.582151793161	0.815726077870181\\
38.7322768974145	0.870896204074718\\
38.8824020016681	0.924543621163558\\
38.9824854045038	0.948063207087067\\
39.0825688073394	0.95704055002313\\
39.1826522101751	0.950377699629662\\
39.2827356130108	0.929666685570304\\
39.4328607172644	0.882132963770736\\
39.5829858215179	0.836244431997827\\
39.6830692243536	0.819980313785848\\
39.7831526271893	0.821229530733106\\
39.883236030025	0.840419708511263\\
39.9833194328607	0.873161409615555\\
40.23352793995	0.96505830576875\\
40.3336113427857	0.987292772913548\\
40.4336947456214	0.994570599325442\\
40.533778148457	0.986083718622112\\
40.6338615512927	0.963755490311662\\
40.7839866555463	0.914904753332408\\
40.884070058382	0.882938940161061\\
40.9841534612177	0.860671875791645\\
41.0842368640534	0.854366176038873\\
41.1843202668891	0.866407904656583\\
41.2844036697248	0.894218030217239\\
41.5846538782319	1.00195754754998\\
41.6847372810676	1.02257692479581\\
41.7848206839033	1.02801020047504\\
41.8849040867389	1.01772885846128\\
41.9849874895746	0.993961479533667\\
42.2852376980817	0.900903807913231\\
42.3853211009174	0.887790989572416\\
42.4854045037531	0.89274618274748\\
42.5854879065888	0.915127800703743\\
42.6855713094245	0.949761494469769\\
42.8857381150959	1.02461623252974\\
42.9858215179316	1.05056972490134\\
43.0859049207673	1.06199967273658\\
43.185988323603	1.05721095254876\\
43.2860717264387	1.03732618085879\\
43.3861551292744	1.0062462744029\\
43.5863219349458	0.937833564331235\\
43.6864053377815	0.917248004351421\\
43.7864887406172	0.914167628896124\\
43.8865721434529	0.930070638705978\\
43.9866555462886	0.961354655535295\\
44.2869057547957	1.07160556538384\\
44.3869891576314	1.08976143471983\\
44.4870725604671	1.09151771076146\\
44.5871559633027	1.07682645489025\\
44.6872393661384	1.04854931558459\\
44.9374478732277	0.960779958823025\\
45.0375312760634	0.940661888073294\\
45.0875729774812	0.93732550309273\\
45.1876563803169	0.945626285893148\\
45.2877397831526	0.971973674995539\\
45.4378648874062	1.03105214044607\\
45.5879899916597	1.08811593577864\\
45.6880733944954	1.11284997253091\\
45.7881567973311	1.1218672051391\\
45.8882402001668	1.11352988345634\\
45.9883236030025	1.08950154013959\\
46.138448707256	1.03479500629223\\
46.2885738115096	0.981571292127448\\
46.3886572143453	0.961352194713953\\
46.4386989157631	0.958325556605203\\
46.488740617181	0.960511038731447\\
46.5888240200167	0.979844915722843\\
46.6889074228524	1.01477938081229\\
46.9391159299416	1.11530626923404\\
47.0391993327773	1.13988009989093\\
47.139282735613	1.14778567949479\\
47.2393661384487	1.13758987449356\\
47.3394495412844	1.11114045816905\\
47.4895746455379	1.05242451022635\\
47.5896580483736	1.01234171147775\\
47.6897414512093	0.982778878010706\\
47.7397831526272	0.974522545957726\\
47.789824854045	0.971518816525105\\
47.8398665554629	0.974044810421468\\
47.9399499582986	0.994859279924711\\
48.0400333611343	1.03191712383727\\
48.2902418682235	1.13715696811826\\
48.3903252710592	1.16243475376776\\
48.4904086738949	1.16992759920133\\
48.5904920767306	1.15844498769604\\
48.6905754795663	1.1300221786669\\
48.8407005838199	1.06777778655631\\
48.9407839866555	1.02584453689957\\
49.0408673894912	0.995246787002685\\
49.0909090909091	0.98684278628231\\
49.1409507923269	0.984002870675283\\
49.1909924937448	0.986995367839803\\
49.2410341951626	0.995674096026463\\
49.3411175979983	1.0275859924994\\
49.4912427022519	1.0957432275328\\
49.5913261050876	1.14024447588695\\
49.6914095079233	1.17381354577948\\
49.791492910759	1.19070103055349\\
49.8415346121768	1.1919128417781\\
49.9416180150125	1.17945141344094\\
50.0417014178482	1.14928237621071\\
50.1918265221017	1.08348252138504\\
50.2919099249375	1.0390129405226\\
50.3919933277731	1.00630028632789\\
50.442035029191	0.99715895411142\\
50.4920767306088	0.993851974854557\\
50.5421184320267	0.996665910332247\\
50.5921601334445	1.00544891169894\\
50.6922435362802	1.03828484373022\\
50.8423686405338	1.10877652914172\\
50.9424520433695	1.15476290730557\\
51.0425354462052	1.18930457079207\\
51.1426188490409	1.20638452474573\\
51.1926605504587	1.20736279020016\\
51.2427022518766	1.20309459368492\\
51.3427856547123	1.17969672721797\\
51.442869057548	1.14009259684421\\
51.6930775646372	1.0238460094982\\
51.743119266055	1.00832986500927\\
51.7931609674729	0.998250193753073\\
51.8432026688907	0.994358801814798\\
51.8932443703086	0.996978183952507\\
51.9432860717264	1.00595879037385\\
52.0433694745621	1.04021498566238\\
52.1934945788157	1.11432892019681\\
52.2935779816514	1.16282003287013\\
52.3936613844871	1.19953974521287\\
52.4437030859049	1.21132924139715\\
52.4937447873228	1.21795708237126\\
52.5437864887406	1.21914390497966\\
52.5938281901585	1.21483270852382\\
52.6939115929942	1.19060824422094\\
52.7939949958299	1.14929982752457\\
53.0442035029191	1.02743107122465\\
53.094245204337	1.01106594504116\\
53.1442869057548	1.00041104783736\\
53.1943286071726	0.99627721357151\\
53.2443703085905	0.999018718142104\\
53.2944120100083	1.00848521984184\\
53.394495412844	1.04465009971864\\
53.5446205170976	1.12294125267606\\
53.6447039199333	1.17424753630249\\
53.744787322769	1.21310411597579\\
53.7948290241868	1.22562148127135\\
53.8448707256047	1.23288720315711\\
53.8949124270225	1.23449344976699\\
53.9449541284404	1.23033891253997\\
53.9949958298582	1.22058066351778\\
54.0950792326939	1.18602821223983\\
54.1951626355296	1.13664740930235\\
54.3452877397832	1.05563501547662\\
54.4453711426188	1.01467911341489\\
54.4954128440367	1.00233928754119\\
54.5454545454545	0.996740695985949\\
54.5954962468724	0.998315980587087\\
54.6455379482902	1.0069658077874\\
54.6955796497081	1.02207190123926\\
54.7956630525438	1.06712187771272\\
55.045871559633	1.19713039214275\\
55.1459549624687	1.22838929788695\\
55.1959966638866	1.23609545392942\\
55.2460383653044	1.23802039095435\\
55.2960800667223	1.23407483752349\\
55.3461217681401	1.22439781158607\\
55.4462051709758	1.18957322496482\\
55.5462885738115	1.13928556389517\\
55.7464553794829	1.03064612011498\\
55.7964970809008	1.01053241364515\\
55.8465387823186	0.996117619877879\\
55.8965804837365	0.988408615285756\\
55.9466221851543	0.987956036797492\\
55.9966638865721	0.99477872966105\\
56.04670558799	1.00836092882928\\
56.1467889908257	1.05154513618287\\
56.3969974979149	1.18424611526694\\
56.4970809007506	1.21879340700288\\
56.5471226021685	1.22839747771444\\
56.5971643035863	1.23231347657087\\
56.6472060050042	1.23042700788828\\
56.697247706422	1.222810382798\\
56.7973311092577	1.19182282215998\\
56.8974145120934	1.14459786587398\\
57.0975813177648	1.0391969428071\\
57.1476230191827	1.0192829597283\\
57.1976647206005	1.00490380266396\\
57.2477064220183	0.997149504685474\\
57.2977481234362	0.9966724608622\\
57.347789824854	1.00360064199201\\
57.3978315262719	1.017521238312\\
57.4979149291076	1.0623955727697\\
57.7981651376147	1.2293229758234\\
57.8982485404504	1.26126737294511\\
57.9482902418682	1.26917578420585\\
57.9983319432861	1.27135461920032\\
58.0483736447039	1.26782383344319\\
58.1484570475396	1.24481496477625\\
58.2485404503753	1.20464422404039\\
58.4987489574646	1.08481074509075\\
58.5988323603003	1.05574384473464\\
58.6488740617181	1.0492036783486\\
58.6989157631359	1.04830716380027\\
58.7989991659716	1.06156315263213\\
59.0492076730609	1.12332758234344\\
59.1492910758966	1.15758047732032\\
59.1993327773144	1.16743995278256\\
59.2493744787323	1.16842114584951\\
59.2994161801501	1.15438395888484\\
59.349457881568	1.11685612811582\\
59.3994995829858	1.04642602516897\\
59.4495412844037	0.935764327592722\\
59.4995829858215	0.784052369151141\\
59.6497080900751	0.234147805618079\\
59.6997497914929	0.106462982264553\\
59.7497914929108	0.0400598456273897\\
59.7998331943286	0.0190537335113774\\
59.8498748957465	0.00562753908796054\\
59.9499582985822	0.000466721078232979\\
60	0.000136041155663236\\
};
\addlegendentry{$\omega_1$}

\addplot [color=mode3]
  table[row sep=crcr]{%
0	1.1342739512088e-06\\
0.400333611342788	0.00331529398349772\\
0.750625521267722	0.0119286996336854\\
2.15179316096747	0.0124720359336976\\
16.7639699749791	0.0106343206766937\\
18.0150125104254	0.0118624137407082\\
20.0667222685571	0.0148728152928328\\
21.3678065054212	0.00913487824044523\\
22.418682235196	0.0132072449006486\\
23.7197664720601	0.0192335637375152\\
24.8206839032527	0.00996405228773511\\
25.5713094245204	0.00653724664655897\\
27.3227689741451	0.0145305485820302\\
29.024186822352	0.00956528375277088\\
30.0750625521268	0.0157986879558578\\
30.975813177648	0.010293003157102\\
31.8265221017515	0.00630620636486867\\
33.4778982485404	0.0149414870008329\\
35.3294412010008	0.0105670410857144\\
36.3803169307756	0.018093704450088\\
37.0308590492077	0.0097138437510651\\
37.6313594662219	0.0040609201652444\\
39.232693911593	0.00781488563960409\\
40.4837364470392	0.0179253662867254\\
41.1843202668891	0.00773325777245759\\
41.7848206839033	0.00320915717553305\\
43.185988323603	0.00683895781037336\\
44.6872393661384	0.0173255555282097\\
46.0884070058382	0.00511714283818066\\
47.8899082568807	0.0157892788522602\\
48.4904086738949	0.0195483056062642\\
49.0909090909091	0.0121478860925208\\
49.791492910759	0.00379961919667693\\
50.6922435362802	0.00479950612977831\\
51.1426188490409	0.00548810741186401\\
51.9933277731443	0.0158195023658152\\
52.5437864887406	0.0192059770568349\\
53.1442869057548	0.0118793888526341\\
53.744787322769	0.00543930174065821\\
54.8457047539616	0.00675696785525304\\
56.6472060050042	0.00491730502945131\\
60	2.03217770433639e-06\\
};
\addlegendentry{$\omega_1-\Omega$}

\end{axis}


\end{tikzpicture}%
	\caption[Modal field profiles at each sideband $n$ for the right-to-left direction]{Non-reciprocal propagation in the RL direction. \textbf{(a)} $E_z$ field profiles of the total field, and its separate components in the first sidebands. \textbf{(b)} Corresponding modal amplitudes of each sideband propagating through the modulated region. Mode $\bra{1}$ remains in state $\bra{1}$, implying a broken PT-symmetry.}
	\label{fig:RLFang}
\end{figure}

\begin{figure}[t!]
	\centering
	\setlength{\figH}{\textwidth}
	\setlength{\figW}{1\textwidth}
	% This file was created by matlab2tikz.
%
%The latest updates can be retrieved from
%  http://www.mathworks.com/matlabcentral/fileexchange/22022-matlab2tikz-matlab2tikz
%where you can also make suggestions and rate matlab2tikz.
%
\pgfplotsset{fangFields/.style={
	width=0.8\figW,
height=0.08\figH,
	scale only axis,
	axis on top,
	title style = {at={(0.4\figH,0.06\figH)}},
	xmin=0,
	xmax=60,
	ymin=0,
	ymax=5,
	clip=false,
	axis background/.style={fill=white},
	colormap={mymap}{[1pt] rgb(0pt)=(0,0,1); rgb(31pt)=(1,1,1); rgb(32pt)=(1,1,1); rgb(63pt)=(1,0,0)},
	colorbar,
	colorbar style={width=.01\linewidth, at={(1.02,0.04\figH)}, anchor=east}	
	}
}

\begin{tikzpicture}
\begin{axis}[%
fangFields,
at={(0\figW,0.95\figH)},
title = {Total field},
point meta min=-21.9924910740817,
point meta max=21.9924910740817,
xticklabels={,,},
colorbar style={title={$V / \mu m$}, width=.01\linewidth, at={(1.02,0.04\figH)}, anchor=east},
ylabel={$\text{y (}\mu\text{m)}$}
]
\addplot [forget plot] graphics [xmin=0, xmax=60, ymin=0, ymax=5] {graphs/fangcavity/TR/totalfield-1.png};
\draw (0,1.95) -- (24.2,1.95) -- (24.2,1.5) -- (35.8,1.5) -- (35.8,1.95) -- (60,1.95);
\draw (0,3.05) -- (24.2,3.05) -- (24.2,3.5) -- (35.8,3.5) -- (35.8,3.05) -- (60,3.05);
\node at (55,6) {$ \longleftarrow \bra{2}$};
\node at (5,6) {$\bra{2} \longleftarrow $};
\node at (-2,7) {\textbf{(a)}};
\end{axis}


\begin{axis}[%
fangFields,
at={(0\figW,0.8\figH)},
point meta min=-8.6982232136118,
point meta max=8.6982232136118,
title = {$\omega_1$},
xticklabels={,,},
ylabel={$\text{y (}\mu\text{m)}$}
]
\addplot [forget plot] graphics [xmin=0, xmax=60, ymin=0, ymax=5] {graphs/fangcavity/TR/fields-1.png};
\draw (0,1.95) -- (24.2,1.95) -- (24.2,1.5) -- (35.8,1.5) -- (35.8,1.95) -- (60,1.95);
\draw (0,3.05) -- (24.2,3.05) -- (24.2,3.5) -- (35.8,3.5) -- (35.8,3.05) -- (60,3.05);
\end{axis}

\begin{axis}[%
fangFields,
at={(0\figW,0.65\figH)},
title = {$\omega_2 = \omega_1 + \Omega$},
point meta min=-21.0798477055797,
point meta max=21.0798477055797,
xticklabels={,,},
ylabel={$\text{y (}\mu\text{m)}$}
]
\addplot [forget plot] graphics [xmin=0, xmax=60, ymin=0, ymax=5] {graphs/fangcavity/TR/fields-2.png};
\draw (0,1.95) -- (24.2,1.95) -- (24.2,1.5) -- (35.8,1.5) -- (35.8,1.95) -- (60,1.95);
\draw (0,3.05) -- (24.2,3.05) -- (24.2,3.5) -- (35.8,3.5) -- (35.8,3.05) -- (60,3.05);
\end{axis}

\begin{axis}[%
fangFields,
at={(0\figW,0.5\figH)},
title = {$\omega_1 - 2\Omega$},
point meta min=-2.78928844263317,
point meta max=2.78928844263317,
xlabel={$\text{x (}\mu\text{m)}$},
ylabel={$\text{y (}\mu\text{m)}$}
]
\addplot [forget plot] graphics [xmin=0, xmax=60, ymin=0, ymax=5] {graphs/fangcavity/TR/fields-3.png};
\draw (0,1.95) -- (24.2,1.95) -- (24.2,1.5) -- (35.8,1.5) -- (35.8,1.95) -- (60,1.95);
\draw (0,3.05) -- (24.2,3.05) -- (24.2,3.5) -- (35.8,3.5) -- (35.8,3.05) -- (60,3.05);
\node at (-2,-3) {\textbf{(b)}};
\end{axis}

% This file was created by matlab2tikz.
%
%The latest updates can be retrieved from
%  http://www.mathworks.com/matlabcentral/fileexchange/22022-matlab2tikz-matlab2tikz
%where you can also make suggestions and rate matlab2tikz.
%
\definecolor{mycolor1}{rgb}{0.00000,0.44700,0.74100}%
\definecolor{mycolor2}{rgb}{0.85000,0.32500,0.09800}%
\definecolor{mycolor3}{rgb}{0.92900,0.69400,0.12500}%
%
\begin{axis}[%
width=0.8\figW,
height=0.4\figH,
at={(0\figW,0\figH)},
scale only axis,
xmin=0,
xmax=60,
ymin=0,
ymax=2,
xlabel = {x ($\mu m$)},
ylabel = {Modal amplitude ($W / \mu m$)},
axis background/.style={fill=white},
legend style={legend cell align=left, align=left, draw=white!15!black}
]
\addplot [color=mode3]
  table[row sep=crcr]{%
0	5.27909466043752e-05\\
0.300250208507087	0.00835897728741486\\
0.450375312760634	0.0313030518030004\\
0.650542118432028	0.0772889461309063\\
0.800667222685568	0.0883356120999963\\
1.4512093411176	0.0970980648934727\\
1.85154295246038	0.0830435699971517\\
2.10175145954963	0.0759551742555473\\
2.50208507089241	0.0958260700853373\\
2.85237698081735	0.107951153476016\\
3.20266889074229	0.100016834228377\\
3.55296080066722	0.0863296263289541\\
4.1534612176814	0.102137635156389\\
4.50375312760634	0.104082242589946\\
4.85404503753128	0.0954998429450882\\
5.604670558799	0.137345944946823\\
6.20517097581318	0.0887422989911641\\
6.65554628857381	0.125299028105893\\
6.80567139282736	0.129288506268658\\
6.9557964970809	0.126657587410449\\
7.20600500417014	0.141996546294152\\
7.45621351125939	0.141935961226501\\
7.60633861551293	0.157007480061942\\
7.75646371976647	0.151500025142099\\
8.05671392827356	0.123823434260594\\
8.15679733110926	0.129304191598052\\
8.45704753961635	0.160116975409153\\
8.60717264386989	0.15043927600243\\
8.70725604670559	0.141342856487604\\
8.85738115095914	0.1381997931203\\
9.25771476230192	0.108222664192432\\
9.60800667222686	0.128145653266593\\
9.80817347789825	0.11232859652057\\
9.95829858215179	0.105421615672327\\
10.208507089241	0.113559812108832\\
10.5087572977481	0.142524818820668\\
10.8090075062552	0.139607445540435\\
11.0091743119266	0.132204612991629\\
11.3094245204337	0.153733537784483\\
11.4095079232694	0.159046270172027\\
11.5596330275229	0.170178652387676\\
11.7097581317765	0.160539617472274\\
11.9599666388657	0.127896109002933\\
12.2602168473728	0.131595806812385\\
12.4103419516264	0.135149475359903\\
12.5604670558799	0.12121760860127\\
12.6105087572978	0.114884917723089\\
12.860717264387	0.125531536239073\\
13.0608840700584	0.114070921596827\\
13.2610508757298	0.101948031873889\\
13.6613844870726	0.112582816218392\\
14.0116763969975	0.099758621591171\\
14.5621351125938	0.132639291877254\\
15.0125104253545	0.113210029793308\\
15.2627189324437	0.132636658087428\\
15.5629691409508	0.118100083038989\\
15.9132610508757	0.0893445269210602\\
16.3135946622185	0.105075087484764\\
16.4637197664721	0.0986648445155041\\
16.6138448707256	0.0913767100056404\\
16.9140950792327	0.105198530599495\\
17.2143452877398	0.0962985429202874\\
17.3644703919933	0.106289418733226\\
17.7147623019183	0.0970747070520801\\
17.8648874061718	0.0945786588746884\\
18.2151793160968	0.132356100499578\\
18.6155129274395	0.121889631365171\\
18.8657214345288	0.10679219238817\\
19.0658882402002	0.124072106047173\\
19.2660550458716	0.135927595004404\\
19.9666388657214	0.118563951628488\\
20.1668056713928	0.128895924357103\\
20.7172643869892	0.124106278247126\\
20.9174311926606	0.136750683274194\\
21.4678899082569	0.135059643061055\\
21.6680567139283	0.117748447308024\\
22.6688907422852	0.127883561608243\\
22.9691409507923	0.107913953781278\\
23.1693077564637	0.119551408113033\\
23.3694745621351	0.111747253501328\\
23.7698081734779	0.0833571445839922\\
24.2702251876564	0.135847737504662\\
24.3703085904921	0.109857577178879\\
24.4703919933278	0.0640181210309905\\
24.5204336947456	0.0668598242927985\\
24.6205170975813	0.115964619504062\\
24.720600500417	0.14462340456803\\
24.8206839032527	0.142882749628832\\
24.9207673060884	0.110558937294826\\
25.0208507089241	0.0622624967911989\\
25.1209341117598	0.0475440114050301\\
25.2710592160133	0.124730282062963\\
25.371142618849	0.145960845794185\\
25.4712260216847	0.134162262518011\\
25.5713094245204	0.100567274409393\\
25.7714762301918	0.0849492636873777\\
25.9216013344454	0.115344615378149\\
26.0216847372811	0.11226509691997\\
26.2218515429525	0.0711427481505851\\
26.3219349457882	0.0754816826868776\\
26.5221017514596	0.126362632831601\\
26.6221851542952	0.124957808665755\\
26.7222685571309	0.104414866852963\\
26.8223519599666	0.0746809818187231\\
27.0725604670559	0.105093927678261\\
27.1226021684737	0.0989825445320847\\
27.2226855713094	0.106568442656226\\
27.3227689741451	0.0927559997656147\\
27.4228523769808	0.0578431713048886\\
27.6230191826522	0.121763762905687\\
27.8231859883236	0.140273472921734\\
27.9232693911593	0.127986648237467\\
28.023352793995	0.0905042551637578\\
28.1234361968307	0.0461971024275414\\
28.2235195996664	0.0740849087587208\\
28.3236030025021	0.111102262923168\\
28.4236864053378	0.128265378150516\\
28.5237698081735	0.118167132960238\\
28.673894912427	0.067942515439313\\
28.7739783152627	0.0897008022864796\\
28.9241034195163	0.141493377366118\\
29.024186822352	0.152233069990686\\
29.1242702251877	0.134699795384165\\
29.3244370308591	0.0684749906562772\\
29.3744787322769	0.07222140984058\\
29.5746455379483	0.132992266865841\\
29.674728940784	0.135292404263566\\
29.7748123436197	0.114866659546841\\
29.8748957464554	0.0828218983375848\\
29.9249374478732	0.0794115676937324\\
30.0250208507089	0.122732403935082\\
30.1251042535446	0.144084840283575\\
30.2251876563803	0.134211462127141\\
30.325271059216	0.0952485294425998\\
30.4253544620517	0.0496705038037391\\
30.4753961634696	0.0422548413242296\\
30.5254378648874	0.0573497083659333\\
30.6255212677231	0.113474411488596\\
30.7256046705588	0.14550715640155\\
30.8256880733945	0.143829372775627\\
30.9257714762302	0.110016757919865\\
31.0258548790659	0.0666548200196502\\
31.0758965804837	0.0644847483057376\\
31.2260216847373	0.102790848065986\\
31.326105087573	0.112318188154163\\
31.4261884904087	0.103003933066525\\
31.6763969974979	0.0472073625473115\\
31.9266055045872	0.1253851978475\\
32.0266889074229	0.133511471670722\\
32.1267723102586	0.118947978467673\\
32.2768974145121	0.0820252419660434\\
32.3769808173478	0.113699907859299\\
32.4770642201835	0.120801363232424\\
32.627189324437	0.101207514701017\\
32.7272727272727	0.0976508945182033\\
32.8273561301084	0.0787351375180236\\
32.8773978315263	0.0853222341188484\\
32.977481234362	0.12597647181677\\
33.0775646371977	0.142251833999445\\
33.1776480400334	0.128634293448506\\
33.3277731442869	0.106079563236364\\
33.4278565471226	0.0762340289251853\\
33.5279399499583	0.101198251484583\\
33.628023352794	0.109645660611868\\
33.7781484570475	0.105499176294295\\
33.8782318598832	0.107177955767895\\
33.9783152627189	0.0873095081291169\\
34.0283569641368	0.0720152719651495\\
34.1284403669725	0.115437910470703\\
34.2285237698082	0.136915287390728\\
34.3286071726439	0.131087498398422\\
34.628857381151	0.0562867439073926\\
34.6788990825688	0.0613078516262604\\
34.8290241868224	0.103200301834306\\
34.929107589658	0.110544984354497\\
35.1292743953294	0.0868013861661154\\
35.2293577981651	0.0697646783044092\\
35.3794829024187	0.106767461485376\\
35.4795663052544	0.137925294098871\\
35.5796497080901	0.14050809114179\\
35.6797331109258	0.111878664387092\\
35.7798165137615	0.0672040392708766\\
35.8298582151793	0.0539669143298127\\
35.929941618015	0.0883427480274079\\
35.9799833194329	0.11081259405745\\
36.0800667222686	0.172341758063801\\
36.1801501251043	0.211487364880512\\
36.2802335279399	0.216841299151696\\
36.3803169307756	0.186645132688739\\
36.5304420350292	0.108808519603684\\
36.6305254378649	0.105596536924701\\
36.7306088407006	0.120209821517555\\
36.8807339449541	0.176252835408377\\
36.9808173477898	0.18700831050699\\
37.1809841534612	0.173814581517512\\
37.2810675562969	0.142667692041613\\
37.4812343619683	0.0550092479801094\\
37.7814845704754	0.138997941036742\\
37.8815679733111	0.148705907394238\\
38.0817347789825	0.128700992696167\\
38.2819015846539	0.0972221445341859\\
38.3319432860717	0.106332021174694\\
38.4320266889074	0.145746459128098\\
38.5321100917431	0.158662462614686\\
38.6321934945788	0.142866712263341\\
38.8323603002502	0.0770828092725466\\
38.8824020016681	0.0986330659630639\\
39.0325271059216	0.183608696929547\\
39.1326105087573	0.214696768417106\\
39.232693911593	0.21479520072711\\
39.3327773144287	0.185401759941207\\
39.4328607172644	0.140801812010572\\
39.6830692243536	0.131388027035541\\
39.8331943286072	0.187438543809904\\
39.9332777314429	0.20458855795016\\
40.1834862385321	0.198129585924569\\
40.2835696413678	0.179374203940682\\
40.533778148457	0.119245975350694\\
40.5838198498749	0.123739620722439\\
40.7339449541284	0.160711718055019\\
40.8340283569641	0.171013561839899\\
40.9341117597998	0.167125983629099\\
41.0842368640534	0.13707941267505\\
41.2343619683069	0.098472020162653\\
41.2844036697248	0.0879306279268874\\
41.4345287739783	0.126437684870986\\
41.534612176814	0.12387450921878\\
41.6346955796497	0.0970466270295276\\
41.7347789824854	0.123813874038831\\
41.9349457881568	0.181916977380716\\
42.0350291909925	0.190095134561375\\
42.3853211009174	0.163727041451693\\
42.535446205171	0.129068218932709\\
42.7356130108424	0.131241472750837\\
42.9858215179316	0.233683050334577\\
43.0859049207673	0.237146932462835\\
43.185988323603	0.210177285026198\\
43.3861551292744	0.127967292521127\\
43.5362802335279	0.153724010351745\\
43.5863219349458	0.157285371432742\\
43.7364470391993	0.199715506912753\\
43.836530442035	0.19918262127797\\
43.9366138448707	0.174081468926566\\
44.0366972477064	0.137881767956031\\
44.18682235196	0.127787723809071\\
44.3869891576314	0.127711400884913\\
44.5871559633027	0.106067499581087\\
44.7873227689741	0.123950913989781\\
44.9374478732277	0.140802803126839\\
45.0875729774812	0.134915919882225\\
45.1876563803169	0.128190859956526\\
45.2877397831526	0.131087944151396\\
45.3878231859883	0.110203127584562\\
45.487906588824	0.0682735282435374\\
45.5379482902419	0.0597050810411304\\
45.6380316930776	0.0814068786535884\\
45.6880733944954	0.101728374310142\\
45.838198498749	0.186442880018227\\
45.9382819015847	0.21467310587024\\
46.0383653044204	0.210101069420588\\
46.138448707256	0.174422744376258\\
46.1884904086739	0.150283248957351\\
46.3386155129274	0.12856902162482\\
46.4386989157631	0.131191386588455\\
46.6388657214345	0.187126828899565\\
46.8890742285238	0.203132256293294\\
46.9891576313595	0.183495021425607\\
47.139282735613	0.121412848797\\
47.2393661384487	0.0893874354833883\\
47.2894078398666	0.0882185678125893\\
47.3894912427022	0.11598200526813\\
47.5396163469558	0.168007286223094\\
47.6396997497915	0.178678671803333\\
47.7397831526272	0.167552916488653\\
47.9899916597164	0.116370999381857\\
48.0900750625521	0.133610787264715\\
48.1901584653878	0.159175561184597\\
48.2902418682235	0.163299018356746\\
48.3903252710592	0.143235245508166\\
48.5404503753128	0.0971391476291927\\
48.5904920767306	0.105928923125013\\
48.7406171809842	0.176944222048974\\
48.8407005838199	0.204221425173856\\
48.9407839866555	0.206491650225331\\
49.0408673894912	0.182752765719272\\
49.1409507923269	0.149349558794519\\
49.2910758965805	0.143567135635124\\
49.3411175979983	0.137621313873666\\
49.3911592994162	0.147657604192162\\
49.5913261050876	0.216718889418658\\
49.6914095079233	0.228182254338783\\
49.791492910759	0.227262120830545\\
49.8915763135947	0.232078282378829\\
49.9916597164304	0.212834087683007\\
50.1918265221017	0.157837849715342\\
50.2418682235196	0.151765326936143\\
50.5921601334445	0.216901293924693\\
50.6922435362802	0.209685779817789\\
50.7923269391159	0.18773202206782\\
50.9924937447873	0.115837140505057\\
51.0425354462052	0.0965666981605935\\
51.1426188490409	0.12791168228518\\
51.2427022518766	0.137934261243267\\
51.3427856547123	0.130220803993168\\
51.5929941618015	0.196012263964533\\
51.6930775646372	0.204807835156544\\
51.7931609674729	0.196506639473128\\
51.8932443703086	0.179209255727365\\
51.9933277731443	0.183071790427974\\
52.4437030859049	0.151993296718977\\
52.4937447873228	0.165631885902691\\
52.6438698915763	0.243221779436894\\
52.743953294412	0.270942993590694\\
52.8440366972477	0.266003662969226\\
52.9441201000834	0.231625314217503\\
53.094245204337	0.159504136254746\\
53.2944120100083	0.172530950480521\\
53.3444537114262	0.182069373048883\\
53.4445371142619	0.217606978360429\\
53.5446205170976	0.228519415512274\\
53.6447039199333	0.210739026779741\\
53.744787322769	0.179316837074119\\
54.0950792326939	0.118027689567867\\
54.1451209341118	0.113065328645504\\
54.2452043369475	0.145177668465777\\
54.3452877397832	0.171181117364434\\
54.4453711426188	0.175460310469084\\
54.5454545454545	0.15511904845372\\
54.7456213511259	0.0938997643831669\\
54.8457047539616	0.0952322822705653\\
55.2960800667223	0.114024436641152\\
55.7964970809008	0.116600092573293\\
56.2969140950792	0.101907940496851\\
56.6472060050042	0.0948274705769947\\
57.3978315262719	0.117242397760428\\
57.9983319432861	0.114293755542278\\
59.0992493744787	0.101713475335522\\
59.349457881568	0.10047640591381\\
59.4495412844037	0.084704482069391\\
59.7497914929108	0.00474155785195052\\
60	3.23208772670114e-05\\
};
\addlegendentry{$\omega_1-2\Omega$}

\addplot [color=mode1]
  table[row sep=crcr]{%
0	0.000101601231406789\\
0.150125104253547	0.00787354734853096\\
0.200166805671394	0.0268273283822253\\
0.250208507089241	0.0731521347519219\\
0.300250208507087	0.163487149490692\\
0.350291909924934	0.307284376781006\\
0.450375312760634	0.71583349747322\\
0.550458715596328	1.11938263001561\\
0.600500417014182	1.26733042633668\\
0.650542118432028	1.37193852956555\\
0.700583819849875	1.43835583331837\\
0.750625521267722	1.47710864200392\\
0.800667222685568	1.49961675322472\\
0.900750625521269	1.5155779349694\\
1.40116763969975	1.53516405427587\\
1.90158465387823	1.53754906555292\\
2.45204336947456	1.5196447311042\\
2.75229357798165	1.51174693512906\\
3.45287739783153	1.53814703400901\\
3.95329441201001	1.53682407628403\\
4.55379482902418	1.51354192532855\\
4.70391993327773	1.51294496807774\\
5.35446205170976	1.53561512117169\\
5.90492076730609	1.53535642048921\\
6.30525437864888	1.51818283726512\\
6.50542118432027	1.50950598606842\\
7.20600500417014	1.54017670836721\\
7.75646371976647	1.53727126071986\\
8.3069224353628	1.51112229233708\\
8.75729774812343	1.49537474948332\\
9.40783986655546	1.50935252804485\\
10.1584653878232	1.48774984701743\\
10.6088407005838	1.4753290767988\\
11.0592160133445	1.48959909301246\\
11.85988323603	1.48329836700599\\
12.7105921601334	1.43260860902925\\
13.4111759799833	1.44238714691377\\
14.512093411176	1.39601888901665\\
14.9624687239366	1.40600918483197\\
15.5629691409508	1.40103968114351\\
15.81317764804	1.39762201991555\\
16.0633861551293	1.37210780831816\\
16.4136780650542	1.33104443336997\\
16.5638031693078	1.32809856966364\\
17.0642201834862	1.3427593280142\\
17.3144286905755	1.34427475101533\\
17.5646371976647	1.3263333865147\\
17.9649708090075	1.29354759586902\\
18.4153461217681	1.28586049451529\\
18.8657214345288	1.29859323153539\\
19.9666388657214	1.25817658903636\\
20.4670558798999	1.19623701870021\\
20.9174311926606	1.19615437227458\\
21.3177648040033	1.19955865277045\\
21.5679733110926	1.17704559227964\\
22.0183486238532	1.1260047670798\\
22.2685571309425	1.13097150883416\\
22.8690575479566	1.15455506218938\\
23.2193494578816	1.13413069890169\\
23.5195996663887	1.12479875154609\\
23.7698081734779	1.11681608994798\\
23.9199332777314	1.09746220014882\\
24.070058381985	1.05669735664545\\
24.2201834862385	0.991859500148315\\
24.3202668890742	0.927876820997135\\
24.4203502919099	0.819151056092601\\
24.5204336947456	0.689202563227923\\
24.5704753961635	0.639753668965284\\
24.6205170975813	0.614875344837898\\
24.6705587989992	0.614474709000945\\
24.720600500417	0.637483920220056\\
24.7706422018349	0.678925051448005\\
24.8707256046706	0.793140390941268\\
24.9708090075063	0.905411920346438\\
25.0208507089241	0.942472309221202\\
25.070892410342	0.962512426370857\\
25.1209341117598	0.964502036195199\\
25.1709758131776	0.948855569794169\\
25.2210175145955	0.915708950784115\\
25.3211009174312	0.809083117097472\\
25.4211843202669	0.68608917847213\\
25.4712260216847	0.639442357936623\\
25.5212677231026	0.61549492323779\\
25.5713094245204	0.623076276060047\\
25.6213511259383	0.656714501340957\\
25.721434528774	0.766481010439314\\
25.8215179316097	0.88631151772644\\
25.8715596330275	0.930962324025643\\
25.9216013344454	0.959886013461926\\
25.9716430358632	0.970882100082981\\
26.0216847372811	0.963148200424392\\
26.0717264386989	0.937234425239801\\
26.1217681401168	0.895075632183719\\
26.2218515429525	0.777462564122942\\
26.3219349457882	0.659394218014235\\
26.371976647206	0.622990283030475\\
26.4220183486239	0.613173767736576\\
26.4720600500417	0.632372774803493\\
26.5221017514596	0.675828599964795\\
26.7222685571309	0.909198700660674\\
26.7723102585488	0.945971173243755\\
26.8223519599666	0.96582575414476\\
26.8723936613845	0.96736648360546\\
26.9224353628023	0.950567006403041\\
26.9724770642202	0.916761790785955\\
27.0725604670559	0.810753978643078\\
27.1726438698916	0.691379830096857\\
27.2226855713094	0.647288492757767\\
27.2727272727273	0.625590796184127\\
27.3227689741451	0.631283468099866\\
27.372810675563	0.662805122597263\\
27.4728940783987	0.772180969922957\\
27.5729774812344	0.884509089673259\\
27.6230191826522	0.925298374922676\\
27.6730608840701	0.950607585638103\\
27.7231025854879	0.958452334095369\\
27.7731442869058	0.948246145227657\\
27.8231859883236	0.920763141244009\\
27.9232693911593	0.82416843112496\\
28.0733944954128	0.656503196545195\\
28.1234361968307	0.627073867844246\\
28.1734778982485	0.623659354771917\\
28.2235195996664	0.647090753943814\\
28.2735613010842	0.691879286374743\\
28.4737281067556	0.911387763966395\\
28.5237698081735	0.942881156814039\\
28.5738115095913	0.957517580386032\\
28.6238532110092	0.954155805279008\\
28.673894912427	0.932991121751357\\
28.7239366138449	0.900162379938756\\
28.8240200166806	0.799368686014475\\
28.9741451209341	0.629213269020433\\
29.024186822352	0.611349103569069\\
29.0742285237698	0.62308961692451\\
29.1242702251877	0.65994619314948\\
29.2243536280234	0.775299795567285\\
29.3244370308591	0.888301363166626\\
29.3744787322769	0.928305684424444\\
29.4245204336947	0.952447804520212\\
29.4745621351126	0.958920193253093\\
29.5246038365304	0.947243762665615\\
29.6246872393661	0.882885472020227\\
29.7247706422018	0.774735454238566\\
29.8248540450375	0.660549898764323\\
29.8748957464554	0.62240289442741\\
29.9249374478732	0.608311570458909\\
29.9749791492911	0.628425369822644\\
30.0250208507089	0.672981239933563\\
30.2251876563803	0.898435351063561\\
30.2752293577982	0.932530926607186\\
30.325271059216	0.95004891212826\\
30.3753127606339	0.949725249398774\\
30.4253544620517	0.935045777851521\\
30.4753961634696	0.904265472865397\\
30.5754795663053	0.803042652545919\\
30.675562969141	0.684710022897036\\
30.7256046705588	0.639316769962001\\
30.7756463719766	0.615492267541129\\
30.8256880733945	0.619071766128513\\
30.8757297748123	0.649359380108756\\
30.975813177648	0.760151512840707\\
31.0758965804837	0.877942785778146\\
31.1259382819016	0.922261232539995\\
31.1759799833194	0.95125122833106\\
31.2260216847373	0.962648092247647\\
31.2760633861551	0.955582734392429\\
31.326105087573	0.930542208202006\\
31.3761467889908	0.88940758198747\\
31.4762301918265	0.774162268369878\\
31.5763135946622	0.658869290718805\\
31.6263552960801	0.624010876894033\\
31.6763969974979	0.61575775710152\\
31.7264386989158	0.636412784922982\\
31.7764804003336	0.681145352774337\\
31.976647206005	0.918238837380464\\
32.0266889074229	0.955582315858194\\
32.0767306088407	0.975756418987309\\
32.1267723102586	0.97727874185518\\
32.1768140116764	0.960028880569595\\
32.2268557130942	0.9252457573767\\
32.3269391159299	0.815376381559126\\
32.4270225187656	0.689658687611278\\
32.4770642201835	0.642157463896055\\
32.5271059216013	0.617890987438585\\
32.5771476230192	0.622728974822948\\
32.627189324437	0.65544188921367\\
32.7272727272727	0.771660731149041\\
32.8273561301084	0.893060470435593\\
32.8773978315263	0.938285098733608\\
32.9274395329441	0.967636603130664\\
32.977481234362	0.978918423114266\\
33.0275229357798	0.971329390132738\\
33.0775646371977	0.945411994528179\\
33.1276063386155	0.903077981919282\\
33.2276897414512	0.784383421076832\\
33.3277731442869	0.663625166258718\\
33.3778148457048	0.625106812010948\\
33.4278565471226	0.612888795960352\\
33.4778982485404	0.629846126145232\\
33.5279399499583	0.671620158508659\\
33.7281067556297	0.904635455622966\\
33.7781484570475	0.942450316956496\\
33.8281901584654	0.963614436062151\\
33.8782318598832	0.96663503037734\\
33.9282735613011	0.951366006415647\\
33.9783152627189	0.918985203852422\\
34.0783986655546	0.814601717728678\\
34.1784820683903	0.693020860290197\\
34.2285237698082	0.645549337239913\\
34.278565471226	0.6190431545199\\
34.3286071726439	0.619340723909268\\
34.3786488740617	0.646036660341792\\
34.4787322768974	0.750088440001001\\
34.5788156797331	0.862891353274044\\
34.6788990825688	0.937979364268919\\
34.7289407839867	0.952556449283797\\
34.7789824854045	0.948762577181569\\
34.8290241868224	0.926769495482752\\
34.8790658882402	0.88808124602123\\
34.9791492910759	0.774042987148206\\
35.0792326939116	0.659268194349885\\
35.1292743953294	0.624781634148704\\
35.1793160967473	0.613398282936359\\
35.2293577981651	0.627606490479117\\
35.279399499583	0.663827780072005\\
35.3794829024187	0.772671792706525\\
35.4795663052544	0.889266497010517\\
35.5296080066722	0.930903062942647\\
35.5796497080901	0.955303468572183\\
35.6296914095079	0.960284569853158\\
35.6797331109258	0.948545857266453\\
35.8798999165972	0.833872721890621\\
35.929941618015	0.824753425982969\\
35.9799833194329	0.830595452637496\\
36.0300250208507	0.849478786072375\\
36.0800667222686	0.881278476398172\\
36.1801501251043	0.978841718733797\\
36.4303586321935	1.27623730133914\\
36.5304420350292	1.35620203720132\\
36.580483736447	1.3815864317265\\
36.6305254378649	1.39621999712057\\
36.6805671392827	1.39975061521549\\
36.7306088407006	1.39215216203493\\
36.7806505421184	1.37372725198842\\
36.8807339449541	1.30730460993896\\
36.9808173477898	1.20992114638175\\
37.1809841534612	0.992500329679011\\
37.2310258548791	0.951667613787784\\
37.2810675562969	0.923281176160813\\
37.3311092577148	0.910085488217092\\
37.3811509591326	0.913496137689556\\
37.4311926605505	0.933248227650424\\
37.4812343619683	0.967482415955097\\
37.581317764804	1.06695194390684\\
37.8315262718932	1.34425134860516\\
37.9316096747289	1.41613294419317\\
37.9816513761468	1.43792799669948\\
38.0316930775646	1.44935381665924\\
38.0817347789825	1.45007543626424\\
38.1317764804003	1.44005147627285\\
38.1818181818182	1.41954196883138\\
38.2819015846539	1.34971488304969\\
38.3819849874896	1.24949763508411\\
38.582151793161	1.02777851029903\\
38.6321934945788	0.993055372576869\\
38.6822351959967	0.971485719673396\\
38.7322768974145	0.964794627785494\\
38.7823185988324	0.973771745872945\\
38.8323603002502	0.997696136352502\\
38.9324437030859	1.081331555155\\
39.2827356130108	1.44337063431196\\
39.3327773144287	1.47180797534676\\
39.3828190158465	1.48931286584558\\
39.4328607172644	1.4953888266615\\
39.5329441201001	1.47587199816475\\
39.6330275229358	1.42192110845571\\
39.7331109257715	1.33580724071598\\
39.9833194328607	1.08016098484922\\
40.0333611342786	1.04505549942235\\
40.0834028356964	1.02278162036544\\
40.1334445371143	1.0155718179561\\
40.1834862385321	1.02433943246545\\
40.23352793995	1.05467545441722\\
40.3336113427857	1.15726680902151\\
40.533778148457	1.3944094910659\\
40.6338615512927	1.48491901665273\\
40.7339449541284	1.5394219334316\\
40.7839866555463	1.55136236824253\\
40.8340283569641	1.55281196746893\\
40.884070058382	1.54394979068285\\
40.9841534612177	1.4974975269695\\
41.0842368640534	1.41937103468182\\
41.3844870725605	1.14156947575599\\
41.4345287739783	1.11570107018368\\
41.4845704753962	1.10249035027483\\
41.534612176814	1.10324490304826\\
41.5846538782319	1.11813504757953\\
41.6346955796497	1.1461539234143\\
41.7347789824854	1.23293009654652\\
41.9849874895746	1.49677828340908\\
42.0850708924103	1.56866399399879\\
42.1351125938282	1.59044746905989\\
42.185154295246	1.60150432007776\\
42.2351959966639	1.60131255979228\\
42.2852376980817	1.58971776911871\\
42.3352793994996	1.56694905055095\\
42.4353628023353	1.49085357101321\\
42.6855713094245	1.23950246408572\\
42.7856547122602	1.17331629510866\\
42.8356964136781	1.15591144793652\\
42.8857381150959	1.15118673036999\\
42.9357798165138	1.15966272509102\\
42.9858215179316	1.18080375794516\\
43.0859049207673	1.25440517304519\\
43.2360300250209	1.40564582967787\\
43.3861551292744	1.55590094085692\\
43.4862385321101	1.61449891357916\\
43.5362802335279	1.62751919588599\\
43.6363636363636	1.62603865062452\\
43.6864053377815	1.6121841233908\\
43.7864887406172	1.5551778597474\\
43.8865721434529	1.46573069277735\\
44.1367806505421	1.21330416534938\\
44.18682235196	1.19204385205792\\
44.2368640533778	1.18400608208562\\
44.2869057547957	1.1898278040621\\
44.3369474562135	1.20900151865067\\
44.4370308590492	1.28049654695193\\
44.5871559633027	1.43189573932644\\
44.7372810675563	1.57537920976677\\
44.837364470392	1.64165112750447\\
44.9374478732277	1.67403847563637\\
44.9874895746455	1.67613430257601\\
45.0375312760634	1.66861574152695\\
45.1376146788991	1.62570393237149\\
45.2376980817348	1.55018756804369\\
45.3878231859883	1.39972088521065\\
45.487906588824	1.30072402175155\\
45.5879899916597	1.23040934987112\\
45.6380316930776	1.21333298313102\\
45.6880733944954	1.21314970071367\\
45.7381150959133	1.23759033474727\\
45.838198498749	1.31708391656248\\
46.138448707256	1.60926885671692\\
46.2385321100917	1.67812672837519\\
46.2885738115096	1.69685477877304\\
46.3386155129274	1.70400987425553\\
46.3886572143453	1.69931708789599\\
46.4386989157631	1.68294473959443\\
46.5387823185988	1.61807069604581\\
46.6388657214345	1.54222197906877\\
46.8890742285238	1.32556851012724\\
46.9891576313595	1.27633120756166\\
47.0391993327773	1.26860789864843\\
47.0892410341952	1.27354482769444\\
47.139282735613	1.29099549421595\\
47.2393661384487	1.35885537572343\\
47.4895746455379	1.61704179957042\\
47.5896580483736	1.68295972523072\\
47.6897414512093	1.71326354223226\\
47.7397831526272	1.71749932188973\\
47.789824854045	1.71102821931976\\
47.8398665554629	1.69398057122839\\
47.9399499582986	1.63070312910689\\
48.2902418682235	1.34340336743607\\
48.3402835696414	1.32072406303487\\
48.3903252710592	1.30871166601923\\
48.4403669724771	1.30836974807942\\
48.4904086738949	1.31987875066989\\
48.5904920767306	1.37498767776518\\
48.6905754795663	1.46068814470526\\
48.8907422852377	1.64815487328941\\
48.9908256880734	1.71550735600004\\
49.0909090909091	1.74899974189274\\
49.1409507923269	1.75094816185484\\
49.1909924937448	1.74257761505469\\
49.2910758965805	1.69567837942208\\
49.3911592994162	1.61384085152059\\
49.6413678065054	1.35949907004647\\
49.6914095079233	1.33475053888396\\
49.7414512093411	1.32031765197787\\
49.791492910759	1.31719282142112\\
49.8415346121768	1.32556272427918\\
49.9416180150125	1.37352168311641\\
50.0417014178482	1.45163564114097\\
50.3419516263553	1.7152299036807\\
50.3919933277731	1.73877414343663\\
50.442035029191	1.75188120990541\\
50.4920767306088	1.75425452380665\\
50.5421184320267	1.74600896807956\\
50.6422018348624	1.70582325429048\\
50.7422852376981	1.64254326935038\\
50.8924103419516	1.5097301648907\\
51.0425354462052	1.38151713688794\\
51.092577147623	1.35195314801874\\
51.1426188490409	1.3330517747797\\
51.1926605504587	1.329971520497\\
51.2427022518766	1.34325534993319\\
51.3427856547123	1.4004365979031\\
51.4929107589658	1.53135047449691\\
51.6430358632193	1.66199575751251\\
51.743119266055	1.72451381044009\\
51.8432026688907	1.75643447465818\\
51.9432860717264	1.75334621656953\\
52.0433694745621	1.71535933025417\\
52.1434528773978	1.64740587126156\\
52.4437030859049	1.39082830209481\\
52.4937447873228	1.36466194097168\\
52.5437864887406	1.34946986047025\\
52.5938281901585	1.34655226017016\\
52.6438698915763	1.35627609033541\\
52.6939115929942	1.37800698936407\\
52.7939949958299	1.45522077295779\\
53.094245204337	1.7195712420718\\
53.1943286071726	1.76650118666937\\
53.2443703085905	1.77570895754097\\
53.2944120100083	1.77478948413184\\
53.3444537114262	1.76374944765543\\
53.4445371142619	1.71325140048072\\
53.5446205170976	1.63199113873119\\
53.744787322769	1.44818716304063\\
53.8448707256047	1.38925901842768\\
53.8949124270225	1.37254454461198\\
53.9449541284404	1.36612369641837\\
53.9949958298582	1.37051550690995\\
54.0450375312761	1.38547042756132\\
54.1451209341118	1.44251978605124\\
54.2452043369475	1.52788932939222\\
54.395329441201	1.66328235266029\\
54.4954128440367	1.72836705853258\\
54.5954962468724	1.76161361025062\\
54.6955796497081	1.75919381494113\\
54.7956630525438	1.72348463217601\\
54.8957464553795	1.66077419234331\\
55.045871559633	1.53277221102032\\
55.1459549624687	1.44855815185198\\
55.2460383653044	1.38799762133164\\
55.2960800667223	1.37351122955922\\
55.3461217681401	1.37112377429032\\
55.396163469558	1.38020644833385\\
55.4962468723937	1.42909120283323\\
55.5963302752294	1.50650108989295\\
55.7964970809008	1.67451208951361\\
55.8965804837365	1.73435822929542\\
55.9966638865721	1.76133761799226\\
56.04670558799	1.76095478222278\\
56.1467889908257	1.7324753685723\\
56.2468723936614	1.67125826432797\\
56.4470391993328	1.50157037725101\\
56.5471226021685	1.42391599433611\\
56.6472060050042	1.37537594141292\\
56.697247706422	1.36668268974267\\
56.7472894078399	1.36957876044467\\
56.7973311092577	1.38478641105536\\
56.8974145120934	1.44673402366185\\
57.2477064220183	1.72550343327348\\
57.347789824854	1.76099450862081\\
57.3978315262719	1.76605364760544\\
57.4979149291076	1.74872886290072\\
57.5979983319433	1.69780422622952\\
57.698081734779	1.62189886321869\\
57.8982485404504	1.46030532235682\\
57.9983319432861	1.41273271720523\\
58.0483736447039	1.40372428755519\\
58.0984153461218	1.40590041302335\\
58.1484570475396	1.41926930934411\\
58.2485404503753	1.47565489268284\\
58.348623853211	1.56150576549112\\
58.5487906588824	1.74104796941542\\
58.6488740617181	1.79970654377056\\
58.7489574645538	1.82332759308847\\
58.7989991659716	1.82034641893443\\
58.8990825688073	1.78622976321945\\
58.999165971643	1.71241307593277\\
59.0492076730609	1.66062836375237\\
59.1492910758966	1.68683773991973\\
59.2493744787323	1.68801040962281\\
59.2994161801501	1.66994517214099\\
59.349457881568	1.62936581347114\\
59.3994995829858	1.55629091763688\\
59.4495412844037	1.44094653563579\\
59.4995829858215	1.27716347081723\\
59.5996663886572	0.82309115722969\\
59.6997497914929	0.345682878218931\\
59.7497914929108	0.174792213855923\\
59.7998331943286	0.0723484411783275\\
59.8498748957465	0.0246004469667156\\
59.9499582985822	0.00266301772485633\\
60	0.000581557466027505\\
};
\addlegendentry{$\omega_2$}

\addplot [color=mode2]
  table[row sep=crcr]{%
0	2.70122228727132e-05\\
0.350291909924934	0.00919264829632027\\
0.450375312760634	0.0306227518932189\\
0.650542118432028	0.0832242419328182\\
0.800667222685568	0.0955578438467128\\
1.40116763969975	0.0976791773005061\\
5.30442035029191	0.111544572583071\\
5.95496246872393	0.135713746408101\\
6.45537948290242	0.126295811206234\\
6.80567139282736	0.129466964767843\\
7.75646371976647	0.16519961369815\\
8.20683903252711	0.169402759942408\\
9.20767306088407	0.206817002729835\\
9.60800667222686	0.214184047477943\\
10.5087572977481	0.253833034822932\\
11.0091743119266	0.260203931863551\\
11.9099249374479	0.302125948074519\\
12.3603002502085	0.307996615925894\\
13.3611342785655	0.349840611983524\\
13.7614678899083	0.356544240702554\\
14.7122602168474	0.397037706283236\\
15.162635529608	0.404540390466074\\
16.0133444537114	0.444022796870627\\
16.5638031693078	0.452492204002269\\
17.3644703919933	0.488594176797115\\
17.9149291075897	0.495860461146393\\
18.8156797331109	0.53386389702348\\
19.2660550458716	0.540079023275503\\
20.2168473728107	0.575729590111891\\
20.6672226855713	0.580763799171343\\
21.6180150125104	0.6120605314225\\
22.0183486238532	0.618963474580383\\
23.2193494578816	0.669709180163871\\
23.9199332777314	0.685363318338212\\
24.1201000834028	0.662448128904011\\
24.2702251876564	0.628551207203337\\
24.4703919933278	0.562808995445636\\
24.6705587989992	0.496073824528416\\
24.7706422018349	0.476551871814159\\
24.8707256046706	0.472762590424921\\
24.9708090075063	0.48510374686856\\
25.1709758131776	0.537831458716262\\
25.371142618849	0.585388888366033\\
25.5212677231026	0.596215282452675\\
25.6713928273561	0.583251315166983\\
26.0717264386989	0.522987888475214\\
26.2218515429525	0.531321481339944\\
26.6722268557131	0.586601787736043\\
26.8223519599666	0.57451014510837\\
27.2727272727273	0.510794570535339\\
27.4228523769808	0.524998787975299\\
27.8231859883236	0.592213729557869\\
27.9733110925771	0.587986870354399\\
28.1234361968307	0.563392636410747\\
28.3736447039199	0.512831575795921\\
28.5237698081735	0.505567486177171\\
28.673894912427	0.523532790142475\\
29.024186822352	0.585889151361044\\
29.1743119266055	0.586510474377732\\
29.3244370308591	0.567275246629833\\
29.6246872393661	0.519041093832513\\
29.7748123436197	0.521038299770275\\
29.9249374478732	0.544134092119862\\
30.1751459549625	0.588003734880445\\
30.325271059216	0.593607377308707\\
30.4753961634696	0.57795952240734\\
30.8757297748123	0.510183293851632\\
31.0258548790659	0.517053997899247\\
31.2260216847373	0.553021038791549\\
31.4261884904087	0.58267149977717\\
31.5763135946622	0.582464179950904\\
31.7264386989158	0.561515967663489\\
32.0266889074229	0.50600160691436\\
32.1768140116764	0.505357679637605\\
32.3269391159299	0.528769215424234\\
32.627189324437	0.586956919447921\\
32.7773144286906	0.592470899780842\\
32.9274395329441	0.57750031255366\\
33.2777314428691	0.527667307546338\\
33.4278565471226	0.534259369666508\\
33.628023352794	0.568704877490887\\
33.8281901584654	0.597198216778779\\
33.9783152627189	0.595960831891475\\
34.1284403669725	0.572126140060405\\
34.5287739783153	0.486346403986808\\
34.6788990825688	0.489739338788283\\
34.8790658882402	0.522248773571256\\
35.0792326939116	0.549280883183677\\
35.2293577981651	0.551324235256594\\
35.5796497080901	0.539164616369654\\
35.6797331109258	0.554875135530075\\
35.7798165137615	0.585030158121626\\
35.929941618015	0.650532071166801\\
36.1301084236864	0.742093703109376\\
36.2301918265221	0.775032156590413\\
36.3302752293578	0.79307900887909\\
36.4303586321935	0.793783971199531\\
36.5304420350292	0.776743415861027\\
36.6305254378649	0.743725763129852\\
36.7806505421184	0.673828044151939\\
36.9808173477898	0.582502845210563\\
37.0809007506255	0.558920651180493\\
37.1809841534612	0.559175946187132\\
37.2810675562969	0.581766994760912\\
37.4812343619683	0.661104511794392\\
37.6313594662219	0.713361167303496\\
37.7314428690575	0.731848113857083\\
37.8315262718932	0.733351842768201\\
37.9316096747289	0.717307986126322\\
38.0316930775646	0.685556328575331\\
38.231859883236	0.594576425578538\\
38.3819849874896	0.534775854471874\\
38.4820683903253	0.516614840138828\\
38.582151793161	0.522097390666481\\
38.6822351959967	0.548637375775392\\
39.0825688073394	0.689518713091118\\
39.1826522101751	0.696457546924627\\
39.2827356130108	0.685601878151992\\
39.3828190158465	0.657867847919952\\
39.5329441201001	0.592476074198942\\
39.7331109257715	0.498196918025712\\
39.8331943286072	0.470492013055576\\
39.9332777314429	0.467377901780289\\
40.0333611342786	0.488960476523744\\
40.1834862385321	0.549720840405293\\
40.3836530442035	0.629549726444282\\
40.4837364470392	0.651590013023061\\
40.5838198498749	0.656294690172018\\
40.6839032527106	0.642617070192088\\
40.7839866555463	0.61179069557771\\
40.9341117597998	0.541939982395036\\
41.1342785654712	0.444559985992015\\
41.2343619683069	0.418222373436457\\
41.3344453711426	0.419053474488422\\
41.4345287739783	0.445160135559675\\
41.6346955796497	0.532479506375488\\
41.7848206839033	0.587828944591685\\
41.8849040867389	0.607263590741454\\
41.9849874895746	0.608906878719203\\
42.0850708924103	0.592182034727955\\
42.185154295246	0.558667003386731\\
42.3352793994996	0.485987458250293\\
42.535446205171	0.389000853902175\\
42.6355296080067	0.366141214682074\\
42.7356130108424	0.371692650300275\\
42.8356964136781	0.402389811015325\\
43.185988323603	0.54728177547328\\
43.2860717264387	0.564273917797486\\
43.3861551292744	0.563054872164798\\
43.4862385321101	0.543334236914717\\
43.5863219349458	0.506862048365093\\
43.7364470391993	0.43011434811585\\
43.8865721434529	0.350722491084206\\
43.9866555462886	0.316727260901438\\
44.0867389491243	0.312335269247356\\
44.18682235196	0.338401020887304\\
44.3369474562135	0.407690110931206\\
44.4870725604671	0.475443705634447\\
44.5871559633027	0.505989215472766\\
44.6872393661384	0.519651169890878\\
44.7873227689741	0.514652211408702\\
44.8874061718098	0.491165629897708\\
44.9874895746455	0.451257531226069\\
45.1876563803169	0.341505389335445\\
45.2877397831526	0.289304503645077\\
45.3878231859883	0.257620922943389\\
45.487906588824	0.260336602145827\\
45.5879899916597	0.293555020718209\\
45.9382819015847	0.448123311725659\\
46.0383653044204	0.466923861615136\\
46.138448707256	0.468101607687935\\
46.2385321100917	0.45092537565224\\
46.3386155129274	0.416673326786182\\
46.488740617181	0.340413219661627\\
46.6889074228524	0.229013784140072\\
46.7889908256881	0.200559255240961\\
46.8390325271059	0.200522873494116\\
46.9391159299416	0.227695229334991\\
47.139282735613	0.327784634251152\\
47.2894078398666	0.390679736473714\\
47.3894912427022	0.415016352716009\\
47.4895746455379	0.421894930517453\\
47.5896580483736	0.410562034811534\\
47.6897414512093	0.381727700024356\\
47.8398665554629	0.311036383614969\\
48.14011676397	0.148716054684144\\
48.1901584653878	0.141112476204754\\
48.2402001668057	0.146366003234206\\
48.3402835696414	0.185369944339072\\
48.5904920767306	0.312792619950258\\
48.6905754795663	0.347728798868857\\
48.790658882402	0.367490876516271\\
48.8907422852377	0.370170621611521\\
48.9908256880734	0.355575839020915\\
49.0909090909091	0.325227898321174\\
49.2410341951626	0.254139744228269\\
49.5412844036697	0.0864862150698755\\
49.5913261050876	0.0873565421835778\\
49.6914095079233	0.127624651883551\\
49.9416180150125	0.253854842722568\\
50.0917431192661	0.3029191552695\\
50.1918265221017	0.317737238283939\\
50.2919099249375	0.316766844401286\\
50.3919933277731	0.300692548669531\\
50.4920767306088	0.269904151162798\\
50.6422018348624	0.200962283747891\\
50.8423686405338	0.082155621239373\\
50.9424520433695	0.0304461145672548\\
51.0425354462052	0.071774695049811\\
51.2927439532944	0.197193741728633\\
51.442869057548	0.246786601164366\\
51.5429524603837	0.264156875230817\\
51.6430358632193	0.267869022348719\\
51.743119266055	0.258416868819118\\
51.8932443703086	0.221333828887623\\
52.0433694745621	0.162981735483854\\
52.2935779816514	0.0461030369780744\\
52.3436196830692	0.0476854397090278\\
52.6939115929942	0.170910630656664\\
52.8440366972477	0.202145016759189\\
52.9941618015012	0.214775342492196\\
53.1442869057548	0.208478685653574\\
53.3444537114262	0.176205095443443\\
53.744787322769	0.0955795713614336\\
53.8949124270225	0.126901589470812\\
54.0950792326939	0.160558189474628\\
54.2452043369475	0.168777817768557\\
54.4954128440367	0.160826161268233\\
54.8957464553795	0.151446520331135\\
56.5971643035863	0.150515643548331\\
59.349457881568	0.13949729924132\\
59.4495412844037	0.114247705357776\\
59.6997497914929	0.0130600195586226\\
59.7998331943286	0.00176906215830286\\
60	3.34742082372941e-05\\
};
\addlegendentry{$\omega_1$};

\end{axis}
\end{tikzpicture}%
	\caption[Broken time-reversal symmetry in the photonic AB waveguide.]{Non-reciprocal propagation under a time-reversed operation. Light of mode $\bra{2}$ is injected from the RL direction, mimicking a time-reversal operation.\textbf{(a)} $E_z$ field profiles of the total field, and its separate components in the first sidebands. \textbf{(b)} Corresponding modal amplitudes of each sideband propagating through the modulated region. Since TR-symmetry is broken, even after propagating through the modulated structure the mode does not transition back to $\ket{1}$, and instead remains in state $\bra{2}$. Note how the field profile at $\omega_1-2\Omega$ exists as a superposition of several modes, since it does not have an allowed solution on the band-structure.}
	\label{fig:TRFang}
\end{figure}   
\clearpage

In Figures \ref{fig:LRFang},\ref{fig:RLFang}, and \ref{fig:TRFang} there is an excellent agreement between the transfer matrix formalism and the simulations. Importantly, there is complete TR and PT-symmetry breaking, implying total non-reciprocal mode conversion based on interference through the photonic AB effect. Of interesting consequence is the generation of an additional sidebands at frequencies that are not defined in the dispersion relation ($\omega_1-\Omega$ and $\omega_1 - 2\Omega$). As such, the field is distributed as a \textit{superposition} of several possible transverse electric modes, which can be safely neglected owing to their low modal amplitude. These spurious modes are `leaky', implying the superposition contains modes close to or above the light cone of the band-structure. However, the existence of the leaky modes gives insight into the amount of energy lost as a result of modulation. Likewise, the effectively vanishing modal amplitudes near $0$ and $60\mskip3mu \mu m$ are due to the absorbing boundary conditions of the PML layer.  In Figure \ref{fig:final}, the result of extracting the frequencies for additional detectors $D_1$ and $D_2$ placed at $1.5$ and $58 \mskip3mu \mu m$ respectively are shown. As is expected, the LR direction completely converts $\ket{1}$ to $\ket{2}$, whilst the RL and TR directions are not affected, as summarised in Table \ref{tab:modess}. In comparison with the indirect modulation of \ref{nonrec}, these modes can be removed using widely available passive modal filters, allowing for broadband optical isolation.



\begin{figure}
	\setlength{\figH}{0.22\textwidth}
	\setlength{\figW}{0.24\textwidth}
	\begin{subfigure}{0.33\textwidth}
		\centering
		% This file was created by matlab2tikz.
%
%The latest updates can be retrieved from
%  http://www.mathworks.com/matlabcentral/fileexchange/22022-matlab2tikz-matlab2tikz
%where you can also make suggestions and rate matlab2tikz.
%
\definecolor{mycolor1}{rgb}{0.00000,0.44700,0.74100}%
\definecolor{mycolor2}{rgb}{0.85000,0.32500,0.09800}%
%
\begin{tikzpicture}

\begin{axis}[%
width=\figW,
height=\figH,
at={(0\figW,0.5\figH)},
scale only axis,
xmin=0,
xmax=0.3,
ymin=0,
ymax=1,
ylabel={Photon flux ($n$)},
xlabel={$\omega$ ($2 \pi c/a$)},
axis background/.style={fill=white}
]
\addplot [color=mode2]
  table[row sep=crcr]{%
0	2.02692564466034e-05\\
0.103292102248673	0.000280380826124693\\
0.107562576760246	0.000190615556102669\\
0.108630195388139	0.00026995448538969\\
0.109697814016032	0.000519930067296892\\
0.111299241957872	0.000215296157155986\\
0.112900669899712	0.000634176485360793\\
0.113968288527605	0.00020327735682879\\
0.115836621126418	0.000942878467160568\\
0.117171144411285	0.000326515510382208\\
0.118772572353125	0.00140039239467527\\
0.120107095637991	0.00035373842841957\\
0.120640904951938	0.00116074002404076\\
0.121441618922858	0.00244261461512596\\
0.122242332893778	0.0013850202900576\\
0.122776142207724	0.000408587791816073\\
0.123309951521671	0.000993757115390403\\
0.124644474806538	0.00579810982243822\\
0.124911379463511	0.00525849068435669\\
0.125978998091404	0.000602444647397116\\
0.126245902748378	0.00183188292919789\\
0.126779712062324	0.0112196277781009\\
0.128648044661137	0.0704843920079685\\
0.128914949318111	0.0701279677450288\\
0.129448758632057	0.0593096830285558\\
0.131317091230871	0.00378945819138909\\
0.131850900544817	0.000631914808392198\\
0.13211780520179	0.000505948784009114\\
0.133185423829684	0.00154458353336451\\
0.13612137505639	0.000304034172873946\\
0.13852351696915	0.000350593787477038\\
0.139591135597043	0.000147372181025895\\
0.141192563538883	0.000559332596227868\\
0.142793991480723	0.000174242982558193\\
0.14412851476559	0.000556450780817563\\
0.145996847364403	0.000234867995221322\\
0.147331370649269	0.00053685754884869\\
0.148932798591109	0.000220053094867056\\
0.150267321875976	0.000575031354170585\\
0.151868749817816	0.000204490570465232\\
0.153203273102682	0.000559917053416736\\
0.155071605701496	0.000241972842460081\\
0.156406128986362	0.000503333759755531\\
0.158007556928202	0.000215640841144671\\
0.159608984870042	0.000468432273565522\\
0.160943508154908	0.000198510235661642\\
0.162278031439775	0.000568700683763135\\
0.163879459381615	0.000186873780516628\\
0.165480887323455	0.000521887940909016\\
0.166815410608321	0.000215236657740814\\
0.168416838550161	0.0005117451543033\\
0.170018266492001	0.000327827659412305\\
0.171352789776868	0.000573903968672962\\
0.172954217718708	0.000366358291039814\\
0.174555645660547	0.000612415457353244\\
0.176157073602387	0.000535231396460611\\
0.179893738800014	0.000796722977213449\\
0.180961357427907	0.000640520416625323\\
0.182562785369747	0.00130448052689025\\
0.183897308654613	0.000778589142653274\\
0.185498736596453	0.0022961112541604\\
0.18683325988132	0.000717084788819378\\
0.187367069195266	0.00226714769104364\\
0.188167783166186	0.00518544185419767\\
0.188701592480133	0.004588007759047\\
0.189769211108026	0.00049136922898918\\
0.190036115765	0.00116294283788032\\
0.190569925078946	0.00558154741010597\\
0.191370639049866	0.0121328790656863\\
0.19163754370684	0.0117059872706191\\
0.192171353020786	0.00638171237143981\\
0.192705162334733	0.000833724042838435\\
0.192972066991706	0.00102874885156701\\
0.193505876305653	0.010927048633822\\
0.194573494933546	0.0392075460899508\\
0.194840399590519	0.0365970710964973\\
0.195908018218413	0.000523095981015853\\
0.196174922875386	0.00960501487696508\\
0.196708732189333	0.0951188568627017\\
0.197509446160252	0.423314770014581\\
0.198843969445119	0.990485168117749\\
0.199110874102092	1\\
0.199377778759066	0.961020912376248\\
0.199911588073012	0.760020992106738\\
0.201513016014852	0.050353278565517\\
0.202046825328799	0.000363268209923717\\
0.202313729985772	0.00430040634764217\\
0.203381348613665	0.0489402815684417\\
0.203648253270639	0.0467393166629728\\
0.205249681212478	0.000153564157706176\\
0.205783490526425	0.00731245491864252\\
0.206584204497345	0.017477998742554\\
0.206851109154318	0.0164576038174333\\
0.208185632439185	0.000176797055538858\\
0.208452537096158	0.000338682134933821\\
0.209787060381025	0.00840029692714084\\
0.210053965037998	0.0076256968243178\\
0.211388488322865	3.05309877179916e-05\\
0.211655392979838	0.000388770979337538\\
0.212989916264705	0.00432350519999902\\
0.213523725578651	0.00271916644939618\\
0.214324439549571	0.00013440060224057\\
0.214858248863518	0.000481469477618512\\
0.215925867491411	0.00258951938434993\\
0.216459676805358	0.00196749403237084\\
0.217527295433251	7.94261203567181e-05\\
0.218328009404171	0.000901339134857571\\
0.219128723375091	0.00155366712176597\\
0.219662532689038	0.00106370195726102\\
0.220463246659957	0.000167831813488117\\
0.221263960630877	0.00042014959947223\\
0.222064674601797	0.000883657179136188\\
0.224466816514557	0.000383177031759274\\
0.22553443514245	0.000457560385735079\\
0.22713586308429	0.000249372568820627\\
0.229004195683103	0.000223745701734224\\
0.266370847659367	9.05383916440794e-05\\
0.266904656973314	0.000557174247925341\\
0.267705370944234	0.0026518422284747\\
0.269039894229101	0.00696769036725153\\
0.269306798886074	0.0071006719864235\\
0.269840608200021	0.00623237800756127\\
0.271708940798834	0.0001630669461965\\
0.27224275011278	7.38143888536769e-05\\
0.273577273397647	0.000644937367476395\\
0.27544560599646	6.39473390915413e-05\\
0.2770470339383	0.00020040586779313\\
0.27864846188014	6.63046866131722e-05\\
0.280516794478953	9.55509025191148e-05\\
0.283185841048686	9.75039627970631e-05\\
0.288257029531179	5.94028880258612e-05\\
0.296264169240379	4.77090192534391e-05\\
0.300000834438005	4.15517847882629e-05\\
};
\addplot [color=mode1]
  table[row sep=crcr]{%
0	2.06393170665287e-06\\
0.101957578963806	0.000160355368395093\\
0.102758292934726	0.000394640117033385\\
0.103559006905646	1.42184153222313e-05\\
0.104626625533539	0.000337256041492262\\
0.105427339504459	4.42557245388109e-05\\
0.106228053475379	0.00050961117456616\\
0.107295672103272	0.000107104910455513\\
0.108096386074192	0.0007081861321796\\
0.109164004702085	0.000123430482289466\\
0.109964718673006	0.000755279969964162\\
0.110765432643925	9.80524098581625e-07\\
0.111833051271819	0.000823007742122073\\
0.112633765242738	4.13321900445407e-05\\
0.113701383870632	0.0012064007916226\\
0.114502097841552	5.56713526180808e-05\\
0.115302811812472	0.00164761357596643\\
0.115569716469445	0.00140055991570009\\
0.116103525783392	8.9182165108781e-05\\
0.116637335097338	0.000799033471166499\\
0.117171144411285	0.00201152434768215\\
0.117704953725231	0.000605233355285062\\
0.117971858382205	1.29486297264503e-05\\
0.118238763039178	0.000395338334635786\\
0.119039477010098	0.00338492021670156\\
0.119840190981018	0.000114749296588057\\
0.120107095637991	0.000725177212089534\\
0.120907809608911	0.00492357216524519\\
0.121708523579831	1.15925391288574e-05\\
0.121975428236805	0.00111471197492641\\
0.122776142207724	0.00675067306027621\\
0.123576856178644	0.00043480759280623\\
0.124110665492591	0.0117182855158933\\
0.124644474806538	0.0175954580703515\\
0.125445188777458	0.000395109982169961\\
0.125978998091404	0.0259693298903021\\
0.126512807405351	0.0423195271822019\\
0.127046616719298	0.00435051494031358\\
0.127313521376271	0.00758492452007831\\
0.127580426033244	0.0700169381062403\\
0.128114235347191	0.428853458885456\\
0.128914949318111	1\\
0.129181853975084	0.978965653730684\\
0.129715663289031	0.600060540093866\\
0.130516377259951	0.042489256760581\\
0.130783281916924	0.00353658621280983\\
0.131050186573897	0.0144415357042833\\
0.131583995887844	0.052381748889758\\
0.131850900544817	0.0454543342138174\\
0.132651614515737	0.000193571093205058\\
0.13291851917271	0.00292130120465184\\
0.133452328486657	0.0141548194756904\\
0.13371923314363	0.0122461041229291\\
0.13451994711455	0.000460051181016885\\
0.13532066108547	0.0107268726981755\\
0.135587565742444	0.00910258049532153\\
0.136388279713364	3.30363866760663e-05\\
0.137188993684283	0.00465783062401037\\
0.137455898341257	0.00378439987225354\\
0.137989707655203	0.000223654181801924\\
0.13852351696915	0.00164475667528485\\
0.139057326283097	0.00372407519795659\\
0.139858040254017	0.00012354369887313\\
0.14092565888191	0.00259793142654874\\
0.14172637285283	6.93578188215582e-05\\
0.141993277509803	0.000296295885228615\\
0.142793991480723	0.00204428857673644\\
0.143594705451643	1.88754072589781e-05\\
0.143861610108616	0.000199485477999195\\
0.144395419422563	0.00129910800239896\\
0.145463038050456	3.4893455505669e-05\\
0.14653065667835	0.00137405419880365\\
0.147331370649269	6.67469440296387e-06\\
0.148398989277163	0.000795328220045688\\
0.149199703248083	3.71591303682806e-05\\
0.150000417219003	0.000897471170834718\\
0.151334940503869	0.000338374715313172\\
0.151868749817816	0.000781321616889263\\
0.152936368445709	6.76187558306118e-05\\
0.153737082416629	0.000657454011505187\\
0.154804701044522	6.70368494035678e-05\\
0.155605415015442	0.00050126531488659\\
0.156406128986362	9.64535975866987e-06\\
0.157740652271229	0.000346079681024403\\
0.158274461585175	4.32532366922977e-06\\
0.159342080213069	0.000360513929722917\\
0.160142794183989	1.30387188232994e-05\\
0.161210412811882	0.000358813123655599\\
0.162011126782802	1.75896139920084e-05\\
0.163078745410695	0.000300704353936698\\
0.163879459381615	2.95506590635153e-05\\
0.164947078009508	0.000262608181982538\\
0.165747791980428	2.75848363189279e-05\\
0.166815410608321	0.000180790797960517\\
0.167616124579241	7.0114675727817e-05\\
0.168416838550161	0.000318008178997964\\
0.169484457178054	5.1387546993853e-05\\
0.170285171148974	0.000228234818772766\\
0.171352789776868	7.09439892521146e-05\\
0.172153503747787	0.000222394462361608\\
0.173221122375681	9.18923666133331e-05\\
0.174288741003574	9.29286526150097e-05\\
0.175356359631467	0.000181701243112897\\
0.191370639049866	0.000184578433084948\\
0.192438257677759	0.000441316747346931\\
0.193505876305653	0.000258316513111367\\
0.195641113561439	0.000914107270121978\\
0.196975636846306	0.000254395875616931\\
0.197509446160252	0.0017247150963462\\
0.199377778759066	0.0100386061881148\\
0.199911588073012	0.00826497169257268\\
0.201246111357879	0.00121397797593681\\
0.201779920671825	0.000387440590282617\\
0.203648253270639	0.000712304747895631\\
0.204982776555505	0.000149381169535001\\
0.224466816514557	3.49675174335928e-05\\
0.22873729102613	5.6534806154529e-05\\
0.233274670194676	7.01903912192492e-06\\
0.237812049363223	4.95226821242145e-05\\
0.242616333188743	1.35944100509988e-05\\
0.247420617014262	1.59989754613399e-05\\
0.260498945205955	9.04671562773629e-07\\
0.270908226827914	0.000207594294210356\\
0.2730434640837	1.56119033378754e-05\\
0.299200120467085	5.11728957763857e-06\\
0.300000834438005	2.7223879450089e-05\\
};
\end{axis}
\end{tikzpicture}%
		\subcaption{RL}
	\end{subfigure}%
	\begin{subfigure}{0.33\textwidth}
		\centering
		% This file was created by matlab2tikz.
%
%The latest updates can be retrieved from
%  http://www.mathworks.com/matlabcentral/fileexchange/22022-matlab2tikz-matlab2tikz
%where you can also make suggestions and rate matlab2tikz.
%
\definecolor{mycolor1}{rgb}{0.00000,0.44700,0.74100}%
\definecolor{mycolor2}{rgb}{0.85000,0.32500,0.09800}%
%
\begin{tikzpicture}

\begin{axis}[%
width=\figW,
height=\figH,
at={(0\figW,0\figH)},
scale only axis,
xmin=0,
xmax=0.3,
ymin=0,
ymax=1,
xlabel={$\omega$ ($2 \pi c/a$)},
axis background/.style={fill=white}
]
\addplot [color=mode1, forget plot]
  table[row sep=crcr]{%
0	3.88673264386519e-05\\
0.0587190245341291	0.000288895213929496\\
0.0605873571329423	0.000407554525789999\\
0.0635233083596487	0.000104902306033106\\
0.0659254502724087	0.000323619183853818\\
0.0683275921851685	6.37155649954035e-05\\
0.0709966387549015	0.000344282085255943\\
0.073131876010688	4.95531779189928e-05\\
0.0760678272373945	0.000488692707976579\\
0.078203064493181	6.99013508840274e-05\\
0.0811390157198875	0.000738307353089551\\
0.083274252975674	6.03439972830522e-05\\
0.0862102042023805	0.000966774114379731\\
0.0886123461151402	0.000110279287079296\\
0.0912813926848735	0.00119415637000886\\
0.09341662994066	4.18238352311029e-05\\
0.0944842485685533	0.000854902627853038\\
0.0955518671964464	0.00179604323857063\\
0.0963525811673664	0.00167486796267857\\
0.0984878184231528	6.88471203496022e-05\\
0.0992885323940729	0.00090853043528738\\
0.100623055678939	0.00288064910562102\\
0.101156864992886	0.00295635254335092\\
0.101690674306833	0.00242743530213541\\
0.103292102248673	6.95029485608956e-05\\
0.104092816219592	0.000754119973114253\\
0.106228053475379	0.00504008049988403\\
0.106761862789326	0.00432803203342691\\
0.108630195388139	0.000105426192973379\\
0.109164004702085	0.000947613905575073\\
0.109964718673006	0.00428541423423412\\
0.111032337300899	0.00845278913092984\\
0.111299241957872	0.00869319375076816\\
0.111566146614845	0.00848371994951602\\
0.112099955928792	0.00679729066571122\\
0.113701383870632	0.000108503225145329\\
0.113968288527605	0.000341980879380976\\
0.114502097841552	0.00270341483409275\\
0.116637335097338	0.01667844870554\\
0.117171144411285	0.0140440989154202\\
0.118772572353125	0.000136660284805856\\
0.119039477010098	0.000410371616097471\\
0.119573286324045	0.00544565689993037\\
0.120374000294965	0.0226680807617432\\
0.121441618922858	0.0451273515122541\\
0.121708523579831	0.0467140129595078\\
0.121975428236805	0.0458105064198076\\
0.122509237550751	0.0364590241432714\\
0.123843760835618	0.000345257283584921\\
0.124110665492591	0.00130793613966573\\
0.124377570149564	0.00885492636316787\\
0.124911379463511	0.0497441765140558\\
0.125712093434431	0.18913220418551\\
0.127046616719298	0.592313419780696\\
0.128381140004164	0.949979994686141\\
0.128914949318111	1\\
0.129181853975084	0.998233506489319\\
0.129715663289031	0.941449581289506\\
0.130516377259951	0.746188777188888\\
0.13291851917271	0.0746684419354082\\
0.13371923314363	0.00562021939329149\\
0.133986137800604	0.000309725835684782\\
0.134253042457577	0.00130392986105687\\
0.134786851771524	0.0151237426976352\\
0.135854470399417	0.0469962225820375\\
0.136388279713364	0.0500184016206293\\
0.13692208902731	0.0428974276549838\\
0.138790421626123	0.00144772637971657\\
0.139057326283097	0.000113224222431541\\
0.13932423094007	0.000352800415001298\\
0.139858040254017	0.00448413965679317\\
0.141192563538883	0.0180653974582243\\
0.141459468195857	0.0187222731595087\\
0.14172637285283	0.0183715240512414\\
0.142260182166776	0.015036217094748\\
0.143861610108616	0.000765594181906692\\
0.144395419422563	0.000160780041085884\\
0.14492922873651	0.00213476037770843\\
0.14653065667835	0.0101519854944379\\
0.147064465992296	0.00961123495512806\\
0.147865179963216	0.00574264264904767\\
0.148932798591109	0.000579080469022886\\
0.149466607905056	9.20797609047508e-05\\
0.150000417219003	0.00119982075952163\\
0.151601845160843	0.00650647183236952\\
0.152135654474789	0.00636520457224066\\
0.152936368445709	0.00403646604253405\\
0.154003987073602	0.000519664576796641\\
0.154537796387549	5.82388821335211e-05\\
0.155071605701496	0.000699153823800369\\
0.156939938300309	0.00464667197681123\\
0.157740652271229	0.00371623167266999\\
0.159342080213069	0.000174740337848256\\
0.159875889527015	0.000137813505997508\\
0.160676603497935	0.0013459868674961\\
0.161744222125828	0.00320381554906368\\
0.162278031439775	0.00335056467496297\\
0.162811840753722	0.00283162744897725\\
0.164680173352535	6.19170065134789e-05\\
0.165480887323455	0.000530729543030128\\
0.167082315265295	0.00247776964275337\\
0.167883029236215	0.00217635061654464\\
0.170018266492001	8.11407369474271e-05\\
0.170818980462921	0.000654598787708238\\
0.172153503747787	0.00189503778019473\\
0.172954217718708	0.00171486090331996\\
0.175089454974494	0.000104864785164915\\
0.176157073602387	0.000777714138052499\\
0.17722469223028	0.00153065072087433\\
0.178292310858174	0.00123067644327723\\
0.179893738800014	0.00013470216006839\\
0.180961357427907	0.000480879325755801\\
0.182562785369747	0.00134901621616046\\
0.18363040399764	0.000904366759798103\\
0.184964927282507	0.000165559268547444\\
0.1860325459104	0.000455645715565334\\
0.18763397385224	0.00121609538014744\\
0.188701592480133	0.000845320344863287\\
0.190036115765	0.000194768298143577\\
0.191103734392893	0.000438096030068946\\
0.192705162334733	0.00111986859121838\\
0.194039685619599	0.000631420797184035\\
0.195374208904466	0.000163279525361881\\
0.196708732189333	0.00077908796899373\\
0.198043255474199	0.0014169294652675\\
0.199377778759066	0.0010723485577786\\
0.200979206700906	0.000595942279109973\\
0.203915157927612	0.000620747971421531\\
0.205783490526425	0.000241965638161146\\
0.208719441753132	0.000643475537537297\\
0.210854679008918	0.00018060762274219\\
0.214057534892598	0.000513309625902902\\
0.215925867491411	0.000148477917433665\\
0.219128723375091	0.000465484630120638\\
0.221263960630877	0.000149010222779467\\
0.22393300720061	0.000476953999740459\\
0.22633514911337	0.000116826618160815\\
0.229271100340077	0.000380836597472234\\
0.231406337595863	8.5974455506399e-05\\
0.23434228882257	0.000345126849571109\\
0.23674443073533	9.30502235432229e-05\\
0.239413477305063	0.000315249265120388\\
0.241815619217822	7.2753509476664e-05\\
0.244484665787556	0.000292138733504865\\
0.247153712357289	9.32220429585851e-05\\
0.249555854270049	0.000273054242184045\\
0.252224900839782	7.76272988718407e-05\\
0.254893947409515	0.000219004705521719\\
0.257296089322275	6.44965757119476e-05\\
0.259965135892008	0.000202992908045152\\
0.262634182461741	8.58989335559279e-05\\
0.265036324374501	0.000188453043735048\\
0.267705370944234	7.55468344884047e-05\\
0.270374417513967	0.000140967971464789\\
0.272776559426727	7.74017951870043e-05\\
0.27544560599646	0.000143621887806145\\
0.278114652566193	0.000104671046546656\\
0.280783699135926	0.000115930333205894\\
0.283185841048686	9.9361801052078e-05\\
0.285854887618419	0.000115845411306381\\
0.288523934188152	0.000116789163551401\\
0.291192980757886	8.36957315482056e-05\\
0.293862027327619	0.000128088581529351\\
0.296531073897352	5.52903256878512e-05\\
0.299467025124058	0.000150828081684651\\
0.300000834438005	0.00015982821120919\\
};
\addplot [color=mode2, forget plot]
  table[row sep=crcr]{%
0	0.000134350006603778\\
0.0659254502724087	0.000546352358206104\\
0.0672599735572752	0.000460862662837735\\
0.0691283061560883	9.55790606635976e-05\\
0.0720642573827948	0.000672134212729381\\
0.0741994946385813	8.7401919838026e-05\\
0.0771354458652878	0.00075397678012834\\
0.0792706831210743	7.01283008002207e-05\\
0.0819397296908073	0.000925324458987964\\
0.0843418716035673	0.000102239839298246\\
0.0870109181733003	0.00115823206292909\\
0.0891461554290869	7.49940352624545e-05\\
0.089946869400007	0.000622957480105324\\
0.0912813926848735	0.0017744005502347\\
0.0920821066557933	0.00160304945766754\\
0.0942173439115799	0.000121089660574336\\
0.0950180578824997	0.000899620005289536\\
0.0963525811673664	0.0024763252773865\\
0.0971532951382863	0.00221721635786709\\
0.0990216277370994	6.88004995530456e-05\\
0.0998223417080195	0.00073837292708534\\
0.101423769649859	0.00335373569011055\\
0.102224483620779	0.00297396420774598\\
0.104092816219592	3.92697161286648e-05\\
0.104626625533539	0.000500658243272945\\
0.105694244161432	0.00319538613993675\\
0.106494958132352	0.00467040498185978\\
0.107028767446299	0.00449833231831542\\
0.107829481417219	0.00259624950391957\\
0.108897100045112	7.86111653929833e-05\\
0.109430909359059	0.000310584416365334\\
0.109964718673006	0.00180181468622176\\
0.111566146614845	0.00739135487676301\\
0.112099955928792	0.00694240725090878\\
0.112900669899712	0.0037230043267944\\
0.113701383870632	0.000412414406023132\\
0.113968288527605	1.85220806636632e-05\\
0.114235193184578	0.000176071948576606\\
0.114769002498525	0.00225987597121224\\
0.116637335097338	0.0147183810942626\\
0.116904239754312	0.0145579495701427\\
0.117438049068258	0.0118070136074451\\
0.119039477010098	2.465222358361e-05\\
0.119306381667071	0.000947501410340168\\
0.119840190981018	0.00718090974939378\\
0.121975428236805	0.0438753132707257\\
0.122509237550751	0.0364499498612483\\
0.124110665492591	0.000622739923811677\\
0.124377570149564	0.00675661841620934\\
0.124911379463511	0.0442116066621001\\
0.125712093434431	0.178829630884551\\
0.127046616719298	0.581835543870001\\
0.128381140004164	0.947534486194421\\
0.128914949318111	1\\
0.129181853975084	0.998877675699094\\
0.129715663289031	0.942079961332503\\
0.130516377259951	0.744211160416836\\
0.132651614515737	0.11315772679601\\
0.133452328486657	0.0164085403245755\\
0.133986137800604	7.10423854644038e-05\\
0.134253042457577	0.00195643439784354\\
0.135053756428497	0.0262844492153576\\
0.135854470399417	0.0474938857403329\\
0.13612137505639	0.0496993558829111\\
0.136388279713364	0.0491230041958695\\
0.13692208902731	0.0408526036414969\\
0.138790421626123	0.000795706382213268\\
0.139057326283097	5.41550979704652e-06\\
0.13932423094007	0.000710933427661375\\
0.14012494491099	0.00851816587544341\\
0.14092565888191	0.0164739381569738\\
0.141459468195857	0.0178341636514858\\
0.141993277509803	0.0154444100365945\\
0.143861610108616	0.000286117622562365\\
0.14412851476559	1.36753117139232e-05\\
0.144395419422563	0.000408616499049996\\
0.145196133393483	0.00428969739499285\\
0.146263752021376	0.00902659670834471\\
0.14653065667835	0.00917987763214145\\
0.147064465992296	0.00804337300542679\\
0.149199703248083	1.9707078568576e-05\\
0.149733512562029	0.000845586908188922\\
0.151334940503869	0.00541018781790625\\
0.151868749817816	0.00530878162286963\\
0.152669463788736	0.00332930208959281\\
0.153737082416629	0.000382642886062268\\
0.154270891730576	2.91624250183808e-05\\
0.154804701044522	0.00058806056570293\\
0.156406128986362	0.00353191533932695\\
0.157206842957282	0.00313738891295468\\
0.159342080213069	5.36897724505714e-05\\
0.159875889527015	0.000466589778830073\\
0.161477317468855	0.00245063100627285\\
0.162278031439775	0.00213316748429437\\
0.164146364038588	6.73618161206591e-05\\
0.164947078009508	0.000415498892222477\\
0.166548505951348	0.00180231145694543\\
0.167349219922268	0.00151769174651983\\
0.169217552521081	8.40223087013037e-05\\
0.170285171148974	0.000626232967441132\\
0.171619694433841	0.00141558579819789\\
0.172687313061734	0.000945225650540671\\
0.174021836346601	0.000114312445260945\\
0.175089454974494	0.00041344848499425\\
0.176690882916334	0.00116429745160396\\
0.178025406201201	0.000549959079481965\\
0.179093024829094	9.4587078521613e-05\\
0.180160643456987	0.000403374882612662\\
0.181762071398827	0.000995137993678785\\
0.183363499340667	0.000299459587668416\\
0.18443111796856	0.000105410871372502\\
0.187367069195266	0.000746947820029709\\
0.189502306451053	0.000103043392124524\\
0.192438257677759	0.000689519088927781\\
0.194573494933546	8.11926997945633e-05\\
0.198043255474199	0.000898373506712069\\
0.200178492729986	0.000580536876829685\\
0.202313729985772	0.000510478358584754\\
0.204448967241559	0.000109648218290737\\
0.207918727782212	0.000336144282479856\\
0.209787060381025	0.000151452203610569\\
0.212456106950758	0.000387339872801595\\
0.214858248863518	0.000162692367967177\\
0.217527295433251	0.000333407369542016\\
0.219929437346011	0.000153228801843541\\
0.222331579258771	0.000331965315724325\\
0.225000625828504	0.000129305397597923\\
0.227402767741264	0.000293873253189636\\
0.230071814310997	0.000105626887377586\\
0.23274086088073	0.000218607833060824\\
0.23514300279349	9.1631430102268e-05\\
0.237812049363223	0.000184056413309541\\
0.240214191275983	8.76970342735817e-05\\
0.242883237845716	0.000156349295140634\\
0.245552284415449	0.000117946598475971\\
0.247954426328209	0.000136546725246101\\
0.250623472897942	0.000117511195575792\\
0.253292519467675	9.42269177877098e-05\\
0.255961566037408	0.000139284877229029\\
0.258630612607141	5.98940084699517e-05\\
0.261566563833848	0.000163126341109265\\
0.27864846188014	4.33498806224897e-05\\
0.285321078304473	4.38910115891922e-05\\
0.291459885414859	9.86946118051169e-05\\
0.300000834438005	2.33733493741894e-05\\
};
\end{axis}
\end{tikzpicture}%
		\subcaption{LR}
	\end{subfigure}%
	\begin{subfigure}{0.33\textwidth}
		\centering
		% This file was created by matlab2tikz.
%
%The latest updates can be retrieved from
%  http://www.mathworks.com/matlabcentral/fileexchange/22022-matlab2tikz-matlab2tikz
%where you can also make suggestions and rate matlab2tikz.
%
\definecolor{mycolor1}{rgb}{0.00000,0.44700,0.74100}%
\definecolor{mycolor2}{rgb}{0.85000,0.32500,0.09800}%
%
\begin{tikzpicture}

\begin{axis}[%
width=\figW,
height=\figH,
at={(0\figW,0\figH)},
scale only axis,
xmin=0,
xmax=0.3,
ymin=0,
ymax=1,
xlabel={$\omega$ ($2 \pi c/a$)},
axis background/.style={fill=white},
legend pos=north west
]
\addplot [color=mode1]
  table[row sep=crcr]{%
0	1.55493558309239e-05\\
0.128648044661137	0.000291401836858496\\
0.130783281916924	0.000272579306686227\\
0.133185423829684	0.000132906235705388\\
0.135854470399417	0.000282632640331348\\
0.137989707655203	0.000107228709113727\\
0.14092565888191	0.000354175823229053\\
0.143060896137696	0.000107132078084549\\
0.145729942707429	0.000529848463097338\\
0.148132084620189	0.00011034827748313\\
0.150801131189922	0.000676029799295597\\
0.153203273102682	8.6709076011271e-05\\
0.155872319672415	0.000725615531831814\\
0.158007556928202	6.95901860032766e-05\\
0.159342080213069	0.000872165130341385\\
0.160409698840962	0.00136635364920878\\
0.161477317468855	0.000885334125582116\\
0.162811840753722	5.24261914098823e-05\\
0.163612554724642	0.000370662827645418\\
0.165747791980428	0.00213950279486941\\
0.166548505951348	0.00151524255748625\\
0.167883029236215	9.99298748551869e-05\\
0.168683743207134	0.000451329014977686\\
0.171085885119894	0.00343490754811349\\
0.171886599090814	0.00224906255381763\\
0.172954217718708	0.000302554660580823\\
0.173488027032654	0.00017879873144433\\
0.174021836346601	0.000919156488057515\\
0.176157073602387	0.00543332780202976\\
0.176690882916334	0.00464728239889034\\
0.178559215515147	0.000121828812656988\\
0.179093024829094	0.000933252851363031\\
0.179893738800014	0.00425022274790998\\
0.180961357427907	0.00837555305559978\\
0.18122826208488	0.00860274235849001\\
0.181495166741854	0.00838172254753555\\
0.1820289760558	0.00668435077258689\\
0.18363040399764	9.27083512791693e-05\\
0.183897308654613	0.000346171492089375\\
0.18443111796856	0.00272661757263548\\
0.186299450567373	0.016515440220733\\
0.18683325988132	0.0155673636769622\\
0.18763397385224	0.00861458379711411\\
0.18843468782316	0.00110254813881161\\
0.188701592480133	0.000130329011890584\\
0.188968497137106	0.000447910380210859\\
0.189502306451053	0.00555166721986233\\
0.190303020421973	0.0228470182196445\\
0.191370639049866	0.0455482939349976\\
0.19163754370684	0.0472468463701474\\
0.191904448363813	0.0464698105071626\\
0.192438257677759	0.0373595815668704\\
0.193772780962626	0.000582682443470706\\
0.194039685619599	0.000953620516230158\\
0.194306590276573	0.00770275184263558\\
0.194840399590519	0.0463805201203464\\
0.195641113561439	0.181350090175954\\
0.196975636846306	0.5789769997712\\
0.198310160131172	0.942252496575551\\
0.198843969445119	0.99825962608984\\
0.199110874102092	1\\
0.199644683416039	0.950558949856845\\
0.200445397386959	0.764740867950573\\
0.202847539299719	0.0874677020756052\\
0.203648253270639	0.00940986069280059\\
0.204182062584585	0.000290886130887014\\
0.204715871898532	0.0111964738588415\\
0.206050395183399	0.0469218182207407\\
0.206317299840372	0.0480789686741683\\
0.206584204497345	0.0466769059702812\\
0.207118013811292	0.0376257976481291\\
0.208719441753132	0.00250926353509517\\
0.209253251067078	6.69887869633179e-05\\
0.209520155724052	0.000969587975592434\\
0.210320869694971	0.00864972503415906\\
0.211121583665892	0.0160331112409842\\
0.211655392979838	0.017175408231457\\
0.212189202293785	0.0148192128118578\\
0.214057534892598	0.000333467470762461\\
0.214324439549571	4.02818042173347e-05\\
0.214591344206545	0.000380854134824471\\
0.215125153520491	0.00253402178169493\\
0.216459676805358	0.00876959050175152\\
0.216726581462331	0.00900869958947359\\
0.216993486119304	0.00877051552680541\\
0.217527295433251	0.00705748228004466\\
0.219128723375091	0.000272039889200215\\
0.219662532689038	0.000179347459904555\\
0.220196342002984	0.00137371627138227\\
0.221530865287851	0.00534614050185867\\
0.222064674601797	0.00552290605915928\\
0.222598483915744	0.0045960448969713\\
0.224466816514557	4.79648292748269e-05\\
0.225267530485477	0.000821715458200867\\
0.226868958427317	0.00380924974414776\\
0.227669672398237	0.00328263718586364\\
0.22953800499705	5.92327265036552e-05\\
0.23033871896797	0.000517587296184674\\
0.23194014690981	0.00275334142803629\\
0.23274086088073	0.0024778958078675\\
0.234876098136516	3.92773102042554e-05\\
0.23594371676441	0.0009665185658283\\
0.237011335392303	0.00207579350585685\\
0.237812049363223	0.00194892818245229\\
0.240214191275983	8.43691858622986e-05\\
0.241281809903876	0.000969422628691374\\
0.242349428531769	0.00168286048158128\\
0.243417047159662	0.00117788989713086\\
0.244751570444529	8.9212386712445e-05\\
0.245819189072422	0.00030498939111645\\
0.247687521671236	0.00135155492492922\\
0.248755140299129	0.000797963957217762\\
0.250089663583995	3.6908011647574e-05\\
0.251157282211889	0.000367919546328865\\
0.252758710153729	0.00110230415849455\\
0.254093233438595	0.0005246527385665\\
0.255427756723462	3.20985488073688e-05\\
0.257029184665301	0.000686157614896654\\
0.258096803293195	0.000878742506423391\\
0.261299659176875	0.000213993016872305\\
0.263167991775688	0.000724239501030954\\
0.266103943002394	0.000184181060117083\\
0.268506084915154	0.000967059785361934\\
0.271975845455807	0.000219003733829215\\
0.27384417805462	0.000384297727108684\\
0.276513224624353	8.83362682626565e-05\\
0.278915366537113	0.000338967445775884\\
0.281584413106846	8.81737632429935e-05\\
0.283986555019606	0.000312097828071956\\
0.286655601589339	8.86890943270213e-05\\
0.289057743502099	0.000300067701469731\\
0.291726790071832	8.27004019099409e-05\\
0.294395836641566	0.000256229286488718\\
0.296797978554325	7.07704273552601e-05\\
0.299467025124058	0.000251704428123389\\
0.300000834438005	0.000162100326521086\\
};
\addlegendentry{$D_1$};
\addplot [color=mode2]
  table[row sep=crcr]{%
0	6.28028652625012e-05\\
0.129982567946004	0.000394692742722524\\
0.131583995887844	0.000698770716094366\\
0.134786851771524	0.00019868256087352\\
0.136655184370337	0.000666974065289372\\
0.139858040254017	0.000256735068325575\\
0.14172637285283	0.000756975321301256\\
0.144662324079536	0.000212968754511822\\
0.14653065667835	0.000885190533250979\\
0.148132084620189	0.000221472399044798\\
0.149199703248083	8.65124002005224e-05\\
0.151868749817816	0.00115568318147807\\
0.154270891730576	0.000157772642941367\\
0.155338510358469	0.00101554189682829\\
0.156406128986362	0.00173331044670522\\
0.157206842957282	0.00147117562233334\\
0.159075175556095	6.96388803462789e-05\\
0.159875889527015	0.000516739486775775\\
0.161477317468855	0.00201506871342882\\
0.162278031439775	0.00171224416975724\\
0.164146364038588	5.78575439533768e-05\\
0.164947078009508	0.000655860005683184\\
0.166548505951348	0.00252726291218996\\
0.167349219922268	0.00210893713455218\\
0.169217552521081	4.73513166596717e-05\\
0.169751361835028	0.000455705499391668\\
0.171619694433841	0.0031529830570276\\
0.172420408404761	0.00251171360091851\\
0.174021836346601	1.91378927785202e-05\\
0.174822550317521	0.000780425841003352\\
0.176690882916334	0.0043952872615558\\
0.17722469223028	0.00384278328508025\\
0.179093024829094	2.19478635807846e-05\\
0.17962683414304	0.000873347255885726\\
0.180694452770934	0.00521815762812161\\
0.181495166741854	0.00752277729454254\\
0.1820289760558	0.00717645931652644\\
0.18282969002672	0.00398844381761898\\
0.183897308654613	4.25200507609969e-05\\
0.18443111796856	0.000826541263138969\\
0.184964927282507	0.00396170238662585\\
0.186566355224347	0.0154554640332349\\
0.187100164538293	0.0146102331768789\\
0.187900878509213	0.00786768019001527\\
0.188701592480133	0.000765609664994615\\
0.188968497137106	1.00751503933072e-05\\
0.18923540179408	0.000583158450087495\\
0.189769211108026	0.0062465476170126\\
0.19083682973592	0.0307335311685133\\
0.19163754370684	0.044838626116368\\
0.191904448363813	0.0455642956425175\\
0.192171353020786	0.0437085502056458\\
0.192705162334733	0.0326095155633743\\
0.194039685619599	8.29826755261998e-05\\
0.194306590276573	0.00426705372360048\\
0.194840399590519	0.0368891203464228\\
0.195641113561439	0.163536508091964\\
0.196708732189333	0.470279547957771\\
0.198310160131172	0.936783366445699\\
0.198843969445119	0.997055164766234\\
0.199110874102092	1\\
0.199377778759066	0.984354536839983\\
0.199911588073012	0.9008230433819\\
0.200979206700906	0.587712085592871\\
0.202580634642745	0.123557976784367\\
0.203381348613665	0.0201415548995225\\
0.203915157927612	0.000406488651291115\\
0.204182062584585	0.00108835887659708\\
0.204715871898532	0.0145567313328479\\
0.205783490526425	0.0458145556500049\\
0.206050395183399	0.0485122389622314\\
0.206584204497345	0.0458241697346764\\
0.207384918468265	0.0276969588192231\\
0.208452537096158	0.00373766110674989\\
0.208986346410105	2.24533793631299e-05\\
0.209253251067078	0.000486565351071278\\
0.209787060381025	0.00475962450493927\\
0.211121583665892	0.0170058642777724\\
0.211388488322865	0.0173118308292897\\
0.211655392979838	0.0166691715286995\\
0.212189202293785	0.013045117629338\\
0.213790630235625	0.000327698357917239\\
0.214057534892598	1.6912870248742e-06\\
0.214324439549571	0.000333045662057163\\
0.214858248863518	0.00251401078911795\\
0.216192772148385	0.00868642775459749\\
0.216459676805358	0.00885659243967174\\
0.216993486119304	0.00778433919550259\\
0.219128723375091	1.08968142209509e-05\\
0.219662532689038	0.000835543978102926\\
0.221263960630877	0.00529807561889384\\
0.221797769944824	0.00516775764115707\\
0.222598483915744	0.003184733474213\\
0.223666102543637	0.000317415799661047\\
0.224199911857584	2.24659696721474e-05\\
0.224733721171531	0.000632607768083604\\
0.22633514911337	0.00357108285710717\\
0.22713586308429	0.00310004865283231\\
0.229004195683103	1.9159038553207e-05\\
0.229804909654023	0.000509891041929667\\
0.231406337595863	0.00256350123591442\\
0.232207051566783	0.00217146493457077\\
0.234075384165596	8.57470921800108e-06\\
0.234876098136516	0.000433689156983164\\
0.236477526078356	0.0019350223537693\\
0.237278240049276	0.00159491485970564\\
0.239146572648089	7.41945127402666e-06\\
0.240214191275983	0.000620431913449071\\
0.241548714560849	0.00151250505035638\\
0.242616333188743	0.00099358717544229\\
0.243950856473609	3.53715982044367e-05\\
0.245018475101502	0.000335135952300725\\
0.246619903043342	0.00120572852227885\\
0.247687521671236	0.000760124777400062\\
0.249022044956102	2.13407546516553e-05\\
0.250089663583995	0.000296490869482025\\
0.251691091525835	0.000967715917518985\\
0.253025614810702	0.000422026775932727\\
0.254093233438595	1.05853251659571e-05\\
0.255427756723462	0.000402060862756404\\
0.256762280008328	0.000786201423783384\\
0.260232040548981	0.000274503218366862\\
0.261833468490821	0.000689916951627456\\
0.265036324374501	0.00023287435433339\\
0.266904656973314	0.000956960594683309\\
0.268239180258181	0.000576948236111097\\
0.269840608200021	8.73507236096582e-05\\
0.278114652566193	0.000142858287752068\\
0.279982985165006	0.000112473223961018\\
0.282118222420793	0.000326971236328966\\
0.285054173647499	0.000110449333495799\\
0.287189410903286	0.000291454661272006\\
0.290125362129992	0.000100987672789321\\
0.292527504042752	0.000210638371687333\\
0.294929645955512	6.09022446864671e-05\\
0.297598692525245	0.000172780313424825\\
0.300000834438005	6.53213967842792e-05\\
};
\addlegendentry{$D_2$};
\end{axis}
\end{tikzpicture}%
		\subcaption{TR}
	\end{subfigure}
	\caption[Fourier components of the optical AB effect]{Frequencies of the optical Aharonov-Bohm effect for detector $D_1$ at $1.5 \mu m$ and detector $D_2$ at $58 \mu m$ for \textbf{(a)} left to right \textbf{(b)} right to left, and \textbf{(c)} time-reversed directions. Note that for all graphs, both $D_1$ and $D_2$ are shown, however there is an overlap in the LR and TR paths.}
	\label{fig:final}
\end{figure}


\begin{table}[b]
	\centering
	\begin{tabular}{|l|l|l|}
		\hline
		Direction & Input         & Output         \\
		\hline
		Left to right (LR) & $\ket{1}$  & $\ket{2}$  \\
		Right to left (RL) & $\bra{1}$ & $\bra{1}$ \\
		Time-reversed (TR) & $\bra{2}$ & $\bra{2}$ \\
		\hline
	\end{tabular}
	\caption[Summary of mode transitions in the photonic AB effect.]{Summary of mode transitions in the photonic AB effect.}
	\label{tab:modess}
\end{table}

 