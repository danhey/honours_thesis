% These \phantomsection are to ensure that the hyperref package hyperlinks to the correct page in the electronic pdf. If you turn hyperref off they don't do anything so they can just stay here.
\phantomsection\addcontentsline{toc}{chapter}{Abstract}
\chapter*{Abstract}

The principle of reciprocity ensures that the transfer of light is symmetric between any two points in both space and time. Non-reciprocal devices that are essential to modern photonics rely exclusively on magneto-optic materials, severely limiting their applicability. In this thesis, it is shown that the influence of a time-dependent medium on light can be described in terms of a gauge potential that is directly connected to the electromagnetic properties of the medium itself. This potential arises from the breaking of reciprocity, and is demonstrated through finite difference time and frequency domain simulations. It is shown how such gauge potentials can be exploited to demonstrate the Aharonov-Bohm effect for light, and a broadband optical isolator based on this technique is designed and numerically validated, showing complete non-reciprocal frequency conversion. For the isolator demonstrated here, an even mode is completely converted to an odd mode in the left to right direction whilst counter-propagating modes are not affected. In doing so, implementations of vastly more efficient finite difference methods are presented, with a two-fold order of magnitude improvement over time-domain simulations. By suitably tailoring the non-reciprocal phase, an emerging gauge field for light can be created which perfectly mimics the dynamics of charged particles in real magnetic fields. Such artificial fields allow for the demonstration of quantum effects typically associated with electrons, and as has been recently shown non-trivial topological properties of light.


%The movement of waves obeys the fundamental principle of reciprocity, that governs symmetry in energy transmission between any two points in space and time. Non-reciprocal devices that are essential to modern photonics rely exclusively on magneto-optic materials, severely limiting their applicability. We show that the manipulation of light flow via synthetic gauge fields emerging from dynamic media

% The manipulation of light flow via effective magnetic fields arising from nonreciprocal phase shifts shows promising results and potential widespread adoption in integrated photonics. It is shown that the influence of a moving medium on light can be described in terms of a vector potential that is directly connected to the velocity and electromagnetic properties of the medium itself. This formulation gives rise to optical analogues of the standard electromagnetic fields, such as an optical Lorentz force, causing photons to travel in a manner similar to charged particles in electric fields. Additionally, a nonreciprocal phase shift is shown to be imparted to light as it travels through a path-dependent moving medium. This report will also discuss current efforts in the understanding and synthesis of optical nonreciprocity outside of bulk magneto-optic materials, as achieved through media with a time-dependence, which forms a crucial step in the realisation of fully integrated photonic devices.

%Photons are weak particles that do not directly couple to magnetic fields. However, it is possible to generate a photonic gauge field by breaking reciprocity such that the phase of light depends on its direction of propagation. This non-reciprocal phase indicates the presence of an effective magnetic field for the light itself. By suitable tailoring of this phase it is possible to demonstrate quantum effects typically associated with electrons, and as has been recently shown, non-trivial topological properties of light. This paper reviews dynamic modulation as a process for breaking the time-reversal symmetry of light and generating a synthetic gauge field, and discusses its role in topological photonics, as well as recent developments in exploring topological photonics in higher dimensions.